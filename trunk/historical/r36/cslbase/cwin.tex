\documentclass[a4paper,11pt]{article}
\title{The CWIN-2 Windows Interface Code}
\author{A C Norman}
\begin{document}
\maketitle

\section{Introduction}
This document is not too concerned about internal data structures or
programming issues, but will attempt to document the behaviour I will
implement and in some cases explain why I did things that way.  The
explanation is needed because while writing the code (and its predecessors)
I found it hard to make scrolling, selection, cut \& paste operations,
font changes, user input and program output all interact properly.


First the fundamentals.  I will support a window that always has both
horizontal and vertical scroll bars visible.

To start with it will contain just text, but parts of that will be
shown in differing colours to mark out where there are prompts. I will
allow for the possibility that in a later version the text in the buffer
can be displayed in various fonts, sizes and with various effects.
The unit of display will be a ``line'' which is represented as a block
of characters and all effects withing the block will be indicated by
embedding control sequences within it. Each line will start off in
a standard font configuration.

The user types in characters, uses the DELETE key, activates scroll bars,
uses the mouse to re-position the caret and/or establish a selection
region, and invoked CUT and PASTE operations.

There is a program running that will, from time to time, request characters
from the user. Before doing so it can select the prompt that should be used.
It can also generate output. There is no automatic supposition about how
long the program delays between I/O requests and how it interleaves reads
and writes.

The stored text lives in a buffer that has finite size, and so after a
program has done enough the buffer will fill up and parts of it will need to
need to be recycled.

\section{Scrolling under user control}
If the whole of the text buffer can fit on within the window vertically
the the vertical scroll bar has no effect, and any attempt to position
it to anywhere other than the ``top'' position has no effect. If there are
more lines of text than will fit on the screen at once the vertical
scroll bar (and related short-cut keys, viz page-up, page-down, home and
end) cause the window to scroll. Scrolling is limited so that at one
extreme the first line of input text is visible at the top of the
window and at the other extreme the final line of text is just fully visible
at the bottom of the window.

If the collection of lines displayed in the window are all short enough
to fit completely across it then the horizontal scroll position is forced
to its home (leftmost) position and attempts to move it have no effect.
Otherwise the user may scroll horizontally up to an extreme where the
longest line that lies within the window comes just half way across the
window.

If the window size of font is changed scrolling can be activated to keep
the conditions described above true.  In particular this means that when
the window is enlarged it will scroll to keep the last line of the text
visible as the last line in the window, and to keep horizontally scrolled
information visible at least half way across the window. Resizing the
window will update the position the the scroll-bar thumbs to reflect
what has happened. There can also be scrolling caused by input and output
operations, but that is discussed in a moment under the heading AUTO-
SCROLLING.

The END operation moves the caret to an extreme end of the
text and scroll the window to make it visible.  HOME scrolls the window
to show the very top of the text, but does not re-position the caret.

\section{Use of the mouse to select regions of text}
The caret is considered to lie between a pair of characters (or right at
the very start or very end of the text). Text forming a ``prompt'' is treated
as representing a single item, and so the mouse can not select a position
within the prompt text. Pressing the left mouse button causes the caret to
be moved to the gap nearest to the mouse position. The place where the mouse
button was depressed becomes the root of a possible selection region.
From when the mouse button was depressed to when it is released the mouse
is captured by the window, in the sense that even if the mouse moves outside
the window's borders its activity is still tracked. If the mouse is moved
with the button down a region of text becomes selected. The text in that
region is displayed in a form of inverse video (exact details depend on the
windows colour scheme in force). If the mouse is dragged above or below the
window while making a selection the window is scrolled in the relevant
direction, and the speed of scrolling increases if the mouse remains outside
the window for a while.  If the mouse moves to the left of the window and
the window has already been scrolled horizontally it will be gradually
restored. If the mouse goes to the right of the window scrolling will be
activated if the line that the mouse is on extends beyond the right hand
border of the current window.

Pressing the (left) mouse button with SHIFT held down results in a selection
being extended from its original root to the new mouse position. Thus
clicking in one place and then shift-clicking in another sets up a selection
from the first to second place.  

The shift-click operation relies upon a previously set selection root. If none
has been established before then shift-click behaves just like an ordinary
click. Once set by an initial click a selection root remains valid until
another mouse click (which re-positions it) or until buffer wrapping deletes
some characters from the buffer (eventually almost any input or output
operation will cause this) or until the user deletes a character from the
text using the DELETE key.  CUT operations can leave a selection root
set at the place where the removed material used to be.

The COPY operation copies all characters from the selection to the clipboard.
If control characters are present they are sent to the clipboard without
adjustment. Lines sent to the clipboard have CR/LF at their end as
termination.  CUT acts as COPY but then the selected region is deleted.
RE-INPUT performs a COPY and then PASTEs the material into the input
buffer(but because it uses COPY it places the selected text into the clipboard
on the way).  SELECT-ALL does what it says (and does not move the caret).
CLEAR throws away all text in the display buffer.  Immediately after a CUT
operation the UNDO\footnote{For a first version of this interface I will
not implement UNDO.} button will re-position the caret to the place where
material was deleted and will perform a PASTE. Only one level of CUT can
be UNDOne.
A DELETE when a selection has been set up will delete the selection (but not
do the COPY operation that a CUT would). Part of an effect will be that
a DELETE may not keep quite so much stuff for UNDO as would a CUT.

COPY operations can fail if the clipboard can not be opened or if it
proves impossible to allocate windows global memory for the text that was to
be copied. In such cases the fact that the COPY failed will not be indicated
to the user, and in the case of a CUT it will not be possible to UNDO the CUT.


\section{Program-generated output}
This is always inserted at the end of the buffer. If the caret is at the end
of the text it moves as new text is inserted, and in this case if the caret
starts off visible the screen is scrolled automatically to keep the caret
position visible. This may involve horizontal as well as vertical scrolling,
but an attempt will be made to delay auto-scrolling so that it does not happen
too often.

Action has to be taken when the buffer becomes full. The buffer can be ``full''
either because the maximum number of characters have been stored in it or
because it holds the maximum number of lines that it is configured for
(at present I will allow for 64K characters and 2K lines).  In normal
circumstances when the buffer overflows the oldest whole line of text
stored in it is discarded. This policy is modified in various circumstances:
\begin{enumerate}
\item If the buffer contains only one (incomplete) line of text then the 
    system throws the entire buffer contents away. If there were unbalanced
    control sequences in the buffer the subsequent display may be mangled,
    for instance by being in the wrong colour or font. 
    This case can only arise if the program being run keeps
    printing large numbers of characters without an intervening newline.
    I really hope this circumstance does not arise! I specify this extreme
    behaviour because it is hard to indicate a safe way of abandoning a
    part-line if I intend in due course to put elaborately interpreted
    in the buffer. A possible safer refinement here would be to discard any
    further characters inserted into the buffer until a newline. But at
    present it does not seem worth while legislating for such a desparate
    and (I hope) unusual situation.
\item If the line that is to be discarded overlaps with a region of text that
    has been selected the system pauses, changing the title bar of the window
    to alert the user to what has happened. This delay is so that the user can
    complete a COPY operation to preserve line that is about to be lost, or
    a PRINT (for the same reason).  When the selection is cancelled for any
    reason the line will become vulnerable and output can continue.
\item If the line for that is to be deleted is not involved in a selection
    but it is visible on the screen and the end of the text (ie the insertion
    point) is not visible, then again the system pauses. This allows for the
    case where a user has scrolled up the display to inspect an early part of
    a transaction, and would not like it to vanish untill the scroll back down
    to the end.  I allow the line to be deleted anyway in the case that all
    the lines of input are visible on the screen: this case can only arise
    when at least one of the lines of output is amazingly long. This is
    because the user could not then scroll the window to give the system a
    hint that it should proceed.
\end{enumerate}

If output is generated and after it has been put on the screen the window
is not scrolled down as far as it will go then output will need to update the
vertical scroll thumb's position. For instance if before the output was
generated the scrollbar thumb was at the 50\% position, afterwards it will
typically be higher up. Discarding old text can also influence the
position where the thumb should be shown.

\section{Keyboard and PASTE input}
If there is a selection active then DELETE or BACK\footnote{On my keyboard
this is a key marked with a left-arrow, positioned above the ENTER key, and
is the one that I usually use to delete characters.} deletes it, and has no
other effect. Otherwise BACK deletes the character before the caret, or if that is the end of a
prompt it deletes the whole prompt. DELETE deletes the character (or prompt)
after the caret. If the caret is at the start of the text then BACK does
nothing, while if it is at the end DELETE does nothing.
After a consecutive series of DELETE operations an UNDO will re-insert the
whole block of deleted characters, but there will be
a limit to the number of characters that will be stored, so this can only
be relied upon for small operations.

The effect of keyboard input depends on whether the caret is at the
end of the text or within the buffer. To a more minor extent it also
depends on whether the caret (while within the text as a whole) identifies
a position within a currently active input line.

When the caret is not at the end of the text, normal characters are inserted
before the caret. The region covered by an insert is recorded so that after
a PASTE operation UNDO will discard what was inserted, while after
individually typed characters each UNDO will delete either a block of
characters typed without repositioning the caret or a block of characters
that end at a newline. As usual there will be a limit to how much undoing can
be done, and the exact rule for where that limit is is not documented and
nobody should ever rely on it.

Inserting characters into the middle of the text like this might overfill the
buffer, and in that case the first line of the buffer gets discarded to make
space. There will be no delay in this even if the first line forms part of
a selection (the starting point of the selection will move to the start of the
next line, and maybe the selection will vanish totally). There will be a
special case if the caret is in the first line when the user tries to insert
more - in this case the insertion will fail. If it is from the keyboard
the system beeps and discards the single offending character. If it is
part of a PASTE then characters are ignored up to the next newline in the
pasted text, which is notionally inserted and then the resulting first line
is then thrown away. I guess in this case it means that if the caret is right
at the start of the buffer all the inserted stuff gets abandoned.

Characters inserted at (or in some cases near) the end of the text can be
used as program input. If the caret is at the end of the text typed-in
chaarcters are placed in a type-ahead buffer until the program requests
a line of input. When that happens characters are accepted from the type-ahead
buffer (and/or the PASTE source) and echoed to the screen until a newline is
seen. If the user types a newline at the end of the input line the characters
in it are moved to a program-input-buffer which is where the program reads
them from.  If the user re-positions the caret and inserts a newline into the
middle of the input line then pre-typed characters in the line but after
where the newline was get pushed back into the type-ahead buffer (and if that
overflows they are lost with a beep). The effect is that the program gets
one line at once and when that line is placed in its input buffer it will just
have been echoed to the screen.  The program-input-buffer will have limited
length and truly ridiculously long input will be silently truncated when
moved into it.  I will feel entitled to reject input activity that I notice
creating an input line that is longer then that limit.

An elaboration on this explanation is that part of the final line in the
text buffer can be an incomplete input line. This can start part way along
the line (eg it will tend to start after the displayed prompt). After various
CUT and DELETE operations or when the program requests input after printing
a line that was not terminated it can start well along the final line. An
incomplete line is created when the program requests a line of input. When
the user inserts a newline into the incomplete line it becomes complete, its
contents are moved elsewhere and there is no longer an incomplete line.

When a PASTE operation copies material into the middle of a document any
prompts are inserted. But if then some of that line is moved out to the
program-input-buffer prompts are discarded during the move. If PASTE puts
stuff right at the end of the buffer it omits any prompts in the pasted
stuff.  But the start of each line of input that is echoed will get a fresh
prompt displayed on it.

When the program that is being run is halted waiting for input and the
screen has been scrolled such that the end of the buffer the window title is
changed to ``waiting for input''.

Pressing any key or performing a PASTE operation always scrolls the window
to make the caret visible. The caret can only have become invisible as a
result of a user-initiated scroll request (or HOME) since except when such a
request has hidden it the window scrolls automatically to keep it visible.

Note that the rules given here indicate that characters are only inserted into
the buffer at two distinct places: where the caret is and at the end of the
buffer. So the implementation can survive if it just caches information about
those two positions.

\section{UNDO --- a summary}
There is an undo buffer that can store a limited number of characters
and a limited number of transactions. A transaction identifies a caret
position or a range within the text, an possibly a sequence of associated
characters:
\begin{enumerate}
\item After a PASTE that happened within the body of the text and did not
    terminate an input line an UNDO discards the inserted material;
\item After a PASTE that put one or more newlines into the input area no
    UNDO will be possible (because some of the inserted text has been passed
    on to the appication code to process);
\item After a CUT followed possibly by operations that move the caret an UNDO
    re-positions the caret and does inserts characters as for a PASTE (but
    that paste is not itself undoable);
\item After a sequence of DELETE keys have been pressed an UNDO will re-insert
    the deleted characters. It can re-instate deleted prompts.
\item Sequences of non-delete characters are collected up to the point of
    a newline. If the newline causes transmission of the characters to
    the program no UNDO is possible. Otherwise each block up to a newline
    is an UNDO unit.
\item Previously stored UNDO operations can become invalid if they overlap
    with a non-undo-able operation or if the text that they relate to gets
    abandoned as the main text buffer rolls, or if the undo stack becomes
    over-full. [Is this hard to implement reliably?]
\end{enumerate}

\section{Auto-scrolling --- a summary}
If the user never re-positions the caret it will remain at the extreme end of
the text buffer. And scrolling will occus to keep it visible because
\begin{enumerate}
\item Program output will occurs with the caret at the end and visible;
\item Input requests will be handled with the caret at the end of the buffer.
\end{enumerate}

If the user scrolls the window back (using the scroll bars) so that the caret
becomes invisible then scrolling on output will cease, but the screen will
jump to make the insertion point visible if a key is pressed and the
program is requesting input (if the program is NOT requesting input yet the
key-stroke just goes in a type-ahead buffer and nothing special happens).

If the user has moved the caret to other than at the end of the buffer
then the window is never scrolled by cwin, but in cases when it might
be interesting to scroll it the title text of the window is updated to
give the user a clue to that fact.

\section{Print and other operations}
The regular PRINT item on the menu should just print the whole of the
contents of the text buffer. It will apply a fixed with limit and truncate
any material that spills off to the right. It will pack lines onto pages
in a simple-minded manner.  PRINT-SELECTION behaves the same way but
only processes text within the selected region of text.
The READ menu operation will insert text of the form
\begin{verbatim}
    IN "<filename>";
\end{verbatim}
into the type-ahead buffer.

\section{Implementation status}
\begin{description}
\item{User-controlled scrolling:} OK
\item{Selection using the mouse:} OK
\item{CUT:} OK
\item{Program output:} OK, but optimisation and find control over
when the screen gets re-painted will need further work, I expect.
\item{Keyboard input:} DELETE OK. Input OK.
\item{Paste:} Paste clumsy in middle of text (works character by character
because multi-char inserts seem to be bug-ridden). Paste at end seems
OK.
\item{UNDO:} Nothing done at all yet, and will need re-work of other buffer
update operations to preserve information.
\item{Print:} Nothing done yet, and the issue of line width of the paper used
relative to the current window size is unclear.
\item{Other operations:} READ is not done yet, but is probably easy.
\end{description}

\end{document}
