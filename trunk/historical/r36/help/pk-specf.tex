\section{Special Functions}
\begin{Introduction}{Special Function Package}
The REDUCE \name{Special Function Package} supplies extended
algebraic and numeric support for a wide class of objects.
This package was released together with REDUCE 3.5 (October 1993)
for the first time, a major update is released with REDUCE 3.6.\\
\\

The functions included in this package are in most cases (unless otherwise
stated) defined and named like in the book by Abramowitz and Stegun:
Handbook of Mathematical Functions, Dover Publications.\\
 \\

The aim is to collect as much information on the special functions
and simplification capabilities as possible,
i.e. algebraic simplifications and numeric (rounded mode) code, limits
of the functions together
with the definitions of the functions, which are in most cases a power 
series, a (definite) integral and/or a differential equation.\\
 \\

\em{What can be found:} Some famous constants, a variety of Bessel functions,
special polynomials,
the Gamma function, the (Riemann) Zeta function, Elliptic Functions, Elliptic
Integrals, 3J symbols (Clebsch-Gordan coefficients) and integral functions.\\
 \\

\em{What is missing:} Mathieu functions, LerchPhi, etc..
The information about the special functions which solve certain
differential equation is very limited.
In several cases numerical approximation is restricted to real
arguments or is missing completely.\\
 \\

The implementation of this package uses REDUCE rule sets to a large extent,
which guarantees a high 'readability' of the functions definitions in the
source file directory. It makes extensions to the special
functions code easy in most cases too. To look at these rules
it may be convenient to use the showrules operator e.g.\\
 \\
 \nameref{showrules} Besseli;\\
.\\

Some evaluations are improved if the special function package is loaded,
e.g. some (infinite) sums and products leading to expressions including
special functions are known in this case.\\
\\

Note: The special function package has to be loaded explicitly by calling
\begin{verbatim}
   load_package specfn;
\end{verbatim}
The functions \nameref{MeijerG} and \nameref{hypergeometric} require
additionally
\begin{verbatim}
   load_package specfn2;
\end{verbatim}

\end{Introduction}

\begin{Concept}{Constants}
\index{Euler's constant}\index{Catalan's constant}\index{Khinchin's constant}
\index{Golden_Ratio}

There are a few constants known to the special function package, namely\\
\\
\name{Euler's constant}  (which can be computed as -\nameref{Psi}(1)) and \\
\name{Khinchin's constant} (which is defined in Khinchin's book \\
	``Continued Fractions'') and \\
\name{Golden_Ratio}  (which can be computed as (1 + sqrt 5)/2)   and \\
\name{Catalan's constant} (which is known as an infinite sum of reciprocal 
powers)

\begin{Examples}
on rounded;
Euler_Gamma;                 &       0.577215664902 \\
Khinchin;                    &       2.68545200107 \\
Catalan                      &       0.915965594177 \\
Golden_Ratio                 &       1.61803398875
\end{Examples}

\end{Concept}

\subsection{Bernoulli Euler Zeta}

\begin{Operator}{BERNOULLI}
The \name{bernoulli} operator returns the nth Bernoulli number.

\begin{Syntax}

\name{Bernoulli}\(\meta{integer}\)

\end{Syntax}

\begin{Examples}
bernoulli 20;                  &       - 174611 / 330 \\
bernoulli 17;                  &       0
\end{Examples}

\begin{Comments}
All Bernoulli numbers with odd indices except for 1 are zero.
\end{Comments}
\end{Operator}

\begin{Operator}{BERNOULLIP}
The \name{BernoulliP} operator returns the nth Bernoulli Polynomial
evaluated at x.

\begin{Syntax}

\name{BernoulliP}\(\meta{integer},\meta{expression}\)

\end{Syntax}

\begin{Examples}
BernoulliP(3,z);            &    z*(2*z^2  - 3*z + 1)/2\\

BernoulliP(10,3);           &    338585 / 66
\end{Examples}

\begin{Comments}
The value of the nth Bernoulli Polynomial at 0 is the nth Bernoulli number. 
\end{Comments}
\end{Operator}

\begin{Operator}{EULER}
The \name{EULER} operator returns the nth Euler number.

\begin{Syntax}
\name{Euler}\(\meta{integer}\)
\end{Syntax}

\begin{Examples}
Euler 20;                         &       370371188237525 \\
Euler 0;                          &       1
\end{Examples}

\begin{Comments}
The \name{Euler} numbers are evaluated by a recursive algorithm which
makes it hard to compute Euler numbers above say 200.

Euler numbers appear in the coefficients of the power series 
representation of 1/cos(z).
\end{Comments}
\end{Operator}

\begin{Operator}{EULERP}
The \name{EulerP} operator returns the nth Euler Polynomial.

\begin{Syntax}
\name{EulerP}\(\meta{integer},\meta{expression}\)
\end{Syntax}

\begin{Examples}
EulerP(2,xx);            &  xx*(xx - 1) \\
EulerP(10,3);            &  2046
\end{Examples}

\begin{Comments}
The Euler numbers are the values of the Euler Polynomials at 1/2
multiplied by 2**n.
\end{Comments}
\end{Operator}

\begin{Operator}{ZETA}
The \name{Zeta} operator returns Riemann's Zeta function, 

  Zeta (z) := sum(1/(k**z),k,1,infinity)

\begin{Syntax}
\name{Zeta}\(\meta{expression}\)
\end{Syntax}

\begin{Examples}
Zeta(2);                 &  pi^2 / 6 \\
on rounded; \\
Zeta 1.01;               &  100.577943338
\end{Examples}

\begin{Comments}
Numerical computation for the Zeta function for arguments close to 1 are 
tedious, because the series is converging very slowly. In this case a formula
(e.g. found in Bender/Orzag: Advanced Mathematical Methods for
Scientists and Engineers, McGraw-Hill) is used.

No numerical approximation for complex arguments is done.
\end{Comments}

\end{Operator}
\subsection{Bessel Functions}

\begin{Operator}{BESSELJ}
The \name{BesselJ} operator returns the Bessel function of the first kind.

\begin{Syntax}
\name{BesselJ}\(\meta{order},\meta{argument}\)
\end{Syntax}

\begin{Examples}
BesselJ(1/2,pi);                &       0 \\
on rounded; \\
BesselJ(0,1);                   &        0.765197686558  \\
\end{Examples}

\end{Operator}

\begin{Operator}{BESSELY}
\index{Weber's function}
The \name{BesselY} operator returns the Bessel function of the second kind.
\begin{Syntax}
\name{BesselY}\(\meta{order},\meta{argument}\)
\end{Syntax}

\begin{Examples}
BesselY (1/2,pi);               &     - sqrt(2) / pi \\
on rounded; \\
BesselY (1,3);                  &      0.324674424792
\end{Examples}

\begin{Comments}
The operator \name{BesselY} is also called Weber's function.
\end{Comments}
\end{Operator}

\begin{Operator}{HANKEL1}
The \name{Hankel1} operator returns the Hankel function of the first kind.

\begin{Syntax}
\name{Hankel1}\(\meta{order},\meta{argument}\)
\end{Syntax}

\begin{Examples}
on complex; \\
Hankel1 (1/2,pi);               &     - i * sqrt(2) / pi \\
Hankel1 (1,pi);                 &   besselj(1,pi) + i*bessely(1,pi)

\end{Examples}

\begin{Comments}
The operator \name{Hankel1} is also called Bessel function of the third kind.
There is currently no numeric evaluation of Hankel functions.
\end{Comments}
\end{Operator}

\begin{Operator}{HANKEL2}
The \name{Hankel2} operator returns the Hankel function of the second kind.

\begin{Syntax}
\name{Hankel2}\(\meta{order},\meta{argument}\)
\end{Syntax}

\begin{Examples}
on complex; \\
Hankel2 (1/2,pi);               &     - i * sqrt(2) / pi \\
Hankel2 (1,pi);                 &   besselj(1,pi) - i*bessely(1,pi)
\end{Examples}

\begin{Comments}
The operator \name{Hankel2} is also called Bessel function of the third kind.
There is currently no numeric evaluation of Hankel functions.
\end{Comments}
\end{Operator}

\begin{Operator}{BESSELI}
The \name{BesselI} operator returns the modified Bessel function I.

\begin{Syntax}
\name{BesselI}\(\meta{order},\meta{argument}\)
\end{Syntax}

\begin{Examples}
on rounded; \\
Besseli (1,1);                  &      0.565159103992
\end{Examples}

\begin{Comments}
The knowledge about the operator \name{BesselI} is currently fairly limited.
\end{Comments}
\end{Operator}

\begin{Operator}{BESSELK}
The \name{BesselK} operator returns the modified Bessel function K.

\begin{Syntax}
\name{BesselK}\(\meta{order},\meta{argument}\)
\end{Syntax}

\begin{Examples}
df(besselk(0,x),x);            &    - besselk(1,x)
\end{Examples}

\begin{Comments}
There is currently no numeric support for the operator \name{BesselK}.
\end{Comments}
\end{Operator}

\begin{Operator}{StruveH}
The \name{StruveH} operator returns Struve's H function.

\begin{Syntax}
\name{StruveH}\(\meta{order},\meta{argument}\)
\end{Syntax}

\begin{Examples}
struveh(-3/2,x);               &    - besselj(3/2,x) / i
\end{Examples}

\end{Operator}

\begin{Operator}{StruveL}
The \name{StruveL} operator returns the modified Struve L function .

\begin{Syntax}
\name{StruveL}\(\meta{order},\meta{argument}\)
\end{Syntax}

\begin{Examples}
struvel(-3/2,x);               &    besseli(3/2,x)
\end{Examples}

\end{Operator}

\begin{Operator}{KummerM}
\index{Confluent Hypergeometric function}
The \name{KummerM} operator returns Kummer's M function.

\begin{Syntax}
\name{KummerM}\(\meta{parameter},\meta{parameter},\meta{argument}\)
\end{Syntax}

\begin{Examples}
kummerm(1,1,x);               &    e^x \\
on rounded; \\
kummerm(1,3,1.3);             &    1.62046942914
\end{Examples}

\begin{Comments}
Kummer's M function is one of the Confluent Hypergeometric functions.
For reference see the \nameref{hypergeometric} operator.
\end{Comments}
\end{Operator}

\begin{Operator}{KummerU}
\index{Confluent Hypergeometric function}
The \name{KummerU} operator returns Kummer's U function.

\begin{Syntax}
\name{KummerU}\(\meta{parameter},\meta{parameter},\meta{argument}\)
\end{Syntax}

\begin{Examples}
df(kummeru(1,1,x),x)           &    - kummeru(2,2,x)
\end{Examples}

\begin{Comments}
Kummer's U function is one of the Confluent Hypergeometric functions.
For reference see the \nameref{hypergeometric} operator.
\end{Comments}
\end{Operator}

\begin{Operator}{WhittakerW}
\index{Confluent Hypergeometric function}
The \name{WhittakerW} operator returns Whittaker's W function.

\begin{Syntax}
\name{WhittakerW}\(\meta{parameter},\meta{parameter},\meta{argument}\)
\end{Syntax}

\begin{Examples}
WhittakerW(2,2,2);     &    \rfrac{4*sqrt(2)*kummeru(\rfrac{1}{2},5,2)}{e}
\end{Examples}

\begin{Comments}
Whittaker's W function is one of the Confluent Hypergeometric functions.
For reference see the \nameref{hypergeometric} operator.
\end{Comments}
\end{Operator}

\subsection{Airy Functions}

\begin{Operator}{Airy_Ai}
The \name{Airy\_Ai} operator returns the Airy Ai function for a given argument.

\begin{Syntax}
\name{Airy\_Ai}\(\meta{argument}\)
\end{Syntax}

\begin{Examples}
on complex;
on rounded;
Airy_Ai(0);              &   0.355028053888          \\
Airy_Ai(3.45 + 17.97i);   &  - 5.5561528511e+9 - 8.80397899932e+9*i  \\
\end{Examples}

\end{Operator}

\begin{Operator}{Airy_Bi}
The \name{Airy\_Bi} operator returns the Airy Bi function for a given
argument.

\begin{Syntax}
\name{Airy\_Bi}\(\meta{argument}\)
\end{Syntax}

\begin{Examples}
Airy_Bi(0);              &   0.614926627446          \\
Airy_Bi(3.45 + 17.97i);  &  8.80397899932e+9 - 5.5561528511e+9*i   \\
\end{Examples}

\end{Operator} 

\begin{Operator}{Airy_Aiprime}
The \name{Airy\_Aiprime} operator returns the Airy Aiprime function for a
given argument.

\begin{Syntax}
\name{Airy\_Aiprime}\(\meta{argument}\)
\end{Syntax}

\begin{Examples}
Airy_Aiprime(0);              &  - 0.258819403793           \\
Airy_Aiprime(3.45+17.97i);&  - 3.83386421824e+19 + 2.16608828136e+19*i \\
\end{Examples}

\end{Operator}

\begin{Operator}{Airy_Biprime}
The \name{Airy\_Biprime} operator returns the Airy Biprime function for a
given argument.

\begin{Syntax}
\name{Airy\_Biprime}\(\meta{argument}\)
\end{Syntax}

\begin{Examples}
Airy_Biprime(0);              &             \\
Airy_Biprime(3.45 + 17.97i); & 3.84251916792e+19 - 2.18006297399e+19*i\\
\end{Examples}

\end{Operator}

\subsection{Jacobi's Elliptic Functions and Elliptic Integrals}

\begin{Operator}{JacobiSN}
The \name{Jacobisn} operator returns the Jacobi Elliptic function sn.

\begin{Syntax}
\name{Jacobisn}\(\meta{expression},\meta{integer}\)
\end{Syntax}

\begin{Examples}
Jacobisn(0.672, 0.36)    & 0.609519691792 \\
Jacobisn(1,0.9)          & 0.770085724907881 \\
\end{Examples}

\end{Operator}

\begin{Operator}{JacobiCN}
The \name{Jacobicn} operator returns the Jacobi Elliptic function cn.

\begin{Syntax}
\name{Jacobicn}\(\meta{expression},\meta{integer}\)
\end{Syntax}

\begin{Examples}
Jacobicn(7.2, 0.6)      & 0.837288298482018  \\
Jacobicn(0.11, 19)      & 0.994403862690043 - 1.6219006985556e-16*i  \\
\end{Examples}

\end{Operator}


\begin{Operator}{JacobiDN}
The \name{Jacobidn} operator returns the Jacobi Elliptic function dn.

\begin{Syntax}
\name{Jacobidn}\(\meta{expression},\meta{integer}\)
\end{Syntax}

\begin{Examples}
Jacobidn(15, 0.683)     & 0.640574162024592 \\
Jacobidn(0,0)           & 1 \\
\end{Examples}

\end{Operator}

\begin{Operator}{JacobiCD}
The \name{Jacobicd} operator returns the Jacobi Elliptic function cd.

\begin{Syntax}
\name{Jacobicd}\(\meta{expression},\meta{integer}\)
\end{Syntax}

\begin{Examples}
Jacobicd(1, 0.34)       & 0.657683337805273 \\
Jacobicd(0.8,0.8)       & 0.925587311582301 \\
\end{Examples}

\end{Operator}

\begin{Operator}{JacobiSD}
The \name{Jacobisd} operator returns the Jacobi Elliptic function sd.

\begin{Syntax}
\name{Jacobisd}\(\meta{expression},\meta{integer}\)
\end{Syntax}

\begin{Examples}
Jacobisd(12, 0.4)       & 0.357189729437272    \\
Jacobisd(0.35,1)        &  - 1.17713873203043  \\
\end{Examples}

\end{Operator}

\begin{Operator}{JacobiND}
The \name{Jacobind} operator returns the Jacobi Elliptic function nd.

\begin{Syntax}
\name{Jacobind}\(\meta{expression},\meta{integer}\)
\end{Syntax}

\begin{Examples}
Jacobind(0.2, 17) & 1.46553203037507 + 0.0000000000334032759313703*i \\
Jacobind(30, 0.001)      & 1.00048958438  \\
\end{Examples}

\end{Operator}

\begin{Operator}{JacobiDC}
The \name{Jacobidc} operator returns the Jacobi Elliptic function dc.

\begin{Syntax}
\name{Jacobidc}\(\meta{expression},\meta{integer}\)
\end{Syntax}

\begin{Examples}
Jacobidc(0.003,1)        & 1 \\
Jacobidc(2, 0.75)        & 6.43472885111  \\
\end{Examples}

\end{Operator}

\begin{Operator}{JacobiNC}
The \name{Jacobinc} operator returns the Jacobi Elliptic function nc.

\begin{Syntax}
\name{Jacobinc}\(\meta{expression},\meta{integer}\)
\end{Syntax}

\begin{Examples}
Jacobinc(1,0)           & 1.85081571768093 \\
Jacobinc(56, 0.4387)    & 39.304842663512  \\
\end{Examples}

\end{Operator}

\begin{Operator}{JacobiSC}
The \name{Jacobisc} operator returns the Jacobi Elliptic function sc.

\begin{Syntax}
\name{Jacobisc}\(\meta{expression},\meta{integer}\)
\end{Syntax}

\begin{Examples}
Jacobisc(9, 0.88)        &  - 1.16417697982095  \\
Jacobisc(0.34, 7)        & 0.305851938390775 - 9.8768100944891e-12*i \\
\end{Examples}

\end{Operator}

\begin{Operator}{JacobiNS}
The \name{Jacobins} operator returns the Jacobi Elliptic function ns.

\begin{Syntax}
\name{Jacobins}\(\meta{expression},\meta{integer}\)
\end{Syntax}

\begin{Examples}
Jacobins(3, 0.9)         & 1.00945801599785 \\
Jacobins(0.887, 15)      & 0.683578280513975 - 0.85023411082469*i \\
\end{Examples}

\end{Operator}

\begin{Operator}{JacobiDS}
The \name{Jacobisn} operator returns the Jacobi Elliptic function ds.

\begin{Syntax}
\name{Jacobids}\(\meta{expression},\meta{integer}\)
\end{Syntax}

\begin{Examples}
Jacobids(98,0.223)       & - 1.061253961477 \\
Jacobids(0.36,0.6)       & 2.76693172243692 \\
\end{Examples}

\end{Operator}

\begin{Operator}{JacobiCS}
The \name{Jacobics} operator returns the Jacobi Elliptic function cs.

\begin{Syntax}
\name{Jacobics}\(\meta{expression},\meta{integer}\)
\end{Syntax}

\begin{Examples}
Jacobics(0, 0.767)       & infinity   \\
Jacobics(1.43, 0)        & 0.141734127352112 \\
\end{Examples}

\end{Operator}

\begin{Operator}{JacobiAMPLITUDE}
The \name{JacobiAmplitude} operator returns the amplitude of u.
\begin{Syntax}
\name{JacobiAmplitude}\(\meta{expression},\meta{integer}\)
\end{Syntax}

\begin{Examples}
JacobiAmplitude(7.239, 0.427)  & 0.0520978301448978 \\
JacobiAmplitude(0,0.1)         & 0 \\
\end{Examples}

\begin{Comments}
Amplitude u = asin(\name{Jacobisn(u,m)})
\end{Comments}
\end{Operator}


\begin{Operator}{AGM_FUNCTION}
The \name{AGM_function} operator returns a list of (N, AGM,
 list of aNtoa0, list of bNtob0, list of cNtoc0) where a0, b0 and c0 
are the initial values; N is the index number of the last term
used to generate the AGM.  AGM is the Arithmetic Geometric Mean.

\begin{Syntax}
\name{AGM_function}\(\meta{integer},\meta{integer},\meta{integer}\)
\end{Syntax}

\begin{Examples}
AGM_function(1,1,1)   & {1,1,{1,1},{1,1},{0,1}}  \\
AGM_function(1, 0.1, 1.3) &
\begin{multilineoutput}{6cm}
\{6,
 2.27985615996629, 
 \{2.27985615996629, 2.27985615996629,
  2.2798561599706, 2.2798624278857, 
  2.28742283656583, 2.55, 1\},
 \{2.27985615996629, 2.27985615996629,
  2.27985615996198, 2.2798498920555, 
  2.27230201920557, 2.02484567313166, 4.1\},
 \{0, 4.30803136219904e-12, 0.0000062679151007581,
  0.00756040868012758, 0.262577163434171, - 1.55, 5.9\}\}
\end{multilineoutput} \\
\end{Examples}

\begin{Comments}
The other Jacobi functions use this function with initial values
a0=1, b0=sqrt(1-m), c0=sqrt(m). 
\end{Comments}
\end{Operator}

\begin{Operator}{LANDENTRANS}
The \name{landentrans} operator generates the descending landen
transformation of the given imput values, returning a list of these
values; initial to final in each case.
\begin{Syntax}
\name{landentrans}\(\meta{expression},\meta{integer}\)
\end{Syntax}

\begin{Examples}
landentrans(0,0.1)    & \{\{0,0,0,0,0\},\{0.1,0.0025041751943776, \\
 & 0.00000156772498954046,6.1444078 9914461e-13,0\}\}  \\
\end{Examples}

\begin{Comments}
The first list ascends in value, and the second descends in value.
\end{Comments}
\end{Operator}


\begin{Operator}{EllipticF}
The \name{EllipticF} operator returns the Elliptic Integral of the
First Kind.
\begin{Syntax}
\name{EllitpicF}\(\meta{expression},\meta{integer}\)
\end{Syntax}

\begin{Examples}
EllipticF(0.3, 8.222)   & 0.3 \\
EllipticF(7.396, 0.1)   & 7.58123216114307 \\ 
\end{Examples}

\begin{Comments}
The Complete Elliptic Integral of the First Kind can be found by
putting the first argument to pi/2 or by using \name{EllipticK}
and the second argument.
\end{Comments}
\end{Operator}


\begin{Operator}{EllipticK}
The \name{EllipticK} operator returns the Elliptic value K.

\begin{Syntax}
\name{EllipticK}\(\meta{integer}\)
\end{Syntax}

\begin{Examples}
EllipticK(0.2)          & 1.65962359861053   \\
EllipticK(4.3)          & 0.808442364282734 - 1.05562492399206*i  \\
EllipticK(0.000481)     & 1.57098526617635    \\
\end{Examples}

\begin{Comments}
The \name{EllipticK} function is the Complete Elliptic Integral of 
the First Kind.
\end{Comments}
\end{Operator}

\begin{Operator}{EllipticKprime}
The \name{EllipticK'} operator returns the Elliptic value K(m).

\begin{Syntax}
\name{EllipticKprime}\(\meta{integer}\)
\end{Syntax}

\begin{Examples}
EllipticKprime(0.2)          & 2.25720532682085 \\
EllipticKprime(4.3)          & 1.05562492399206 \\
EllipticKprime(0.000481)     & 5.206621921966   \\
\end{Examples}

\begin{Comments}
The \name{EllipticKprime} function is the Complete Elliptic Integral of 
the First Kind of (1-m).
\end{Comments}
\end{Operator}


\begin{Operator}{EllipticE}
The \name{EllipticE} operator used with two arguments
returns the Elliptic Integral of the Second Kind.
\begin{Syntax}
\name{EllipticE}\(\meta{expression},\meta{integer}\)
\end{Syntax}

\begin{Examples}
EllipticE(1.2,0.22)     & 1.15094019180949 \\
EllipticE(0,4.35)       & 0                \\
EllipticE(9,0.00719)    & 8.98312465929145  \\
\end{Examples}

\begin{Comments}
The Complete Elliptic Integral of the Second Kind can be obtained by
using just the second argument, or by using pi/2 as the first argument.
\end{Comments}

The \name{EllipticE} operator used with one argument
returns the Elliptic value E.
\begin{Syntax}
\name{EllipticE}\(\meta{integer}\)
\end{Syntax}

\begin{Examples}
EllipticE(0.22)             & 1.48046637439519  \\
EllipticE(pi/2, 0.22)       & 1.48046637439519  \\
\end{Examples}

\end{Operator}

\begin{Operator}{EllipticTHETA}
The \name{EllipticTheta} operator returns one of the four Theta 
functions. It cannot except any number other than 1,2,3 or 4 as
its first argument.

\begin{Syntax}
\name{EllipticTheta}\(\meta{integer},\meta{expression},\meta{integer}\)
\end{Syntax}

\begin{Examples}
EllipticTheta(1, 1.4, 0.72)    & 0.91634775373  \\
EllipticTheta(2, 3.9, 6.1 )    & -48.0202736969 + 20.9881034377 i \\
EllipticTheta(3, 0.67, 0.2)    & 1.0083077448   \\
EllipticTheta(4, 8, 0.75)      & 0.894963369304 \\
EllipticTheta(5, 1, 0.1)       & ***** In EllipticTheta(a,u,m); a = 1,2,3 or 4. 
  \\
\end{Examples}

\begin{Comments}
Theta functions are important because every one of the Jacobian
Elliptic functions can be expressed as the ratio of two theta functions.
\end{Comments}
\end{Operator}

\begin{Operator}{JacobiZETA}
The \name{JacobiZeta} operator returns the Jacobian function Zeta.

\begin{Syntax}
\name{JacobiZeta}\(\meta{expression},\meta{integer}\)
\end{Syntax}

\begin{Examples}
JacobiZeta(3.2, 0.8)    &  - 0.254536403439 \\
JacobiZeta(0.2, 1.6)    &  0.171766095970451 - 0.0717028569800147*i  \\
\end{Examples}

\begin{Comments}
The Jacobian function Zeta is related to the Jacobian function Theta.
But it is significantly different from Riemann's Zeta Function \nameref{Zeta}.
\end{Comments}
\end{Operator}

\subsection{Gamma and Related Functions}
\begin{Operator}{POCHHAMMER}

The \name{Pochhammer} operator implements the Pochhammer notation 
(shifted factorial).

\begin{Syntax}
\name{Pochhammer}\(\meta{expression},\meta{expression}\)
\end{Syntax}

\begin{Examples}
pochhammer(17,4);               &   116280 \\

pochhammer(1/2,z);              & 
 \rfrac{factorial(2*z)}{(2^{2*z}*factorial(z))}
\end{Examples}

\begin{Comments}
A number of complex rules for \name{Pochhammer} are inactive, because they
cause a huge system load in algebraic mode. If one wants to use more rules
for the simplification of Pochhammer's notation, one can do: 
\\
let special!*pochhammer!*rules;

\end{Comments}
\end{Operator}

\begin{Operator}{GAMMA}
The \name{Gamma} operator returns the Gamma function.

\begin{Syntax}

\name{Gamma}\(\meta{expression}\)

\end{Syntax}

\begin{Examples}
gamma(10);             &       362880    \\
gamma(1/2);            &       sqrt(pi)
\end{Examples}
\end{Operator}

\begin{Operator}{BETA}
The \name{Beta} operator returns the Beta function defined by

   Beta (z,w) := defint(t**(z-1)* (1 - t)**(w-1),t,0,1)  .

\begin{Syntax}

\name{Beta}\(\meta{expression},\meta{expression}\)

\end{Syntax}

\begin{Examples}
Beta(2,2);                  &       1 / 6 \\
Beta(x,y);                  &   gamma(x)*gamma(y) / gamma(x + y)
\end{Examples}

\begin{Comments}
The operator \name{Beta} is simplified towards the \nameref{GAMMA} operator.
\end{Comments}
\end{Operator}

\begin{Operator}{PSI}
\index{Euler's constant}
The \name{Psi} operator returns the Psi (or DiGamma) function.

   Psi(x) := df(Gamma(z),z)/ Gamma (z)

\begin{Syntax}

\name{Gamma}\(\meta{expression}\)

\end{Syntax}

\begin{Examples}
Psi(3);                 & (2*log(2) + psi(1/2) + psi(1) + 3)/2 \\
on rounded; \\
- Psi(1);               &  0.577215664902
\end{Examples}

\begin{Comments}
Euler's constant can be found as  - Psi(1). 
\end{Comments}
\end{Operator}

\begin{Operator}{POLYGAMMA}
The \name{Polygamma} operator returns the Polygamma function.

   Polygamma(n,x) := df(Psi(z),z,n);

\begin{Syntax}
\name{Polygamma}\(\meta{integer},\meta{expression}\)
\end{Syntax}

\begin{Examples}
 Polygamma(1,2);        &  (pi^2  - 6) / 6\\
on rounded; \\
Polygamma(1,2.35);      &  0.52849689109
\end{Examples}

\begin{Comments}
The Polygamma function is used for simplification of the \nameref{ZETA} 
function for some arguments.
\end{Comments}
\end{Operator}

\subsection{Miscellaneous Functions}

\begin{Operator}{DILOG extended}
\index{Spence's Integral}
The package \name{specfn} supplies an extended support for the
\nameref{dilog} operator  which implements the \nameindex{dilogarithm function}.

dilog(x) := - defint(log(t)/(t - 1),t,1,x);

\begin{Syntax}
\name{Dilog}\(\meta{order},\meta{expression}\)
\end{Syntax}

\begin{Examples}
defint(log(t)/(t - 1),t,1,x);  &  - dilog (x) \\
dilog 2;                       &  - pi^2 /12 \\

on rounded; \\
Dilog 20;                      &  - 5.92783972438
\end{Examples}

\begin{Comments}
The operator \name{Dilog} is sometimes called Spence's Integral for n = 2.
\end{Comments}
\end{Operator}

\begin{Operator}{Lambert_W function}
Lambert's W function is the inverse of the function w * e^w.
It is used in the \nameref{solve} package for equations containing
exponentials and logarithms.

\begin{Syntax}
\name{Lambert\_W}\(\meta{z}\)
\end{Syntax}

\begin{Examples}
Lambert_W(-1/e);      &  -1 \\
solve(w + log(w),w);  & {w=lambert\_w(1)}\\
on rounded; \\
Lambert_W(-0.05);     &  - 0.0527059835515
\end{Examples}

\begin{Comments}
The current implementation will compute the principal branch in
rounded mode only.
\end{Comments}
\end{Operator}

\subsection{Orthogonal Polynomials}

\begin{Operator}{ChebyshevT}
The \name{ChebyshevT} operator computes the nth Chebyshev T Polynomial (of the
first kind).

\begin{Syntax}
\name{ChebyshevT}\(\meta{integer},\meta{expression}\)
\end{Syntax}

\begin{Examples}
ChebyshevT(3,xx);                 &   xx*(4*xx^2  - 3) \\

ChebyshevT(3,4);                  &       244
\end{Examples}

\begin{Comments}
Chebyshev's T polynomials are computed using the recurrence relation:

ChebyshevT(n,x) := 2x*ChebyshevT(n-1,x) - ChebyshevT(n-2,x)  with \\
ChebyshevT(0,x) := 0 and ChebyshevT(1,x) := x
\end{Comments}
\end{Operator}

\begin{Operator}{ChebyshevU}
The \name{ChebyshevU} operator returns the nth Chebyshev U Polynomial (of the
second kind).

\begin{Syntax}
\name{ChebyshevU}\(\meta{integer},\meta{expression}\)
\end{Syntax}

\begin{Examples}
ChebyshevU(3,xx);                 & 4*x*(2*x^2  - 1) \\

ChebyshevU(3,4);                  &      496
\end{Examples}

\begin{Comments}
Chebyshev's U polynomials are computed using the recurrence relation:

ChebyshevU(n,x) := 2x*ChebyshevU(n-1,x) - ChebyshevU(n-2,x)  with \\
ChebyshevU(0,x) := 0 and ChebyshevU(1,x) := 2x

\end{Comments}
\end{Operator}

\begin{Operator}{HermiteP}
The \name{HermiteP} operator returns the nth Hermite Polynomial.

\begin{Syntax}
\name{HermiteP}\(\meta{integer},\meta{expression}\)
\end{Syntax}

\begin{Examples}
HermiteP(3,xx);                 & 4*xx*(2*xx^2  - 3) \\
HermiteP(3,4);                  &      464
\end{Examples}

\begin{Comments}
Hermite polynomials are computed using the recurrence relation:
  \\
HermiteP(n,x) := 2x*HermiteP(n-1,x) - 2*(n-1)*HermiteP(n-2,x)  with \\
HermiteP(0,x) := 1 and HermiteP(1,x) := 2x

\end{Comments}
\end{Operator}


\begin{Operator}{LaguerreP}
The \name{LaguerreP} operator computes the nth Laguerre Polynomial.
The two argument call of LaguerreP is a (common) abbreviation of
LaguerreP(n,0,x).

\begin{Syntax}
\name{LaguerreP}\(\meta{integer},\meta{expression}\)  or\\
\name{LaguerreP}\(\meta{integer},\meta{expression},\meta{expression}\)
\end{Syntax}

\begin{Examples}
LaguerreP(3,xx);                        &  (- xx^3  + 9*xx^2  - 18*xx + 6)/6\\

LaguerreP(2,3,4);                       &       -2
\end{Examples}

\begin{Comments}
Laguerre polynomials are computed using the recurrence relation:

LaguerreP(n,a,x) := (2n+a-1-x)/n*LaguerreP(n-1,a,x) - 
			(n+a-1) * LaguerreP(n-2,a,x) with \\
 \\
LaguerreP(0,a,x) := 1 and LaguerreP(2,a,x) := -x+1+a
\end{Comments}
\end{Operator}

\begin{Operator}{LegendreP}
The binary \name{LegendreP} operator computes the nth Legendre
Polynomial which is
a special case of the nth Jacobi Polynomial with \\
 \\
LegendreP(n,x) := JacobiP(n,0,0,x)\\
  \\
The ternary form returns the associated Legendre Polynomial (see below).

\begin{Syntax}
\name{LegendreP}\(\meta{integer},\meta{expression}\)      or\\
\name{LegendreP}\(\meta{integer},\meta{expression},\meta{expression}\)
\end{Syntax}

\begin{Examples}
LegendreP(3,xx);                &\rfrac{xx*(5*xx^2  - 3)}{2}\\

LegendreP(3,2,xx);              &15*xx*( - xx^2  + 1)
\end{Examples}

\begin{Comments}
The ternary form of the operator \name{LegendreP} is the associated
Legendre Polynomial defined as \\ 
\\
  P(n,m,x) = (-1)**m * (1-x**2)**(m/2) * df(LegendreP(n,x),x,m)
\end{Comments}
\end{Operator}

\begin{Operator}{JacobiP}
The \name{JacobiP} operator computes the nth Jacobi Polynomial.

\begin{Syntax}
\name{JacobiP}\(\meta{integer},\meta{expression},\meta{expression},
   \meta{expression}\)
\end{Syntax}

\begin{Examples}
JacobiP(3,4,5,xx);      & \rfrac{7*(65*xx^3  - 13*xx^2  - 13*xx + 1)}{8}\\

JacobiP(3,4,5,6);       &       94465/8
\end{Examples}

\end{Operator}

\begin{Operator}{GegenbauerP}
\index{ultraspherical polynomials}
The \name{GegenbauerP} operator computes Gegenbauer's (ultraspherical)
polynomials.

\begin{Syntax}
\name{GegenbauerP}\(\meta{integer},\meta{expression},\meta{expression}\)
\end{Syntax}

\begin{Examples}
GegenbauerP(3,2,xx);              &       4*xx*(8*xx^2  - 3)\\

GegenbauerP(3,2,4);               &       2000
\end{Examples}

\end{Operator}

\begin{Operator}{SolidHarmonicY}
\index{Solid harmonic polynomials}
The \name{SolidHarmonicY} operator computes Solid harmonic (Laplace)
polynomials.

\begin{Syntax}

\name{SolidHarmonicY}\(\meta{integer},\meta{integer},
\meta{expression},\meta{expression},\meta{expression},\meta{expression}\)

\end{Syntax}

\begin{Examples}

SolidHarmonicY(3,-2,x,y,z,r2); &

\rfrac{sqrt(105)*z*(-2*i*x*y + x^2 - y^2)}{4*sqrt(pi)*sqrt(2)}\\

\end{Examples}

\end{Operator}

\begin{Operator}{SphericalHarmonicY}
\index{Spherical harmonic polynomials}
The \name{SphericalHarmonicY} operator computes Spherical harmonic (Laplace)
polynomials. These are special cases of the
solid harmonic polynomials, \nameref{SolidHarmonicY}.

\begin{Syntax}

\name{SphericalHarmonicY}\(\meta{integer},\meta{integer},
\meta{expression},\meta{expression}\)

\end{Syntax}

\begin{Examples}
SphericalHarmonicY(3,2,theta,phi); &
 
\rfrac{sqrt(105)*cos(theta)*sin(theta)^2*(cos(phi)^2+2*cos(phi)*sin(phi)*i-
sin(phi)^2)}{4*sqrt(pi)*sqrt(2)}\\

\end{Examples}

\end{Operator}


\subsection{Integral Functions}
\index{sine integral function}
\begin{Operator}{Si}
\index{Sine integral function}\index{integral function}
The \name{Si} operator returns the Sine Integral function.

\begin{Syntax}
\name{Si}\(\meta{expression}\)
\end{Syntax}

\begin{Examples}
limit(Si(x),x,infinity);          &       pi / 2 \\
on rounded; \\
Si(0.35);                         &       0.347626790989
\end{Examples}

\begin{Comments}
The numeric values for the operator \name{Si} are computed via the
power series representation, which limits the argument range.
\end{Comments}
\end{Operator}

\begin{Operator}{Shi}
\index{hyperbolic sine integral function}\index{integral function}
The \name{Shi} operator returns the hyperbolic Sine Integral function.

\begin{Syntax}
\name{Shi}\(\meta{expression}\)
\end{Syntax}

\begin{Examples}
df(shi(x),x);                      &     sinh(x) / x \\
on rounded; \\
Shi(0.35);                         &    0.352390716351
\end{Examples}

\begin{Comments}
The numeric values for the operator \name{Shi} are computed via the
power series representation, which limits the argument range.
\end{Comments}
\end{Operator}

\begin{Operator}{s_i}
\index{sine integral function}\index{integral function}
The \name{s_i} operator returns the Sine Integral function si.

\begin{Syntax}
\name{s_i}\(\meta{expression}\)
\end{Syntax}

\begin{Examples}
s_i(xx);          &       (2*Si(xx) - pi) / 2 \\
df(s_i(x),x);     &       sin(x) / x
\end{Examples}

\begin{Comments}
The operator name \name{s_i} is simplified towards \nameref{SI}.
Since REDUCE is not case sensitive by default the name ``si'' can't be
used.
\end{Comments}
\end{Operator}

\begin{Operator}{Ci}
\index{cosine integral function}
The \name{Ci} operator returns the Cosine Integral function.

\begin{Syntax}
\name{Ci}\(\meta{expression}\)
\end{Syntax}

\begin{Examples}
defint(cos(t)/t,t,x,infinity);     &    - ci (x) \\
on rounded; \\
Ci(0.35);                          &    - 0.50307556932
\end{Examples}

\begin{Comments}
The numeric values for the operator \name{Ci} are computed via the
power series representation, which limits the argument range.
\end{Comments}
\end{Operator}

\begin{Operator}{Chi}
\index{hyperbolic cosine integral function}\index{integral function}
The \name{Chi} operator returns the Hyperbolic Cosine Integral function.

\begin{Syntax}
\name{Chi}\(\meta{expression}\)
\end{Syntax}

\begin{Examples}
defint((cosh(t)-1)/t,t,0,x);       &   - log(x) + psi(1) + chi(x)\\
on rounded; \\
Chi(0.35);                         &   - 0.44182471827
\end{Examples}

\begin{Comments}
The numeric values for the operator \name{Chi} are computed via the
power series representation, which limits the argument range.
\end{Comments}
\end{Operator}

\begin{Operator}{ERF extended}
\index{error function}
The special function package supplies an extended support for the
\nameref{erf} operator  which implements the \nameindex{error function} \\
 \\
  defint(e**(-x**2),x,0,infinity) * 2/sqrt(pi) \\
.\\
\begin{Syntax}
\name{erf}\(\meta{expression}\)
\end{Syntax}

\begin{Examples}
erf(-x);                        &   - erf(x)\\
on rounded; \\
erf(0.35);                      &   0.379382053562
\end{Examples}

\begin{Comments}
The numeric values for the operator \name{erf} are computed via the
power series representation, which limits the argument range.
\end{Comments}
\end{Operator}

\begin{Operator}{erfc}
\index{error function}\index{complementary error function}
The \name{erfc} operator returns the complementary Error function\\
 \\
  1 - defint(e**(-x**2),x,0,infinity) * 2/sqrt(pi) \\
.\\
\begin{Syntax}
\name{erfc}\(\meta{expression}\)
\end{Syntax}

\begin{Examples}
erfc(xx);                     &      - erf(xx) + 1
\end{Examples}

\begin{Comments}
The operator \name{erfc} is simplified towards the \nameref{erf} operator.
\end{Comments}
\end{Operator}

\begin{Operator}{Ei}
\index{exponential integral function}
The \name{Ei} operator returns the Exponential Integral function.

\begin{Syntax}
\name{Ei}\(\meta{expression}\)
\end{Syntax}

\begin{Examples}
df(ei(x),x);               &   \rfrac{e^x}{x}\\
on rounded; \\
Ei(0.35);                  &    - 0.0894340019184
\end{Examples}

\begin{Comments}
The numeric values for the operator \name{Ei} are computed via the
power series representation, which limits the argument range.
\end{Comments}
\end{Operator}

\begin{Operator}{Fresnel_C}
The \name{Fresnel_C} operator represents Fresnel's Cosine function.

\begin{Syntax}
\name{Fresnel_C}\(\meta{expression}\)
\end{Syntax}

\begin{Examples}
int(cos(t^2*pi/2),t,0,x);     &   fresnel\_c(x) \\
on rounded; \\
fresnel_c(2.1);                  &   0.581564135061
\end{Examples}

\begin{Comments}
The operator \name{Fresnel_C} has a limited numeric evaluation of 
large values of its argument.
\end{Comments}
\end{Operator}

\begin{Operator}{Fresnel_S}
The \name{Fresnel_S} operator represents Fresnel's Sine Integral function.

\begin{Syntax}
\name{Fresnel_S}\(\meta{expression}\)
\end{Syntax}

\begin{Examples}
int(sin(t^2*pi/2),t,0,x);     &   fresnel\_s(x) \\
on rounded; \\
fresnel_s(2.1);                  &   0.374273359378
\end{Examples}

\begin{Comments}
The operator \name{Fresnel_S} has a limited numeric evaluation of 
large values of its argument.
\end{Comments}
\end{Operator}

\subsection{Combinatorial Operators}

\begin{Operator}{BINOMIAL}
The \name{Binomial} operator returns the Binomial coefficient if both
parameter are integer and expressions involving the Gamma function otherwise.

\begin{Syntax}

\name{Binomial}\(\meta{integer},\meta{integer}\)

\end{Syntax}

\begin{Examples}
Binomial(49,6);                &  13983816 \\

Binomial(n,3);                 &  \rfrac{gamma(n + 1)}{6*gamma(n - 2)}
\end{Examples}

\begin{Comments}
The operator \name{Binomial} evaluates the Binomial coefficients from 
the explicit form and therefore it is not the best algorithm if you
want to compute many binomial coefficients with big indices in which
case a recursive algorithm is preferable.
\end{Comments}
\end{Operator}

\begin{Operator}{STIRLING1}
The \name{Stirling1} operator returns the Stirling Numbers S(n,m) of the first
kind, i.e. the number of permutations of n symbols which have exactly m cycles
(divided by (-1)**(n-m)).

\begin{Syntax}
\name{Stirling1}\(\meta{integer},\meta{integer}\)
\end{Syntax}

\begin{Examples}
Stirling1 (17,4);              &       -87077748875904 \\
Stirling1 (n,n-1);             &  \rfrac{-gamma(n+1)}{2*gamma(n-1)}
\end{Examples}

\begin{Comments}
The operator \name{Stirling1} evaluates the Stirling numbers of the
first kind by rulesets for special cases or by a computing the closed
form, which is a series involving the operators \nameref{BINOMIAL}
and \nameref{STIRLING2}.
\end{Comments}
\end{Operator}

\begin{Operator}{STIRLING2}
The \name{Stirling1} operator returns the Stirling Numbers S(n,m) of the 
second kind, i.e. the number of ways of partitioning a set of n elements
into m non-empty subsets.

\begin{Syntax}
\name{Stirling2}\(\meta{integer},\meta{integer}\)
\end{Syntax}

\begin{Examples}
Stirling2 (17,4);              &       694337290 \\
Stirling2 (n,n-1);             &   \rfrac{gamma(n+1)}{2*gamma(n-1)}
\end{Examples}

\begin{Comments}
The operator \name{Stirling2} evaluates the Stirling numbers of the
second kind by rulesets for special cases or by a computing the closed
form. 
\end{Comments}
\end{Operator}

\subsection{3j and  6j symbols}

\begin{Operator}{ThreejSymbol}
The \name{ThreejSymbol} operator implements the 3j symbol.

\begin{Syntax}

\name{ThreejSymbol}\(\meta{list of j1,m1},\meta{list of j2,m2},
\meta{list of j3,m3}\)

\end{Syntax}

\begin{Examples}

ThreejSymbol({j+1,m},{j+1,-m},{1,0}); & 

\rfrac{( - 1)^j *(abs(j - m + 1) - abs(j + m + 1))}
	{ 2*sqrt(2*j^3  + 9*j^2  + 13*j + 6)*( - 1)^m}\\

\end{Examples}
\end{Operator}

\begin{Operator}{Clebsch_Gordan}
The \name{Clebsch\_Gordan} operator implements the Clebsch\_Gordan
coefficients. This is closely related to the \nameref{Threejsymbol}.

\begin{Syntax}

\name{Clebsch\_Gordan}\(\meta{list of j1,m1},\meta{list of j2,m2},
\meta{list of j3,m3}\)

\end{Syntax}

\begin{Examples}
 Clebsch_Gordan({2,0},{2,0},{2,0});  & \rfrac{-2}{sqrt(14)}\\
\end{Examples}
\end{Operator}

\begin{Operator}{SixjSymbol}
The \name{SixjSymbol} operator implements the 6j symbol.
\begin{Syntax}

\name{SixjSymbol}\(\meta{list of j1,j2,j3},\meta{list of l1,l2,l3}\)

\end{Syntax}
\begin{Examples}

SixjSymbol({7,6,3},{2,4,6}); & \rfrac{1}{14*sqrt(858)}\\

\end{Examples}

\begin{Comments}
The operator \name{SixjSymbol} uses the \nameref{ineq} package in order
to find minima and maxima for the summation index.
\end{Comments}
\end{Operator}

\subsection{Miscellaneous}


\begin{Operator}{HYPERGEOMETRIC}
\index{hypergeometric function}
\index{generalized hypergeometric function}
The \name{Hypergeometric} operator  provides simplifications for the 
generalized hypergeometric functions. 
The \name{Hypergeometric} operator is included in the package specfn2.

\begin{Syntax}
\name{hypergeometric}\(\meta{list of parameters},\meta{list of parameters},
   \meta{argument}\)
\end{Syntax}

\begin{Examples}
load specfn2;\\
hypergeometric ({1/2,1},{3/2},-x^2); & \rfrac{atan(x)}{x}\\
hypergeometric ({},{},z);      &   e^z
\end{Examples}

\begin{Comments}
The special case where the length of the first list is equal to 2 and
the length of the second list is equal to 1 is often called
``the hypergeometric function'' (notated as 2F1(a1,a2,b;x)).
\end{Comments}
\end{Operator}

\begin{Operator}{MeijerG}
The \name{MeijerG} operator provides simplifications for Meijer's G  
function. The simplifications are performed towards polynomials,
elementary or
special functions or (generalized) \nameref{hypergeometric} functions.

The \name{MeijerG} operator is included in the package specfn2.

\begin{Syntax}
\name{MeijerG}\(\meta{list of parameters},\meta{list of parameters},
   \meta{argument}\)
\end{Syntax}
The first element of the lists has to be the list containing the
first group (mostly called ``m'' and ``n'') of parameters. This passes
the  four parameters of a Meijer's G function implicitly via the
length of the lists.

\begin{Examples}
load specfn2;\\
MeijerG({{},1},{{0}},x);       &   heaviside(-x+1)\\
MeijerG({{}},{{1+1/4},1-1/4},(x^2)/4) * sqrt pi;
		& \rfrac{sqrt(2)*sin(x)*x^2}{4*sqrt(x)}
\end{Examples}

\begin{Comments}
Many well-known functions can be written as G functions,
e.g. exponentials, logarithms, trigonometric functions, Bessel functions
and hypergeometric functions.
The formulae can be found e.g. in \\
A.P.Prudnikov, Yu.A.Brychkov, O.I.Marichev: 
Integrals and Series, Volume 3: More special functions,
Gordon and Breach Science Publishers (1990).
\end{Comments}
\end{Operator}

\begin{Operator}{Heaviside}

The \name{Heaviside} operator returns the Heaviside function. \\

Heaviside(~w) => if (w < 0) then 0 else 1 \\
    when numberp w;

\begin{Syntax}
\name{Heaviside}\(\meta{argument}\)
\end{Syntax}

\begin{Comments}
This operator is often included in the result of the simplification
of a generalized \nameref{hypergeometric} function or a
\nameref{MeijerG} function.

No simplification is done for this function.
\end{Comments}
\end{Operator}

\begin{Operator}{erfi}

The \name{erfi} operator returns the error function of an imaginary argument.

erfi(~x) => 2/sqrt(pi) * defint(e**(t**2),t,0,x); 

\begin{Syntax}
\name{erfi}\(\meta{argument}\)
\end{Syntax}

\begin{Comments}
This operator is sometimes included in the result of the simplification
of a generalized \nameref{hypergeometric} function or a
\nameref{MeijerG} function.

No simplification is done for this function.
\end{Comments}
\end{Operator}

