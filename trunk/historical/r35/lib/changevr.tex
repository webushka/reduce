\newcommand{\definedas}{\stackrel{\triangle}{=}}
\newcommand{\spc}{\;\;\;\;\;}
\newcommand{\mspc}{\;\;\;\;\;\;\;\;\;\;}
\newcommand{\lspc}{\;\;\;\;\;\;\;\;\;\;\;\;\;\;\;}

\documentstyle[fleqn,11pt]{article}

\renewcommand{\baselinestretch}{1.1}
\newcommand{\eg}[1]{\begin{quote}{\tt #1} \end{quote}}
\setlength{\textwidth}{15cm}
\addtolength{\oddsidemargin}{-1cm}

\title{                    CHANGEVAR,  \\
                         A REDUCE Facility   \\
                              to   \\
                    Perform Change of Independent Variable(s)\\
                               in \\
                    Differential Equations\\[2cm]
        }
\author{
          G. \"{U}\c{c}oluk
             \thanks{Email address: A07917 @ TRMETU.BITNET}
                     \\
                               Department of Physics \\
                               Middle East Technical University \\
                               Ankara, Turkey
       }
\date{October 1989}

\begin{document}
\maketitle
\newpage

\section{Introduction}
The mathematics behind the change of independent variable(s) in differential
equations is quite straightforward. It is basically the application of the chain
rule. If the dependent variable of the differential equation is $F$, the
independent variables are $x_{i}$ and the new independent variables are
$u_{i}$ (where ${\scriptstyle i=1\ldots n}$) then the first derivatives are:
\[
    \frac{\partial F}{\partial x_{i}} = \frac{\partial F}{\partial u_{j}}
                                        \frac{\partial u_{j}}{\partial x_{i}}
\]
We assumed Einstein's summation convention. Here the problem is to
calculate the $\partial u_{j}/\partial x_{i}$ terms if the change of variables
is given by
\[
    x_{i} = f_{i}(u_{1},\ldots,u_{n})
\]
The first thought might be solving the above given equations for $u_{j}$ and
then differentiating them with respect to $x_{i}$, then again making use of the
equations above, substituting new variables for the old ones in the calculated
derivatives. This is not always a  preferable way to proceed. Mainly because
the functions $f_{i}$ may not always be easily invertible. Another approach
that makes use of the Jacobian is better. Consider the above given equations
which relate the old variables to the new ones. Let us differentiate them:
\begin{eqnarray*}
  \frac{\partial x_{j}}{\partial x_{i}} & = &
        \frac{\partial f_{j}}{\partial x_{i}}   \\
  \delta_{ij} & = &
        \frac{\partial f_{j}}{\partial u_{k}}
        \frac{\partial u_{k}}{\partial x_{i}}
\end{eqnarray*}
The first derivative is nothing but the $(j,k)$ th entry of the Jacobian matrix.

So if we speak in matrix language
\[ {\bf 1 = J \cdot D} \]
where we defined the Jacobian
\[ {\bf J}_{ij} \definedas  \frac{\partial f_{i}}{\partial u_{j}} \]
and the matrix of the derivatives we wanted to obtain as
\[ {\bf D}_{ij} \definedas  \frac{\partial u_{i}}{\partial x_{j}}. \]
If the Jacobian has a non-vanishing determinant then it is invertible and
we are able to write from the matrix equation above:
\[ {\bf  D = J^{-1}} \]
so finally we have what we want
\[
   \frac{\partial u_{i}}{\partial x_{j}} = \left[{\bf J^{-1}}\right]_{ij}
\]

The higher derivatives are obtained by the successive application of the chain
rule and using the definitions of the old variables in terms of the new ones. It

can be easily verified that the only derivatives that are needed to be
calculated are the first order ones which are obtained above.

\section{How to Use CHANGEVAR}
{\bf This facility requires the matrix package to be present in the session}.
So if it is not autoloaded in your REDUCE implementation, {\bf load it}.
On the PC version of REDUCE 3.3 this is done by:
\eg{LOAD MATR;}
in the REDUCE environment. Then simply read in the file {\tt CHVAR} by a
{\tt IN} statement:
\eg{IN CHVAR\$}
Now the REDUCE function {\tt CHANGEVAR} is ready to use.
{\bf Note: The file is named CHVAR, but the function has the name CHANGEVAR}.
The function {\tt CHANGEVAR} has (at least) four different arguments.
Here we give a list them:
\begin{itemize}
\item {\bf FIRST ARGUMENT} \\
     Is a list of the dependent variables of the differential equation.
     They shall be enclosed in a pair of curly braces and separated by commas.
     If there is only one dependent variable there is no need for the curly
     braces.
\item {\bf SECOND ARGUMENT}  \\
     Is a list of the {\bf new} independent variables. Similar to what is said
     for the first argument, these shall also be separated by commas,
     enclosed in curly braces and the curly braces can be omitted if there is
     only one new variable.
\item {\bf THIRD ARGUMENT}  \\
     Is a list of equations separated by commas, where each of the equation
     is of the form
      \eg{{\em old variable} = {\em a function in new variables}}
     The left hand side cannot be a non-kernel structure. In this argument
     the functions which give the old variables in terms of the new ones are
     introduced. It is possible to omit totally the curly braces which enclose
     the list. {\bf Please note that only for this argument it is allowed to
     omit the curly braces even if the list has \underline{more than one}
     items}.
\item {\bf LAST ARGUMENT}  \\
     Is a list of algebraic expressions which evaluates to  differential
     equations, separated by commas, enclosed in curly braces.
     So, variables in which differential equations are already stored may be
     used freely. Again it is possible to omit the curly braces if there is
     only {\bf one} differential equation.
\end{itemize}

If the last argument is a list then the result of {\tt CHANGEVAR} is a list too.


It is possible to display the entries of the inverse Jacobian, explained in the
introduction. To do so, turn {\tt ON} the flag {DISPJACOBIAN} by a statement:
\eg{ON DISPJACOBIAN;}



\section{AN EXAMPLE\ldots\ldots The 2-dim. Laplace Equation}
The 2-dimensional Laplace equation in cartesian coordinates is:
\[
   \frac{\partial^{2} u}{\partial x^{2}} +
   \frac{\partial^{2} u}{\partial y^{2}} = 0
\]
Now assume we want to obtain the polar coordinate form of Laplace equation.
The change of variables is:
\[
   x = r \cos \theta, \mspc  y = r \sin \theta
\]
The solution using {\tt CHANGEVAR}  (of course after it is properly loaded)
is as follows
\eg{CHANGEVAR(\{u\},\{r,theta\},\{x=r*cos theta,y=r*sin theta\}, \\
    \hspace*{2cm}     \{df(u(x,y),x,2)+df(u(x,y),y,2)\} )}
Here we could omit the curly braces in the first and last arguments (because
those lists have only one member) and the curly braces in the third argument
(because they are optional), but you cannot leave off the curly braces in the
second argument. So one could equivalently write
\eg{CHANGEVAR(u,\{r,theta\},x=r*cos theta,y=r*sin theta,        \\
    \hspace*{2cm}     df(u(x,y),x,2)+df(u(x,y),y,2) )}

If you have tried out the above example, you will notice that the denominator
contains a $\cos^{2} \theta + \sin^{2} \theta$ which is actually equal to $1$.
This has of course nothing to do with the {\tt CHANGEVAR} facility introduced
here. One has to be overcome these pattern matching problems by the conventional

methods REDUCE provides (a {\tt LET} statement, for example, will fix it).

Secondly you will notice that your {\tt u(x,y)} operator has changed to
{\tt u(r,theta)} in the result. Nothing magical  about this. That is just what
we do with pencil and paper. {\tt u(r,theta)} represents the  the transformed
dependent variable.

\section{ANOTHER EXAMPLE\ldots\ldots An Euler Equation}
Consider a differential equation which is of Euler type, for instance:
\[
   x^{3}y''' - 3 x^{2}y'' + 6 x y' - 6 y = 0
\]
Where prime denotes differentiation with respect to $x$.
As it is well known Euler type of equations are solved by a change of variable:
\[
   x = e^{u}
\]
So our  {\tt CHANGEVAR} call reads as follows:
\eg{CHANGEVAR(y, u, x=e**u, x**3*df(y(x),x,3)-   \\
    \hspace*{2cm}   3*x**2*df(y(x),x,2)+6*x*df(y(x),x)-6*y(x))}

\end{document}



