\chapter{Built-in Prefix Operators}
In the following subsections are descriptions of the most useful prefix
\index{Prefix}
operators built into {\REDUCE} that are not defined in other sections (such
as substitution operators). Some are fully defined internally as
procedures; others are more nearly abstract operators, with only some of
their properties known to the system.

In many cases, an operator is described by a prototypical header line as
follows. Each formal parameter is given a name and followed by its allowed
type. The names of classes referred to in the definition are printed in
lower case, and parameter names in upper case. If a parameter type is not
commonly used, it may be a specific set enclosed in brackets {\tt \{} \ldots
{\tt \}}.
Operators that accept formal parameter lists of arbitrary length have the
parameter and type class enclosed in square brackets indicating that zero
or more occurrences of that argument are permitted. Optional parameters
and their type classes are enclosed in angle brackets.

\section{Numerical Operators}\index{Numerical operator}
{\REDUCE} includes a number of functions that are analogs of those found
in most numerical systems.  With numerical arguments, such functions
return the expected result.  However, they may also be called with
non-numerical arguments.  In such cases, except where noted, the system
attempts to simplify the expression as far as it can.  In such cases, a
residual expression involving the original operator usually remains.
These operators are as follows:

\subsection{ABS}
{\tt ABS}\ttindex{ABS} returns the absolute value
of its single argument, if that argument has a numerical value.
A non-numerical argument is returned as an absolute value, with an overall
numerical coefficient taken outside the absolute value operator. For example:
\begin{verbatim}
	abs(-3/4)     ->  3/4
	abs(2a)       ->  2*ABS(A)
	abs(i)        ->  1
	abs(-x)       ->  ABS(X)
\end{verbatim}

\subsection{CEILING}\ttindex{CEILING}
This operator returns the ceiling (i.e., the least integer greater than
the given argument) if its single argument has a numerical value.  A
non-numerical argument is returned as an expression in the original
operator.  For example:

\begin{verbatim}
	ceiling(-5/4) ->  -1
	ceiling(-a)   ->  CEILING(-A)
\end{verbatim}

\subsection{CONJ}\ttindex{CONJ}
This returns the complex conjugate
of an expression, if that argument has an numerical value.  A
non-numerical argument is returned as an expression in the operators
{\tt REPART}\ttindex{REPART} and {\tt IMPART}\ttindex{IMPART}. For example:
\begin{verbatim}
	conj(1+i)     -> 1-I
	conj(a+i*b)   -> REPART(A) - REPART(B)*I - IMPART(A)*I
			 - IMPART(B)
\end{verbatim}

\subsection{FACTORIAL}\ttindex{FACTORIAL}

If the single argument of {\tt FACTORIAL} evaluates to a non-negative
integer, its factorial is returned.  Otherwise an expression involving
{\tt FACTORIAL} is returned. For example:
\begin{verbatim}
	factorial(5)  ->  120
	factorial(a)  ->  FACTORIAL(A)
\end{verbatim}

\subsection{FIX}\ttindex{FIX}
This operator returns the fixed value (i.e., the integer part of
the given argument) if its single argument has a numerical value.  A
non-numerical argument is returned as an expression in the original
operator.  For example:

\begin{verbatim}
	fix(-5/4)   ->  -1
	fix(a)      ->  FIX(A)
\end{verbatim}

\subsection{FLOOR}\ttindex{FLOOR}
This operator returns the floor (i.e., the greatest integer less than
the given argument) if its single argument has a numerical value.  A
non-numerical argument is returned as an expression in the original
operator.  For example:

\begin{verbatim}
	floor(-5/4)   ->  -2
	floor(a)      ->  FLOOR(A)
\end{verbatim}

\subsection{IMPART}\ttindex{IMPART}
This operator returns the imaginary part of an expression, if that argument
has an numerical value.  A non-numerical argument is returned as an expression
in the operators {\tt REPART}\ttindex{REPART} and {\tt IMPART}.  For example:
\begin{verbatim}
	impart(1+i)   -> 1
	impart(a+i*b) -> REPART(B) + IMPART(A)
\end{verbatim}

\subsection{MAX/MIN}

{\tt MAX} and {\tt MIN}\ttindex{MAX}\ttindex{MIN} can take an arbitrary
number of expressions as their arguments.  If all arguments evaluate to
numerical values, the maximum or minimum of the argument list is returned.
If any argument is non-numeric, an appropriately reduced expression is
returned.  For example:
\begin{verbatim}
	max(2,-3,4,5) ->  5
	min(2,-2)     ->  -2.
	max(a,2,3)    ->  MAX(A,3)
	min(x)        ->  X
\end{verbatim}
{\tt MAX} or {\tt MIN} of an empty list returns 0.

\subsection{NEXTPRIME}\ttindex{NEXTPRIME}

{\tt NEXTPRIME} returns the next prime greater than its integer argument,
using a probabilistic algorithm.  A type error occurs if the value of the
argument is not an integer.  For example:
\begin{verbatim}
	nextprime(5)      ->  7
	nextprime(-2)     ->  2
	nextprime(-7)     -> -5
	nextprime 1000000 -> 1000003
\end{verbatim}
whereas {\tt nextprime(a)} gives a type error.

\subsection{RANDOM}\ttindex{RANDOM}

{\tt random(}{\em n\/}{\tt)} returns a random number $r$ in the range $0
\leq r < n$.  A type error occurs if the value of the argument is not a
positive integer in algebraic mode, or positive number in symbolic mode.
For example:
\begin{verbatim}
	random(5)         ->    3
	random(1000)      ->  191
\end{verbatim}
whereas {\tt random(a)} gives a type error.

\subsection{RANDOM\_NEW\_SEED}\ttindex{RANDOM\_NEW\_SEED}

{\tt random\_new\_seed(}{\em n\/}{\tt)} reseeds the random number generator
to a sequence determined by the integer argument $n$.  It can be used to
ensure that a repeatable pseudo-random sequence will be delivered
regardless of any previous use of {\tt RANDOM}, or can be called early in
a run with an argument derived from something variable (such as the time
of day) to arrange that different runs of a REDUCE program will use
different random sequences.  When a fresh copy of REDUCE is first created
it is as if {\tt random\_new\_seed(1)} has been obeyed.

A type error occurs if the value of the argument is not a positive integer.

\subsection{REPART}\ttindex{REPART}
This returns the real part of an expression, if that argument has an
numerical value.  A non-numerical argument is returned as an expression in
the operators {\tt REPART} and {\tt IMPART}\ttindex{IMPART}.  For example:
\begin{verbatim}
	repart(1+i)   -> 1
	repart(a+i*b) -> REPART(A) - IMPART(B)
\end{verbatim}

\subsection{ROUND}\ttindex{ROUND}
This operator returns the rounded value (i.e, the nearest integer) of its
single argument if that argument has a numerical value.  A non-numeric
argument is returned as an expression in the original operator.  For
example:
\begin{verbatim}
	round(-5/4)   ->  -1
	round(a)      ->  ROUND(A)
\end{verbatim}

\subsection{SIGN}\ttindex{SIGN}
{\tt SIGN} tries to evaluate the sign of its argument. If this
is possible {\tt SIGN} returns one of 1, 0 or -1.  Otherwise, the result
is the original form or a simplified variant. For example:
\begin{verbatim}
	sign(-5)      ->  -1
	sign(-a^2*b)  ->  -SIGN(B)
\end{verbatim}
Note that even powers of formal expressions are assumed to be
positive only as long as the switch {\tt COMPLEX} is off.

\section{Mathematical Functions}

{\REDUCE} knows that the following represent mathematical functions
\index{Mathematical function} that can
take arbitrary scalar expressions as their single argument:
\begin{verbatim}
       ACOS ACOSH ACOT ACOTH ACSC ACSCH ASEC ASECH ASIN ASINH
       ATAN ATANH ATAN2 COS COSH COT COTH CSC CSCH DILOG EI EXP
       HYPOT LN LOG LOGB LOG10 SEC SECH SIN SINH SQRT TAN TANH
\end{verbatim}
\ttindex{ACOS}\ttindex{ACOSH}\ttindex{ACOT}
\ttindex{ACOTH}\ttindex{ACSC}\ttindex{ACSCH}\ttindex{ASEC}
\ttindex{ASECH}\ttindex{ASIN}
\ttindex{ASINH}\ttindex{ATAN}\ttindex{ATANH}
\ttindex{ATAN2}\ttindex{COS}
\ttindex{COSH}\ttindex{COT}\ttindex{COTH}\ttindex{CSC}
\ttindex{CSCH}\ttindex{DILOG}\ttindex{Ei}\ttindex{EXP}
\ttindex{HYPOT}\ttindex{LN}\ttindex{LOG}\ttindex{LOGB}\ttindex{LOG10}
\ttindex{SEC}\ttindex{SECH}\ttindex{SIN}
\ttindex{SINH}\ttindex{SQRT}\ttindex{TAN}\ttindex{TANH}
where {\tt LOG} is the natural logarithm (and equivalent to {\tt LN}),
and {\tt LOGB} has two arguments of which the second is the logarithmic base.

The derivatives of all these functions are also known to the system.

{\REDUCE} knows various elementary identities and properties
of these functions. For example:
\begin{verbatim}
      cos(-x) = cos(x)              sin(-x) = - sin (x)
      cos(n*pi) = (-1)^n            sin(n*pi) = 0
      log(e)  = 1                   e^(i*pi/2) = i
      log(1)  = 0                   e^(i*pi) = -1
      log(e^x) = x                  e^(3*i*pi/2) = -i
\end{verbatim}

Beside these identities, there are a lot of simplifications
for elementary functions
defined in the {\REDUCE} system as rulelists. In order to
view these, the SHOWRULES operator can be used, e.g.

\begin{verbatim}
      SHOWRULES tan;

{tan(~n*arbint(~i)*pi + ~(~ x)) => tan(x) when fixp(n),

 tan(~x)

  => trigquot(sin(x),cos(x)) when knowledge_about(sin,x,tan)

 ,

      ~x + ~(~ k)*pi
 tan(----------------)
	    ~d

	     x                             k    1
  =>  - cot(---) when x freeof pi and abs(---)=---,
	     d                             d    2

      ~(~ w) + ~(~ k)*pi           w + remainder(k,d)*pi
 tan(--------------------) => tan(-----------------------)
	    ~(~ d)                           d

			       k
 when w freeof pi and ratnump(---) and fixp(k)
			       d

	   k
  and abs(---)>=1,
	   d

 tan(atan(~x)) => x,

			     2
 df(tan(~x),~x) => 1 + tan(x) }

\end{verbatim}

For further simplification, especially of expressions involving
trigonometric functions, see the TRIGSIMP\ttindex{TRIGSIMP} package
documentation.

Functions not listed above may be defined in the special functions
package SPECFN\ttindex{SPECFN}.

The user can add further rules for the reduction of expressions involving
these operators by using the {\tt LET}\ttindex{LET} command.

% The square root function can be input using the name {\tt SQRT}, or the
% power operation {\tt \verb|^|(1/2)}.  On output, unsimplified square roots
% are normally represented by the operator {\tt SQRT} rather than a
% fractional power.

In many cases it is desirable to expand product arguments of logarithms,
or collect a sum of logarithms into a single logarithm.  Since these are
inverse operations, it is not possible to provide rules for doing both at
the same time and preserve the {\REDUCE} concept of idempotent evaluation.
As an alternative, REDUCE provides two switches {\tt EXPANDLOGS}
\ttindex{EXPANDLOGS} and {\tt COMBINELOGS}\ttindex{COMBINELOGS} to carry
out these operations.  Both are off by default.  Thus to expand {\tt
LOG(X*Y)} into a sum of logs, one can say
\begin{verbatim}
	ON EXPANDLOGS; LOG(X*Y);
\end{verbatim}
and to combine this sum into a single log:
\begin{verbatim}
	ON COMBINELOGS; LOG(X) + LOG(Y);
\end{verbatim}

At the present time, it is possible to have both switches on at once,
which could lead to infinite recursion.  However, an expression is
switched from one form to the other in this case.  Users should not rely
on this behavior, since it may change in the next release.

The current version of {\REDUCE} does a poor job of simplifying surds.  In
particular, expressions involving the product of variables raised to
non-integer powers do not usually have their powers combined internally,
even though they are printed as if those powers were combined.  For
example, the expression
\begin{verbatim}
	x^(1/3)*x^(1/6);
\end{verbatim}
will print as
\begin{verbatim}
	SQRT(X)
\end{verbatim}
but will have an internal form containing the two exponentiated terms.
If you now subtract {\tt sqrt(x)} from this expression, you will {\em not\/}
get zero.  Instead, the confusing form
\begin{verbatim}
	SQRT(X) - SQRT(X)
\end{verbatim}
will result.  To combine such exponentiated terms, the switch
{\tt COMBINEEXPT}\ttindex{COMBINEEXPT} should be turned on.

The square root function can be input using the name {\tt SQRT}, or the
power operation {\tt \verb|^|(1/2)}.  On output, unsimplified square roots
are normally represented by the operator {\tt SQRT} rather than a
fractional power.  With the default system switch settings, the argument
of a square root is first simplified, and any divisors of the expression
that are perfect squares taken outside the square root argument.  The
remaining expression is left under the square root.
% However, if the switch {\tt REDUCED}\ttindex{REDUCED} is on,
% multiplicative factors in the argument of the square root are also
% separated, becoming individual square roots. Thus with {\tt REDUCED} off,
Thus the expression
\begin{verbatim}
	 sqrt(-8a^2*b)
 \end{verbatim}
becomes
\begin{verbatim}
	2*a*sqrt(-2*b).
\end{verbatim}
% whereas with {\tt REDUCED} on, it would become
% \begin{verbatim}
%         2*a*i*sqrt(2)*sqrt(b) .
% \end{verbatim}
% The switch {\tt REDUCED}\ttindex{REDUCED} also applies to other rational
% powers in addition to square roots.

Note that such simplifications can cause trouble if {\tt A} is eventually
given a value that is a negative number.  If it is important that the
positive property of the square root and higher even roots always be
preserved, the switch {\tt PRECISE}\ttindex{PRECISE} should be set on
(the default value).
This causes any non-numerical factors taken out of surds to be represented
by their absolute value form.
With % both {\tt REDUCED} and
{\tt PRECISE} on then, the above example would become
\begin{verbatim}
	2*abs(a)*sqrt(-2*b).
\end{verbatim}

The statement that {\REDUCE} knows very little about these functions
applies only in the mathematically exact {\tt off rounded} mode.  If
{\tt ROUNDED}\ttindex{ROUNDED} is on, any of the functions
\begin{verbatim}
	ACOS ACOSH ACOT ACOTH ACSC ACSCH ASEC ASECH ASIN ASINH
	ATAN ATANH ATAN2 COS COSH COT COTH CSC CSCH EXP HYPOT
	LN LOG LOGB LOG10 SEC SECH SIN SINH SQRT TAN TANH
\end{verbatim}
\ttindex{ACOS}\ttindex{ACOSH}\ttindex{ACOT}\ttindex{ACOTH}
\ttindex{ACSC}\ttindex{ACSCH}\ttindex{ASEC}\ttindex{ASECH}
\ttindex{ASIN}\ttindex{ASINH}\ttindex{ATAN}\ttindex{ATANH}
\ttindex{ATAN2}\ttindex{COS}\ttindex{COSH}\ttindex{COT}
\ttindex{COTH}\ttindex{CSC}\ttindex{CSCH}\ttindex{EXP}\ttindex{HYPOT}
\ttindex{LN}\ttindex{LOG}\ttindex{LOGB}\ttindex{LOG10}\ttindex{SEC}
\ttindex{SECH}\ttindex{SIN}\ttindex{SINH}\ttindex{SQRT}\ttindex{TAN}
\ttindex{TANH}
when given a numerical argument has its value calculated to the current
degree of floating point precision.  In addition, real (non-integer
valued) powers of numbers will also be evaluated.

If the {\tt COMPLEX} switch is turned on in addition to {\tt ROUNDED},
these functions will also calculate a real or complex result, again to
the current degree of floating point precision,
if given complex arguments.  For example, with {\tt on rounded,complex;}
\begin{verbatim}
	2.3^(5.6i)   ->   -0.0480793490914 - 0.998843519372*I
	cos(2+3i)    ->   -4.18962569097 - 9.10922789376*I
\end{verbatim}

\section{DF Operator}
The operator {\tt DF}\ttindex{DF} is used to represent partial
differentiation\index{Differentiation} with respect
to one or more variables. It is used with the syntax:
\begin{verbatim}
     DF(EXPRN:algebraic[,VAR:kernel<,NUM:integer>]):algebraic.
\end{verbatim}
The first argument is the expression to be differentiated. The remaining
arguments specify the differentiation variables and the number of times
they are applied.

The number {\tt NUM} may be omitted if it is 1.  For example,
\begin{quote}
\begin{tabbing}
{\tt            df(y,x1,2,x2,x3,2)} \= = $\partial^{5}y/\partial x_{1}^{2} \
 \partial x_{2}\partial x_{3}^{2}.$\kill
{\tt            df(y,x)} \> = $\partial y/\partial x$ \\
{\tt            df(y,x,2)} \> = $\partial^{2}y/\partial x^{2}$ \\
{\tt            df(y,x1,2,x2,x3,2)} \> = $\partial^{5}y/\partial x_{1}^{2} \
 \partial x_{2}\partial x_{3}^{2}.$
\end{tabbing}
\end{quote}
The evaluation of {\tt df(y,x)} proceeds as follows: first, the values of
{\tt Y} and {\tt X} are found.  Let us assume that {\tt X} has no assigned
value, so its value is {\tt X}.  Each term or other part of the value of
{\tt Y} that contains the variable {\tt X} is differentiated by the
standard rules.  If {\tt Z} is another variable, not {\tt X} itself, then
its derivative with respect to {\tt X} is taken to be 0, unless {\tt Z}
has previously been declared to {\tt DEPEND} on {\tt X}, in which
case the derivative is reported as the symbol {\tt df(z,x)}.


\subsection{Adding Differentiation Rules}

The {\tt LET}\ttindex{LET} statement can be used to introduce
rules for differentiation of user-defined operators.  Its general form is
\begin{verbatim}
	FOR ALL <var1>,...,<varn>
	     LET DF(<operator><varlist>,<vari>)=<expression>
\end{verbatim}
where {\tt <varlist>} ::= ({\tt <var1>},\dots,{\tt <varn>}), and
{\tt <var1>},...,{\tt <varn>} are the dummy variable arguments of
{\tt <operator>}.

An analogous form applies to infix operators.

{\it Examples:}
\begin{verbatim}
	for all x let df(tan x,x)= 1 + tan(x)^2;
\end{verbatim}
(This is how the tan differentiation rule appears in the {\REDUCE}
source.)
\begin{verbatim}
	for all x,y let df(f(x,y),x)=2*f(x,y),
			df(f(x,y),y)=x*f(x,y);
\end{verbatim}
Notice that all dummy arguments of the relevant operator must be declared
arbitrary by the {\tt FOR ALL} command, and that rules may be supplied for
operators with any number of arguments.  If no differentiation rule
appears for an argument in an operator, the differentiation routines will
return as result an expression in terms of {\tt DF}\ttindex{DF}.  For
example, if the rule for the differentiation with respect to the second
argument of {\tt F} is not supplied, the evaluation of {\tt df(f(x,z),z)}
would leave this expression unchanged. (No {\tt DEPEND} declaration
is needed here, since {\tt f(x,z)} obviously ``depends on'' {\tt Z}.)

Once such a rule has been defined for a given operator, any future
differentiation\index{Differentiation} rules for that operator must be
defined with the same number of arguments for that operator, otherwise we
get the error message
\begin{verbatim}
	Incompatible DF rule argument length for <operator>
\end{verbatim}

\section{INT Operator}
{\tt INT}\ttindex{INT} is an operator in {\REDUCE} for indefinite
integration\index{Integration}\index{Indefinite integration} using a
combination of the Risch-Norman algorithm and pattern matching.  It is
used with the syntax:
\begin{verbatim}
   INT(EXPRN:algebraic,VAR:kernel):algebraic.
\end{verbatim}
This will return correctly the indefinite integral for expressions comprising
polynomials, log functions, exponential functions and tan and atan. The
arbitrary constant is not represented. If the integral cannot be done in
closed terms, it returns a formal integral for the answer in one of two ways:
\begin{enumerate}
\item It returns the input, {\tt INT(\ldots,\ldots)} unchanged.

\item It returns an expression involving {\tt INT}s of some
      other functions (sometimes more complicated than
      the original one, unfortunately).
\end{enumerate}
Rational functions can be integrated when the denominator is factorizable
by the program. In addition it will attempt to integrate expressions
involving error functions, dilogarithms and other trigonometric
expressions. In these cases it might not always succeed in finding the
solution, even if one exists.

{\it Examples:}
\begin{verbatim}
	int(log(x),x) ->  X*(LOG(X) - 1),
	int(e^x,x)    ->  E**X.
\end{verbatim}
The program checks that the second argument is a variable and gives an
error if it is not.

{\it Note:} If the {\tt int} operator is called with 4 arguments,
{\REDUCE} will implicitly call the definite integration package (DEFINT)
and this package will interpret the third and fourth arguments as the lower
and upper limit of integration, respectively.  For details, consult
the documentation on the DEFINT package.


\subsection{Options}

The switch {\tt TRINT} when on will trace the operation of the algorithm. It
produces a great deal of output in a somewhat illegible form, and is not
of much interest to the general user. It is normally off.

If the switch {\tt FAILHARD} is on the algorithm will terminate with an
error if the integral cannot be done in closed terms, rather than return a
formal integration form. {\tt FAILHARD} is normally off.

The switch {\tt NOLNR} suppresses the use of the linear properties of
integration in cases when the integral cannot be found in closed terms.
It is normally off.

\subsection{Advanced Use}

If a function appears in the integrand that is not one of the functions
{\tt EXP, ERF, TAN, ATAN, LOG, DILOG}\ttindex{EXP}\ttindex{ERF}
\ttindex{TAN}\ttindex{ATAN}\ttindex{LOG}\ttindex{DILOG}
then the algorithm will make an
attempt to integrate the argument if it can, differentiate it and reach a
known function.  However the answer cannot be guaranteed in this case.  If
a function is known to be algebraically independent of this set it can be
flagged transcendental by
\begin{verbatim}
	flag('(trilog),'transcendental);
\end{verbatim}
in which case this function will be added to the permitted field
descriptors for a genuine decision procedure. If this is done the user is
responsible for the mathematical correctness of his actions.

The standard version does not deal with algebraic extensions. Thus
integration of expressions involving square roots and other like things
can lead to trouble.  A contributed package that supports integration of
functions involving square roots is available, however
(ALGINT\extendedmanual{, chapter~\ref{ALGINT}}).
In addition there is a definite integration
package, DEFINT\extendedmanual{( chapter~\ref{DEFINT})}.

\subsection{References}

	A. C. Norman \& P. M. A. Moore, ``Implementing the New Risch
		Algorithm'', Proc. 4th International Symposium on Advanced
		Comp. Methods in Theor. Phys., CNRS, Marseilles, 1977.

	S. J. Harrington, ``A New Symbolic Integration System in Reduce'',
		Comp. Journ. 22 (1979) 2.

	A. C. Norman \& J. H. Davenport, ``Symbolic Integration --- The Dust
		Settles?'', Proc. EUROSAM 79, Lecture Notes in Computer
		Science 72, Springer-Verlag, Berlin Heidelberg New York
		(1979) 398-407.

%\subsection{Definite Integration} \index{Definite integration}
%
%If {\tt INT} is used with the syntax
%
%\begin{verbatim}
%   INT(EXPRN:algebraic,VAR:kernel,LOWER:algebraic,UPPER:algebraic):algebraic.
%\end{verbatim}
%
%The definite integral of {\tt EXPRN} with respect to {\tt VAR} is
%calculated between the limits {\tt LOWER} and {\tt UPPER}.  In the present
%system, this is calculated either by pattern matching, or by first finding
%the indefinite integral, and then substituting the limits into this.

\section{LENGTH Operator}
{\tt LENGTH}\ttindex{LENGTH} is a generic operator for finding the
length of various objects in the system.  The meaning depends on the type
of the object.  In particular, the length of an algebraic expression is
the number of additive top-level terms its expanded representation.

{\it Examples:}
\begin{verbatim}
	length(a+b)    ->  2
	length(2)      ->  1.
\end{verbatim}
Other objects that support a length operator include arrays, lists and
matrices. The explicit meaning in these cases is included in the description
of these objects.

\section{MAP Operator}\ttindex{MAP}

The {\tt MAP} operator applies a uniform evaluation pattern to all members
of a composite structure: a matrix, a list, or the arguments of an
operator expression.  The evaluation pattern can be a unary procedure, an
operator, or an algebraic expression with one free variable.

It is used with the syntax:
\begin{verbatim}
   MAP(U:function,V:object)
\end{verbatim}
Here {\tt object} is a list, a matrix or an operator expression.
{\tt Function} can be one of the following:
\begin{enumerate}
\item the name of an operator for a single argument: the operator
is evaluated once with each element of {\tt object} as its single argument;
\item an algebraic expression with exactly one free variable, that is
a variable preceded by the tilde symbol. The expression
is evaluated for each element of {\tt object}, where the element is
substituted for the free variable;
\item a replacement rule of the form {\tt var => rep}
where {\tt var} is a variable (a kernel without a subscript)
and {\tt rep} is an expression that contains {\tt var}.
{\tt Rep} is evaluated for each element of {\tt object} where
the element is substituted for  {\tt var}. {\tt Var} may be
optionally preceded by a tilde.
\end{enumerate}
The rule form  for {\tt function} is needed when more than
one free variable occurs.

Examples:
\begin{verbatim}
	map(abs,{1,-2,a,-a})  ->  {1,2,ABS(A),ABS(A)}
	map(int(~w,x), mat((x^2,x^5),(x^4,x^5))) ->

		[  3     6 ]
		[ x     x  ]
		[----  ----]
		[ 3     6  ]
		[          ]
		[  5     6 ]
		[ x     x  ]
		[----  ----]
		[ 5     6  ]

	map(~w*6, x^2/3 = y^3/2 -1) -> 2*X^2=3*(Y^3-2)
\end{verbatim}

You can use {\tt MAP} in nested expressions. However, you cannot
apply {\tt MAP} to a non-composed object, e.g. an identifier or a number.


\section{MKID Operator}\ttindex{MKID}
In many applications, it is useful to create a set of identifiers for
naming objects in a consistent manner. In most cases, it is sufficient to
create such names from two components. The operator {\tt MKID} is provided
for this purpose. Its syntax is:
\begin{verbatim}
MKID(U:id,V:id|non-negative integer):id
\end{verbatim}
for example
\begin{verbatim}
	mkid(a,3)      -> A3
	mkid(apple,s)  -> APPLES
\end{verbatim}
while {\tt mkid(a+b,2)} gives an error.

The {\tt SET}\ttindex{SET} operator can be used to give a value to the
identifiers created by {\tt MKID}, for example
\begin{verbatim}
	set(mkid(a,3),3);
\end{verbatim}
will give {\tt A3} the value 2.

\section{PF Operator}\ttindex{PF}

{\tt PF(<exp>,<var>)} transforms the expression {\tt <exp>} into a list of
partial fractions with respect to the main variable, {\tt <var>}.  {\tt PF}
does a complete partial fraction decomposition, and as the algorithms used
are fairly unsophisticated (factorization and the extended Euclidean
algorithm), the code may be unacceptably slow in complicated cases.

{\it Example:}
Given {\tt 2/((x+1)\verb|^|2*(x+2))} in the workspace,
{\tt pf(ws,x);} gives the result
\begin{verbatim}
	    2      - 2         2
	{-------,-------,--------------} .
	  X + 2   X + 1    2
			  X  + 2*X + 1
\end{verbatim}

If you want the denominators in factored form, use {\tt off exp;}.
Thus, with {\tt 2/((x+1)\verb|^|2*(x+2))} in the workspace, the commands
{\tt off exp; pf(ws,x);} give the result
\begin{verbatim}
	    2      - 2       2
	{-------,-------,----------} .
	  X + 2   X + 1          2
			  (X + 1)
\end{verbatim}

To recombine the terms, {\tt FOR EACH \ldots SUM} can be used.  So with
the above list in the workspace, {\tt for each j in ws sum j;} returns the
result
\begin{verbatim}
	     2
     ------------------
		     2
      (X + 2)*(X + 1)
\end{verbatim}

Alternatively, one can use the operations on lists to extract any desired
term.

\section{SELECT Operator}\ttindex{SELECT}
\ttindex{map}\ttindex{list}

The {\tt SELECT} operator extracts from a list,
or from the arguments of an n--ary operator, elements corresponding
to a boolean predicate. It is used with the syntax:
\begin{verbatim}
  SELECT(U:function,V:list)
\end{verbatim}

{\tt Function} can be one of the following forms:
\begin{enumerate}
\item the name of an operator for a single argument: the operator
is evaluated once with each element of {\tt object} as its single argument;
\item an algebraic expression with exactly one free variable, that is
a variable preceded by the tilde symbol. The expression
is evaluated for each element of \meta{object}, where the element is
substituted for the free variable;
\item a replacement rule of the form \meta{var $=>$ rep}
where {\tt var} is a variable (a kernel without subscript)
and {\tt rep} is an expression that contains {\tt var}.
{\tt Rep} is evaluated for each element of {\tt object} where
the element is substituted for  {\tt var}. {\tt var} may be
optionally preceded by a tilde.
\end{enumerate}
The rule form  for {\tt function} is needed when more than
one free variable occurs.

The result of evaluating {\tt function} is
interpreted as a boolean value corresponding to the conventions of
{\REDUCE}. These values are composed with the leading operator of the
input expression.

{\it Examples:}
\begin{verbatim}
	select( ~w>0 , {1,-1,2,-3,3}) -> {1,2,3}
	select(evenp deg(~w,y),part((x+y)^5,0):=list)
	       -> {X^5 ,10*X^3*Y^2 ,5*X*Y^4}
	select(evenp deg(~w,x),2x^2+3x^3+4x^4) -> 4X^4 + 2X^2
\end{verbatim}


