\documentstyle[11pt,reduce]{article}
\title{Algebraic Number Fields}
\date{}
\author{Eberhard Schr\"{u}fer \\
Institute SCAI.Alg   \\
German National Research Center for Information Technology (GMD) \\
Schloss Birlinghoven \\
D-53754 Sankt Augustin \\
Germany       \\[0.05in]
Email: schruefer@gmd.de}
\begin{document}
\maketitle

\index{algebraic number fields}
\index{algebraic numbers}
\index{ARNUM package}
Algebraic numbers are the solutions of an irreducible polynomial over
some ground domain.  \index{i} The algebraic number $i$ (imaginary
unit), \index{imaginary unit} for example, would be defined by the
polynomial $i^2 + 1$.  The arithmetic of algebraic number $s$ can be
viewed as a polynomial arithmetic modulo the defining polynomial.

Given a defining polynomial for an algebraic number $a$
\begin{eqnarray*}
a^n ~ + ~ {p _{n-1}} {a ^ {n -1}} ~ + ~ ... ~ + ~ {p_0}
\end{eqnarray*}

All algebraic numbers which can be built up from $a$ are then of the form:
\begin{eqnarray*}
{r_{n-1}} {a ^{n-1}} ~+~ {r_{n-2}} {a ^{n-2}} ~+~ ... ~+~ {r_0}
\end{eqnarray*}
where the $r_j$'s are rational numbers.

\index{+ ! algebraic numbers}
The operation of addition is defined by
\begin{eqnarray*}
({r_{n-1}} {a ^{n-1}} ~+~ {r_{n-2}} {a ^{n-2}} ~+~ ...) ~ + ~
({s_{n-1}} {a ^{n-1}} ~+~ {s_{n-2}} {a ^{n-2}} ~+~ ...) ~ =  \\
({r_{n-1}+s_{n-1}}) {a ^{n-1}} ~+~ ({r_{n-2}+s_{n-2}}) {a ^{n-2}} ~+~ ...
\end{eqnarray*}

\index{* ! algebraic numbers}
Multiplication of two algebraic numbers can be performed by normal
polynomial multiplication followed by a reduction of the result with the
help of the defining polynomial.

\begin{eqnarray*}
({r_{n-1}} {a ^{n-1}} + {r_{n-2}} {a ^{n-2}} + ...) ~ \times ~
({s_{n-1}} {a ^{n-1}} + {s_{n-2}} {a ^{n-2}} + ...) = \\
 {r_{n-1}} {s ^{n-1}}{a^{2n-2}} +  ... ~ {\bf modulo} ~
a^n ~ + ~ {p _{n-1}} {a ^ {n -1}} ~ + ~ ... ~ + ~ {p_0} \\
= ~~~{q_{n-1}} a^{n-1} ~ + ~ {q _{n-2}} {a ^ {n -2}} ~ + ~ ...
\end{eqnarray*}

\index{/ ! algebraic numbers}
Division of two algebraic numbers r and s yields another algebraic number q.

$ \frac{r}{s} = q$ or $ r = q s $.

The last equation written out explicitly reads

\begin{eqnarray*}
\lefteqn{({r_{n-1}} {a^{n-1}} + {r_{n-2}} {a^{n-2}} + \ldots)} \\
& = & ({q_{n-1}} {a^{n-1}} + {q_{n-2}} {a^{n-2}} + \ldots) \times
({s_{n-1}} {a^{n-1}} + {s_{n-2}} {a^{n-2}} + \ldots) \\
& & {\bf modulo} (a^n + {p _{n-1}} {a^{n -1}} + \ldots) \\
& = & ({t_{n-1}} {a^{n-1}} + {t_{n-2}} {a^{n-2}} + \ldots)
\end{eqnarray*}

The $t_i$ are linear in the $q_j$.  Equating equal powers of $a$ yields a
linear system for the quotient coefficients $q_j$.

With this, all field operations for the algebraic numbers are available.  The
translation into algorithms is straightforward.  For an implementation we
have to decide on a data structure for an algebraic number.  We have chosen
the representation REDUCE normally uses for polynomials, the so-called
standard form.  Since our polynomials have in general rational coefficients,
we must allow for a rational number domain inside the algebraic number.

\begin{tabbing}
\s{algebraic number} ::= \\
\hspace{.25in} \= {\tt :ar:} . \s{univariate polynomial over the rationals}
\\[0.05in]

\s{univariate polynomial over the rationals} ::= \\
\> \s{variable} .** \s{ldeg} .* \s{rational} .+ \s{reductum} \\[0.05in]

\s{ldeg} ::= integer \\[0.3in]

\s{rational} ::= \\
\> {\tt :rn:} . \s{integer numerator} . \s{integer denominator} :
integer \\[0.05in]

\s{reductum} ::= \s{univariate polynomial} : \s{rational} : nil
\end{tabbing}

This representation allows us to use the REDUCE functions for adding and
multiplying polynomials on the tail of the tagged algebraic number.  Also,
the routines for solving linear equations can easily be used for the
calculation of quotients.  We are still left with the problem of
introducing a particular algebraic number.  In the current version this is
done by giving the defining polynomial to the statement {\bf defpoly}.  The
\index{DEFPOLY statement}
algebraic number sqrt(2), for example, can be introduced by
\begin{verbatim}
   defpoly sqrt2**2 - 2;
\end{verbatim}

This statement associates a simplification function for the
translation of the variable in the defining polynomial into its tagged
internal form and also generates a power reduction rule used by the
operations {\bf times} and {\bf quotient} for the reduction of their
result modulo the defining polynomial.  A basis for the representation
of an algebraic number is also set up by the statement.  In the
working version, the basis is a list of powers of the indeterminate of
the defining polynomial up to one less then its degree.  Experiments
with integral bases, however, have been very encouraging, and these
bases might be available in a later version.  If the defining
polynomial is not monic, it will be made so by an appropriate
substitution.

\example \index{ARNUM package ! example}

\begin{verbatim}
     defpoly sqrt2**2-2;

     1/(sqrt2+1);

     sqrt2 - 1

     (x**2+2*sqrt2*x+2)/(x+sqrt2);

     x + sqrt2

     on gcd;

     (x**3+(sqrt2-2)*x**2-(2*sqrt2+3)*x-3*sqrt2)/(x**2-2);

       2
     (x  - 2*x - 3)/(x - sqrt2)

     off gcd;

     sqrt(x**2-2*sqrt2*x*y+2*y**2);

     abs(x - sqrt2*y)
\end{verbatim}

Until now we have dealt with only a single algebraic number.  In practice
this is not sufficient as very often several algebraic numbers appear in an
expression.  There are two possibilities for handling this: one can use
multivariate extensions \cite{Davenport:81} or one can construct a defining
polynomial that contains all specified extensions.  This package implements
the latter case (the so called primitive representation).  The algorithm we
use for the construction of the primitive element is the same as given by
Trager \cite{Trager:76}.  In the implementation, multiple extensions can be
given as a list of equations to the statement {\bf defpoly}, which, among other
things, adds the new extension to the previously defined one.  All
algebraic numbers are then expressed in terms of the primitive element.


\example\index{ARNUM package ! example}

\begin{verbatim}
   defpoly sqrt2**2-2,cbrt5**3-5;

   *** defining polynomial for primitive element:

     6       4        3        2
   a1  - 6*a1  - 10*a1  + 12*a1  - 60*a1 + 17

   sqrt2;

             5             4              3              2
   48/1187*a1  + 45/1187*a1  - 320/1187*a1  - 780/1187*a1  +


   735/1187*a1 - 1820/1187

   sqrt2**2;

   2
\end{verbatim}
\newpage
We can provide factorization of polynomials over the algebraic number
domain by using Trager's algorithm.  The polynomial to be factored is first
mapped to a polynomial over the integers by computing the norm of the
polynomial, which is the resultant with respect to the primitive element of
the polynomial and the defining polynomial.  After factoring over the
integers, the factors over the algebraic number field are recovered by GCD
calculations.

\example\index{ARNUM package ! example}

\begin{verbatim}
   defpoly a**2-5;

   on factor;

   x**2 + x - 1;

   (x + (1/2*a + 1/2))*(x - (1/2*a - 1/2))
\end{verbatim}
\index{SPLIT\_FIELD function}
We have also incorporated a function {\bf split\_field} for the calculation
of a primitive element of minimal degree for which a given polynomial splits
into linear factors.  The algorithm as described in Trager's article is
essentially a repeated primitive element calculation.

\example\index{ARNUM package ! example}

\begin{verbatim}
   split_field(x**3-3*x+7);

   *** Splitting field is generated by:

     6        4        2
   a2  - 18*a2  + 81*a2  + 1215



            4          2
   {1/126*a2  - 5/42*a2  - 1/2*a2 + 2/7,


              4          2
    - (1/63*a2  - 5/21*a2  + 4/7),


           4          2
   1/126*a2  - 5/42*a2  + 1/2*a2 + 2/7}


   for each j in ws product (x-j);

    3
   x  - 3*x + 7
\end{verbatim}

A more complete description can be found in \cite{Bradford:86}.
\bibliography{arnum}
\bibliographystyle{plain}
\end{document}

