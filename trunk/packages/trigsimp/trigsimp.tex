\documentclass[11pt]{article}

\usepackage{reduce}

\title{\texttt{TRIGSIMP} \\
A \REDUCE{} Package for the Simplification and Factorization of
Trigonometric and Hyperbolic Expressions}

\author{Wolfram Koepf \\
	Andreas Bernig \\
	Herbert Melenk \\
	ZIB Berlin \\
	~ \\
	Revised by Francis Wright \\
	QMW London \\
	E-mail: \texttt{F.J.Wright@Maths.QMW.ac.uk}}

\date{17 January 1999}

\begin{document}
\maketitle

\section{Introduction}

The \REDUCE{} package TRIGSIMP is a useful tool for all kinds of
problems related to trigonometric and hyperbolic simplification and
factorization.  There are three operators included in TRIGSIMP:
trigsimp, trigfactorize and triggcd.  The first is for simplifying
trigonometric or hyperbolic expressions and has many options, the
second is for factorizing them and the third is for finding the
greatest common divisor of two trigonometric or hyperbolic
polynomials.  This package is automatically loaded when one of these
operators is used.


\section{Simplifying trigonometric expressions}

As there is no normal form for trigonometric and hyperbolic
expressions, the same function can convert in many different
directions, e.g.\ $\sin(2x) \leftrightarrow 2\sin(x)\cos(x)$.  The
user has the possibility to give several parameters to the operator
\texttt{trigsimp} in order to influence the transformations.  It is
possible to decide whether or not a rational expression involving
trigonometric and hyperbolic functions vanishes.

To simplify an expression \texttt{f}, one uses
\texttt{trigsimp(f[,options])}.  For example:
\begin{verbatim}
trigsimp(sin(x)^2+cos(x)^2);

1
\end{verbatim}
The possible options (where $^*$ denotes the default) are:
\begin{enumerate}
\item \texttt{sin}$^*$ or \texttt{cos};
\item \texttt{sinh}$^*$ or \texttt{cosh};
\item \texttt{expand}$^*$, \texttt{combine} or \texttt{compact};
\item \texttt{hyp}, \texttt{trig} or \texttt{expon};
\item \texttt{keepalltrig};
\item \texttt{tan} and/or \texttt{tanh};
\item target arguments of the form \textit{variable} /
\textit{positive integer}.
\end{enumerate}
From each of the first four groups one can use at most one option,
otherwise an error message will occur.  Options can be given in any
order.

The first group fixes the preference used while transforming a
trigonometric expression:
\begin{verbatim}
trigsimp(sin(x)^2);

      2
sin(x)

trigsimp(sin(x)^2, cos);

         2
 - cos(x)  + 1
\end{verbatim}
The second group is the equivalent for the hyperbolic functions.

The third group determines the type of transformation.  With the
default, \texttt{expand}, an expression is transformed to use only
simple variables as arguments:
\begin{verbatim}
trigsimp(sin(2x+y));

                                 2
2*cos(x)*cos(y)*sin(x) - 2*sin(x) *sin(y) + sin(y)
\end{verbatim}
With \texttt{combine}, products of trigonometric functions are
transformed to trig\-onometric functions involving sums of variables:
\begin{verbatim}
trigsimp(sin(x)*cos(y), combine);

 sin(x - y) + sin(x + y)
-------------------------
            2
\end{verbatim}
With \texttt{compact}, the \REDUCE{} operator \texttt{compact}
\cite{hearns} is applied to \texttt{f}.  This often leads to a simple
form, but in contrast to \texttt{expand} one does not get a normal
form. For example:
\begin{verbatim}
trigsimp((1-sin(x)^2)^20*(1-cos(x)^2)^20, compact);

      40       40
cos(x)  *sin(x)
\end{verbatim}

With an option from the fourth group, the input expression is
transformed to trigonometric, hyperbolic or exponential form
respectively:
\begin{verbatim}
trigsimp(sin(x), hyp);

 - sinh(i*x)*i

trigsimp(sinh(x), expon);

  2*x
 e    - 1
----------
      x
   2*e

trigsimp(e^x, trig);

cos(i*x) - sin(i*x)*i
\end{verbatim}

Usually, \texttt{tan}, \texttt{cot}, \texttt{sec}, \texttt{csc} are
expressed in terms of \texttt{sin} and \texttt{cos}.  It can sometimes
be useful to avoid this, which is handled by the option
\texttt{keepalltrig}:
\begin{verbatim}
trigsimp(tan(x+y), keepalltrig);

  - (tan(x) + tan(y))
----------------------
  tan(x)*tan(y) - 1
\end{verbatim}
Alternatively, the options \texttt{tan} and/or \texttt{tanh} can be
given to convert the output to the specified form as far as possible:
\begin{verbatim}
trigsimp(tan(x+y), tan);

  - (tan(x) + tan(y))
----------------------
  tan(x)*tan(y) - 1
\end{verbatim}
By default, the other functions used will be \texttt{cos} and/or
\texttt{cosh}, unless the other desired functions are also specified
in which case this choice will be respected.

The final possibility is to specify additional target arguments for
the trigonometric or hyperbolic functions, each of which should have
the form of a variable divided by a positive integer.  These
additional arguments are treated as if they had occurred within the
expression to be simplified, and their denominators are used in
determining the overall denominator to use for each variable in the
simplified form:
\begin{verbatim}
trigsimp(csc x - cot x + csc y - cot y, x/2, y/2, tan);

     x          y
tan(---) + tan(---)
     2          2
\end{verbatim}

It is possible to use the options of different groups simultaneously:
\begin{verbatim}
trigsimp(sin(x)^4, cos, combine);

 cos(4*x) - 4*cos(2*x) + 3
---------------------------
             8
\end{verbatim}

Sometimes, it is necessary to handle an expression in separate steps:
\begin{verbatim}
trigsimp((sinh(x)+cosh(x))^n+(cosh(x)-sinh(x))^n, expon);

  1   n    n*x
(----)  + e
   x
  e

trigsimp(ws, hyp);

2*cosh(n*x)

trigsimp((cosh(a*n)*sinh(a)*sinh(p)+cosh(a)*sinh(a*n)*sinh(p)+
    sinh(a - p)*sinh(a*n))/sinh(a));

cosh(a*n)*sinh(p) + cosh(p)*sinh(a*n)

trigsimp(ws, combine);

sinh(a*n + p)
\end{verbatim}

The \texttt{trigsimp} operator can be applied to equations, lists and
matrices (and compositions thereof) as well as scalar expressions, and
automatically maps itself recursively over such non-scalar data
structures:
\begin{verbatim}
trigsimp( { sin(2x) = cos(2x) } );

                            2
{2*cos(x)*sin(x)= - 2*sin(x)  + 1}
\end{verbatim}


\section{Factorizing trigonometric expressions}

With \texttt{trigfactorize(p,x)} one can factorize the trigonometric
or hyperbolic polynomial \texttt{p} in terms of trigonometric
functions of the argument \texttt{x}.  The output has the same format
as that from the standard \REDUCE{} operator \texttt{factorize}.  For
example:
\begin{verbatim}
trigfactorize(sin(x), x/2);

             x            x
{{2,1},{sin(---),1},{cos(---),1}}
             2            2
\end{verbatim}
If the polynomial is not coordinated or balanced \cite{art}, the
output will equal the input.  In this case, changing the value for
\texttt{x} can help to find a factorization, e.g.
\begin{verbatim}
trigfactorize(1+cos(x), x);

{{cos(x) + 1,1}}

trigfactorize(1+cos(x), x/2);

             x
{{2,1},{cos(---),2}}
             2
\end{verbatim}
The polynomial can consist of both trigonometric and hyperbolic functions:
\begin{verbatim}
trigfactorize(sin(2x)*sinh(2x), x);

{{4,1}, {sinh(x),1}, {cosh(x),1}, {sin(x),1}, {cos(x),1}}
\end{verbatim}

The \texttt{trigfactorize} operator respects the standard \REDUCE{}
\texttt{factorize} switch \texttt{nopowers} -- see the \REDUCE{}
manual for details.  Turning it on gives the behaviour that was
standard before \REDUCE~3.7:
\begin{verbatim}
on nopowers;

trigfactorize(1+cos(x), x/2);

        x        x
{2,cos(---),cos(---)}
        2        2
\end{verbatim}


\section{GCDs of trigonometric expressions}

The operator \texttt{triggcd} is essentially an application of the
algorithm behind \texttt{trigfactorize}.  With its help the user can
find the greatest common divisor of two trigonometric or hyperbolic
polynomials.  It uses the method described in \cite{art}.  The syntax
is \texttt{triggcd(p,q,x)}, where \texttt{p} and \texttt{q} are the
trigonometric polynomials and \texttt{x} is the argument to use.  For
example:
\begin{verbatim}
triggcd(sin(x), 1+cos(x), x/2);

     x
cos(---)
     2

triggcd(sin(x), 1+cos(x), x);

1
\end{verbatim}
The polynomials $p$ and $q$ can consist of both trigonometric and
hyperbolic functions:
\begin{verbatim}
triggcd(sin(2x)*sinh(2x), (1-cos(2x))*(1+cosh(2x)), x);

cosh(x)*sin(x)
\end{verbatim}


\section{Further Examples}

With the help of this package the user can create identities:
\begin{verbatim}
trigsimp(tan(x)*tan(y));

 sin(x)*sin(y)
---------------
 cos(x)*cos(y)

trigsimp(ws, combine);
\end{verbatim}

{\samepage\begin{verbatim}
 cos(x - y) - cos(x + y)
-------------------------
 cos(x - y) + cos(x + y)
\end{verbatim}}

\begin{verbatim}
trigsimp((sin(x-a)+sin(x+a))/(cos(x-a)+cos(x+a)));

 sin(x)
--------
 cos(x)

trigsimp(cosh(n*acosh(x))-cos(n*acos(x)), trig);

0

trigsimp(sec(a-b), keepalltrig);

  csc(a)*csc(b)*sec(a)*sec(b)
-------------------------------
 csc(a)*csc(b) + sec(a)*sec(b)

trigsimp(tan(a+b), keepalltrig);

  - (tan(a) + tan(b))
----------------------
  tan(a)*tan(b) - 1

trigsimp(ws, keepalltrig, combine);

tan(a + b)
\end{verbatim}

Some difficult expressions can be simplified:
\begin{verbatim}
df(sqrt(1+cos(x)), x, 4);

            5           4            3       2            3
( - 4*cos(x)  - 4*cos(x)  - 20*cos(x) *sin(x)  + 12*cos(x)

             2       2            2                   4
  - 24*cos(x) *sin(x)  + 20*cos(x)  - 15*cos(x)*sin(x)

                    2                       4            2
  + 12*cos(x)*sin(x)  + 8*cos(x) - 15*sin(x)  + 16*sin(x) )/

(16*sqrt(cos(x) + 1)

         4           3           2
 *(cos(x)  + 4*cos(x)  + 6*cos(x)  + 4*cos(x) + 1))

on rationalize;
trigsimp(ws);

 sqrt(cos(x) + 1)
------------------
        16

off rationalize;
load_package taylor;

taylor(sin(x+a)*cos(x+b), x, 0, 4);

cos(b)*sin(a) + (cos(a)*cos(b) - sin(a)*sin(b))*x

                                    2
 - (cos(a)*sin(b) + cos(b)*sin(a))*x

    2*( - cos(a)*cos(b) + sin(a)*sin(b))   3
 + --------------------------------------*x
                     3

    cos(a)*sin(b) + cos(b)*sin(a)   4      5
 + -------------------------------*x  + O(x )
                  3

trigsimp(ws, combine);

 sin(a - b) + sin(a + b)                               2
------------------------- + cos(a + b)*x - sin(a + b)*x
            2

    2*cos(a + b)   3    sin(a + b)   4      5
 - --------------*x  + ------------*x  + O(x )
         3                  3
\end{verbatim}

Certain integrals whose evaluation was not possible in \REDUCE{}
(without preprocessing) are now computable:
\begin{verbatim}
int(trigsimp(sin(x+y)*cos(x-y)*tan(x)), x);

       2
(cos(x) *x - cos(x)*sin(x) - 2*cos(y)*log(cos(x))*sin(y)

          2
  + sin(x) *x)/2

int(trigsimp(sin(x+y)*cos(x-y)/tan(x)), x);

                                   x  2
(cos(x)*sin(x) - 2*cos(y)*log(tan(---)  + 1)*sin(y)
                                   2

                      x
  + 2*cos(y)*log(tan(---))*sin(y) + x)/2
                      2
\end{verbatim}
Without the package, the integration fails, and in the second case one
does not receive an answer for many hours.

\begin{verbatim}
trigfactorize(sin(2x)*cos(y)^2, y/2);

{{2*cos(x)*sin(x),1},

       y          y
 {cos(---) - sin(---),2},
       2          2

       y          y
 {cos(---) + sin(---),2}}
       2          2
\end{verbatim}
\begin{verbatim}
trigfactorize(sin(y)^4-x^2, y);

        2               2
{{sin(y)  + x,1},{sin(y)  - x,1}}

trigfactorize(sin(x)*sinh(x), x/2);

{{4,1},

        x
 {sinh(---),1},
        2

        x
 {cosh(---),1},
        2

       x
 {sin(---),1},
       2

       x
 {cos(---),1}}
       2

triggcd(-5+cos(2x)-6sin(x), -7+cos(2x)-8sin(x), x/2);

       x        x
2*cos(---)*sin(---) + 1
       2        2

triggcd(1-2cosh(x)+cosh(2x), 1+2cosh(x)+cosh(2x), x/2);

        x  2
2*sinh(---)  + 1
        2
\end{verbatim}

\begin{thebibliography}{99}

\bibitem{art}
Roach, Kelly: Difficulties with Trigonometrics. Notes of a talk.

\bibitem{hearns}
Hearn, A.C.: COMPACT User Manual.

\end{thebibliography}
\end{document}
