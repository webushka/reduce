\documentclass[11pt]{article}
% The fontenc package would be useful to suppoty "{" and "}" characters
% in a \texttt context, however it is not always automatically available.
% So when you see warnings about those characters here is a place you could
% change to get rid of it.
%\usepackage[T1]{fontenc}
\usepackage{amsmath,amsfonts,color}
\usepackage[pdftex,colorlinks=true,linkcolor=blue,extension=pdf]{hyperref}

\newcommand{\Reduce}{\textsc{Reduce}}
\newcommand{\Cuba}{\textsc{Cuba}}

\title{{\Reduce} interface to the {\Cuba} integration library}
\author{Kostas N. Oikonomou\\ AT\&T Labs Research, Middletown, NJ, U.S.A.
  \\ \small\texttt{ko@research.att.com}}

\begin{document}

\maketitle

\section{Introduction}

The \texttt{cuba} package is an interface between {\Reduce} (CSL) and the
{\Cuba} library for multi-dimensional numerical integration.  The libary can be
found at \url{http://www.feynarts.de/cuba}.  It offers a choice of four
independent methods: Vegas, Suave, Divonne, and Cuhre. The first three are Monte
Carlo-based, the fourth is a deterministic algorithm.  It is recommended to read
the {\Cuba} manual, and, optionally, to look at the other documentation provided
on the site.

The integrals are evaluated \emph{only} over hyper-rectangles\footnote{{\Cuba}
  itself evaluates all integrals over the unit hypercube, but the {\Reduce}
  interface provides a small extension, allowing the user to integrate over an
  arbitrary hyper-rectangle.}.  As an example of what can be done using the
\texttt{cuba} package in {\Reduce}, say $f$ is a function $\mathbb{R}^3\to
\mathbb{R}$ and we want to compute
\begin{equation*}
  \int_{a_1}^{b_1} \int_{a_2}^{b_2} \int_{a_3}^{b_3} f(x_1,x_2,x_3)\, dx_1 \,
  dx_2 \, dx_3
\end{equation*}
using the \texttt{Vegas} algorithm, one of the choices provided by {\Cuba}.
This is done by saying
\begin{table*}[h]
  \centering
  \ttfamily
  \begin{tabular}{l}
    load\_package cuba; \\
    on rounded; \\
    cuba\_int(f,{{a1,b1},{a2,b2},{a3,b3}},Vegas);
  \end{tabular}
\end{table*}\\
if the ``Vegas'' algorithm is to be used.  Although quite a bit of effort has
gone into making the package work even when not in rounded mode, it is probably
best to have \texttt{on rounded}.

The {\Reduce} function \texttt{f} defining the integrand is assumed to take a
3-element \emph{list} $x$ as input and return the value $f(x)$ of the
integrand at the point $x\in\mathbb{R}^3$.  If so, \texttt{cuba\_int(...)} will
return a list of the form
\begin{center}
  \texttt{\{value, error, probability, number of regions,
            number of evaluations, status\}}
\end{center}
where \texttt{value} is the value of the integral, \texttt{error} is an
indication of the probable error, and \texttt{status} indicates whether the
algorithm terminated successfully or not.  Consult the {\Cuba} manual for the
other quantities.


\section{Installation}

At present the Reduce parts of this package can only build using the CSL
version of Reduce, but in that context get compiled automatically as part
of the full standard system. However the code for {\Cuba} and the C-coded
interface between that and Reduce has to be built by hand, and the current
arrangements make that work when all the Reduce sources have been installed
and Reduce is built from scratch.

In that case you should identify the directory \verb+packages/foreign/cuba+
in the Reduce source tree and select it as current. Ensure that the command
\verb+wget+ is available on your platform and then You can then go
\verb+make+ to fetch {\Cuba} from its home site, compile it and then create
the dynamic library that forms a link between Cuba and Reduce.

This should work on any sufficiently modern Unix-like system, including
either the 32 or 64-bit version of Cygwin. The term "modern" here refers
to Linux systems using releases from no older then the very end of 2011:
any such will probably provide a version of the gcc C compiler (ie one
from 4.6.x onwards) sufficient for Cuba. This corresponds to Ubuntu from
release 11.10 onwards or Fedora from about version 15.

To use the Cuba package on Windows you must run a Cygwin version of CSL
Reduce not a native windows one. That means that if you want the benefit of
a GUI you must have an X server running and the environment variable
DISPLAY set up for it. Passing the command-line flag ``\verb+--cygwin+'' to
the CSL version of Reduce should cause a suitable version of the system to
be loaded, and this probably needs to be done from the command line of
a cygwin terminal. This limitation is because the main Cuba library does not
support native Windows.

Anybody with either and older version of an operating system or one other
then (Free)BSD, OSX, Linux or Cygwin may need to identify a C compiler
that can handle Cuba (any that support enough of the features of the 2011
C standard should suffice) and edit "Makefile" to set the C compiler and
any flags or options that it needs. Slightly more extreme alterations will
be needed if the linking command that makes the dynamic interface library
needs changing.


\section{The interface}

Currently, the interface provides the functions listed in Table \ref{tab:intf}.
The table gives minimal explanations, consult the {\Cuba} manual for details.
\begin{table}[h]
  \centering
  \begin{tabular}{|l|p{0.45\textwidth}|}\hline
    \texttt{cuba\_gen\_par(name,value)} & Set the generally-applicable parameter
      \emph{name} (a string) to \emph{value} \\
    \texttt{cuba\_vegas\_par(name,value)}  &   Set a Vegas-specific parameter \\
    \texttt{cuba\_suave\_par(name,value)}  &   Set a Suave-specific parameter \\
    \texttt{cuba\_divonne\_par(name,value)} &  Set a Divonne-specific parameter \\
    \texttt{cuba\_cuhre\_par(name,value)}   &  Set a Cuhre-specific parameter \\
    \hline
    \texttt{cuba\_verbosity(v)}  & For $v=0,1,2$ \texttt{cuba\_int} will provide
    more informative output \\
    \hline
    \texttt{cuba\_set\_flags\_bit(i)}      & Set the $i$th bit of the global
    \texttt{flags} \\
    \texttt{cuba\_clear\_flags\_bit(i)}    & Clear the $i$th bit of the global
    \texttt{flags} \\
    \texttt{cuba\_statefile(fname)} & file \texttt{fname} will be used for
    checkpointing a long-running integration \\
    \hline
    \texttt{cuba\_int(f,$\{\{a_1,b_1\},\dots\}\}$,alg)}    & Integrate
    the {\Reduce} function $f$ over the hyper-rectangle
    $\{a_1,b_1\}\times\cdots\times\{a_m,b_m\}$ using algorithm \texttt{alg} \\
      \hline
  \end{tabular}
  \caption{\label{tab:intf}Functionality of the {\Reduce} interface to the
    {\Cuba} library.}
\end{table}

There are some features of {\Cuba} that are not handled by this version of the
interface: vector integrands, i.e. functions from $\mathbb{R}^n\to\mathbb{R}^m$
with $m>1$, integration routines that can do more than $2^{32}$ evaluations, and
some of the parallelization features.
\clearpage

\section {The \texttt{cuba} package}

\subsection{Structure}

This is not of interest to most users, but the package consists of the following
files\footnote{If the list of files and comments is confusing, refer to the
  Acknowledgments.}:
\begin{table*}[h]
  \centering
  \begin{tabular}{|l|p{0.6\textwidth}|} \hline
    \texttt{redcuba.c} & Builds \texttt{libredcuba.so}, a ``glue'' library
    between the actual {\Cuba} library \texttt{libcuba.a} and {\Reduce}/CSL \\
    \texttt{C\_call\_CSL.h} &  The ``procedural'' interface from C to CSL, used
    in the above \\ \hline
    \texttt{cuba.red}       & The module defining the {\Cuba} package \\
    \texttt{cuba\_main.red} &  The {\Reduce} module (symbolic procedures) implementing
    the interface \\ 
    \texttt{alg\_intf.red}  & Utilities for interfacing between algebraic and
     symbolic modes \\
    \texttt{cuba.tst} & A {\Reduce} test file. \\
    \hline
  \end{tabular}
\end{table*}

\subsection{Debugging}

To debug the interface, there is a variable \texttt{DEBUG} in
\texttt{redcuba.c}, normally set to 0.  By setting it to 1 or 2 and re-making
\texttt{libredcuba.so} the package will produce various debugging messages that
should be useful.



\subsubsection*{Acknowledgments}
Thanks to Arthur Norman for his invaluable support in navigating the intricacies
of {\Reduce}, algebraic and symbolic mode, RLISP, Standard Lisp, CSL, etc.


\end{document}
