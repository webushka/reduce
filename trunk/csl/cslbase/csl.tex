% Generated using doxtract Thu Nov 17 21:35:46 2011
\documentclass[a4paper,11pt]{article}
\title{CSL reference}
\author{A C Norman}
\begin{document}
\maketitle
\section{Introduction}
This is reference material for CSL. The Lisp identifiers mentioned here
are the ones that are initially present in a raw CSL image. Some
proportion of them are not really intended to be used by end-users but
are merely the internal components of some feature.

\section{Command-line options}
The items shown here are the ones that are recognized on the CSL command
line. In general an option that requires an argument can be written as either
{\ttfamily -x yyy} or as {\ttfamily -xyyy}. Arguments should be case
insensitive.
\begin{description} 

\item [{\ttfamily --}]
If the application is run in console mode then its standard output could
be redirected to a file using shell facilities. But the {\ttfamily --}
directive (followed by a file name) redirects output within the Lisp rather
than outside it. If this is done a very limited capability for sending
progress or status reports to stderr (or the title-bar when running in windowed
mode) remains via the {\ttfamily report!-right} function.
  
The {\ttfamily -w} option may frequently make sense in such cases, but if that
is not used and the system tries to run in a window it will create it
starting off minimised.

\item [{\ttfamily --help}]
It is probably obvious what this option does! Note that on Windows the
application was linked as a windows binary so it carefully creates a
console to display the help text in, and organizes a delay to give
people a chance to read it.

\item [{\ttfamily --my-path}]
At some time I had felt the need for this option, but I now forget what I
expected to use it for! It leads the executable to display the fully
rooted name of the directory it was in and then terminate. It may be useful
in some script?

\item [{\ttfamily --texmacs}]
If CSL/Reduce is launched from texmacs this command-line flag should be
used to arrange that the {\ttfamily texmacs} flag is set in
{\ttfamily lispsystem!*}, and the code may then do special things.

\item [{\ttfamily -a}]
{\ttfamily -a} is a curious option, not intended for general or casual use.
If given it causes the {\ttfamily (batchp)} function to return the opposite
result from normal!  Without ``{attfamily -a}'' {\ttfamily (batchp)} returns
{\ttfamily T} either if at least one file was specified on the command line,
or if the standard input is ``not a tty'' (under some operating systems this
makes sense -- for instance the standard input might not be a ``tty'' if it
is provided via file redirection).  Otherwise (ie primary input is directly
from a keyboard) {\ttfamily (batchp)} returns {\ttfamily nil}.  Sometimes
this judgement about how ``batch'' the current run is will be wrong or
unhelpful, so {\ttfamily -a} allows the user to coax the system into better
behaviour.  I hope that this is never used!

\item [{\ttfamily -b}]
{\ttfamily -b} tells the system to avoid any attempt to recolour prompts
and input text. It will mainly be needed on X terminals that have been
set up so that they use colours that make the defaults here unhelpful.
Specifically white-on-black and so on.
{\ttfamily -b} can be followed by colour specifications to make things yet
more specific. It is supposed to be the idea that three colours can be
specified after it for output, input and prompts, with the letters KRGYbMCW
standing for blacK, Red, Green, Yellow, blue, Magenta, Cyan and White.
This may not fully work yet!

\item [{\ttfamily -c}]
Displays a notice relating to the authorship of CSL. Note that this
is an authorship statement not a Copyright notice, because if any
(L)GPL code is involved that would place requirements on what was
displayed in a Copyright Notice.

\item [{\ttfamily -d}]
A command line entry {\ttfamily -Dname=value} or {\ttfamily -D name=value}
sets the value of the named lisp variable to the value (as a string).
Note that the value set is a {\em string} so if you wish to retrieve
it and use it as a symbold or number within your code you will have to
perform some conversion.

\item [{\ttfamily -e}]
A ``spare'' option used from time to time to activate experiments within
CSL.

\item [{\ttfamily -f}]
At one stage CSL could run as a socket server, and {\ttfamily -f portnumber}
activated that mode. {\ttfamily -f-} used a default port, 1206 (a number
inspired by an account number on Titan that I used in the 1960s). The code
that supports this may be a useful foundation to others who want to make a
network service out of this code-base, but is currently disabled.

\item [{\ttfamily -g}]
In line with the implication of this option for C compilers, this enables
a debugging mode. It sets a lisp variable {\ttfamily !*backtrace} and
arranges that all backtraces are displayed notwithstanding use of
{\ttfamily errorset}.

\item [{\ttfamily -h}]
This option is a left-over. When the X-windows version of the code first
started to use Xft it viewed that as optional and could allow a build even when
it was not available. And then even if Xft was detected and liable to be used
by default it provided this option to disable its use. The remnants of the
switch that disabled use of Xft (relating to fonts living on the Host or
the Server) used this switch, but it now has no effect.

\item [{\ttfamily -i}]
CSL and Reduce use image files to keep both initial heap images and
``fasl'' loadable modules. By default if the executable launched has some name,
say xxx, then an image file xxx.img is used. But to support greater
generality {\ttfamily -i} introduces a new image, {\ttfamily -i-} indicates
the default one and a sequence of such directives list image files that are
searched in the order given. These are read-only. The similar option
{\ttfamily -o} equally introduces image files that are scanned for input, but
that can also be used for output. Normally there would only be one
{\ttfamily -o} directive.

\item [{\ttfamily -j}]
Follow this directive with a file-name, and a record of all the files read
during the Lisp run will be dumped there with a view that it can be included
in a Makefile to document dependencies.

\item [{\ttfamily -k}]
{\ttfamily -K nnn} sets the size of heap to be used.  If it is given then that much
memory will be allocated and the heap will never expand.  Without this
option a default amount is used, and (on many machines) it will grow
if space seems tight.
  
The extended version of this option is {\ttfamily -K nnn/ss} and then ss is the
number of ``CSL pages'' to be allocated to the Lisp stack. The default
value (which is 1) should suffice for almost all users, and it should
be noted that the C stack is separate from and independent of this one and
it too could overflow.
  
A suffix K, M or G on the number indicates units of kilobytes, megabytes or
gigabytes, with megabytes being the default. So {\ttfamily -K200M} might
represent typical usage for common-sized computations. In general CSL
will automatically expand its heap, and so it should normally never be
necessary to use this option.

\item [{\ttfamily -l}]
This is to send a copy of the standard output to a named log file. It is
very much as if the Lisp function {\ttfamily (spool ``logfile'')} had been
invoked at the start of the run.

\item [{\ttfamily -m}]
Memory trace mode. An option that represents an experiment from the past,
and no longer reliably in use. It make it possible to force an
exception at stages whene reference to a specified part of memory was made
and that could be useful for some low level debugging. It is not supported
at present.

\item [{\ttfamily -n}]
Normally when the system is started it will run a ``restart function'' as
indicated in its heap image. There can be cases where a heap image has been
created in a bad way such that the saved restart function always fails
abruptly, and hence working out what was wrong becomes hard. In such cases
it may be useful to give the {\ttfamily -n} option that forces CSL to
ignore any startup function and merely always begin in a minimal Lisp-style
read-eval-print loop. This is intended for experts to do disaster recovery
and diagnosis of damaged image files.

\item [{\ttfamily -o}]
See {\ttfamily -i}. This specifies an image file used for output via
{\ttfamily faslout} and {\ttfamily reserve}.

\item [{\ttfamily -p}]
If a suitable profile option gets implemented one day this will activate it,
but for now it has no effect.

\item [{\ttfamily -q}]
This option sets {\ttfamily !*echo} to {\ttfamily nil} and switches off
garbage collector messages to give a slightly quieter run.

\item [{\ttfamily -r}]
The random-number generator in CSL is normally initialised to a value
based on the time of day and is hence not reproducible from run to run.
In many cases that behavious is desirable, but for debugging it can be useful
to force a seed. The directive {\ttfamily -r nnn,mmm} sets the seed to
up to 64 bits taken from the values nnn and mmm. The second value if optional,
and specifying {\ttfamily -r0}  explicitly asks for the non-reproducible
behaviour (I hope). Note that the main Reduce-level random number source is
coded at a higher level and does not get reset this way -- this is the
lower level CSL generator.

\item [{\ttfamily -s}]
Sets the Lisp variable {\ttfamily !*plap} and hence the compiler generates
an assembly listing.

\item [{\ttfamily -t}]
{\ttfamily -t name} reports the time-stamp on the named module, and then
exits. This is for use in perl scripts and the like, and is
needed because the stamps on modules within an image or
library file are not otherwise instantly available.
  
Note that especially on windowed systems it may be necessary to use this
with {\ttfamily -- filename} since the information generated here goes to
the default output, which in some cases is just the screen.

\item [{\ttfamily -u}]
See {\ttfamily -d}, but this forcibly undefines a symbol. There are probably
very very few cases where it is useful since I do not have a large
number of system-specific predefined names.

\item [{\ttfamily -v}]
An option to make things mildly more verbose. It displays more of a banner
at startup and switches garbage collection messages on.

\item [{\ttfamily -w}]
On a typical system if the system is launched it creates a new window and uses
its own windowed intarface in that. If it is run such that at startup the
standard input or output are associated with a file or pipe, or under X the
variable {\ttfamily DISPLAY} is not set it will try to start up in console
mode. The flag {\ttfamily -w} indicates that the system should run in console
more regadless, while {\ttfamily -w+} attempts a window even if that seems
doomed to failure. When running the system to obey a script it will often make
sense to use the {\ttfamily -w} option. Note that on Windows the system is
provided as two separate (but almost identical) binaries. For example the
file {\ttfamily csl.exe} is linked in windows mode. A result is that if
launched from the command line it detaches from its console, and if launched
by double-clicking it does not create a console. It is in fact very ugly when
double clicking on an application causes an unwanted console window to appear.
In contrast {\ttfamily csl.com} is a console mode version of just the same
program, so when launched from a command line it can communicate with the
console in the ordinary expected manner.

\item [{\ttfamily -x}]
{\ttfamily -x} is an option intended for use only by system
support experts -- it disables trapping if segment violations by
errorset and so makes it easier to track down low level disasters --
maybe!  This can be valuable when running under a debugger since if the
code traps signals in its usual way and tries to recover it can make it a lot
harder to find out just what was going wrong.

\item [{\ttfamily -y}]
{\ttfamily -y } sets the variable {\ttfamily !*hankaku}, which causes the
lisp reader convert a Zenkaku code to Hankaku one when read. I leave this
option decoded on the command line even if the Kanji support code is not
otherwise compiled into CSL just so I can reduce conditional compilation.
This was part of the Internationalisation effort for CSL bu this is no longer
supported.

\item [{\ttfamily -z}]
When bootstrapping it is necessary to start up the system for one initial time
without the benefit of any image file at all. The option {\ttfamily -z} makes
this happen, so when it is specified the system starts up with a minimal
environment and only those capabilities that are present in the CSL
kernel. It will normally make sense to start loading some basic Lisp
definitions rather rapidly. The files {\ttfamily compat.lsp},
{\ttfamily extras.lsp} and {\ttfamily compiler.lsp} have Lisp source for the
main things I use, and once they are loaded the Lisp compiler can be used
to compile itself.

\end{description}

\section{Predefined variables}
\begin{description}

\item [{\ttfamily !!fleps1}]
There is a function safe!-fp!-plus that performs floating point
arithmetic but guarantees never to raise an exception. This value was
at one stage related to when small values created there got truncated to zero,
but the current code does not use the Lisp variable at all and instead does
things based on the bitwise representation of the numbers.

\item [{\ttfamily !\$eof!\$}]
The value of this variable is a pseudo-character returned from various
read functions to signal end-of-file.

\item [{\ttfamily !\$eol!\$}]
The value of this variable is an end-of-line character.

\item [{\ttfamily !*plap}]
Not yet written

\item [{\ttfamily !*applyhook!*}]
If this is set it might be supposed to be the name of a function used
by the interpreter as a callbackm but at presnet it does not actually do
anything!

\item [{\ttfamily !*break!-loop!*}]
If the value of this is a symbol that is defined as a function of one
argument then it is called during the processing on an error. This has not
been used in anger and so its whole status may be dubious!

\item [{\ttfamily !*carcheckflag}]
In general CSL arranges that every {\ttfamily car} or {\ttfamily cdr} access
is checked for validity. Once upon a time setting this variable to nil
turned such checks off in the hope of gaining a little speed. But it no
longer does that. It may have a minor effect on array access primitives.

\item [{\ttfamily !*comp}]
When set each function is compiled (into bytecodes) as it gets defined.

\item [{\ttfamily !*debug!-io!*}]
An I/O channel intended to be used for diagnostic interactions.

\item [{\ttfamily !*echo}]
When this is non-nil characters that are read from an input file are
echoed to the standard output. This gives a more comlete transcript in
a log file, but can sometimes amount to over-verbose output.

\item [{\ttfamily !*error!-messages!*}]
Has the value nil and does not do anything!

\item [{\ttfamily !*error!-output!*}]
An I/O channel intended for diagnostic output.

\item [{\ttfamily !*evalhook!*}]
See {\ttfamily !*applyhook!*}. This also does not do anything at present.

\item [{\ttfamily !*gc!-hook!*}]
If this is set to have as its value that is a function of one argument then
that function is called with {\ttfamily nil} on every minor entry to the
garbage collection, and with argument {\ttfamily t} at the end of a ``genuine''
full garbage collection.

\item [{\ttfamily !*hankaku}]
This was concerned with internationalisation to support a Japanese
locale but has not been activated for some while. In the fullness of time I
hope to migrate CSL to use an UTF8 representation of Unicode characters
internally, but that upgrade is at present an ideal and a project not
a reality. Volunteers to help welcomed.

\item [{\ttfamily !*loop!-print!*}]
Probably not used at present.

\item [{\ttfamily !*lower}]
Not yet written

\item [{\ttfamily !*macroexpand!-hook!*}]
Not yet written

\item [{\ttfamily !*math!-output!*}]
Not yet written

\item [{\ttfamily !*native\_code}]
Not yet written

\item [{\ttfamily !*notailcall}]
Not yet written

\item [{\ttfamily !*package!*}]
Not yet written

\item [{\ttfamily !*pgwd}]
Not yet written

\item [{\ttfamily !*pretty!-symmetric}]
Not yet written

\item [{\ttfamily !*prinl!-fn!*}]
Not yet written

\item [{\ttfamily !*prinl!-index!*}]
Not yet written

\item [{\ttfamily !*prinl!-visited!-nodes!*}]
Not yet written

\item [{\ttfamily !*print!-array!*}]
Not yet written

\item [{\ttfamily !*print!-length!*}]
Not yet written

\item [{\ttfamily !*print!-level!*}]
Not yet written

\item [{\ttfamily !*pwrds}]
Not yet written

\item [{\ttfamily !*query!-io!*}]
Not yet written

\item [{\ttfamily !*quotes}]
Not yet written

\item [{\ttfamily !*raise}]
Not yet written

\item [{\ttfamily !*redefmsg}]
Not yet written

\item [{\ttfamily !*resources!*}]
Not yet written

\item [{\ttfamily !*savedef}]
Not yet written

\item [{\ttfamily !*spool!-output!*}]
Not yet written

\item [{\ttfamily !*standard!-input!*}]
Not yet written

\item [{\ttfamily !*standard!-output!*}]
Not yet written

\item [{\ttfamily !*terminal!-io!*}]
Not yet written

\item [{\ttfamily !*trace!-output!*}]
Not yet written

\item [{\ttfamily !@cslbase}]
Not yet written

]pendingrpars]  \item [{\ttfamily pendingrpars}]
Not yet written

\item [{\ttfamily blank}]
The value of this variable is an space or blank character. This
might otherwise be written as ''{\ttfamily ! }''.

\item [{\ttfamily bn}]
Not yet written

\item [{\ttfamily bufferi}]
Not yet written

\item [{\ttfamily bufferp}]
Not yet written

\item [{\ttfamily common!-lisp!-mode}]
Not yet written

\item [{\ttfamily crbuf!*}]
Not yet written

\item [{\ttfamily emsg!*}]
Not yet written

\item [{\ttfamily eof!*}]
Not yet written

\item [{\ttfamily esc!*}]
The value of this variable is the character ``escape''. As a non-printing
character use of this is to be viewed as delicate.

\item [{\ttfamily indblanks}]
Not yet written

\item [{\ttfamily indentlevel}]
Not yet written

\item [{\ttfamily initialblanks}]
Not yet written

\item [{\ttfamily lispsystem!*}]
Not yet written

\item [{\ttfamily lmar}]
Not yet written

\item [{\ttfamily load!-source}]
Not yet written

\item [{\ttfamily nil}]
Not yet written

\item [{\ttfamily ofl!*}]
Not yet written

\item [{\ttfamily program!*}]
Not yet written

\item [{\ttfamily rmar}]
Not yet written

\item [{\ttfamily rparcount}]
Not yet written

\item [{\ttfamily s!:gensym!-serial}]
Not yet written

\item [{\ttfamily stack}]
Not yet written

\item [{\ttfamily t}]
Not yet written

\item [{\ttfamily tab}]
The value of this variable is a tab character.

\item [{\ttfamily thin!*}]
Not yet written

\item [{\ttfamily ttype!*}]
Not yet written

/*!! flags [04] Flags and Properties
  
Most of tags here are probably not much use to end-users, but I am noting them
as a matter of completeness.
  

\end{description}

Items that can appear in {\ttfamily lispsystem!*}
  
There is a global variable called {\ttfamily lispsystem!*} whose value is
reset in the process of CSL starting up. An effect of this is that if the
user changes its value those changes do not survice a preserving and
re-loading a heap image: this is deliberate since the heap image may be
re-loaded on a different instance of CSL possibly on a quite different
computer of with a different configuration. The value of {\ttfamily
lispsystem!*} is a list of items, where each item is either an atomic tag
of a pair whose first component is a key. In general it would be unwise
to rely on exactly what information is present without review of the code
that sets it up. The information may be of interest to anybody but some tags
and keys are reflections of experiments rather than fully stable facilities.
\begin{description}

\item[{\ttfamily (c!-code . count)}]
This will be present if code has been optimised into C through the source
files u01.c to u60.c, and in that case the value tells you how many functions
have been optimised in this manner.
  

\item[{\ttfamily  common!-lisp}]
For a project some while ago a limited Common Lisp compatibility mode was
being developed, and this tag indicated that it was active. In that case all
entries are in upper case and the variable is called {\ttfamily *FEATURES*}
rather than {\ttfamily lispsystem!*}. But note that this Lisp has never even
aspired to be a full Common Lisp, since its author considers Common Lisp to
have been a sad mistake that must bear significant responsibility for the
fact that interest in Lisp has faded dramatically since its introduction.
  

\item[{\ttfamily (compiler!-command . command)}]
The value associated with this key is a string that was used to compile the
files of C code making up CSL. It should contain directives to set up
search paths and predefined symbols. It is intended to be used in an
experiment that generates C code synamically, uses a command based on this
string to compile it and then dynamically links the resulting code in with
the running system.

\item[{\ttfamily csl}]
A simple tag intended to indicate that this Lisp system is CSL and not any
other. This can of course only work properly if all other Lisp systems
agree not to set this tag! In the context of Reduce I note that the PSL
Lisp system sets a tag {\ttfamily psl} on {\ttfamily lispsystem!*} and
the realistic use of this is to discriminate between CSL and PSL hosted
copies of Reduce.

\item[{\ttfamily debug}]
If CSL was compiled with debugging options this is present, and one can imagine
various bits of code being more cautious or more verbose if it is detected.

\item[{\ttfamily  (executable . name)}]
The value is the fully rooted name of the executable file that was launched.

\item[{\ttfamily fox}]
Used to be present if the FOX GUI toolkit was detected and incorporated as
part of CSL, but now probably never used!

\item[{\ttfamily (linker . type)}]
Intended for use in association with {\ttfamily compiler!-command}, the value
is {\ttfamily win32} on Windows, {\ttfamily x86\_64} on 64-bit Linux and
other things on other systems, as detected using the program {\ttfamily
objtype.c}.

\item[{\ttfamily  (name . name)}]
Some indication of the platform. For instance on one system I use it
is {\ttfamily linux-gnu:x86\_64} and on anther it is just {\ttfamily win32}.

\item[{\ttfamily  (native . tag)}]
One of the many experiments within CSL that were active at one stage but are
not current involved compilation directly into machine code. The strong
desire to ensure that image files coudl be used on a cross-platform basis
led to saved compiled code being tagged with a numeric ``native code tag'',
and this key/value pair identified the value to be used on the current
machine.

\item[{\ttfamily  (opsys . operating-system)}]
Some crude indication of the host operating system.

\item [operating system identity]
The name of the current operating system is put on the list. Exactly what
form is not explicitly defined!

\item[{\ttfamily pipes}]
In the earlier days of CSL there were computers where pipes were not
supported, so this tag notes when they are present and hance the facility
to create sub-tasks through them can be used.

\item[{\ttfamily  record\_get}]
An an extension to the CSL profiling scheme it it possible to compile
a special version that tracks and counts each use of property-list access
functions. This can be useful because there are ways to give special
treatment to a small number of flags and a small number of properties. The
special-case flage end up stored as a bitmap in the symbol-header so avoid
need for property-list searching. But of course recording this extra
information slows things down. This tag notes when the slow version is
in use. It might be used to trigger a display of statistics at the end of
a calculation.

\item[{\ttfamily reduce}]
This is intended to report if the initial heap image is for Reduce rather than
merely for Lisp.

\item[{\ttfamily  (shortname . name)}]
Gives the short name of the current executable, without its full path.

\item[{\ttfamily showmath}]
If the ``showmath'' capability has been compiled into CSL this will be present
so that Lisp code can know it is reasonable to try to use it.

\item[{\ttfamily  sixty!-four}]
Present if the Lisp was compiled for a 64-bit computer.

\item[{\ttfamily termed}]
Present if a cursor-addressable console was detected.

\item[{\ttfamily texmacs}]
Present if the system was launched with the {\ttfamily --texmacs} flag.
The intent is that this should only be done when it has been launched with
texmacs as a front-end.

\item[{\ttfamily  (version . ver)}]
The CSL version number.

\item[{\ttfamily win32}, {\ttfamily win64}]
Any windows system puts {\ttfamily win32} in {\ttfamily lispsystem!*}.
If 64-bit windows is is use then {\ttfamily win64} is also included

\item[{\ttfamily windowed}]
Present if CSL is running in its own window rather than in console mode.

\end{description}

\section{Flags and Properties}
\begin{description}

\item [{\ttfamily lose}]
If a name is flagged as {ttfamily lose} then a subsequent attempt to
define or redefine it will be ignored.

\item [{\ttfamily s!:ppchar} and {\ttfamily s!:ppformat}]
These are used in the prettyprint code found in {\ttfamily extras.red}. A
name is given a property {\ttfamily s!:ppformat} if in prettyprinted display
its first few arguments should appear on the same line as it if at all
possible. The {\ttfamily s!:ppchar} property is used to make the display of
bracket characters a little more tide in the source code.

\item [{\ttfamily switch}]
In the Reduce parser some names are ``switches'', and then directives such
as {\ttfamily on xxx} and {\ttfamily off xx} have the effect of setting or
clearing the value of a variable {\ttfamily !*xxx}. This is managed by
setting the {\ttfamily switch} flag om {\ttfamily xxx}. CSL sets some
things as switches ready for when they may be used by the Reduce parser.

\item [{\ttfamily !$\sim$magic!-internal!-symbol!$\sim$}]
CSL does not have a clear representation for functions that is separated from
the representation of an identifier, and so when you ask to get the value
of a raw function you get an identifier (probably a gensym) and this
tag is used to link such values with the symbols they were originally
extracted from.

\end{description}

\section{Functions and Special Forms}
  
Each line here shows a name and then one of the words {\itshape expr},
{\itshape fexpr} or {\itshape macro}. In some cases there can also be special
treatment of functions by the compiler so that they get compiled in-line.
\begin{description}

\item[{\ttfamily abs} {\itshape expr}]
Not yet written

acons expr
Not yet written

acos expr
Not yet written

acosd expr
Not yet written

acosh expr
Not yet written

acot expr
Not yet written

acotd expr
Not yet written

acoth expr
Not yet written

acsc expr
Not yet written

acscd expr
Not yet written

acsch expr
Not yet written

add1 expr
Not yet written

and fexpr
Not yet written

append expr
Not yet written

apply expr
Not yet written

apply0 expr
Not yet written

apply1 expr
Not yet written

apply2 expr
Not yet written

apply3 expr
Not yet written

asec expr
Not yet written

asecd expr
Not yet written

asech expr
Not yet written

ash expr
Not yet written

ash1 expr
Not yet written

asin expr
Not yet written

asind expr
Not yet written

asinh expr
Not yet written

assoc expr
Not yet written

assoc!*!* expr
Not yet written

atan expr
Not yet written

atan2 expr
Not yet written

atan2d expr
Not yet written

atand expr
Not yet written

atanh expr
Not yet written

atom expr
Not yet written

atsoc expr
Not yet written

batchp expr
Not yet written

binary\_close\_input expr
Not yet written

binary\_close\_output expr
Not yet written

binary\_open\_input expr
Not yet written

binary\_open\_output expr
Not yet written

binary\_prin1 expr
Not yet written

binary\_prin2 expr
Not yet written

binary\_prin3 expr
Not yet written

binary\_prinbyte expr
Not yet written

binary\_princ expr
Not yet written

binary\_prinfloat expr
Not yet written

binary\_read2 expr
Not yet written

binary\_read3 expr
Not yet written

binary\_read4 expr
Not yet written

binary\_readbyte expr
Not yet written

binary\_readfloat expr
Not yet written

binary\_select\_input expr
Not yet written

binary\_terpri expr
Not yet written

binopen expr
Not yet written

boundp expr
Not yet written

bps!-getv expr
Not yet written

bps!-putv expr
Not yet written

bps!-upbv expr
Not yet written

bpsp expr
Not yet written

break!-loop expr
Not yet written

byte!-getv expr
Not yet written

bytecounts expr
Not yet written

c\_out expr
Not yet written

carcheck expr
Not yet written

catch fexpr
Not yet written

cbrt expr
Not yet written

ceiling expr
Not yet written

char!-code expr
Not yet written

char!-downcase expr
Not yet written

char!-upcase expr
Not yet written

chdir expr
Not yet written

check!-c!-code expr
Not yet written

checkpoint expr
Not yet written

cl!-equal expr
Not yet written

close expr
Not yet written

close!-library expr
Not yet written

clrhash expr
Not yet written

code!-char expr
Not yet written

codep expr
Not yet written

compile expr
Not yet written

compile!-all expr
Not yet written

compress expr
Not yet written

cond fexpr
Not yet written

cons expr
Not yet written

consp expr
Not yet written

constantp expr
Not yet written

contained expr
Not yet written

convert!-to!-evector expr
Not yet written

copy expr
Not yet written

copy!-module expr
Not yet written

copy!-native expr
Not yet written

cos expr
Not yet written

cosd expr
Not yet written

cosh expr
Not yet written

cot expr
Not yet written

cotd expr
Not yet written

coth expr
Not yet written

create!-directory expr
Not yet written

csc expr
Not yet written

cscd expr
Not yet written

csch expr
Not yet written

date expr
Not yet written

dated!-name expr
Not yet written

datelessp expr
Not yet written

datestamp expr
Not yet written

de fexpr
Not yet written

define!-in!-module expr
Not yet written

deflist expr
Not yet written

deleq expr
Not yet written

delete expr
Not yet written

delete!-file expr
Not yet written

delete!-module expr
Not yet written

difference expr
Not yet written

digit expr
Not yet written

directoryp expr
Not yet written

divide expr
Not yet written

dm fexpr
Not yet written

do macro
Not yet written

do!* macro
Not yet written

dolist macro
Not yet written

dotimes macro
Not yet written

double!-execute expr
Not yet written

egetv expr
Not yet written

eject expr
Not yet written

enable!-backtrace expr
Not yet written

enable!-errorset expr
Not yet written

encapsulatedp expr
Not yet written

endp expr
Not yet written

eputv expr
Not yet written

eq expr
Not yet written

eq!-safe expr
Not yet written

eqcar expr
Not yet written

eql expr
Not yet written

eqlhash expr
Not yet written

eqn expr
Not yet written

equal expr
Not yet written

equalcar expr
Not yet written

equalp expr
Not yet written

error expr
Not yet written

error1 expr
Not yet written

errorset expr
Not yet written

eupbv expr
Not yet written

eval expr
Not yet written

eval!-when fexpr
Not yet written

evectorp expr
Not yet written

evenp expr
Not yet written

evlis expr
Not yet written

exp expr
Not yet written

expand expr
Not yet written

explode expr
Not yet written

explode2 expr
Not yet written

explode2lc expr
Not yet written

explode2lcn expr
Not yet written

explode2n expr
Not yet written

explode2uc expr
Not yet written

explode2ucn expr
Not yet written

explodebinary expr
Not yet written

explodec expr
Not yet written

explodecn expr
Not yet written

explodehex expr
Not yet written

exploden expr
Not yet written

explodeoctal expr
Not yet written

expt expr
Not yet written

faslout expr
Not yet written

fetch!-url expr
Not yet written

fgetv32 expr
Not yet written

fgetv64 expr
Not yet written

file!-length expr
Not yet written

file!-readablep expr
Not yet written

file!-writeablep expr
Not yet written

filedate expr
Not yet written

filep expr
Not yet written

fix expr
Not yet written

fixp expr
Not yet written

flag expr
Not yet written

flagp expr
Not yet written

flagp!*!* expr
Not yet written

flagpcar expr
Not yet written

float expr
Not yet written

floatp expr
Not yet written

floor expr
Not yet written

fluid expr
Not yet written

fluidp expr
Not yet written

flush expr
Not yet written

format macro
Not yet written

fp!-evaluate expr
Not yet written

fputv32 expr
Not yet written

fputv64 expr
Not yet written

frexp expr
Not yet written

funcall expr
Not yet written

funcall!* expr
Not yet written

function fexpr
Not yet written

gcdn expr
Not yet written

gctime expr
Not yet written

gensym expr
Not yet written

gensym1 expr
Not yet written

gensym2 expr
Not yet written

gensymp expr
Not yet written

geq expr
Not yet written

get expr
Not yet written

get!* expr
Not yet written

get!-current!-directory expr
Not yet written

get!-lisp!-directory expr
Not yet written

getd expr
Not yet written

getenv expr
Not yet written

gethash expr
Not yet written

getv expr
Not yet written

getv16 expr
Not yet written

getv32 expr
Not yet written

getv8 expr
Not yet written

global expr
Not yet written

globalp expr
Not yet written

go fexpr
Not yet written

greaterp expr
Not yet written

hash!-table!-p expr
Not yet written

hashcontents expr
Not yet written

hashtagged!-name expr
Not yet written

hypot expr
Not yet written

iadd1 expr
Not yet written

idapply expr
Not yet written

idifference expr
Not yet written

idp expr
Not yet written

iequal expr
Not yet written

if fexpr
Not yet written

igeq expr
Not yet written

igreaterp expr
Not yet written

ileq expr
Not yet written

ilessp expr
Not yet written

ilogand expr
Not yet written

ilogor expr
Not yet written

ilogxor expr
Not yet written

imax expr
Not yet written

imin expr
Not yet written

iminus expr
Not yet written

iminusp expr
Not yet written

indirect expr
Not yet written

inorm expr
Not yet written

input!-libraries fexpr
Not yet written

instate!-c!-code expr
Not yet written

integerp expr
Not yet written

internal!-open expr
Not yet written

intern expr
Not yet written

intersection expr
Not yet written

ionep expr
Not yet written

iplus expr
Not yet written

iplus2 expr
Not yet written

iquotient expr
Not yet written

iremainder expr
Not yet written

irightshift expr
Not yet written

is!-console expr
Not yet written

isub1 expr
Not yet written

itimes expr
Not yet written

itimes2 expr
Not yet written

izerop expr
Not yet written

last expr
Not yet written

lastcar expr
Not yet written

lastpair expr
Not yet written

lcmn expr
Not yet written

length expr
Not yet written

lengthc expr
Not yet written

leq expr
Not yet written

lessp expr
Not yet written

let!* fexpr
Not yet written

library!-members expr
Returns a list of all the modules that could potentially be loaded using
{\ttfamily load!-module}. See {\ttfamily list!-modules} to get a human
readable display that looks more like the result of listing a directory, or
{\ttfamily modulep} for checking the state of a particular named module.
  

library!-name expr
Not yet written

linelength expr
Not yet written

list fexpr
Not yet written

list!* fexpr
Not yet written

list!-directory expr
Not yet written
  

list!-modules expr
This prints a human-readable display of the modules present in the current
image files. This will include ``InitialImage'' which is the heap-image
loaded at system startup. For example
\begin{verbatim}
> (list!-modules)
  
File d:\csl\csl.img (dirsize 8  length 155016, Writable):
  compat       Sat Jul 26 10:20:08 2008  position 556   size: 9320
  compiler     Sat Jul 26 10:20:08 2008  position 9880  size: 81088
  InitialImage Sat Jul 26 10:20:09 2008  position 90972 size: 64040
  
nil
\end{verbatim}
  
See {\ttfamily library!-members} and {\ttfamily modulep} for functions that
make it possible for Lisp code to discover about the loadable modules that are
available.

list!-to!-string expr
Not yet written

list!-to!-symbol expr
Not yet written

list!-to!-vector expr
Not yet written

list2 expr
Not yet written

list2!* expr
Not yet written

list3 expr
Not yet written

list3!* expr
Not yet written

list4 expr
Not yet written

liter expr
Not yet written

ln expr
Not yet written

load!-module expr
Not yet written

load!-source expr
Not yet written

log expr
Not yet written

log10 expr
Not yet written

logand expr
Not yet written

logb expr
Not yet written

logeqv expr
Not yet written

lognot expr
Not yet written

logor expr
Not yet written

logxor expr
Not yet written

lose!-precision expr
Not yet written

lposn expr
Not yet written

lsd expr
Not yet written

macro!-function expr
Not yet written

macroexpand expr
Not yet written

macroexpand!-1 expr
Not yet written

make!-bps expr
Not yet written

make!-function!-stream expr
Not yet written

make!-global expr
Not yet written

make!-native expr
Not yet written

make!-random!-state expr
Not yet written

make!-simple!-string expr
Not yet written

make!-special expr
Not yet written

map expr
Not yet written

mapc expr
Not yet written

mapcan expr
Not yet written

mapcar expr
Not yet written

mapcon expr
Not yet written

maphash expr
Not yet written

maple\_atomic\_value expr
Not yet written

maple\_component expr
Not yet written

maple\_integer expr
Not yet written

maple\_length expr
Not yet written

maple\_string\_data expr
Not yet written

maple\_tag expr
Not yet written

maplist expr
Not yet written

mapstore expr
Not yet written

math!-display expr
Not yet written

max expr
Not yet written

max2 expr
Not yet written

md5 expr
Not yet written

md60 expr
Not yet written

member expr
Not yet written

member!*!* expr
Not yet written

memq expr
Not yet written

min expr
Not yet written

min2 expr
Not yet written

minus expr
Not yet written

minusp expr
Not yet written

mkevect expr
Not yet written

mkfvect32 expr
Not yet written

mkfvect64 expr
Not yet written

mkhash expr
Not yet written

mkquote expr
Not yet written

mkvect expr
Not yet written

mkvect16 expr
Not yet written

mkvect32 expr
Not yet written

mkvect8 expr
Not yet written

mkxvect expr
Not yet written

mod expr
Not yet written

modular!-difference expr
Not yet written

modular!-expt expr
Not yet written

modular!-minus expr
Not yet written

modular!-number expr
Not yet written

modular!-plus expr
Not yet written

modular!-quotient expr
Not yet written

modular!-reciprocal expr
Not yet written

modular!-times expr
Not yet written

modulep expr
This takes a single argument and checks whether there is a loadable module
of that name. If there is not then {\ttfamily nil} is returned, otherwise a
string that indicates the date-stamp on the module is given. See
{\ttfamily datelessp} for working with such dates, and {\ttfamily
library!-members} for finding a list of all modules that are available.
  

mpi\_allgather expr
Not yet written

mpi\_alltoall expr
Not yet written

mpi\_barrier expr
Not yet written

mpi\_bcast expr
Not yet written

mpi\_comm\_rank expr
Not yet written

mpi\_comm\_size expr
Not yet written

mpi\_gather expr
Not yet written

mpi\_iprobe expr
Not yet written

mpi\_irecv expr
Not yet written

mpi\_isend expr
Not yet written

mpi\_probe expr
Not yet written

mpi\_recv expr
Not yet written

mpi\_scatter expr
Not yet written

mpi\_send expr
Not yet written

mpi\_sendrecv expr
Not yet written

mpi\_test expr
Not yet written

mpi\_wait expr
Not yet written

msd expr
Not yet written

native!-address expr
Not yet written

native!-getv expr
Not yet written

native!-putv expr
Not yet written

native!-type expr
Not yet written

nconc expr
Not yet written

ncons expr
Not yet written

neq expr
Not yet written

noisy!-setq fexpr
Not yet written

not expr
Not yet written

nreverse expr
Not yet written

null expr
Not yet written

numberp expr
Not yet written

oblist expr
Not yet written

oddp expr
Not yet written

oem!-supervisor expr
Not yet written

onep expr
Not yet written

open expr
Not yet written

open!-library expr
Not yet written

open!-url expr
Not yet written

or fexpr
Not yet written

orderp expr
Not yet written

ordp expr
Not yet written

output!-library fexpr
Not yet written

pagelength expr
Not yet written

pair expr
Not yet written

pairp expr
Not yet written

parallel expr
Not yet written

peekch expr
Not yet written

pipe!-open expr
Not yet written

plist expr
Not yet written

plus fexpr
Not yet written

plus2 expr
Not yet written

plusp expr
Not yet written

posn expr
Not yet written

preserve expr
Not yet written

prettyprint expr
Not yet written

prin expr
Not yet written

prin1 expr
Not yet written

prin2 expr
Not yet written

prin2a expr
Not yet written

prinbinary expr
Not yet written

princ expr
Not yet written

princ!-downcase expr
Not yet written

princ!-upcase expr
Not yet written

princl expr
Not yet written

prinhex expr
Not yet written

prinl expr
Not yet written

prinoctal expr
Not yet written

prinraw expr
Not yet written

print expr
Not yet written

print!-config!-header expr
Not yet written

print!-csl!-headers expr
Not yet written

print!-imports expr
Not yet written

printc expr
Not yet written

printcl expr
Not yet written

printl expr
Not yet written

printprompt expr
Not yet written

prog fexpr
Not yet written

prog1 fexpr
Not yet written

prog2 fexpr
Not yet written

progn fexpr
Not yet written

protect!-symbols expr
Not yet written

protected!-symbol!-warn expr
Not yet written

psetq macro
Not yet written

put expr
Not yet written

putc expr
Not yet written

putd expr
Not yet written

puthash expr
Not yet written

putv expr
Not yet written

putv!-char expr
Not yet written

putv16 expr
Not yet written

putv32 expr
Not yet written

putv8 expr
Not yet written

qcaar expr
Not yet written

qcadr expr
Not yet written

qcar expr
Not yet written

qcdar expr
Not yet written

qcddr expr
Not yet written

qcdr expr
Not yet written

qgetv expr
Not yet written

qputv expr
Not yet written

quote fexpr
Not yet written

quotient expr
Not yet written

random!-fixnum expr
Not yet written

random!-number expr
Not yet written

rassoc expr
Not yet written

rational expr
Not yet written

rdf expr
Not yet written

rds expr
Not yet written

read expr
Not yet written

readb expr
Not yet written

readch expr
Not yet written

readline expr
Not yet written

reclaim expr
Not yet written

remainder expr
Not yet written

remd expr
Not yet written

remflag expr
Not yet written

remhash expr
Not yet written

remob expr
Not yet written

remprop expr
Not yet written

rename!-file expr
Not yet written

representation expr
Not yet written

resource!-exceeded expr
Not yet written

resource!-limit expr
Not yet written

restart!-csl expr
Not yet written

restore!-c!-code expr
Not yet written

return fexpr
Not yet written

reverse expr
Not yet written

reversip expr
Not yet written

round expr
Not yet written

rplacw expr
Not yet written

rseek expr
Not yet written

rtell expr
Not yet written

s!:blankcount macro
Not yet written

s!:blanklist macro
Not yet written

s!:blankp macro
Not yet written

s!:depth macro
Not yet written

s!:do!-bindings expr
Not yet written

s!:do!-endtest expr
Not yet written

s!:do!-result expr
Not yet written

s!:do!-updates expr
Not yet written

s!:endlist expr
Not yet written

s!:expand!-do expr
Not yet written

s!:expand!-dolist expr
Not yet written

s!:expand!-dotimes expr
Not yet written

s!:explodes expr
Not yet written

s!:finishpending expr
Not yet written

s!:format expr
Not yet written

s!:indenting macro
Not yet written

s!:make!-psetq!-assignments expr
Not yet written

s!:make!-psetq!-bindings expr
Not yet written

s!:make!-psetq!-vars expr
Not yet written

s!:newframe macro
Not yet written

s!:oblist expr
Not yet written

s!:oblist1 expr
Not yet written

s!:overflow expr
Not yet written

s!:prindent expr
Not yet written

s!:prinl0 expr
Not yet written

s!:prinl1 expr
Not yet written

s!:prinl2 expr
Not yet written

s!:prvector expr
Not yet written

s!:putblank expr
Not yet written

s!:putch expr
Not yet written

s!:quotep expr
Not yet written

s!:setblankcount macro
Not yet written

s!:setblanklist macro
Not yet written

s!:setindenting macro
Not yet written

s!:stamp expr
Not yet written

s!:top macro
Not yet written

safe!-fp!-pl expr
Not yet written

safe!-fp!-pl0 expr
Not yet written

safe!-fp!-plus expr
Not yet written

safe!-fp!-quot expr
Not yet written

safe!-fp!-times expr
Not yet written

sample expr
Not yet written

sassoc expr
Not yet written

schar expr
Not yet written

scharn expr
Not yet written

sec expr
Not yet written

secd expr
Not yet written

sech expr
Not yet written

seprp expr
Not yet written

set expr
Not yet written

set!-autoload expr
Not yet written

set!-help!-file expr
Not yet written

set!-print!-precision expr
Not yet written

set!-small!-modulus expr
Not yet written

setpchar expr
Not yet written

setq fexpr
Not yet written

silent!-system expr
Not yet written

simple!-string!-p expr
Not yet written

simple!-vector!-p expr
Not yet written

sin expr
Not yet written

sind expr
Not yet written

sinh expr
Not yet written

smemq expr
Not yet written

sort expr
Not yet written

sortip expr
Not yet written

spaces expr
Not yet written

special!-char expr
Not yet written

special!-form!-p expr
Not yet written

spool expr
Not yet written

sqrt expr
Not yet written

stable!-sort expr
Not yet written

stable!-sortip expr
Not yet written

start!-module expr
Not yet written

startup!-banner expr
Not yet written

stop expr
Not yet written

streamp expr
Not yet written

stringp expr
Not yet written

sub1 expr
Not yet written

subla expr
Not yet written

sublis expr
Not yet written

subst expr
Not yet written

superprinm expr
Not yet written

superprintm expr
Not yet written

sxhash expr
Not yet written

symbol!-argcode expr
Not yet written

symbol!-argcount expr
Not yet written

symbol!-env expr
Not yet written

symbol!-fastgets expr
Not yet written

symbol!-fn!-cell expr
Not yet written

symbol!-function expr
Not yet written

symbol!-make!-fastget expr
Not yet written

symbol!-name expr
Not yet written

symbol!-protect expr
Not yet written

symbol!-restore!-fns expr
Not yet written

symbol!-set!-definition expr
Not yet written

symbol!-set!-env expr
Not yet written

symbol!-set!-native expr
Not yet written

symbol!-value expr
Not yet written

symbolp expr
Not yet written

system expr
Not yet written

tagbody fexpr
Not yet written

tan expr
Not yet written

tand expr
Not yet written

tanh expr
Not yet written

terpri expr
Not yet written

threevectorp expr
Not yet written

throw fexpr
Not yet written

time expr
Not yet written

times fexpr
Not yet written

times2 expr
Not yet written

tmpnam expr
Not yet written

trace expr
Not yet written

trace!-all expr
Not yet written

traceset expr
Not yet written

traceset1 expr
Not yet written

truename expr
Not yet written

truncate expr
Not yet written

ttab expr
Not yet written

tyo expr
Not yet written

undouble!-execute expr
Not yet written

unfluid expr
Not yet written

unglobal expr
Not yet written

union expr
Not yet written

unless fexpr
Not yet written

unmake!-global expr
Not yet written

unmake!-special expr
Not yet written

unreadch expr
Not yet written

untrace expr
Not yet written

untraceset expr
Not yet written

untraceset1 expr
Not yet written

unwind!-protect fexpr
Not yet written

upbv expr
Not yet written

user!-homedir!-pathname expr
Not yet written

vectorp expr
Not yet written

verbos expr
Not yet written

when fexpr
Not yet written

where!-was!-that expr
Not yet written

window!-heading expr
Not yet written

writable!-libraryp expr
Not yet written

write!-module expr
Not yet written

wrs expr
Not yet written

xassoc expr
Not yet written

xcons expr
Not yet written

xdifference expr
Not yet written

xtab expr
Not yet written

zerop expr
Not yet written

!$\sim$block fexpr
Not yet written

!$\sim$let fexpr
Not yet written

!$\sim$tyi expr
Not yet written

\end{description}

\end{document}
