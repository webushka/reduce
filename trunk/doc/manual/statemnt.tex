\chapter{Statements}

A statement\index{Statement} is any combination of reserved words and
expressions, and has the syntax \index{Proper statement}
\begin{verbatim}
        <statement> ::= <expression>|<proper statement>
\end{verbatim}
A {\REDUCE} program consists of a series of commands which are statements
followed by a terminator:\index{Terminator}\index{Semicolon}
\index{Dollar sign}
\begin{verbatim}
        <terminator> ::= ;|$
\end{verbatim}
The division of the program into lines is arbitrary. Several statements
can be on one line, or one statement can be freely broken onto several
lines. If the program is run interactively, statements ending with ; or \$
are not processed until an end-of-line character is encountered. This
character can vary from system to system, but is normally the \key{Return}
key on an ASCII terminal.  Specific systems may also use additional keys
as statement terminators.

If a statement is a proper statement\index{Proper statement}, the
appropriate action takes place.

Depending on the nature of the proper statement some result or response may
or may not be printed out, and the response may or may not depend on the
terminator used.

If a statement is an expression, it is evaluated. If the terminator is a
semicolon, the result is printed. If the terminator is a dollar sign, the
result is not printed. Because it is not usually possible to know in
advance how large an expression will be, no explicit format statements are
offered to the user. However, a variety of output declarations are
available so that the output can be produced in different forms. These
output declarations are explained in Section~\ref{sec-output}.

The following sub-sections describe the types of proper statements
\index{Proper statement} in {\REDUCE}.

\section{Assignment Statements}

These statements\index{Assignment} have the syntax
\begin{verbatim}
    <assignment statement> ::= <expression> := <expression>
\end{verbatim}
The {\tt <expression>} on the left side is normally the name of a variable, an
operator symbol with its list of arguments filled in, or an array name with
the proper number of integer subscript values within the array bounds. For
example:
\begin{quote}
\begin{tabbing}
{\tt a1 := b + c} \\
{\tt h(l,m) := x-2*y} \hspace{1in} \= (where {\tt h} is an operator) \\
{\tt k(3,5) := x-2*y} \> (where {\tt k} is a 2-dim. array)
\end{tabbing}
\end{quote}
More general assignments\index{Assignment} such as {\tt a+b := c} are also
allowed.  The effect of these is explained in Section~\ref{sec-gensubs}.

An assignment statement causes the expression on the right-hand-side to be
evaluated.  If the left-hand-side is a variable, the value of the
right-hand-side is assigned to that unevaluated variable.  If the
left-hand-side is an operator or array expression, the arguments of that
operator or array are evaluated, but no other simplification done.  The
evaluated right-hand-side is then assigned to the resulting expression.
For example, if {\tt A} is a single-dimensional array, {\tt a(1+1) := b}
assigns the value {\tt B} to the array element {\tt a(2)}.

If a semicolon is used as the terminator when an assignment
\index{Assignment} is issued as a command (i.e. not as a part of a group
statement or procedure or other similar construct), the left-hand side
symbol of the assignment statement is printed out, followed by a
``{\tt :=}'', followed by the value of the expression on the right.

It is also possible to write a multiple assignment statement:
\index{Multiple assignment statement}
\begin{verbatim}
    <expression> := ... := <expression> := <expression>
\end{verbatim}
In this form, each {\tt <expression>} but the last is set to the value of
the last {\tt <expression>}.  If a semicolon is used as a terminator, each
expression except the last is printed followed by a ``{\tt :=}'' ending
with the value of the last expression.


\subsection{Set Statement}

In some cases, it is desirable to perform an assignment in which {\em both\/}
the left- and right-hand sides of an assignment\index{Assignment} are
evaluated.  In this case, the {\tt SET}\ttindex{SET} statement can be used
with the syntax:

\begin{verbatim}
        SET(<expression>,<expression>);
\end{verbatim}
For example, the statements
\begin{verbatim}
        j := 23;
        set(mkid(a,j),x);
\end{verbatim}
assigns the value {\tt X} to {\tt A23}.

\section{Group Statements}

The group statement\index{Group statement} is a construct used where
{\REDUCE} expects a single statement, but a series of actions needs to be
performed.  It is formed by enclosing one or more statements (of any kind)
between the symbols {\tt $<<$} and {\tt $>>$}, separated by semicolons or
dollar signs -- it doesn't matter which.  The statements are executed one
after another.

Examples will be given in the sections on {\tt IF}\ttindex{IF} and other
types of statements in which the {\tt $<<$} \ldots {\tt $>>$} construct is
useful.

If the last statement in the enclosed group has a value, then that is also
the value of the group statement.  Care must be taken not to have a
semicolon or dollar sign after the last grouped statement, if the value of
the group is relevant: such an extra terminator causes the group to have
the value NIL or zero.

\section{Conditional Statements}

The conditional statement\index{Conditional statement} has the following
syntax:

\begin{verbatim}
 <conditional statement> ::=
    IF <boolean expression> THEN <statement> [ELSE <statement>]
\end{verbatim}

The boolean expression is evaluated. If this is {\em true}, the first
{\tt <statement>} is executed.  If it is {\em false}, the second is.

{\it Examples:}
\begin{verbatim}
        if x=5 then a:=b+c else d:=e+f

        if x=5 and numberp y
           then <<ff:=q1; a:=b+c>>
           else <<ff:=q2; d:=e+f>>
\end{verbatim}
Note the use of the group statement\index{Group statement}.
\\
Conditional statements associate to the right; i.e.,\ttindex{IF}
\begin{verbatim}
        IF <a> THEN <b> ELSE IF <c> THEN <d> ELSE <e>
\end{verbatim}
is equivalent to:
\begin{verbatim}
        IF <a> THEN <b> ELSE (IF <c> THEN <d> ELSE <e>)
\end{verbatim}
In addition, the construction
\begin{verbatim}
        IF <a> THEN IF <b> THEN <c> ELSE <d>
\end{verbatim}
parses as
\begin{verbatim}
        IF <a> THEN (IF <b> THEN <c> ELSE <d>).
\end{verbatim}
If the value of the conditional statement\index{Conditional
statement} is of primary interest, it is often called a conditional
expression instead.  Its value is the value of whichever statement was
executed. (If the executed statement has no value, the conditional
expression has no value or the value 0, depending on how it is used.)

{\it Examples:}
\begin{verbatim}
        a:=if x<5 then 123 else 456;
        b:=u + v^(if numberp z then 10*z  else 1) + w;
\end{verbatim}
If the value is of no concern, the {\tt ELSE} clause may be omitted if no
action is required in the {\em false\/} case.
\begin{verbatim}
        if x=5 then a:=b+c;
\end{verbatim}
Note:  As explained in Section~\ref{sec-boolean},a
if a scalar or numerical expression is used in place of
the boolean expression -- for example, a variable is written there -- the
{\em true\/} alternative is followed unless the expression has the value 0.

\section{FOR Statements}

The {\tt FOR} statement is used to define a variety of program
loops\index{Loop}.  Its general syntax is as follows:\ttindex{UNTIL}
\ttindex{DO}\ttindex{PRODUCT}\ttindex{SUM}\ttindex{COLLECT}\ttindex{JOIN}
\begin{small}
\[ \mbox{\tt FOR} \left\{ \begin{array}{@{}ccc@{}}
    \mbox{\tt \meta{var} := \meta{number} } \left\{ \begin{array}{@{}c@{}}
    \mbox{\tt STEP \meta{number} UNTIL} \\
    \mbox{\tt :}
    \end{array}
    \right\} \mbox{\tt \meta{number}} \\[3mm]
    \multicolumn{1}{c}{\mbox{\tt EACH \meta{var}
       \(\left\{
          \begin{tabular}{@{}c@{}}
              IN \\ ON
          \end{tabular}
       \right\}\)
       \meta{list}}}
    \end{array}
    \right\} \mbox{\tt \meta{action} \meta{exprn}} \]
\end{small}%
%
where
\begin{center}
\tt       \meta{action} ::= do|product|sum|collect|join.
\end{center}
The assignment\index{Assignment} form of the {\tt FOR} statement defines an
iteration over the indicated numerical range.  If expressions that do not
evaluate to numbers are used in the designated places, an error will
result.

The {\tt FOR EACH}\ttindex{FOR EACH} form of the {\tt FOR} statement is
designed to iterate down a list.  Again, an error will occur if a list is
not used.

The action {\tt DO}\ttindex{DO} means that {\tt <exprn>} is simply
evaluated and no value kept; the statement returning 0 in this case (or no
value at the top level). {\tt COLLECT} means that the results of
evaluating {\tt <exprn>} each time are linked together to make a list,
and {\tt JOIN} means that the values of {\tt <exprn>} are themselves
lists that are joined to make one list (similar to {\tt CONC} in Lisp).
Finally, {\tt PRODUCT}\ttindex{PRODUCT} and {\tt SUM}\ttindex{SUM}
form the respective combined value out of the values of {\tt <exprn>}.

In all cases, {\tt <exprn>} is evaluated algebraically within the
scope of the current value of {\tt <var>}.  If {\tt <action>} is
{\tt DO}\ttindex{DO}, then nothing else happens.  In other cases, {\tt
<action>} is a binary operator that causes a result to be built up and
returned by {\tt FOR}.  In those cases, the loop\index{Loop} is
initialized to a default value ({\tt 0} for {\tt SUM},\ttindex{SUM} {\tt
1} for {\tt PRODUCT},\ttindex{PRODUCT} and an empty list for the other
actions).  The test for the end condition is made before any action is
taken.  As in Pascal, if the variable is out of range in the assignment
case, or the {\tt <list>} is empty in the {\tt FOR EACH}\ttindex{FOR EACH}
case, {\tt <exprn>} is not evaluated at all.

{\it Examples:}
\begin{enumerate}
\item If {\tt A}, {\tt B} have been declared to be arrays, the following
stores $5^{2}$ through $10^{2}$ in {\tt A(5)} through {\tt A(10)}, and at
the same time stores the cubes in the {\tt B} array:
\begin{verbatim}
   for i := 5 step 1 until 10 do <<a(i):=i^2; b(i):=i^3>>
\end{verbatim}
\item As a convenience, the common construction
\begin{verbatim}
        STEP 1 UNTIL
\end{verbatim}
may be abbreviated to a colon. Thus, instead of the above we could write:
\begin{verbatim}
        for i := 5:10 do <<a(i):=i^2; b(i):=i^3>>
\end{verbatim}
\item The following sets {\tt C} to the sum of the squares of 1,3,5,7,9;
and {\tt D} to the expression {\tt x*(x+1)*(x+2)*(x+3)*(x+4):}
\begin{verbatim}
        c := for j:=1 step 2 until 9 sum j^2;
        d := for k:=0 step 1 until 4 product (x+k);
\end{verbatim}
\item The following forms a list of the squares of the elements of the list
{\tt \{a,b,c\}:}\ttindex{FOR EACH}
\begin{verbatim}
        for each x in {a,b,c} collect x^2;
\end{verbatim}
\item The following forms a list of the listed squares of the elements of the
list {\tt \{a,b,c\}}
(i.e., {\tt \{\{A\verb|^|2\},\{B\verb|^|2\},\{C\verb|^|2\}\}):}
\begin{verbatim}
        for each x in {a,b,c} collect {x^2};
\end{verbatim}
\item The following also forms a list of the squares of the elements of
the list {\tt \{a,b,c\},} since the {\tt JOIN} operation joins the
individual lists into one list:\ttindex{FOR EACH}
\begin{verbatim}
        for each x in {a,b,c} join {x^2};
\end{verbatim}
\end{enumerate}
The control variable used in the {\tt FOR} statement is actually a new
variable, not related to the variable of the same name outside the {\tt
FOR} statement.  In other words, executing a statement {\tt for i:=} \ldots
doesn't change the system's assumption that $i^{2} = -1$.
Furthermore, in algebraic mode, the value of the control variable is
substituted in {\tt <exprn>} only if it occurs explicitly in that
expression.  It will not replace a variable of the same name in the value
of that expression.  For example:
\begin{verbatim}
        b := a; for a := 1:2 do write b;
\end{verbatim}
prints {\tt A} twice, not 1 followed by 2.

\section{WHILE \ldots DO}

The\ttindex{WHILE} {\tt FOR \ldots DO}\ttindex{DO} feature allows easy
coding of a repeated operation in which the number of repetitions is known
in advance.  If the criterion for repetition is more complicated, {\tt
WHILE \ldots DO} can often be used.  Its syntax is:
\begin{verbatim}
        WHILE <boolean expression> DO <statement>
\end{verbatim}
The {\tt WHILE \ldots DO} controls the single statement following {\tt DO}.
If several statements are to be repeated, as is almost always the case,
they must be grouped using the $<<$ \ldots $>>$ or {\tt BEGIN \ldots END}
as in the example below.

The {\tt WHILE} condition is tested each time {\em before\/} the action
following the {\tt DO} is attempted.  If the condition is false to begin
with, the action is not performed at all.  Make sure that what is to be
tested has an appropriate value initially.

{\it Example:}

Suppose we want to add up a series of terms, generated one by one, until
we reach a term which is less than 1/1000 in value.  For our simple
example, let us suppose the first term equals 1 and each term is obtained
from the one before by taking one third of it and adding one third its
square. We would write:
\begin{verbatim}
        ex:=0; term:=1;
        while num(term - 1/1000) >= 0  do
                <<ex := ex+term; term:=(term + term^2)/3>>;
        ex;
\end{verbatim}
As long as {\tt TERM} is greater than or equal to ({\tt >=}) 1/1000 it will
be added to {\tt EX} and the next {\tt TERM} calculated.  As soon as {\tt
TERM} becomes less than 1/1000 the {\tt WHILE} test fails and the {\tt
TERM} will not be added.


\section{REPEAT \ldots UNTIL}

\ttindex{REPEAT} {\tt REPEAT \ldots  UNTIL} is very similar in purpose to
{\tt WHILE \ldots DO}.  Its syntax is:
\begin{verbatim}
        REPEAT <statement> UNTIL <boolean expression>
\end{verbatim}
(PASCAL users note: Only a single statement -- usually a group statement
-- is allowed between the {\tt REPEAT} and the {\tt UNTIL.)}

There are two essential differences:
\begin{enumerate}
\item The test is performed {\em after\/} the controlled statement (or group of
statements) is executed, so the controlled statement is always executed at
least once.

\item The test is a test for when to stop rather than when to continue, so its
``polarity'' is the opposite of that in {\tt WHILE \ldots DO.}
\end{enumerate}

As an example, we rewrite the example from the {\tt WHILE \ldots DO} section:
\begin{samepage}
\begin{verbatim}
        ex:=0; term:=1;
        repeat <<ex := ex+term; term := (term + term^2)/3>>
           until num(term - 1/1000) < 0;
        ex;
\end{verbatim}
\end{samepage}
In this case, the answer will be the same as before, because in neither
case is a term added to {\tt EX} which is less than 1/1000.

\section{Compound Statements}

\index{Compound statement}Often the desired process can best (or only) be
described as a series of steps to be carried out one after the other.  In
many cases, this can be achieved by use of the group statement\index{Group
statement}.  However, each step often provides some intermediate
result, until at the end we have the final result wanted.  Alternatively,
iterations on the steps are needed that are not possible with constructs
such as {\tt WHILE}\ttindex{WHILE} or {\tt REPEAT}\ttindex{REPEAT}
statements.  In such cases the steps of the process must be
enclosed between the words {\tt BEGIN} and {\tt END}\ttindex{BEGIN \ldots
END} forming what is technically called a {\em block\/}\index{Block} or
{\em compound\/} statement.  Such a compound statement can in fact be used
wherever a group statement appears.  The converse is not true: {\tt BEGIN
\ldots END} can be used in ways that {\tt $<<$} \ldots {\tt $>>$} cannot.

If intermediate results must be formed, local variables must be provided
in which to store them. {\em Local\/} means that their values are deleted as
soon as the block's operations are complete, and there is no conflict with
variables outside the block that happen to have the same name.  Local
variables are created by a {\tt SCALAR}\ttindex{SCALAR} declaration
immediately after the {\tt BEGIN}:
\begin{verbatim}
     scalar a,b,c,z;
\end{verbatim}
If more convenient, several {\tt SCALAR} declarations can be given one after
another:
\begin{verbatim}
     scalar a,b,c;
     scalar z;
\end{verbatim}
In place of {\tt SCALAR} one can also use the declarations
{\tt INTEGER}\ttindex{INTEGER} or {\tt REAL}\ttindex{REAL}.  In the present
version of {\REDUCE} variables declared {\tt INTEGER} are expected to have
only integer values, and are initialized to 0. {\tt REAL}
variables on the other hand are currently treated as algebraic mode {\tt
SCALAR}s.

{\it CAUTION:} {\tt INTEGER}, {\tt REAL} and {\tt SCALAR} declarations can
only be given immediately after a {\tt BEGIN}.  An error will result if
they are used after other statements in a block (including {\tt ARRAY} and
{\tt OPERATOR} declarations, which are global in scope), or outside the
top-most block (e.g., at the top level).  All variables declared {\tt
SCALAR} are automatically initialized to zero in algebraic mode ({\tt NIL}
in symbolic mode).

Any symbols not declared as local variables in a block refer to the
variables of the same name in the current calling environment. In
particular, if they are not so declared at a higher level (e.g., in a
surrounding block or as parameters in a calling procedure), their values can
be permanently changed.

Following the {\tt SCALAR}\ttindex{SCALAR} declaration(s), if any, write the
statements to be executed, one after the other, separated by delimiters
(e.g., {\tt ;} or {\tt \$}) (it doesn't matter which).  However, from a
stylistic point of view, {\tt ;} is preferred.

The last statement in the body, just before {\tt END}, need not have a
terminator (since the {\tt BEGIN \ldots END} are in a sense brackets
confining the block statements).  The last statement must also be the
command {\tt RETURN}\ttindex{RETURN} followed by the variable or
expression whose value is to be the value returned by the procedure.  If
the {\tt RETURN} is omitted (or nothing is written after the word
{\tt RETURN}) the procedure will have no value or the value zero, depending
on how it is used (and {\tt NIL} in symbolic mode).  Remember to put a
terminator after the {\tt END}.

{\it Example:}

Given a previously assigned integer value for {\tt N}, the following block
will compute the Legendre polynomial of degree {\tt N} in the variable
{\tt X}:
\begin{verbatim}
        begin scalar seed,deriv,top,fact;
           seed:=1/(y^2 - 2*x*y +1)^(1/2);
           deriv:=df(seed,y,n);
           top:=sub(y=0,deriv);
           fact:=for i:=1:n product i;
           return top/fact
        end;
\end{verbatim}

\subsection{Compound Statements with GO TO}

It is possible to have more complicated structures inside the {\tt BEGIN
\ldots END}\ttindex{BEGIN \ldots END} brackets than indicated in the
previous example.  That the individual lines of the program need not be
assignment\index{Assignment} statements, but could be almost any other
kind of statement or command, needs no explanation.  For example,
conditional statements, and {\tt WHILE}\ttindex{WHILE} and {\tt REPEAT}
\ttindex{REPEAT} constructions, have an obvious role in defining more
intricate blocks.

If these structured constructs don't suffice, it is possible to use labels
\index{Label} and {\tt GO} {\tt TO}s\ttindex{GO TO} within a compound
statement,\index{Compound statement} and also to use {\tt RETURN}
\ttindex{RETURN} in places within the block other than just before the
{\tt END}.  The following subsections discuss these matters in detail.
For many readers the following example, presenting one possible definition
of a process to calculate the factorial of {\tt N} for preassigned {\tt N}
will suffice:

{\it Example:}
\begin{verbatim}
        begin scalar m;
            m:=1;
         l: if n=0 then return m;
            m:=m*n;
            n:=n-1;
            go to l
        end;
\end{verbatim}

\subsection{Labels and GO TO Statements}

\index{Label}\ttindex{GO TO}Within a {\tt BEGIN \ldots END} compound
statement it is possible to label statements, and transfer to them out of
sequence using {\tt GO} {\tt TO} statements.  Only statements on the top
level inside compound statements can be labeled, not ones inside
subsidiary constructions like {\tt $<<$} \ldots {\tt $>>$}, {\tt IF} \ldots
{\tt THEN} \ldots , {\tt WHILE} \ldots {\tt DO} \ldots , etc.

Labels and {\tt GO TO} statements have the syntax:
\begin{verbatim}
        <go to statement> ::= GO TO <label> | GOTO <label>
        <label> ::= <identifier>
        <labeled statement> ::= <label>:<statement>
\end{verbatim}
Note that statement names cannot be used as labels.

While {\tt GO TO} is an unconditional transfer, it is frequently used
in conditional statements such as
\begin{verbatim}
        if x>5 then go to abcd;
\end{verbatim}
giving the effect of a conditional transfer.

Transfers using {\tt GO TO}s can only occur within the block in which the
{\tt GO TO} is used.  In other words, you cannot transfer from an inner
block to an outer block using a {\tt GO TO}.  However, if a group statement
occurs within a compound statement, it is possible to jump out of that group
statement to a point within the compound statement using a {\tt GO TO}.

\subsection{RETURN Statements}

The value corresponding to a {\tt BEGIN \ldots END} compound statement,
\ttindex{BEGIN \ldots END} such as a procedure body, is normally 0 ({\tt
NIL} in symbolic mode).  By executing a {\tt RETURN}\ttindex{RETURN}
statement in the compound statement a different value can be returned.
After a {\tt RETURN} statement is executed, no further statements within
the compound statement are executed.

{\tt Examples:}
\begin{verbatim}
        return x+y;
        return m;
        return;
\end{verbatim}
Note that parentheses are not required around the {\tt x+y}, although they
are permitted.  The last example is equivalent to {\tt return 0} or {\tt
return nil}, depending on whether the block is used as part of an
expression or not.

Since {\tt RETURN}\ttindex{RETURN} actually moves up only one
block\index{Block} level, in a sense the casual user is not expected to
understand, we tabulate some cautions concerning its use.
\begin{enumerate}
\item {\tt RETURN} can be used on the top level inside the compound
statement, i.e. as one of the statements bracketed together by the {\tt
BEGIN \ldots END}\ttindex{BEGIN \ldots END}

\item {\tt RETURN} can be used within a top level {\tt $<<$} \ldots {\tt
$>>$} construction within the compound statement.  In this case, the {\tt
RETURN} transfers control out of both the group statement and the compound
statement.

\item {\tt RETURN} can be used within an {\tt IF} \ldots {\tt THEN} \ldots
{\tt ELSE} \ldots on the top level within the compound statement.
\end{enumerate}
NOTE:  At present, there is no construct provided to permit early
termination of a {\tt FOR}\ttindex{FOR}, {\tt WHILE}\ttindex{WHILE},
or {\tt REPEAT}\ttindex{REPEAT} statement.  In particular, the use of
{\tt RETURN} in such cases results in a syntax error.  For example,
\begin{verbatim}
        begin scalar y;
           y := for i:=0:99 do if a(i)=x then return b(i);
           ...
\end{verbatim}
will lead to an error.

