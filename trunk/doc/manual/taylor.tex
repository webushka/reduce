
\subsection{Basic Use}

The most important operator is `\verb+TAYLOR+'. \index{TAYLOR operator}
It is used as follows:
\hypertarget{operator:TAYLOR}{}
\begin{verbatim}
  TAYLOR(EXP:algebraic,
         VAR:kernel,VAR0:algebraic,ORDER:integer[,...])
         :algebraic.
\end{verbatim}
where \f{EXP} is the expression to be expanded. It can be any \REDUCE{}
object, even an expression containing other Taylor kernels. \f{VAR} is
the kernel with respect to which \f{EXP} is to be expanded. \f{VAR0}
denotes the point about which and \f{ORDER} the order up to which
expansion is to take place. If more than one \f{(VAR, VAR0, ORDER)} triple
is specified \f{TAYLOR} will expand its first argument independently
with respect to each variable in turn. For example,
\begin{verbatim}
  taylor(e^(x^2+y^2),x,0,2,y,0,2);
\end{verbatim}
will calculate the Taylor expansion up to order $X^{2}*Y^{2}$:
\begin{verbatim}
       2    2    2  2      3  3
  1 + y  + x  + y *x  + O(x ,y )
\end{verbatim}
Note that once the expansion has been done it is not possible to
calculate higher orders.
Instead of a kernel, \f{VAR} may also
be a list of kernels. In this case expansion will take place in a way
so that the \emph{sum} of the degrees of the kernels does not exceed
\f{ORDER}.
If \f{VAR0} evaluates to the special identifier \f{INFINITY}, expansion is
done in a series in 1/VAR instead of \f{VAR}.

The expansion is performed variable per variable, i.e.\ in the example
above by first expanding $\exp(x^{2}+y^{2})$ with respect to $x$ and
then expanding every coefficient with respect to $y$.

\index{IMPLICIT\_TAYLOR operator}\index{INVERSE\_TAYLOR}
\hypertarget{operator:IMPLICIT_TAYLOR}{}
There are two
extra operators to compute the Taylor expansions of implicit and
inverse functions:
\begin{verbatim}
  IMPLICIT_TAYLOR(F:algebraic,
                  VAR:kernel,DEPVAR:kernel,
                  VAR0:algebraic,DEPVAR0:algebraic,
                  ORDER:integer)
           :algebraic
\end{verbatim}
takes a function F depending on two variables VAR and DEPVAR and
computes the Taylor series of the implicit function DEPVAR(VAR)
given by the equation F(VAR,DEPVAR) = 0, around the point VAR0.  
(Violation of the necessary condition F(VAR0,DEPVAR0)=0 causes an error.)
For example,
\begin{verbatim}
  implicit_taylor(x^2 + y^2 - 1,x,y,0,1,5);
\end{verbatim}
gives the output
\begin{verbatim}
       1   2    1   4      6
  1 - ---*x  - ---*x  + O(x )
       2        8
\end{verbatim}

\hypertarget{operator:INVERSE_TAYLOR}{}
The operator
\begin{verbatim}
  INVERSE_TAYLOR(F:algebraic,VAR:kernel,DEPVAR:kernel,
                 VAR0:algebraic,ORDER:integer)
         : algebraic
\end{verbatim}
takes a function F depending on VAR1 and computes the Taylor series of
the inverse of F with respect to VAR2. For example,
\begin{verbatim}
  inverse_taylor(exp(x)-1,x,y,0,8);
\end{verbatim}
yields
\begin{verbatim}
       1   2    1   3    1   4    1   5                  9
  y - ---*y  + ---*y  - ---*y  + ---*y  + (3 terms) + O(y )
       2        3        4        5
\end{verbatim}


\index{TAYLORPRINTTERMS variable}\hypertarget{reserved:TAYLORPRINTTERMS}{}
When a Taylor kernel is printed, only a certain number of (non-zero)
coefficients are shown. If there are more, an expression of the form
\f{($n$ terms)} is printed to indicate how many non-zero
terms have been suppressed. The number of terms printed is given by
the value of the shared algebraic variable \f{TAYLORPRINTTERMS}.
Allowed values are integers and the special identifier \f{ALL}. The
latter setting specifies that all terms are to be printed. The default
setting is $5$.

\index{PART operator}\index{PART}
The \f{PART} operator can be used to extract subexpressions of a
Taylor expansion in the usual way. All terms can be accessed,
irregardless of the value of the variable \f{TAYLORPRINTTERMS}.


\index{TAYLORKEEPORIGINAL switch}
If the switch \hyperlink{switch:TAYLORKEEPORIGINAL}{\f{TAYLORKEEPORIGINAL}}
is set to \f{ON} the
original expression EXP is kept for later reference.
It can be recovered by means of the operator

\hypertarget{operator:TAYLORORIGINAL}{}
\hspace*{2em} {\tt TAYLORORIGINAL}(EXP:{\em exprn}):{\em exprn}

An error is signalled if EXP is not a Taylor kernel or if the original
expression was not kept, i.e.\ if \f{TAYLORKEEPORIGINAL} was
\f{OFF} during expansion.  The template of a Taylor kernel, i.e.\
the list of all variables with respect to which expansion took place
together with expansion point and order can be extracted using
\ttindex{TAYLORTEMPLATE}.

\hypertarget{operator:TAYLORTEMPLATE}{}
\hspace*{2em} {\tt TAYLORTEMPLATE}(EXP:{\em exprn}):{\em list}

This returns a list of lists with the three elements (VAR,VAR0,ORDER).
As with \f{TAYLORORIGINAL},
an error is signalled if EXP is not a Taylor kernel.

The operator
\hypertarget{operator:TAYLORTOSTANDARD}{}\\
\hspace*{2em} {\tt TAYLORTOSTANDARD}(EXP:{\em exprn}):{\em exprn}

converts all Taylor kernels in EXP into standard form and
\ttindex{TAYLORTOSTANDARD} resimplifies the result.

The boolean operator
\hypertarget{operator:TAYLORSERIESP}{}\\
\hspace*{2em} {\tt TAYLORSERIESP}(EXP:{\em exprn}):{\em boolean}

may be used to determine if EXP is a Taylor kernel.
\ttindex{TAYLORSERIESP} (Note that this operator is subject to the same
restrictions as, e.g., \f{ORDP} or \f{NUMBERP}, i.e.\ it may only be used in
boolean expressions in \f{IF} or \f{LET} statements. 

Finally there is

\hypertarget{operator:TAYLORCOMBINE}{}
\hspace*{2em} {\tt TAYLORCOMBINE}(EXP:{\em exprn}):{\em exprn}

which tries to combine all Taylor kernels found in EXP into one.
\ttindex{TAYLORCOMBINE}
Operations currently possible are:
\index{Taylor series ! arithmetic}
\begin{itemize}
  \item Addition, subtraction, multiplication, and division.
  \item Roots, exponentials, and logarithms.
  \item Trigonometric and hyperbolic functions and their inverses.
\end{itemize}
Application of unary operators like \f{LOG} and \f{ATAN} will
nearly always succeed. For binary operations their arguments have to be
Taylor kernels with the same template. This means that the expansion
variable and the expansion point must match. Expansion order is not so
important, different order usually means that one of them is truncated
before doing the operation.

\ttindex{TAYLORKEEPORIGINAL} \ttindex{TAYLORCOMBINE}
If \hyperlink{switch:TAYLORKEEPORIGINAL}{\f{TAYLORKEEPORIGINAL}} is set to \f{ON} and if all Taylor
kernels in \f{exp} have their original expressions kept
\hyperlink{switch:TAYLORCOMBINE}{\f{TAYLORCOMBINE}} will also combine these and store the result
as the original expression of the resulting Taylor kernel.
\index{TAYLORAUTOEXPAND switch}
There is also the switch \hyperlink{switch:TAYLORAUTOEXPAND}{\f{TAYLORAUTOEXPAND}} (see below).

There are a few restrictions to avoid mathematically undefined
expressions: it is not possible to take the logarithm of a Taylor
kernel which has no terms (i.e. is zero), or to divide by such a
beast.  There are some provisions made to detect singularities during
expansion: poles that arise because the denominator has zeros at the
expansion point are detected and properly treated, i.e.\ the Taylor
kernel will start with a negative power.  (This is accomplished by
expanding numerator and denominator separately and combining the
results.)  Essential singularities of the known functions (see above)
are handled correctly.

\index{Taylor series ! differentiation}
Differentiation of a Taylor expression is possible.  If you
differentiate with respect to one of the Taylor variables the order
will decrease by one.

\index{Taylor series ! substitution}
Substitution is a bit restricted: Taylor variables can only be replaced
by other kernels.  There is one exception to this rule: you can always
substitute a Taylor variable by an expression that evaluates to a
constant.  Note that \REDUCE{} will not always be able to determine
that an expression is constant.

\index{Taylor series ! integration}
Only simple taylor kernels can be integrated. More complicated
expressions that contain Taylor kernels as parts of themselves are
automatically converted into a standard representation by means of the
\hyperlink{operator:TAYLORTOSTANDARD}{\f{TAYLORTOSTANDARD}} operator. 
In this case a suitable warning is printed.

\index{Taylor series ! reversion} It is possible to revert a Taylor
series of a function $f$, i.e., to compute the first terms of the
expansion of the inverse of $f$ from the expansion of $f$. This is
done by the operator

\hypertarget{operator:TAYLORREVERT}{}
\hspace*{2em} {\tt TAYLORREVERT}(EXP:{\em exprn},OLDVAR:{\em kernel},
                                 NEWVAR:{\em kernel}):{\em exprn}

EXP must evaluate to a Taylor kernel with OLDVAR being one of its
expansion variables. Example:
\begin{verbatim}
  taylor (u - u**2, u, 0, 5)$
  taylorrevert (ws, u, x);
\end{verbatim}
gives
\begin{verbatim}
       2      3      4       5      6
  x + x  + 2*x  + 5*x  + 14*x  + O(x )
\end{verbatim}

This package introduces a number of new switches:
\begin{description}

\index{TAYLORAUTOCOMBINE switch}
\item[\sw{TAYLORAUTOCOMBINE}] \hypertarget{switch:TAYLORAUTOCOMBINE}{}causes
    Taylor expressions to be automatically combined during the simplification
    process.  This is equivalent to applying \f{TAYLORCOMBINE} to
    every expression that contains Taylor kernels.
    Default is \f{ON}.

\index{TAYLORAUTOEXPAND switch}
\item[\sw{TAYLORAUTOEXPAND}] \hypertarget{switch:TAYLORAUTOEXPAND}{} makes Taylor expressions ``contagious''
    in the sense that \f{TAYLORCOMBINE} tries to Taylor expand
    all non-Taylor subexpressions and to combine the result with the
    rest. Default is \f{OFF}.

\index{TAYLORKEEPORIGINAL switch}\hypertarget{switch:TAYLORKEEPORIGINAL}{}
\item[\sw{TAYLORKEEPORIGINAL}] forces the
    package to keep the original expression, i.e.\ the expression
    that was Taylor expanded.  All operations performed on the
    Taylor kernels are also applied to this expression  which can
    be recovered using the operator \f{TAYLORORIGINAL}.
    Default is \f{OFF}.

\index{TAYLORPRINTORDER switch}\hypertarget{switch:TAYLORPRINTORDER}{}
\item[\sw{TAYLORPRINTORDER}] causes the
    remainder to be printed in big-$O$ notation.  Otherwise, three
    dots are printed. Default is \f{ON}.

\index{VERBOSELOAD switch}
\item[\sw{VERBOSELOAD}] will cause
    \REDUCE{} to print some information when the Taylor package is
    loaded.  This switch is already present in \textsf{PSL} systems.
    Default is \f{OFF}.

\end{description}
\index{defaults ! TAYLOR package}

\subsection{Caveats}
\index{caveats ! TAYLOR package}

\f{TAYLOR} should always detect non-analytical expressions in
its first argument. As an example, consider the function $xy/(x+y)$
that is not analytical in the neighborhood of $(x,y) = (0,0)$: Trying
to calculate
\begin{verbatim}
  taylor(x*y/(x+y),x,0,2,y,0,2);
\end{verbatim}
causes an error
\begin{verbatim}
***** Not a unit in argument to QUOTTAYLOR
\end{verbatim}
Note that it is not generally possible to apply the standard \REDUCE{}
operators to a Taylor kernel. For example, \f{PART}, \f{COEFF},
or \f{COEFFN} cannot be used. Instead, the expression at hand has
to be converted to standard form first using the \f{TAYLORTOSTANDARD}
operator.

\subsection{Warning messages}
\index{warnings ! TAYLOR package}
\begin{description}

\item[\msg{*** Cannot expand further... truncation done}]\mbox{}\\
    You will get this warning if you try to expand a Taylor kernel to
    a higher order.

\item[\msg{*** Converting Taylor kernels to standard representation}]\mbox{}\\
    This warning appears if you try to integrate an expression
    containing Taylor kernels.

\end{description}

\subsection{Error messages}
\index{errors ! TAYLOR package}
\begin{description}

\item[\msg{***** Branch point detected in ...}]\mbox{}\\
    This occurs if you take a rational power of a Taylor kernel
    and raising the lowest order term of the kernel to this
    power yields a non analytical term (i.e.\ a fractional power).

\item[\msg{***** Cannot replace part ... in Taylor kernel}]\mbox{}\\
\index{PART Operator}%
    The \f{PART} operator can only be used to either replace the
    template of a Taylor kernel (part 2) or the original expression
    that is kept for reference (part 3).    

\item[\msg{***** Computation loops (recursive definition?): ...}]\mbox{}\\
    Most probably the expression to be expanded contains an operator
    whose derivative involves the operator itself.

\item[\msg{***** Error during expansion (possible singularity)}]\mbox{}\\
    The expression you are trying to expand caused an error.
    As far as I know this can only happen if it contains a function
    with a pole or an essential singularity at the expansion point.
    (But one can never be sure.)

\item[\msg{***** Essential singularity in ...}]\mbox{}\\
    An essential singularity was detected while applying a
    special function to a Taylor kernel.

\item[\msg{***** Expansion point lies on branch cut in ...}]\mbox{}\\
    The only functions with branch cuts this package knows of are
    (natural) logarithm, inverse circular and hyperbolic tangent and
    cotangent.  The branch cut of the logarithm is assumed to lie on
    the negative real axis.  Those of the arc tangent and arc
    cotangent functions are chosen to be compatible with this: both
    have essential singularities at the points $\pm i$.  The branch
    cut of arc tangent is the straight line along the imaginary axis
    connecting $+1$ to $-1$ going through $\infty$ whereas that of arc
    cotangent goes through the origin.  Consequently, the branch cut
    of the inverse hyperbolic tangent resp.\ cotangent lies on the
    real axis and goes from $-1$ to $+1$, that of the latter across
    $0$, the other across $\infty$.

    The error message can currently only appear when you try to
    calculate the inverse tangent or cotangent of a Taylor
    kernel that starts with a negative degree.
    The case of a logarithm of a Taylor kernel whose constant term
    is a negative real number is not caught since it is
    difficult to detect this in general.

\item[\msg{***** Input expression non-zero at given point}]\mbox{}\\
    Violation of the necessary condition F(VAR0,DEPVAR0)=0 for the arguments of
    \f{IMPLICIT\_TAYLOR}.

\item[\msg{***** Invalid substitution in Taylor kernel: ...}]\mbox{}\\
    You tried to substitute a variable that is already present in the
    Taylor kernel or on which one of the Taylor variables depend.

\item[\msg{***** Not a unit in ...}]\mbox{}\\
    This will happen if you try to divide by or take the logarithm of
    a Taylor series whose constant term vanishes.

\item[\msg{***** Not implemented yet (...)}]\mbox{}\\
    Sorry, but I haven't had the time to implement this feature.
    Tell me if you really need it, maybe I have already an improved
    version of the package.

\item[\msg{***** Reversion of Taylor series not possible: ...}]\mbox{}\\
\ttindex{TAYLORREVERT}
    You tried to call the \f{TAYLORREVERT} operator with
    inappropriate arguments. The second half of this error message
    tells you why this operation is not possible.

\item[\msg{***** Taylor kernel doesn't have an original part}]\mbox{}\\
\ttindex{TAYLORORIGINAL} \ttindex{TAYLORKEEPORIGINAL}
    The Taylor kernel upon which you try to use \f{TAYLORORIGINAL}
    was created with the switch \f{TAYLORKEEPORIGINAL}
    set to \f{OFF}
    and does therefore not keep the original expression.

\item[\msg{***** Wrong number of arguments to TAYLOR}]\mbox{}\\
    You try to use the operator \f{TAYLOR} with a wrong number of
    arguments.

\item[\msg{***** Zero divisor in TAYLOREXPAND}]\mbox{}\\
    A zero divisor was found while an expression was being expanded.
    This should not normally occur.

\item[\msg{***** Zero divisor in Taylor substitution}]\mbox{}\\
    That's exactly what the message says.  As an example consider the
    case of a Taylor kernel containing the term \f{1/x} and you try
    to substitute \f{x} by \f{0}.

\item[\msg{***** ... invalid as kernel}]\mbox{}\\
    You tried to expand with respect to an expression that is not a
    kernel.

\item[\msg{***** ... invalid as order of Taylor expansion}]\mbox{}\\
    The order parameter you gave to \f{TAYLOR} is not an integer.

\item[\msg{***** ... invalid as Taylor kernel}]\mbox{}\\
\ttindex{TAYLORORIGINAL} \ttindex{TAYLORTEMPLATE}
    You tried to apply \f{TAYLORORIGINAL} or \f{TAYLORTEMPLATE}
    to an expression that is not a Taylor kernel.

\item[\msg{***** ... invalid as Taylor Template element}]\mbox{}\\
    You tried to substitute the \f{TAYLORTEMPLATE} part of a Taylor
    kernel with a list a incorrect form. For the correct form see the
    description of the \f{TAYLORTEMPLATE} operator.

\item[\msg{***** ... invalid as Taylor variable}]\mbox{}\\
    You tried to substitute a Taylor variable by an expression that is
    not a kernel.

\item[\msg{***** ... invalid as value of TaylorPrintTerms}]\mbox{}\\
\ttindex{TAYLORPRINTTERMS}
    You have assigned an invalid value to \hyperlink{reserved:TAYLORPRINTTERMS}{\f{TAYLORPRINTTERMS}}.
    Allowed values are: an integer or the special identifier
    \f{ALL}.

\item[\msg{TAYLOR PACKAGE (...): this can't happen ...}]\mbox{}\\
    This message shows that an internal inconsistency was detected.
    This is not your fault, at least as long as you did not try to
    work with the internal data structures of \REDUCE. Send input
    and output to me, together with the version information that is
    printed out.

\end{description}

\subsection{Comparison to other packages}

At the moment there is only one \REDUCE{} package that I know of:
the truncated power series package by Alan Barnes and Julian Padget.
In my opinion there are two major differences:
\begin{itemize}
  \item The interface. They use the domain mechanism for their power
        series, I decided to invent a special kind of kernel. Both
        approaches have advantages and disadvantages: with domain
        modes, it is easier
        to do certain things automatically, e.g., conversions.
  \item The concept of a truncated series. Their idea is to remember
        the original expression and to compute more coefficients when
        more of them are needed. My approach is to truncate at a
        certain order and forget how the unexpanded expression
        looked like.  I think that their method is more widely
        usable, whereas mine is more efficient when you know in
        advance exactly how many terms you need.
\end{itemize}

