\chapter{User Contributed Packages} \index{User packages}
\label{chap-user}
The complete {\REDUCE} system includes a number of packages contributed by
users that are provided as a service to the user community.  Questions
regarding these packages should be directed to their individual authors.

All such packages have been precompiled as part of the installation process.
However, many must be specifically loaded before they can be used. (Those
that are loaded automatically are so noted in their description.) You should
also consult the user notes for your particular implementation for further
information on whether this is necessary.  If it is, the relevant command is
\f{LOAD\_PACKAGE},\ttindex{LOAD\_PACKAGE} which takes a list of one or
more package names as argument, for example:

\begin{verbatim}
        load_package algint;
\end{verbatim}
although this syntax may vary from implementation to implementation.

Nearly all these packages come with separate documentation and test files
(except those noted here that have no additional documentation), which is
included, along with the source of the package, in the {\REDUCE} system
distribution.  These items should be studied for any additional details on
the use of a particular package.

\let\origsectionmark=\sectionmark
%\def\sectionmark#1{%
%  \def\xyzzy##1:##2\relax{\origsectionmark{##1}}%
%  \xyzzy#1\relax}
\def\sectionmark#1{}


The packages available in the current release of {\REDUCE} are as follows:

\section{ALGINT: Integration of square roots} \ttindex{ALGINT}
\label{ALGINT}

This package, which is an extension of the basic integration package
distributed with {\REDUCE}, will analytically integrate a wide range of
expressions involving square roots where the answer exists in that class
of functions. It is an implementation of the work described in J.H.
Davenport, ``On the Integration of Algebraic Functions", LNCS 102,
Springer Verlag, 1981.  Both this and the source code should be consulted
for a more detailed description of this work.

\hypertarget{switch:ALGINT}{}
The \texttt{ALGINT} package is loaded automatically when the switch \sw{ALGINT}
is turned on.  
One enters an expression for integration, as with the regular integrator,
for example:
\begin{verbatim}
        int(sqrt(x+sqrt(x**2+1))/x,x);
\end{verbatim}
If one later wishes to integrate expressions without using the facilities of
this package, the switch \sw{ALGINT} \ttindex{ALGINT} should be turned
off. 

\hypertarget{switch:TRA}{}
The switches supported by the standard integrator (e.g., \sw{TRINT})
\ttindex{TRINT} are also supported by this package.  In addition, the
switch \sw{TRA}, \ttindex{TRA} if on, will give further tracing
information about the specific functioning of the algebraic integrator.

There is no additional documentation for this package.

Author: James H. Davenport.

\section{APPLYSYM: Infinitesimal symmetries of differential equations}
\ttindex{APPLYSYM}

This package provides programs APPLYSYM, QUASILINPDE and DETRAFO for
applying infinitesimal symmetries of differential equations, the
generalization of special solutions and the calculation of symmetry and
similarity variables.

Author: Thomas Wolf.

\section{ARNUM: An algebraic number package} \ttindex{ARNUM}

This package provides facilities for handling algebraic numbers as
polynomial coefficients in {\REDUCE} calculations. It includes facilities for
introducing indeterminates to represent algebraic numbers, for calculating
splitting fields, and for factoring and finding greatest common divisors
in such domains.

Author: Eberhard Schr\"ufer.

\section{ASSERT: Dynamic Verification of Assertions on Function Types}
\ttindex{ASSERT}
\label{ASSERT}

ASSERT admits to add to symbolic mode RLISP code assertions (partly)          
specifying \emph{types} of the arguments and results of RLISP expr
procedures. These types can be associated with functions testing the
validity of the respective arguments during runtime.

Author: Thomas Sturm.

\section{ASSIST: Useful utilities for various applications} \ttindex{ASSIST}
\label{ASSIST}\hypertarget{ASSIST}{}

ASSIST contains a large number of additional general purpose functions
that allow a user to better adapt \REDUCE\ to various calculational
strategies and to make the programming task more straightforward and more
efficient.

Author: Hubert Caprasse.

\section{AVECTOR: A vector algebra and calculus package} \ttindex{AVECTOR}

This package provides REDUCE with the ability to perform vector algebra
using the same notation as scalar algebra.  The basic algebraic operations
are supported, as are differentiation and integration of vectors with
respect to scalar variables, cross product and dot product, component
manipulation and application of scalar functions (e.g. cosine) to a vector
to yield a vector result.

Author: David Harper.

\section{BIBASIS: A Package for Calculating Boolean Involutive Bases}
\ttindex{BIBASIS} \label{BIBASIS}

Authors: Yuri A. Blinkov and Mikhail V. Zinin


\section{BOOLEAN: A package for boolean algebra} \ttindex{BOOLEAN}

This package supports the computation with boolean expressions in the
propositional calculus.  The data objects are composed from algebraic
expressions connected by the infix boolean operators {\bf and}, {\bf or},
{\bf implies}, {\bf equiv}, and the unary prefix operator {\bf not}.
{\bf Boolean} allows you to simplify expressions built from these
operators, and to test properties like equivalence, subset property etc.

Author: Herbert Melenk.


\section{CDIFF: A package for the geometry of Differential Equations}
\ttindex{CDIFF}\label{CDIFF}


Authors: P. Gragert, P.H.M. Kersten, G. Post and G. Roelofs, R. Vitolo.


\section{CALI: A package for computational commutative algebra}
\ttindex{CALI}

This package contains algorithms for computations in commutative algebra
closely related to the Gr\"obner algorithm for ideals and modules.  Its
heart is a new implementation of the Gr\"obner algorithm that also allows
for the computation of syzygies.  This implementation is also applicable to
submodules of free modules with generators represented as rows of a matrix.

Author: Hans-Gert Gr\"abe.

\section{CAMAL: Calculations in celestial mechanics} \ttindex{CAMAL}
\label{CAMAL}

This packages implements in REDUCE the Fourier transform procedures of the
CAMAL package for celestial mechanics.

Author: John P. Fitch.

\section{CHANGEVR: Change of Independent Variable(s) in DEs}
\ttindex{CHANGEVR}

This package provides facilities for changing the independent variables in
a differential equation. It is basically the application of the chain rule.

Author: G. \"{U}\c{c}oluk.

\section{COMPACT: Package for compacting expressions} \ttindex{COMPACT}

COMPACT is a package of functions for the reduction of a polynomial in the
presence of side relations.  COMPACT applies the side relations to the
polynomial so that an equivalent expression results with as few terms as
possible.  For example, the evaluation of
\begin{verbatim}
     compact(s*(1-sin x^2)+c*(1-cos x^2)+sin x^2+cos x^2,
             {cos x^2+sin x^2=1});
\end{verbatim}
yields the result\pagebreak[1]
\begin{samepage}
\begin{verbatim}
              2           2
        SIN(X) *C + COS(X) *S + 1 .
\end{verbatim}

Author:  Anthony C. Hearn.
\end{samepage}

\section{CRACK: Solving overdetermined systems of PDEs or ODEs}
\ttindex{CRACK}

CRACK is a package for solving overdetermined systems of partial or
ordinary differential equations (PDEs, ODEs).  Examples of programs which
make use of CRACK (finding symmetries of ODEs/PDEs, first integrals, an
equivalent Lagrangian or a "differential factorization" of ODEs) are
included.  The application of symmetries is also possible by using the
APPLYSYM package.

Authors: Andreas Brand, Thomas Wolf.

\section{CVIT: Fast calculation of Dirac gamma matrix traces}
\ttindex{CVIT}
\label{CVIT}

This package provides an alternative method for computing traces of Dirac
gamma matrices, based on an algorithm by Cvitanovich that treats gamma
matrices as 3-j symbols.

Authors: V.Ilyin, A.Kryukov, A.Rodionov, A.Taranov.

\section{DEFINT: A definite integration interface}
\ttindex{DEFINT}
\label{DEFINT}

This package finds the definite integral of an expression in a stated
interval.  It uses several techniques, including an innovative approach
based on the Meijer G-function, and contour integration.

Authors: Kerry Gaskell, Stanley M. Kameny, Winfried Neun.

\section{DESIR: Differential linear homogeneous equation solutions in the
              neighborhood of irregular and regular singular points}
\ttindex{DESIR}

This package enables the basis of formal solutions to be computed for an
ordinary homogeneous differential equation with polynomial coefficients
over Q of any order, in the neighborhood of zero (regular or irregular
singular point, or ordinary point).

Documentation for this package is in plain text.

Authors: C. Dicrescenzo, F. Richard-Jung, E. Tournier.

\section{DFPART: Derivatives of generic functions}
\ttindex{DFPART}

This package supports computations with total and partial derivatives of
formal function objects.  Such computations can be useful in the context
of differential equations or power series expansions.

Author: Herbert Melenk.

\section{DUMMY: Canonical form of expressions with dummy variables}
\ttindex{DUMMY}

This package allows a user to find the canonical form of expressions
involving dummy variables. In that way, the simplification of
polynomial expressions can be fully done. The indeterminates are general
operator objects endowed with as few properties as possible. In that way
the package may be used in a large spectrum of applications.

Author: Alain Dresse.

\section{EXCALC: A differential geometry package} \ttindex{EXCALC}

EXCALC is designed for easy use by all who are familiar with the calculus
of Modern Differential Geometry. The program is currently able to handle
scalar-valued exterior forms, vectors and operations between them, as well
as non-scalar valued forms (indexed forms). It is thus an ideal tool for
studying differential equations, doing calculations in general relativity
and field theories, or doing simple things such as calculating the
Laplacian of a tensor field for an arbitrary given frame.

Author: Eberhard Schr\"ufer.

\section{FIDE: Finite difference method for partial differential equations}
\ttindex{FIDE}

This package performs  automation of  the process of numerically
solving  partial  differential  equations  systems  (PDES)  by  means of
computer algebra.  For PDES solving, the finite difference method is applied.
The  computer  algebra  system  REDUCE  and  the  numerical  programming
language FORTRAN  are used in the presented methodology. The main aim of
this methodology is to speed up the process of preparing numerical
programs for  solving PDES.  This process is quite often, especially for
complicated systems, a tedious and time consuming task.

Documentation for this package is in plain text.

Author: Richard Liska.

\section{FPS: Automatic calculation of formal power series} \ttindex{FPS}

This package can expand a specific class of functions into their
corresponding Laurent-Puiseux series.

Authors: Wolfram Koepf and Winfried Neun.

\section{GCREF: A Graph Cross Referencer}
\ttindex{GCREF}\label{GCREF}

This package reuses the code of the RCREF package to create a graph displaying
the interdependency of procedures in a Reduce source code file. 

Authors: A. Dolzmann, T. Sturm.


\section{GENTRAN: A code generation package} \ttindex{GENTRAN}
\label{GENTRAN}

GENTRAN is an automatic code GENerator and TRANslator. It constructs
complete numerical programs based on sets of algorithmic specifications
and symbolic expressions. Formatted FORTRAN, RATFOR, PASCAL or C code can be
generated through a series of interactive commands or under the control of
a template processing routine. Large expressions can be automatically
segmented into subexpressions of manageable size, and a special
file-handling mechanism maintains stacks of open I/O channels to allow
output to be sent to any number of files simultaneously and to facilitate
recursive invocation of the whole code generation process.

Author: Barbara L. Gates.

\section{GNUPLOT: Display of functions and surfaces}
\ttindex{PLOT}\ttindex{GNUPLOT}

This package is an interface to the popular GNUPLOT package.
It allows you to display functions in 2D and surfaces in 3D
on a variety of output devices including X terminals, PC monitors, and
postscript and Latex printer files.

NOTE: The GNUPLOT package may not be included in all versions of REDUCE.

Author: Herbert Melenk.

\section{GROEBNER: A Gr\"obner basis package} \ttindex{GROEBNER}
\label{GROEBNER}

GROEBNER \ttindex{GROEBNER} is a package for the computation of Gr\"obner
Bases using the Buchberger algorithm and related methods
for polynomial ideals and modules.  It can be used over a variety of
different coefficient domains, and for different variable and term
orderings.

Gr\"obner Bases can be used for various purposes in commutative
algebra, e.g. for elimination of variables,\index{Variable elimination}
converting surd expressions to implicit polynomial form,
computation of dimensions, solution of polynomial equation systems 
\index{Polynomial equations} etc. 
The package is also used internally by the {\tt SOLVE} \ttindex{SOLVE} 
operator.

Authors: Herbert Melenk, H.M. M\"oller and Winfried Neun.


\section{GUARDIAN: Guarded Expressions in Practice}
\ttindex{GUARDIAN}\label{GUARDIAN}

Computer algebra systems typically drop some degenerate cases when
evaluating expressions, e.g., $x/x$ becomes $1$ dropping the case
$x=0$. We claim that it is feasible in practice to compute also the
degenerate cases yielding {\em guarded expressions}. We work over real
closed fields but our ideas about handling guarded expression can be
easily transferred to other situations. Using formulas as guards
provides a powerful tool for heuristically reducing the combinatorial
explosion of cases: equivalent, redundant, tautological, and
contradictive cases can be detected by simplification and quantifier
elimination. Our approach allows to simplify the expressions on the
basis of simplification knowledge on the logical side. The method
described in this paper is implemented in the {\sc reduce} package
{\sc guardian}.

Authors: Andreas Dolzmann and Thomas Sturm.

\section{IDEALS: Arithmetic for polynomial ideals} \ttindex{IDEALS}

This package implements the basic arithmetic for polynomial ideals by
exploiting the Gr\"obner bases package of REDUCE.  In order to save
computing time all intermediate Gr\"obner bases are stored internally such
that time consuming repetitions are inhibited.

Author: Herbert Melenk.

\section{INEQ: Support for solving inequalities} \ttindex{INEQ}

This package supports the operator {\bf ineq\_solve} that 
tries to solves single inequalities and sets of coupled inequalities.

Author: Herbert Melenk.

\section{INVBASE: A package for computing involutive bases} \ttindex{INVBASE}

Involutive bases are a new tool for solving problems in connection with
multivariate polynomials, such as solving systems of polynomial equations
and analyzing polynomial ideals.  An involutive basis of polynomial ideal
is nothing but a special form of a redundant Gr\"obner basis.  The
construction of involutive bases reduces the problem of solving polynomial
systems to simple linear algebra.

Authors: A.Yu. Zharkov and Yu.A. Blinkov.

\section{LAPLACE: Laplace transforms} \ttindex{LAPLACE}

This package can calculate ordinary and inverse Laplace transforms of
expressions.  Documentation is in plain text.

Authors: C. Kazasov, M. Spiridonova, V. Tomov.

\section{LIE: Functions for the classification of real n-dimensional Lie
algebras}
\ttindex{LIE}

{\bf LIE} is a package of functions for the classification of real
n-dimensional Lie algebras.  It consists of two modules: {\bf liendmc1}
and {\bf lie1234}.  With the help of the functions in the {\bf liendmcl}
module, real n-dimensional Lie algebras $L$ with a derived algebra
$L^{(1)}$ of dimension 1 can be classified.

Authors: Carsten and Franziska Sch\"obel.

\section{LIMITS: A package for finding limits} \ttindex{LIMITS}


This package loads automatically.

Author: Stanley L. Kameny.


\index{LIMITS package}
LIMITS is a fast limit package for REDUCE for functions which are
continuous except for computable poles and singularities, based on some
earlier work by Ian Cohen and John P. Fitch.  The Truncated Power Series
package is used for non-critical points, at which the value of the
function is the constant term in the expansion around that point.
\index{l'H\^opital's rule}
l'H\^opital's rule is used in critical cases, with preprocessing of
$\infty - \infty$ forms and reformatting of product forms in order
to apply l'H\^opital's rule.  A limited amount of bounded arithmetic
is also employed where applicable.

\subsection{Normal entry points}
\ttindex{LIMIT}\hypertarget{operator:LIMIT}{}
\vspace{.1in}
\noindent \texttt{LIMIT}(EXPRN:{\em algebraic}, VAR:{\em kernel},
LIMPOINT:{\em algebraic}):{\em algebraic}
\vspace{.1in}

This is the standard way of calling limit, applying all of the methods. The
result is the limit of EXPRN as VAR approaches LIMPOINT.


\subsection{Direction-dependent limits}

\ttindex{LIMIT+} \ttindex{LIMIT-}
\hypertarget{operator:LIMIT+}{}
\hypertarget{operator:LIMIT-}{}
\vspace{.1in}
\noindent \texttt{LIMIT!+}(EXPRN:{\em algebraic}, VAR:{\em kernel},
LIMPOINT:{\em algebraic}):{\em algebraic} \\
\noindent \texttt{LIMIT!-}(EXPRN:{\em algebraic}, VAR:{\em kernel},
LIMPOINT:{\em algebraic}):{\em algebraic}
\vspace{.1in}

If the limit depends upon the direction of approach to the \texttt{LIMPOINT},
the functions \texttt{LIMIT!+} and \texttt{LIMIT!-} may be used.  They are
defined by:
\begin{quote}
\begin{tabular}{l}
 \texttt{LIMIT!+ (LIMIT!-)} (EXP,VAR,LIMPOINT) $\rightarrow$\texttt{LIMIT}(EXP*,$\epsilon$,0), \\
  \qquad EXP*=sub(VAR=VAR+(-)$\epsilon^2$,EXP)
\end{tabular}
\end{quote}

\subsection{Diagnostic Functions}

\ttindex{LIMIT0}
\hypertarget{operator:LIMIT0}{}
\vspace{.1in}
\noindent \texttt{LIMIT0}(EXPRN:{\em algebraic}, VAR:{\em kernel},
LIMPOINT:{\em algebraic}):{\em algebraic}
\vspace{.1in}

This function will use all parts of the limits package, but it does not
combine log terms before taking limits, so it may fail if there is a sum
of log terms which have a removable singularity in some of the terms.

\ttindex{LIMIT1}
\vspace{.1in}
\noindent \texttt{LIMIT1}(EXPRN:{\em algebraic}, VAR:{\em kernel},
LIMPOINT:{\em algebraic}):{\em algebraic}
\vspace{.1in}

\index{TPS package}
This function uses the TPS branch only, and will fail if the limit point is
singular.

\ttindex{LIMIT2}
\hypertarget{operator:LIMIT2}{}
\begin{quote}
\begin{tabular}{l@{}l}
\texttt{LIMIT2(} & TOP:{\em algebraic}, \\
&BOT:{\em algebraic}, \\
&VAR:{\em kernel}, \\
&LIMPOINT:{\em algebraic}):{\em algebraic}
\end{tabular}
\end{quote}

This function applies l'H\^opital's rule to the quotient (TOP/BOT).



\section{LINALG: Linear algebra package} \ttindex{LINALG}
\label{LINALG}

This package provides a selection of functions that are useful 
in the world of linear algebra.

Author: Matt Rebbeck.


\section{LPDO: Linear Partial Differential Operators}
\ttindex{LPDO}\label{LPDO}

Author: Thomas Sturm\

\section{MODSR: Modular solve and roots} \ttindex{MODSR}

This package supports solve (M\_SOLVE) and roots (M\_ROOTS) operators for
modular polynomials and modular polynomial systems.  The moduli need not
be primes. M\_SOLVE requires a modulus to be set.  M\_ROOTS takes the
modulus as a second argument. For example:

\begin{verbatim}
on modular; setmod 8;
m_solve(2x=4);            ->  {{X=2},{X=6}}
m_solve({x^2-y^3=3});
    ->  {{X=0,Y=5}, {X=2,Y=1}, {X=4,Y=5}, {X=6,Y=1}}
m_solve({x=2,x^2-y^3=3}); ->  {{X=2,Y=1}}
off modular;
m_roots(x^2-1,8);         ->  {1,3,5,7}
m_roots(x^3-x,7);         ->  {0,1,6}
\end{verbatim}

There is no further documentation for this package.

Author: Herbert Melenk.

\section{NCPOLY: Non--commutative polynomial ideals}
\ttindex{NCPOLY}

This package allows the user to set up automatically a consistent
environment for computing in an algebra where the non--commutativity is
defined by Lie-bracket commutators.  The package uses the {REDUCE} {\bf
noncom} mechanism for elementary polynomial arithmetic; the commutator
rules are automatically computed from the Lie brackets.

Authors: Herbert Melenk and Joachim Apel.

\section{NORMFORM: Computation of matrix normal forms} \ttindex{NORMFORM}
\label{NORMFORM}

This package contains routines for computing the following
normal forms of matrices:
\begin{itemize}
\item smithex\_int
\item smithex
\item frobenius
\item ratjordan
\item jordansymbolic
\item jordan.
\end{itemize}

Author: Matt Rebbeck.

\section{NUMERIC: Solving numerical problems}
\ttindex{NUM\_SOLVE}\index{Newton's method}\ttindex{NUM\_ODESOLVE}
\ttindex{BOUNDS}\index{Chebyshev fit}
\ttindex{NUM\_MIN}\index{Minimum}\ttindex{NUM\_INT}\index{Quadrature}
This package implements basic algorithms of numerical analysis.
These include:
\begin{itemize}
\item solution of algebraic equations by Newton's method
\begin{verbatim}
    num_solve({sin x=cos y, x + y = 1},{x=1,y=2})
\end{verbatim}
\item solution of ordinary differential equations
\begin{verbatim}
    num_odesolve(df(y,x)=y,y=1,x=(0 .. 1), iterations=5)
\end{verbatim}
\item bounds of a function over an interval
\begin{verbatim}
    bounds(sin x+x,x=(1 .. 2));
\end{verbatim}
\item minimizing a function (Fletcher Reeves steepest descent)
\begin{verbatim}
    num_min(sin(x)+x/5, x);
\end{verbatim}
\item Chebyshev curve fitting
\begin{verbatim}
    chebyshev_fit(sin x/x,x=(1 .. 3),5);
\end{verbatim}
\item numerical quadrature
\begin{verbatim}
    num_int(sin x,x=(0 .. pi));
\end{verbatim}
\end{itemize}

Author: Herbert Melenk.

\section[ODESOLVE: Ordinary differential equations solver]%
        {ODESOLVE: \protect\\ Ordinary differential equations solver}
\ttindex{ODESOLVE}

The ODESOLVE package is a solver for ordinary differential equations.  At
the present time it has very limited capabilities.  It can handle only a
single scalar equation presented as an algebraic expression or equation,
and it can solve only first-order equations of simple types, linear
equations with constant coefficients and Euler equations.  These solvable
types are exactly those for which Lie symmetry techniques give no useful
information.  For example, the evaluation of
\begin{verbatim}
        depend(y,x);
        odesolve(df(y,x)=x**2+e**x,y,x);
\end{verbatim}
yields the result
\begin{verbatim}
               X                    3
            3*E  + 3*ARBCONST(1) + X
        {Y=---------------------------}
                        3
\end{verbatim}

Main Author: Malcolm A.H. MacCallum.

Other contributors: Francis Wright, Alan Barnes.

\section{ORTHOVEC: Manipulation of scalars and vectors}
\ttindex{ORTHOVEC}

ORTHOVEC is a collection of REDUCE procedures and operations which
provide a simple-to-use environment for the manipulation of scalars and
vectors.  Operations include addition, subtraction, dot and cross
products, division, modulus, div, grad, curl, laplacian, differentiation,
integration, and Taylor expansion.

Author: James W. Eastwood.

\section{PHYSOP: Operator calculus in quantum theory}
\ttindex{PHYSOP}

This package has been designed to meet the requirements of theoretical
physicists looking for a computer algebra tool to perform complicated
calculations in quantum theory with expressions containing operators.
These operations consist mainly of the calculation of commutators between
operator expressions and in the evaluations of operator matrix elements in
some abstract space.

Author: Mathias Warns.

\section{PM: A REDUCE pattern matcher} \ttindex{PM}

PM is a general pattern matcher similar in style to those found in systems
such as SMP and Mathematica, and is based on the pattern matcher described
in Kevin McIsaac, ``Pattern Matching Algebraic Identities'', SIGSAM Bulletin,
19 (1985), 4-13.

Documentation for this package is in plain text.

Author: Kevin McIsaac.

\section{RANDPOLY: A random polynomial generator} \ttindex{RANDPOLY}

This package is based on a port of the Maple random polynomial
generator together with some support facilities for the generation
of random numbers and anonymous procedures.

Author: Francis J. Wright.

\section{REACTEQN: Support for chemical reaction equation systems}
\ttindex{REACTEQN}

This package allows a user to transform chemical reaction systems into
ordinary differential equation systems (ODE) corresponding to the laws of
pure mass action.

Documentation for this package is in plain text.

Author: Herbert Melenk.

\documentclass[a4paper]{article}

\usepackage[dvipdfm]{graphicx}
\usepackage[dvipdfm]{color}
\usepackage[dvipdfm]{hyperref}

\usepackage{reduce}


\title{REDUCE Support for Reaction Equation Systems}
\author{Herbert Melenk \\
	Konrad-Zuse-Zentrum f\"ur Informationstechnik Berlin \\
	Takustra{\ss}e 7 \\
	D--14195 Berlin--Dahlem \\
	Germany \\
	e-mail: melenk@zib.de \\
	January 1991}
\date{}

\setlength{\parindent}{0cm}

\begin{document}

\maketitle


The REDUCE package REACTEQN allows one to transform chemical reaction 
systems into ordinary differential equation systems (ode) 
corresponding to the laws of pure mass action.  \\
A single reaction equation is an expression of the form

 \meta{n1}\meta{s1} + \meta{n2}\meta{s2} + \ldots $->$ \meta{n3}\meta{s3} + \meta{n4}\meta{s4} + \ldots

 or

 \meta{n1}\meta{s1} + \meta{n2}\meta{s2} + \ldots \meta{} \meta{n3}\meta{s3} + \meta{n4}\meta{s4} + \ldots

where the \meta{si} are arbitrary names of species (REDUCE symbols)
and the \meta{ni} are positive integer numbers. The number 1
can be omitted. The connector $->$ describes a one way reaction,
while \meta{\ } describes a forward and backward reaction. \\
\ \\
A reaction system is a list of reaction equations, each of them
optionally followed by one or two expressions for the rate
constants. A rate constant can a number, a symbol or an 
arbitrary REDUCE expression. If a rate constant is missing,
an automatic constant of the form RATE(n) (where n is an
integer counter) is generated. For double reactions the
first constant is used for the forward direction, the second
one for the backward direction. \\
\ \\
The names of the species are collected in a list bound to
the REDUCE variable SPECIES. This list is automatically filled
during the processing of a reaction system. The species enter
in an order corresponding to their appearance in the reaction
system and the resulting ode's will be ordered in the same manner.  \\
\ \\
If a list of species is preassigned to the variable
SPECIES either explicitly or from previous operations, the 
given order will be maintained and will dominate the formatting
process. So the ordering of the result can be easily influenced
by the user. \\
\ \\

Syntax:

 reac2ode \{ \meta{reaction} {[},\meta{rate} {[},\meta{rate}{]}{]} 

 {[},\meta{reaction} {[},\meta{rate} {[},\meta{rate}{]}{]}{]} 

 .... 

 \};

where two rates are applicable only for \meta{} reactions. \\
\ \\
 Result is a system of explicit ordinary differential
 equations with polynomial righthand sides. As side
 effect the following variables are set: \\
\ \\
 lists:



 rates: list of the rates in the system



 species: list of the species in the system



 matrices:



 inputmat: matrix of the input coefficients



 outputmat: matrix of the output coefficients

In the matrices the row number corresponds to the input reaction 
number, while the column number corresponds to the species index.
Note: if the rates are numerical values, it will be in most cases
appropriate to select a REDUCE evaluation mode for floating
point numbers. That is \\
\ \\

 REDUCE 3.3: on float,numval;

 REDUCE 3.4: on rounded;

Inputmat and outputmat can be used for linear algebra type 
investigations of the reaction system. The classical reaction 
matrix is the difference of these matrices; however, the two 
matrices contain more information than their differences because 
the appearance of a species on both sides is not reflected by 
the reaction matrix. \\
\ \\
EXAMPLES:

\% Example taken from Feinberg (Chemical Engineering):


 species := \{A1,A2,A3,A4,A5\};


 reac2ode \{ A1 + A4 $<>$ 2A1, rho, beta,

 A1 + A2 $<>$ A3, gamma, epsilon,

 A3 $<>$ A2 + A5, theta, mue\};
 

 2

\{DF(A1,T)=RHO{*}A1{*}A4 - BETA{*}A1 - GAMMA{*}A1{*}A2 + EPSILON{*}A3,



DF(A2,T)= - GAMMA{*}A1{*}A2 + EPSILON{*}A3 + THETA{*}A3 - MUE{*}A2{*}A5,



DF(A3,T)=GAMMA{*}A1{*}A2 - EPSILON{*}A3 - THETA{*}A3 + MUE{*}A2{*}A5,


 2

DF(A4,T)= - RHO{*}A1{*}A4 + BETA{*}A1 ,


DF(A5,T)=THETA{*}A3 - MUE{*}A2{*}A5\}


\% the corresponding matrices:


 inputmat;

\begin{verbatim}

                [ 1 0 0 1 0 ]

                [           ]

                [ 1 1 0 0 0 ]

                [           ]

                [ 0 0 1 0 0 ]

\end{verbatim}

 outputmat;

\begin{verbatim}

                [ 2 0 0 0 0 ]

                [           ] 

                [ 0 0 1 0 0 ] 

                [           ] 

                [ 0 1 0 0 1 ] 

\end{verbatim}


\% computation of the classical reaction matrix as difference

\% of output and input matrix:


 reactmat := outputmat-inputmat;

\begin{verbatim}

             [ 1   0   0   -1  0 ]

             [                   ] 

REACTMAT :=  [ -1  -1  1   0   0 ] 

             [                   ] 

             [ 0   1   -1  0   1 ]

\end{verbatim}

\% Example with automatic generation of rate constants

\% and automatic extraction of species


 species := \{\};

 reac2ode \{ A1 + A4 $<>$ 2A1, 

 A1 + A2 $<>$ A3,

 a3 $<>$ A2 + A5\};


new species: A1

new species: A4

new species: A3

new species: A2

new species: A5



 2

\{DF(A1,T)= - A1 {*}RATE(2) + A1{*}A4{*}RATE(1) - A1{*}A2{*}RATE(3) + 


 A3{*}RATE(4),


 2

DF(A4,T)=A1 {*}RATE(2) - A1{*}A4{*}RATE(1),


DF(A2,T)= - A1{*}A2{*}RATE(3) - A2{*}A5{*}RATE(6) + A3{*}RATE(5) + A3{*}RATE(4),


DF(A3,T)=A1{*}A2{*}RATE(3) + A2{*}A5{*}RATE(6) - A3{*}RATE(5) - A3{*}RATE(4),


DF(A5,T)= - A2{*}A5{*}RATE(6) + A3{*}RATE(5)\}



\% Example with rates computed from numerical expressions


 species := \{\};

 reac2ode \{ A1 + A4 $<>$ 2A1, 17.3{*} 22.4\^{}1.5,

 0.04{*} 22.4\^{}1.5 \};


new species: A1

new species: A4


 2

\{DF(A1,T)= - 4.24065{*}A1 + 1834.08{*}A1{*}A4,


 2

DF(A4,T)=4.24065{*}A1 - 1834.08{*}A1{*}A4\}

\end{document}



\section{REDLOG: Extend \REDUCE{} to a computer logic system}
\ttindex{REDLOG}

The name REDLOG stand for REDuce LOGic system. Redlog implements
symbolic algorithms on first-order formulas with respect to
user-chosen first-order languages and theories. The available domains
include real numbers, integers, complex numbers, p-adic numbers,
quantified propositional calculus, term algebras.

Documentation for this package can be found \href{http://redlog.eu/}{online}.

Authors: Andreas Dolzmann and Thomas Sturm

\section{RESET: Code to reset REDUCE to its initial state} \ttindex{RESET}

This package defines a command RESETREDUCE that works through the
history of previous commands, and clears any values which have been
assigned, plus any rules, arrays and the like.  It also sets the various
switches to their initial values.  It is not complete, but does work for
most things that cause a gradual loss of space.  It would be relatively
easy to make it interactive, so allowing for selective resetting.

There is no further documentation on this package.

Author: John Fitch.

\section{RESIDUE: A residue package} \ttindex{RESIDUE}

This package supports the calculation of residues of arbitrary
expressions.

Author: Wolfram Koepf.

\section{RLFI: REDUCE LaTeX formula interface} \ttindex{RLFI}

This package adds \LaTeX{} syntax to REDUCE.  Text generated by REDUCE in
this mode can be directly used in \LaTeX{} source documents.  Various
mathematical constructions are supported by the interface including
subscripts, superscripts, font changing, Greek letters, divide-bars,
integral and sum signs, derivatives, and so on.

Author: Richard Liska.

\section{ROOTS: A REDUCE root finding package} \ttindex{ROOTS}

This root finding package can be used to find some or all of the roots of a
univariate polynomial with real or complex coefficients, to the accuracy
specified by the user.

It is designed so that it can be used as an independent package, or it may
be called from {\tt SOLVE} if {\tt ROUNDED} is on. For example,
the evaluation of
\begin{verbatim}
        on rounded,complex;
        solve(x**3+x+5,x);
\end{verbatim}
yields the result
\begin{verbatim}
    {X= - 1.51598,X=0.75799 + 1.65035*I,X=0.75799 - 1.65035*I}
\end{verbatim}

This package loads automatically.

Author: Stanley L. Kameny.

\section[RSOLVE: Rational/integer polynomial solvers]%
        {RSOLVE: \protect\\ Rational/integer polynomial solvers}
\ttindex{RSOLVE}

This package provides operators that compute the exact rational zeros
of a single univariate polynomial using fast modular methods.  The
algorithm used is that described by R. Loos (1983): Computing rational
zeros of integral polynomials by $p$-adic expansion, \textit{SIAM J.
Computing}, \textbf{12}, 286--293.

Author: Francis J. Wright.

\section{SCOPE: REDUCE source code optimization package} \ttindex{SCOPE}
\label{SCOPE}

SCOPE is a package for the production of an optimized form of a set of
expressions.  It applies an heuristic search for common (sub)expressions
to almost any set of proper REDUCE assignment statements.  The
output is obtained as a sequence of assignment statements.  GENTRAN is
used to facilitate expression output.

Author:  J.A. van Hulzen.

\section{SETS: A basic set theory package} \ttindex{SETS}

The SETS package provides algebraic-mode support for set operations on
lists regarded as sets (or representing explicit sets) and on implicit
sets represented by identifiers.

Author: Francis J. Wright.

\section{SPDE: Finding symmetry groups of {PDE}'s}
\ttindex{SPDE}

The package SPDE provides a set of functions which may be used to
determine the symmetry group of Lie- or point-symmetries of a given system
of partial differential equations. In many cases the determining system is
solved completely automatically. In other cases the user has to provide
additional input information for the solution algorithm to terminate.

Author: Fritz Schwarz.

\section{SPECFN: Package for special functions} \ttindex{SPECFN}

\index{Gamma function}       \ttindex{Gamma}
\index{Digamma function}     \ttindex{Digamma}
\index{Polygamma functions}  \ttindex{Polygamma}
\index{Pochhammer's symbol}  \ttindex{Pochhammer}
\index{Euler numbers}        \ttindex{Euler}
\index{Bernoulli numbers}    \ttindex{Bernoulli}
\index{Zeta function (Riemann's)}  \ttindex{Zeta}
\index{Bessel functions} \ttindex{BesselJ} \ttindex{BesselY}
                         \ttindex{BesselK} \ttindex{BesselI}
\index{Hankel functions} \ttindex{Hankel1} \ttindex{Hankel2}
\index{Kummer functions} \ttindex{KummerM} \ttindex{KummerU}
\index{Struve functions} \ttindex{StruveH} \ttindex{StruveL}
\index{Lommel functions} \ttindex{Lommel1} \ttindex{Lommel2}
\index{Polygamma functions} \ttindex{Polygamma}
\index{Beta function}       \ttindex{Beta}
\index{Whittaker functions} \ttindex{WhittakerM}
                            \ttindex{WhittakerW}
\index{Dilogarithm function}   \ttindex{Dilog}
\index{Psi function}           \ttindex{Psi}
\index{Orthogonal polynomials} 
\index{Hermite polynomials}    \ttindex{HermiteP}
\index{Jacobi's polynomials}   \ttindex{JacobiP}
\index{Legendre polynomials}   \ttindex{LegendreP}
\index{Laguerre polynomials}   \ttindex{LaguerreP}
\index{Chebyshev polynomials}  \ttindex{ChebyshevT} \ttindex{ChebyshevU}
\index{Gegenbauer polynomials} \ttindex{GegenbauerP}
\index{Euler polynomials}      \ttindex{EulerP}
\index{Binomial coefficients}  \ttindex{Binomial}
\index{Stirling numbers} \ttindex{Stirling1} \ttindex{Stirling2}
\index{Spherical and Solid Harmonics} \ttindex{SphericalHarmonicY}
\ttindex{SolidHarmonicY}
\index{Jacobi Elliptic Functions and Integrals}
\ttindex{Jacobiamplitude} \ttindex{Jacobisn} \ttindex{Jacobidn}
\ttindex{Jacobicn} \ttindex{EllipticF} \ttindex{EllipticE}
\ttindex{EllipticTheta} \ttindex{JacobiZeta}
\index{Airy functions} \ttindex{Airy\_Ai} \ttindex{Airy\_Bi}
\ttindex{Airy\_Aiprime} \ttindex{Airy\_Biprime}
\index{3j and 6j symbols} \index{Clebsch Gordan coefficients}
\ttindex{ThreejSymbol} \ttindex{SixjSymbol} \ttindex{Clebsch\_Gordan}

This special function package is separated into two portions to make it
easier to handle.  The packages are called SPECFN and SPECFN2.  The first
one is more general in nature, whereas the second is devoted to special
special functions.  Documentation for the first package can be found in
the file specfn.tex in the ``doc'' directory, and examples in specfn.tst
and specfmor.tst in the examples directory.

The package SPECFN is designed to provide algebraic and numerical
manipulations of several common special functions, namely:

\begin{itemize}
\item Bernoulli Numbers and Euler Numbers;
\item Stirling Numbers;
\item Binomial Coefficients;
\item Pochhammer notation;
\item The Gamma function;
\item The Psi function and its derivatives;
\item The Riemann Zeta function;
\item The Bessel functions J and Y of the first and second kind;
\item The modified Bessel functions I and K;
\item The Hankel functions H1 and H2;
\item The Kummer hypergeometric functions M and U;
\item The Beta function, and Struve, Lommel and Whittaker functions;
\item The Airy functions;
\item The Exponential Integral, the Sine and Cosine Integrals;
\item The Hyperbolic Sine and Cosine Integrals;
\item The Fresnel Integrals and the Error function;
\item The Dilog function;
\item Hermite Polynomials;
\item Jacobi Polynomials;
\item Legendre Polynomials;
\item Spherical and Solid Harmonics;
\item Laguerre Polynomials;
\item Chebyshev Polynomials;
\item Gegenbauer Polynomials;
\item Euler  Polynomials;
\item Bernoulli Polynomials.
\item Jacobi Elliptic Functions and Integrals;
\item 3j symbols, 6j symbols and Clebsch Gordan coefficients;
\end{itemize}

Author:  Chris Cannam, with contributions from Winfried Neun, Herbert
Melenk, Victor Adamchik, Francis Wright and several others.

\section{SPECFN2: Package for special special functions} \ttindex{SPECFN2}

\index{Generalized Hypergeometric functions} 
\index{Meijer's G function}

This package provides algebraic manipulations of generalized
hypergeometric functions and Meijer's G function.  Generalized
hypergeometric functions are simplified towards special functions and
Meijer's G function is simplified towards special functions or generalized
hypergeometric functions.

Author: Victor Adamchik, with major updates by Winfried Neun.


The (generalised) hypergeometric functions  
\begin{displaymath}
_pF_q \left( {{a_1, \ldots , a_p} \atop {b_1, \ldots ,b_q}} \Bigg\vert z \right)
\end{displaymath}
are defined in textbooks on special functions as
\begin{displaymath}
_pF_q \left( {{a_1, \ldots , a_p} \atop {b_1, \ldots ,b_q}} \Bigg\vert z \right)
  = \sum_{n=0}^{\infty}\frac{(a_{1})_{n}\dots(a_{p})_{n}}{(b_{1})_{n}\dots(b_{q})_{n}}\frac{z^{n}}{n!}
\end{displaymath}w
where $(a)_{n}$ is the Pochhammer symbol
\begin{displaymath}
 (a)_{n} = \prod_{k=0}^{n-1} (a+k)
\end{displaymath}

The function 
\begin{displaymath}
G_{p q}^{m n} \left( z \  \Bigg\vert \  {(a_p) \atop (b_q)} \right)
\end{displaymath}
has been studied by C.~S.~Meijer beginning in 1936 and has been
called Meijer's G function later on. The complete definition of Meijer's
G function can be found in \cite{Prudnikov:90}.
Many well-known functions can be written as G functions,
e.g. exponentials, logarithms, trigonometric functions, Bessel functions
and hypergeometric functions.

Several hundreds of particular values can be found in \cite{Prudnikov:90}.


\subsection{\REDUCE{} operator HYPERGEOMETRIC}
\hypertarget{operator:HYPERGEOMETRIC}{}

The operator {\tt hypergeometric} expects 3 arguments, namely the 
list of upper parameters (which may be empty), the list of lower
parameters (which may be empty too), and the argument, e.g the input:
\begin{verbatim}
hypergeometric ({},{},z);
\end{verbatim}
yields the output
\begin{verbatim}
 z
e
\end{verbatim}
and the input
\begin{verbatim}
hypergeometric ({1/2,1},{3/2},-x^2);
\end{verbatim}
gives
\begin{verbatim}
 atan(abs(x))
--------------
    abs(x)
\end{verbatim}


\subsection{Extending the HYPERGEOMETRIC operator}

Since hundreds of particular cases for the generalised hypergeometric
functions can be found in the literature, one cannot expect that all
cases are known to the \texttt{hypergeometric} operator.
Nevertheless the set of special cases can be augmented by adding
rules to the \REDUCE{} system, {\em e.g.}
\begin{verbatim}
let {hypergeometric({1/2,1/2},{3/2},-(~x)^2) => asinh(x)/x};
\end{verbatim}


\section{\REDUCE{} operator {\tt meijerg}}
\hypertarget{operator:MEIJERG}{}

The operator \texttt{meijerg} expects 3 arguments, namely the 
list of upper parameters (which may be empty), the list of lower
parameters (which may be empty too), and the argument.

The first element of the lists has to be the list of the
first n or m respective parameters, e.g. to describe 
\begin{displaymath}
G_{1 1}^{1 0} \left( x \  \Bigg\vert \  {1 \atop 0} \right)
\end{displaymath}
one has to write 
\begin{verbatim}

MeijerG({{},1},{{0}},x); % and the result is:

 sign( - x + 1) + sign(x + 1)
------------------------------
              2

\end{verbatim}
and for
\begin{displaymath}
G_{0 2}^{1 0} \left( \frac{x^2}{4} \  \Bigg\vert \ {\atop  {1+ \frac{1}{4} },
{1-\frac{1}{4}}} \right)
\end{displaymath}
\begin{verbatim}

MeijerG({{}},{{1+1/4},1-1/4},(x^2)/4) * sqrt pi;


                    2                    2
 sqrt(pi)*sqrt(-----------)*sin(abs(x))*x
                abs(x)*pi
-------------------------------------------
                     4


\end{verbatim}




\section{SUM: A package for series summation} \ttindex{SUM}
\hypertarget{operator:SUM}{}
\hypertarget{operator:PROD}{}

This package implements the Gosper algorithm for the summation of series.
It defines operators {\tt SUM} and {\tt PROD}.  The operator {\tt SUM}
returns the indefinite or definite summation of a given expression, and
{\tt PROD} returns the product of the given expression.

This package loads automatically.

Author: Fujio Kako.

\section{SYMMETRY: Operations on symmetric matrices} \ttindex{SYMMETRY}

This package computes symmetry-adapted bases and block diagonal forms of
matrices which have the symmetry of a group.  The package is the
implementation of the theory of linear representations for small finite
groups such as the dihedral groups.

Author: Karin Gatermann.

\section{TAYLOR: Manipulation of Taylor series}
\ttindex{TAYLOR}
\index{Taylor Series} \index{TAYLOR package}
\index{Laurent series} \index{Puiseux series}

This package carries out the Taylor expansion of an expression in one or
more variables and efficient manipulation of the resulting Taylor series.
Capabilities include basic operations (addition, subtraction,
multiplication and division) and also application of certain algebraic and
transcendental functions.

Author: Rainer Sch\"opf.


\subsection{Basic Use}

The most important operator is `\verb+TAYLOR+'. \ttindextype{TAYLOR}{operator}
It is used as follows:
\hypertarget{operator:TAYLOR}{}
\begin{verbatim}
  TAYLOR(EXP:algebraic,
         VAR:kernel,VAR0:algebraic,ORDER:integer[,...])
         :algebraic.
\end{verbatim}
where \f{EXP} is the expression to be expanded. It can be any \REDUCE{}
object, even an expression containing other Taylor kernels. \f{VAR} is
the kernel with respect to which \f{EXP} is to be expanded. \f{VAR0}
denotes the point about which and \f{ORDER} the order up to which
expansion is to take place. If more than one \f{(VAR, VAR0, ORDER)} triple
is specified \f{TAYLOR} will expand its first argument independently
with respect to each variable in turn. For example,
\begin{verbatim}
  taylor(e^(x^2+y^2),x,0,2,y,0,2);
\end{verbatim}
will calculate the Taylor expansion up to order $X^{2}*Y^{2}$:
\begin{verbatim}
       2    2    2  2      3  3
  1 + y  + x  + y *x  + O(x ,y )
\end{verbatim}
Note that once the expansion has been done it is not possible to
calculate higher orders.
Instead of a kernel, \f{VAR} may also
be a list of kernels. In this case expansion will take place in a way
so that the \emph{sum} of the degrees of the kernels does not exceed
\f{ORDER}.
If \f{VAR0} evaluates to the special identifier \f{INFINITY}, expansion is
done in a series in 1/VAR instead of \f{VAR}.

The expansion is performed variable per variable, i.e.\ in the example
above by first expanding $\exp(x^{2}+y^{2})$ with respect to $x$ and
then expanding every coefficient with respect to $y$.

\ttindextype{IMPLICIT\_TAYLOR}{operator}\ttindex{IMPLICIT\_TAYLOR}
\hypertarget{operator:IMPLICIT_TAYLOR}{}
There are two
extra operators to compute the Taylor expansions of implicit and
inverse functions:
\begin{verbatim}
  IMPLICIT_TAYLOR(F:algebraic,
                  VAR:kernel,DEPVAR:kernel,
                  VAR0:algebraic,DEPVAR0:algebraic,
                  ORDER:integer)
           :algebraic
\end{verbatim}
takes a function F depending on two variables VAR and DEPVAR and
computes the Taylor series of the implicit function DEPVAR(VAR)
given by the equation F(VAR,DEPVAR) = 0, around the point VAR0.  
(Violation of the necessary condition F(VAR0,DEPVAR0)=0 causes an error.)
For example,
\begin{verbatim}
  implicit_taylor(x^2 + y^2 - 1,x,y,0,1,5);
\end{verbatim}
gives the output
\begin{verbatim}
       1   2    1   4      6
  1 - ---*x  - ---*x  + O(x )
       2        8
\end{verbatim}

\hypertarget{operator:INVERSE_TAYLOR}{}
The operator
\begin{verbatim}
  INVERSE_TAYLOR(F:algebraic,VAR:kernel,DEPVAR:kernel,
                 VAR0:algebraic,ORDER:integer)
         : algebraic
\end{verbatim}
takes a function F depending on VAR1 and computes the Taylor series of
the inverse of F with respect to VAR2. For example,
\begin{verbatim}
  inverse_taylor(exp(x)-1,x,y,0,8);
\end{verbatim}
yields
\begin{verbatim}
       1   2    1   3    1   4    1   5                  9
  y - ---*y  + ---*y  - ---*y  + ---*y  + (3 terms) + O(y )
       2        3        4        5
\end{verbatim}


\ttindextype{TAYLORPRINTTERMS}{variable}\hypertarget{reserved:TAYLORPRINTTERMS}{}
When a Taylor kernel is printed, only a certain number of (non-zero)
coefficients are shown. If there are more, an expression of the form
\f{($n$ terms)} is printed to indicate how many non-zero
terms have been suppressed. The number of terms printed is given by
the value of the shared algebraic variable \f{TAYLORPRINTTERMS}.
Allowed values are integers and the special identifier \f{ALL}. The
latter setting specifies that all terms are to be printed. The default
setting is $5$.

\ttindextype{PART}{operator}\ttindex{PART}
The \f{PART} operator can be used to extract subexpressions of a
Taylor expansion in the usual way. All terms can be accessed,
irregardless of the value of the variable \f{TAYLORPRINTTERMS}.


\ttindextype{TAYLORKEEPORIGINAL}{switch}
If the switch \hyperlink{switch:TAYLORKEEPORIGINAL}{\f{TAYLORKEEPORIGINAL}}
is set to \f{ON} the
original expression EXP is kept for later reference.
It can be recovered by means of the operator

\hypertarget{operator:TAYLORORIGINAL}{}
\hspace*{2em} \texttt{TAYLORORIGINAL}(EXP:{\em exprn}):{\em exprn}

An error is signalled if EXP is not a Taylor kernel or if the original
expression was not kept, i.e.\ if \f{TAYLORKEEPORIGINAL} was
\f{OFF} during expansion.  The template of a Taylor kernel, i.e.\
the list of all variables with respect to which expansion took place
together with expansion point and order can be extracted using
\ttindex{TAYLORTEMPLATE}.

\hypertarget{operator:TAYLORTEMPLATE}{}
\hspace*{2em} \texttt{TAYLORTEMPLATE}(EXP:{\em exprn}):{\em list}

This returns a list of lists with the three elements (VAR,VAR0,ORDER).
As with \f{TAYLORORIGINAL},
an error is signalled if EXP is not a Taylor kernel.

The operator
\hypertarget{operator:TAYLORTOSTANDARD}{}\\
\hspace*{2em} \texttt{TAYLORTOSTANDARD}(EXP:{\em exprn}):{\em exprn}

converts all Taylor kernels in EXP into standard form and
\ttindex{TAYLORTOSTANDARD} resimplifies the result.

The boolean operator
\hypertarget{operator:TAYLORSERIESP}{}\\
\hspace*{2em} \texttt{TAYLORSERIESP}(EXP:{\em exprn}):{\em boolean}

may be used to determine if EXP is a Taylor kernel.
\ttindex{TAYLORSERIESP} (Note that this operator is subject to the same
restrictions as, e.g., \f{ORDP} or \f{NUMBERP}, i.e.\ it may only be used in
boolean expressions in \f{IF} or \f{LET} statements. 

Finally there is

\hypertarget{operator:TAYLORCOMBINE}{}
\hspace*{2em} \texttt{TAYLORCOMBINE}(EXP:{\em exprn}):{\em exprn}

which tries to combine all Taylor kernels found in EXP into one.
\ttindex{TAYLORCOMBINE}
Operations currently possible are:
\index{Taylor series!arithmetic}
\begin{itemize}
  \item Addition, subtraction, multiplication, and division.
  \item Roots, exponentials, and logarithms.
  \item Trigonometric and hyperbolic functions and their inverses.
\end{itemize}
Application of unary operators like \f{LOG} and \f{ATAN} will
nearly always succeed. For binary operations their arguments have to be
Taylor kernels with the same template. This means that the expansion
variable and the expansion point must match. Expansion order is not so
important, different order usually means that one of them is truncated
before doing the operation.

\ttindex{TAYLORKEEPORIGINAL} \ttindex{TAYLORCOMBINE}
If \hyperlink{switch:TAYLORKEEPORIGINAL}{\f{TAYLORKEEPORIGINAL}} is set to \f{ON} and if all Taylor
kernels in \f{exp} have their original expressions kept
\hyperlink{operator:TAYLORCOMBINE}{\f{TAYLORCOMBINE}} will also combine these and store the result
as the original expression of the resulting Taylor kernel.
\ttindextype{TAYLORAUTOEXPAND}{switch}
There is also the switch \hyperlink{switch:TAYLORAUTOEXPAND}{\f{TAYLORAUTOEXPAND}} (see below).

There are a few restrictions to avoid mathematically undefined
expressions: it is not possible to take the logarithm of a Taylor
kernel which has no terms (i.e. is zero), or to divide by such a
beast.  There are some provisions made to detect singularities during
expansion: poles that arise because the denominator has zeros at the
expansion point are detected and properly treated, i.e.\ the Taylor
kernel will start with a negative power.  (This is accomplished by
expanding numerator and denominator separately and combining the
results.)  Essential singularities of the known functions (see above)
are handled correctly.

\index{Taylor series!differentiation}
Differentiation of a Taylor expression is possible.  If you
differentiate with respect to one of the Taylor variables the order
will decrease by one.

\index{Taylor series!substitution}
Substitution is a bit restricted: Taylor variables can only be replaced
by other kernels.  There is one exception to this rule: you can always
substitute a Taylor variable by an expression that evaluates to a
constant.  Note that \REDUCE{} will not always be able to determine
that an expression is constant.

\index{Taylor series!integration}
Only simple taylor kernels can be integrated. More complicated
expressions that contain Taylor kernels as parts of themselves are
automatically converted into a standard representation by means of the
\hyperlink{operator:TAYLORTOSTANDARD}{\f{TAYLORTOSTANDARD}} operator. 
In this case a suitable warning is printed.

\index{Taylor series!reversion} It is possible to revert a Taylor
series of a function $f$, i.e., to compute the first terms of the
expansion of the inverse of $f$ from the expansion of $f$. This is
done by the operator

\hypertarget{operator:TAYLORREVERT}{}
\hspace*{2em} \texttt{TAYLORREVERT}(EXP:{\em exprn},OLDVAR:{\em kernel},
                                 NEWVAR:{\em kernel}):{\em exprn}

EXP must evaluate to a Taylor kernel with OLDVAR being one of its
expansion variables. Example:
\begin{verbatim}
  taylor (u - u**2, u, 0, 5)$
  taylorrevert (ws, u, x);
\end{verbatim}
gives
\begin{verbatim}
       2      3      4       5      6
  x + x  + 2*x  + 5*x  + 14*x  + O(x )
\end{verbatim}

This package introduces a number of new switches:
\begin{description}

\ttindextype{TAYLORAUTOCOMBINE}{switch}
\item[\sw{TAYLORAUTOCOMBINE}] \hypertarget{switch:TAYLORAUTOCOMBINE}{}causes
    Taylor expressions to be automatically combined during the simplification
    process.  This is equivalent to applying \f{TAYLORCOMBINE} to
    every expression that contains Taylor kernels.
    Default is \f{ON}.

\ttindextype{TAYLORAUTOEXPAND}{switch}
\item[\sw{TAYLORAUTOEXPAND}] \hypertarget{switch:TAYLORAUTOEXPAND}{} makes Taylor expressions ``contagious''
    in the sense that \f{TAYLORCOMBINE} tries to Taylor expand
    all non-Taylor subexpressions and to combine the result with the
    rest. Default is \f{OFF}.

\ttindextype{TAYLORKEEPORIGINAL}{switch}\hypertarget{switch:TAYLORKEEPORIGINAL}{}
\item[\sw{TAYLORKEEPORIGINAL}] forces the
    package to keep the original expression, i.e.\ the expression
    that was Taylor expanded.  All operations performed on the
    Taylor kernels are also applied to this expression  which can
    be recovered using the operator \f{TAYLORORIGINAL}.
    Default is \f{OFF}.

\ttindextype{TAYLORPRINTORDER}{switch}\hypertarget{switch:TAYLORPRINTORDER}{}
\item[\sw{TAYLORPRINTORDER}] causes the
    remainder to be printed in big-$O$ notation.  Otherwise, three
    dots are printed. Default is \f{ON}.

\ttindextype{VERBOSELOAD}{switch}
\item[\sw{VERBOSELOAD}] will cause
    \REDUCE{} to print some information when the Taylor package is
    loaded.  This switch is already present in \textsf{PSL} systems.
    Default is \f{OFF}.

\end{description}
\index{Defaults! TAYLOR package}

\subsection{Caveats}
\index{Caveats!TAYLOR package}

\f{TAYLOR} should always detect non-analytical expressions in
its first argument. As an example, consider the function $xy/(x+y)$
that is not analytical in the neighborhood of $(x,y) = (0,0)$: Trying
to calculate
\begin{verbatim}
  taylor(x*y/(x+y),x,0,2,y,0,2);
\end{verbatim}
causes an error
\begin{verbatim}
***** Not a unit in argument to QUOTTAYLOR
\end{verbatim}
Note that it is not generally possible to apply the standard \REDUCE{}
operators to a Taylor kernel. For example, \f{PART}, \f{COEFF},
or \f{COEFFN} cannot be used. Instead, the expression at hand has
to be converted to standard form first using the \f{TAYLORTOSTANDARD}
operator.

\subsection{Warning messages}
\index{Warnings!TAYLOR package}
\begin{description}

\item[\msg{*** Cannot expand further... truncation done}]\mbox{}\\
    You will get this warning if you try to expand a Taylor kernel to
    a higher order.

\item[\msg{*** Converting Taylor kernels to standard representation}]\mbox{}\\
    This warning appears if you try to integrate an expression
    containing Taylor kernels.

\end{description}

\subsection{Error messages}
\index{Errors!TAYLOR package}
\begin{description}

\item[\msg{***** Branch point detected in ...}]\mbox{}\\
    This occurs if you take a rational power of a Taylor kernel
    and raising the lowest order term of the kernel to this
    power yields a non analytical term (i.e.\ a fractional power).

\item[\msg{***** Cannot replace part ... in Taylor kernel}]\mbox{}\\
\ttindextype{PART}{Operator}%
    The \f{PART} operator can only be used to either replace the
    template of a Taylor kernel (part 2) or the original expression
    that is kept for reference (part 3).    

\item[\msg{***** Computation loops (recursive definition?): ...}]\mbox{}\\
    Most probably the expression to be expanded contains an operator
    whose derivative involves the operator itself.

\item[\msg{***** Error during expansion (possible singularity)}]\mbox{}\\
    The expression you are trying to expand caused an error.
    As far as I know this can only happen if it contains a function
    with a pole or an essential singularity at the expansion point.
    (But one can never be sure.)

\item[\msg{***** Essential singularity in ...}]\mbox{}\\
    An essential singularity was detected while applying a
    special function to a Taylor kernel.

\item[\msg{***** Expansion point lies on branch cut in ...}]\mbox{}\\
    The only functions with branch cuts this package knows of are
    (natural) logarithm, inverse circular and hyperbolic tangent and
    cotangent.  The branch cut of the logarithm is assumed to lie on
    the negative real axis.  Those of the arc tangent and arc
    cotangent functions are chosen to be compatible with this: both
    have essential singularities at the points $\pm i$.  The branch
    cut of arc tangent is the straight line along the imaginary axis
    connecting $+1$ to $-1$ going through $\infty$ whereas that of arc
    cotangent goes through the origin.  Consequently, the branch cut
    of the inverse hyperbolic tangent resp.\ cotangent lies on the
    real axis and goes from $-1$ to $+1$, that of the latter across
    $0$, the other across $\infty$.

    The error message can currently only appear when you try to
    calculate the inverse tangent or cotangent of a Taylor
    kernel that starts with a negative degree.
    The case of a logarithm of a Taylor kernel whose constant term
    is a negative real number is not caught since it is
    difficult to detect this in general.

\item[\msg{***** Input expression non-zero at given point}]\mbox{}\\
    Violation of the necessary condition F(VAR0,DEPVAR0)=0 for the arguments of
    \f{IMPLICIT\_TAYLOR}.

\item[\msg{***** Invalid substitution in Taylor kernel: ...}]\mbox{}\\
    You tried to substitute a variable that is already present in the
    Taylor kernel or on which one of the Taylor variables depend.

\item[\msg{***** Not a unit in ...}]\mbox{}\\
    This will happen if you try to divide by or take the logarithm of
    a Taylor series whose constant term vanishes.

\item[\msg{***** Not implemented yet (...)}]\mbox{}\\
    Sorry, but I haven't had the time to implement this feature.
    Tell me if you really need it, maybe I have already an improved
    version of the package.

\item[\msg{***** Reversion of Taylor series not possible: ...}]\mbox{}\\
\ttindex{TAYLORREVERT}
    You tried to call the \f{TAYLORREVERT} operator with
    inappropriate arguments. The second half of this error message
    tells you why this operation is not possible.

\item[\msg{***** Taylor kernel doesn't have an original part}]\mbox{}\\
\ttindex{TAYLORORIGINAL} \ttindex{TAYLORKEEPORIGINAL}
    The Taylor kernel upon which you try to use \f{TAYLORORIGINAL}
    was created with the switch \f{TAYLORKEEPORIGINAL}
    set to \f{OFF}
    and does therefore not keep the original expression.

\item[\msg{***** Wrong number of arguments to TAYLOR}]\mbox{}\\
    You try to use the operator \f{TAYLOR} with a wrong number of
    arguments.

\item[\msg{***** Zero divisor in TAYLOREXPAND}]\mbox{}\\
    A zero divisor was found while an expression was being expanded.
    This should not normally occur.

\item[\msg{***** Zero divisor in Taylor substitution}]\mbox{}\\
    That's exactly what the message says.  As an example consider the
    case of a Taylor kernel containing the term \f{1/x} and you try
    to substitute \f{x} by \f{0}.

\item[\msg{***** ... invalid as kernel}]\mbox{}\\
    You tried to expand with respect to an expression that is not a
    kernel.

\item[\msg{***** ... invalid as order of Taylor expansion}]\mbox{}\\
    The order parameter you gave to \f{TAYLOR} is not an integer.

\item[\msg{***** ... invalid as Taylor kernel}]\mbox{}\\
\ttindex{TAYLORORIGINAL} \ttindex{TAYLORTEMPLATE}
    You tried to apply \f{TAYLORORIGINAL} or \f{TAYLORTEMPLATE}
    to an expression that is not a Taylor kernel.

\item[\msg{***** ... invalid as Taylor Template element}]\mbox{}\\
    You tried to substitute the \f{TAYLORTEMPLATE} part of a Taylor
    kernel with a list a incorrect form. For the correct form see the
    description of the \f{TAYLORTEMPLATE} operator.

\item[\msg{***** ... invalid as Taylor variable}]\mbox{}\\
    You tried to substitute a Taylor variable by an expression that is
    not a kernel.

\item[\msg{***** ... invalid as value of TaylorPrintTerms}]\mbox{}\\
\ttindex{TAYLORPRINTTERMS}
    You have assigned an invalid value to \hyperlink{reserved:TAYLORPRINTTERMS}{\f{TAYLORPRINTTERMS}}.
    Allowed values are: an integer or the special identifier
    \f{ALL}.

\item[\msg{TAYLOR PACKAGE (...): this can't happen ...}]\mbox{}\\
    This message shows that an internal inconsistency was detected.
    This is not your fault, at least as long as you did not try to
    work with the internal data structures of \REDUCE. Send input
    and output to me, together with the version information that is
    printed out.

\end{description}

\subsection{Comparison to other packages}

At the moment there is only one \REDUCE{} package that I know of:
the truncated power series package by Alan Barnes and Julian Padget.
In my opinion there are two major differences:
\begin{itemize}
  \item The interface. They use the domain mechanism for their power
        series, I decided to invent a special kind of kernel. Both
        approaches have advantages and disadvantages: with domain
        modes, it is easier
        to do certain things automatically, e.g., conversions.
  \item The concept of a truncated series. Their idea is to remember
        the original expression and to compute more coefficients when
        more of them are needed. My approach is to truncate at a
        certain order and forget how the unexpanded expression
        looked like.  I think that their method is more widely
        usable, whereas mine is more efficient when you know in
        advance exactly how many terms you need.
\end{itemize}



\section{TPS: A truncated power series package} \ttindex{TPS} \ttindex{PS}

This package implements formal Laurent series expansions in one variable
using the domain mechanism of REDUCE.  This means that power series
objects can be added, multiplied, differentiated etc.,  like other first
class objects in the system.  A lazy evaluation scheme is used and thus
terms of the series are not evaluated until they are required for printing
or for use in calculating terms in other power series.  The series are
extendible giving the user the impression that the full infinite series is
being manipulated.  The errors that can sometimes occur using series that
are truncated at some fixed depth (for example when a term in the required
series depends on terms of an intermediate series beyond the truncation
depth) are thus avoided.

Authors:  Alan Barnes and Julian Padget.

\section{TRI: TeX REDUCE interface} \ttindex{TRI}

This package provides facilities written in REDUCE-Lisp for typesetting
REDUCE formulas using \TeX.  The \TeX-REDUCE-Interface incorporates three
levels of \TeX output: without line breaking, with line breaking, and
with line breaking plus indentation.

Author: Werner Antweiler.

\section{TRIGSIMP: Simplification and factorization of trigonometric
and hyperbolic functions} \ttindex{TRIGSIMP}
\label{TRIGSIMP}

TRIGSIMP is a useful tool for all kinds of trigonometric and hyperbolic
simplification and factorization.  There are three procedures included in
TRIGSIMP: trigsimp, trigfactorize and triggcd.  The first is for finding
simplifications of trigonometric or hyperbolic expressions with many
options, the second for factorizing them and the third for finding the
greatest common divisor of two trigonometric or hyperbolic polynomials.

Author: Wolfram Koepf.

\section{WU: Wu algorithm for polynomial systems} \ttindex{WU}

This is a simple implementation of the Wu algorithm implemented in REDUCE
working directly from ``A Zero Structure Theorem for
Polynomial-Equations-Solving,'' Wu Wen-tsun, Institute of Systems Science,
Academia Sinica, Beijing.

Author: Russell Bradford.

\section{XCOLOR: Color factor in some field theories}
\ttindex{XCOLOR}

This package calculates the color factor in non-abelian gauge field
theories using an algorithm due to Cvitanovich.

Documentation for this package is in plain text.

Author: A. Kryukov.

\section{XIDEAL: Gr\"obner Bases for exterior algebra} \ttindex{XIDEAL}

XIDEAL constructs Gr\"obner bases for solving the left ideal membership
problem: Gr\"obner left ideal bases or GLIBs. For graded ideals, where each
form is homogeneous in degree, the distinction between left and right
ideals vanishes. Furthermore, if the generating forms are all homogeneous,
then the Gr\"obner bases for the non-graded and graded ideals are
identical. In this case, XIDEAL is able to save time by truncating the
Gr\"obner basis at some maximum degree if desired.

Author: David Hartley.

\section{ZEILBERG: Indefinite and definite summation}
\ttindex{ZEILBERG}

This package is a careful implementation of the Gosper and Zeilberger
algorithms for indefinite and definite summation of hypergeometric terms,
respectively.  Extensions of these algorithms are also included that are
valid for ratios of products of powers, factorials, $\Gamma$ function
terms, binomial coefficients, and shifted factorials that are
rational-linear in their arguments.

Authors: Gregor St\"olting and Wolfram Koepf.

\section{ZTRANS: \texorpdfstring{$Z$}{\textit{Z}}-transform package}
\ttindex{ZTRANS}

This package is an implementation of the $Z$-transform of a sequence.
This is the discrete analogue of the Laplace Transform.

Authors: Wolfram Koepf and Lisa Temme.

\let\sectionmark=\origsectionmark
