

%\begin{abstract}
  The SETS package for \REDUCE 3.5 and later versions provides
  algebraic-mode support for set operations on lists regarded as sets
  (or representing explicit sets) and on implicit sets represented by
  identifiers.  It provides the set-valued infix operators (with
  synonyms) {\tt union}, {\tt intersection} ({\tt intersect}) and {\tt
  setdiff} (\verb|\|, {\tt minus}) and the Boolean-valued infix
  operators (predicates) {\tt member}, {\tt subset\_eq}, {\tt subset},
  {\tt set\_eq}.  The union and intersection operators are n-ary and
  the rest are binary.  A list can be explicitly converted to the
  canonical set representation by applying the operator {\tt mkset}.
  (The package also provides an operator not specifically related to
  set theory called {\tt evalb} that allows the value of any
  Boolean-valued expression to be displayed in algebraic mode.)
%\end{abstract}


\subsection{Introduction}

REDUCE has no specific representation for a set, neither in algebraic
mode nor internally, and any object that is mathematically a set is
represented in REDUCE as a list.  The difference between a set and a
list is that in a set the ordering of elements is not significant and
duplicate elements are not allowed (or are ignored).  Hence a list
provides a perfectly natural and satisfactory representation for a set
(but not vice versa).  Some languages, such as Maple, provide
different internal representations for sets and lists, which may allow
sets to be processed more efficiently, but this is not \emph{necessary}.

This package supports set theoretic operations on lists and represents
the results as normal algebraic-mode lists, so that all other REDUCE
facilities that apply to lists can still be applied to lists that have
been constructed by explicit set operations.  The algebraic-mode set
operations provided by this package have all been available in
symbolic mode for a long time, and indeed are used internally by the
rest of REDUCE, so in that sense set theory facilities in REDUCE are
far from new.  What this package does is make them available in
algebraic mode, generalize their operation by extending the arity of
union and intersection, and allow their arguments to be implicit sets
represented by unbound identifiers.  It performs some simplifications
on such symbolic set-valued expressions, but this is currently rather
{\it ad hoc\/} and is probably incomplete.

For examples of the operation of the SETS package see (or run) the
test file {\tt sets.tst}.  This package is experimental and
developments are under consideration; if you have suggestions for
improvements (or corrections) then please send them to me (FJW),
preferably by email.  The package is intended to be run under
\REDUCE 3.5 and later versions; it may well run correctly under earlier
versions although I cannot provide support for such use.


\subsection{Infix operator precedence}

The set operators are currently inserted into the standard REDUCE
precedence list (see page 28, \S2.7, of the REDUCE 3.6 manual) as
follows:
\begin{verbatim}
or and not member memq = set_eq neq eq >= > <= < subset_eq
subset freeof + - setdiff union intersection * / ^ .
\end{verbatim}


\subsection{Explicit set representation and {\tt mkset}}
\ttindex{MKSET}

Explicit sets are represented by lists, and this package does not
require any restrictions at all on the forms of lists that are
regarded as sets.  Nevertheless, duplicate elements in a set
correspond by definition to the same element and it is conventional
and convenient to represent them by a single element, i.e.\ to remove
any duplicate elements.  I will call this a normal representation.
Since the order of elements in a set is irrelevant it is also
conventional and may be convenient to sort them into some standard
order, and an appropriate ordering of a normal representation gives a
canonical representation.  This means that two identical sets have
identical representations, and therefore the standard REDUCE equality
predicate ({\tt =}) correctly determines set equality; without a
canonical representation this is not the case.

Pre-processing of explicit set-valued arguments of the set-valued
operators to remove duplicates is always done because of the obvious
efficiency advantage if there were any duplicates, and hence explicit
sets appearing in the values of such operators will never contain any
duplicate elements.  Such sets are also currently sorted, mainly
because the result looks better.  The ordering used satisfies the {\tt
ordp} predicate used for most sorting within REDUCE, except that
explicit integers are sorted into increasing numerical order rather
than the decreasing order that satisfies {\tt ordp}.

Hence explicit sets appearing in the result of any set operator are
currently returned in a canonical form.  Any explicit set can also be
put into this form by applying the operator {\tt mkset} to the list
representing it.  For example
\begin{verbatim}
mkset {1,2,y,x*y,x+y};

{x + y,x*y,y,1,2}
\end{verbatim}

The empty set is represented by the empty list \verb|{}|.


\subsection{Union and intersection}
\ttindex{UNION}\ttindex{INTERSECTION}

The operator {\tt intersection} (the name used internally) has the
shorter synonym {\tt intersect}.  These operators will probably most
commonly be used as binary infix operators applied to explicit sets,
e.g.
\begin{verbatim}
{1,2,3} union {2,3,4};

{1,2,3,4}

{1,2,3} intersect {2,3,4};

{2,3}
\end{verbatim}
They can also be used as n-ary operators with any number of arguments,
in which case it saves typing to use them as prefix operators (which
is possible with all REDUCE infix operators), e.g.
\begin{verbatim}
{1,2,3} union {2,3,4} union {3,4,5};

{1,2,3,4,5}

intersect({1,2,3}, {2,3,4}, {3,4,5});

{3}
\end{verbatim}
For completeness, they can currently also be used as unary operators,
in which case they just return their arguments (in canonical form),
and so act as slightly less efficient versions of {\tt mkset} (but
this may change), e.g.
\begin{verbatim}
union {1,5,3,5,1};

{1,3,5}
\end{verbatim}


\subsection{Symbolic set expressions}

If one or more of the arguments evaluates to an unbound identifier
then it is regarded as representing a symbolic implicit set, and the
union or intersection will evaluate to an expression that still
contains the union or intersection operator.  These two operators are
symmetric, and so if they remain symbolic their arguments will be
sorted as for any symmetric operator.  Such symbolic set expressions
are simplified, but the simplification may not be complete in
non-trivial cases.  For example:
\begin{verbatim}
a union b union {} union b union {7,3};

{3,7} union a union b

a intersect {};

{}
\end{verbatim}

In implementations of REDUCE that provide fancy display using
mathematical notation, such as PSL-REDUCE~3.6 for MS-Windows, the
empty set, union, intersection and set difference are all displayed
using their conventional mathematical symbols, namely $\emptyset$,
$\cup$, $\cap$, $\setminus$.

A symbolic set expression is a valid argument for any other set
operator, e.g.
\begin{verbatim}
a union (b intersect c);

b intersection c union a
\end{verbatim}

Intersection distributes over union, which is not applied by default
but is implemented as a rule list assigned to the variable {\tt
set\_distribution\_rule}, e.g.
\begin{verbatim}
a intersect (b union c);

(b union c) intersection a

a intersect (b union c) where set_distribution_rule;

a intersection b union a intersection c
\end{verbatim}


\subsection{Set difference}
\ttindex{SETDIFF}
\ttindextype{\char`\\}{operator (\texttt{SETDIFF})}

The set difference operator is represented by the symbol \verb|\| and
is always output using this symbol, although it can also be input using
either of the two names {\tt setdiff} (the name used internally) or
{\tt minus} (as used in Maple).  It is a binary operator, its operands
may be any combination of explicit or implicit sets, and it may be
used in an argument of any other set operator.  Here are some
examples:
\begin{verbatim}
{1,2,3} \ {2,4};

{1,3}

{1,2,3} \ {};

{1,2,3}

a \ {1,2};

a\{1,2}

a \ a;

{}

a \ {};

a

{} \ a;

{}
\end{verbatim}


\subsection{Predicates on sets}
\ttindex{EVALB}

These are all binary infix operators.  Currently, like all REDUCE
predicates, they can only be used within conditional statements ({\tt
if}, {\tt while}, {\tt repeat}) or within the argument of the {\tt
evalb} operator provided by this package, and they cannot remain
symbolic -- a predicate that cannot be evaluated to a Boolean value
causes a normal REDUCE error.

The {\tt evalb} operator provides a convenient shorthand for an {\tt
if} statement designed purely to display the value of any Boolean
expression (not only predicates defined in this package).  It has some
similarity with the {\tt evalb} function in Maple, except that the
values returned by {\tt evalb} in REDUCE (the identifiers {\tt true}
and {\tt false}) have no significance to REDUCE itself.  Hence, in
REDUCE, use of {\tt evalb} is {\em never\/} necessary.
\begin{verbatim}
if a = a then true else false;

true

evalb(a = a);

true

if a = b then true else false;

false

evalb(a = b);

false

evalb 1;

true

evalb 0;

false
\end{verbatim}
I will use the {\tt evalb} operator in preference to an explicit {\tt
if} statement for purposes of illustration.


\subsubsection{Set membership}
\ttindex{MEMBER}

Set membership is tested by the predicate {\tt member}.  Its left
operand is regarded as a potential set element and its right operand
{\em must\/} evaluate to an explicit set.  There is currently no sense
in which the right operand could be an implicit set; this would
require a mechanism for declaring implicit set membership (akin to
implicit variable dependence) which is currently not implemented.  Set
membership testing works like this:
\begin{verbatim}
evalb(1 member {1,2,3});

true

evalb(2 member {1,2} intersect {2,3});

true

evalb(a member b);

***** b invalid as list
\end{verbatim}


\subsubsection{Set inclusion}
\ttindex{SUBSET\_EQ}

Set inclusion is tested by the predicate {\tt subset\_eq} where {\tt a
subset\_eq b} is true if the set $a$ is either a subset of or equal to
the set $b$; strict inclusion is tested by the predicate {\tt subset}
where {\tt a subset b} is true if the set $a$ is {\em strictly\/} a
subset of the set $b$ and is false is $a$ is equal to $b$.  These
predicates provide some support for symbolic set expressions, but this
is not yet correct as indicated below.  Here are some examples:
\begin{verbatim}
evalb({1,2} subset_eq {1,2,3});

true

evalb({1,2} subset_eq {1,2});

true

evalb({1,2} subset {1,2});

false


evalb(a subset a union b);

true

evalb(a\b subset a);

true

evalb(a intersect b subset a union b);  %%% BUG

false
\end{verbatim}

An undecidable predicate causes a normal REDUCE error, e.g.
\begin{verbatim}
evalb(a subset_eq {b});

***** Cannot evaluate a subset_eq {b} as Boolean-valued set
 expression

evalb(a subset_eq b);  %%% BUG

false
\end{verbatim}


\subsubsection{Set equality}
\ttindex{SET\_EQ}

As explained above, equality of two sets in canonical form can be
reliably tested by the standard REDUCE equality predicate ({\tt =}).
This package also provides the predicate {\tt set\_eq} to test
equality of two sets not represented canonically.  The two predicates
behave identically for operands that are symbolic set expressions
because these are always evaluated to canonical form (although
currently this is probably strictly true only in simple cases).  Here
are some examples:
\begin{verbatim}
evalb({1,2,3} = {1,2,3});

true

evalb({2,1,3} = {1,3,2});

false

evalb(mkset{2,1,3} = mkset{1,3,2});

true

evalb({2,1,3} set_eq {1,3,2});

true

evalb(a union a = a\{});

true
\end{verbatim}


\subsection{Installation}

The source file {\tt sets.red} can be read into REDUCE when required
using {\tt IN}.  If the ``professional'' version is being used this
should be done with {\tt ON COMP} set, but it is much better to
compile the code as a {\tt FASL} file using {\tt FASLOUT} and then
load it with {\tt LOAD\_PACKAGE} (or {\tt LOAD}).  See the REDUCE
manual and implementation-specific guide for further details.

This package has to redefine the REDUCE internal procedure {\tt
mk!*sq} and a warning about this can be expected and ignored.  I
believe (and hope!) that this redefinition is safe and will not have
any unexpected consequences for the rest of REDUCE.


\subsection{Possible future developments}

\begin{itemize}
\item Unary union/intersection to implement repeated
  union/intersection on a set of sets.
\item More symbolic set algebra, canonical forms for set expressions,
  more complete simplification.
\item Better support for Boolean variables via a version (evalb10?)
  of {\tt evalb} that returns 1/0 instead of {\tt true}/{\tt false},
  or predicates that return 1/0 directly.
\end{itemize}
