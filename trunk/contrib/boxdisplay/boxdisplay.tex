\documentstyle[11pt]{article}

\newcommand{\REDUCE}{{\sf REDUCE}}
\newcommand{\MACSYMA}{{\sf MACSYMA}}
\newcommand{\MAPLE}{{\sf MAPLE}}
\newcommand{\Mathematica}{{\sf Mathematica}}
\newcommand{\PSL}{{\sf PSL}}
\renewcommand{\today}{Apr 27, 1990}

\begin{document}

\title{Two-dimensional display of algebraic expressions
       in \REDUCE}

\author{Rainer Sch\"opf\\
        Institut f\"ur Theoretische Physik\\
        der Universit\"at Heidelberg\\
        Philosophenweg 16\\
        D-6900 Heidelberg\\
        Federal Republic of Germany\\
        Email: \verb|<BK4@DHDURZ1.BITNET>|}

\maketitle

\begin{abstract}
  This short note describes a method to display algebraic expressions
  in a device independent way.
\end{abstract}

\section{Introduction}

The idea came from Donald E. Knuth's typesetting program \TeX.
This program handles everything in terms of boxes and glue (the space
between boxes).

\section{Structure of algebraic expressions}

An algebraic expression consists either of a single undividable
object (i.e.\ a symbol denoting a variable) or of an operator ($op$)
and any number of operands ($arg_{1}\ldots arg_{n}$, $n\geq 0$) which
are again algebraic expressions.

\section{Display boxes}

Every algebraic expression is represented by a so-called
{\em display box\/} which looks like follows:\\
\begin{figure}[htb]
  \setlength{\unitlength}{1mm}
  \begin{picture}(120,60)
    \put(40,10){\line(1,0){30}}
    \put(40,10){\line(0,1){50}}
    \put(70,10){\line(0,1){50}}
    \put(40,60){\line(1,0){30}}
    \put(40,30){\line(1,0){30}}
    \put(40,30){\circle*{2}}
    \put(39,30){\makebox(0,0)[r]{Reference point $\longrightarrow$}}
    \put(55,31){\makebox(0,0)[b]{Baseline}}
    \put(40,7){\makebox(30,0)[t]{$\longleftarrow$%
                                 \hfill width\hfill
                                 $\longrightarrow$}}
    \put(80,10){\makebox(0,20){depth}}
    \put(80,30){\makebox(0,30){height}}
    \put(80,60){\makebox(0,0)[t]{$\Bigg\uparrow$}}
    \put(80,34){\makebox(0,0)[b]{$\Bigg\downarrow$}}
    \put(80,29){\makebox(0,0)[t]{$\Big\uparrow$}}
    \put(80,12){\makebox(0,0)[b]{$\Big\downarrow$}}
  \end{picture}
  \caption{A display box.}
\end{figure}

The display box is constructed by recursively descending the
corresponding algebraic expression.  We impose the restriction that the
display boxes of the operands are entirely contained in the box of the
whole expression.  Furthermore, the operands' display boxes must not
overlap.  This restriction is not crucial but will turn out to be
useful later on.

As an example consider the algebraic expression
\begin{displaymath}
  (a + b) ^{2}
\end{displaymath}
In LISP prefix form this may be represented as
\begin{center} \tt
  (expt (plus a b) 2)
\end{center}
Drawing the boxes we get
\begin{center}
  \setlength{\fboxsep}{0.2pt}
  \fbox{%
    $\fbox{$(\fbox{$a$} + \fbox{$b$})$}^{\fbox{\footnotesize$2$}}$}
\end{center}
This example shows some nice features of the proposed scheme
\begin{itemize}
  \item It is possible to provide special kinds of display boxes for
    special operators, e.g. in this case \verb|expt|.
  \item It is possible to use different sizes for different output
    devices, e.g., a smaller font for superscripts if the display
    device supports it.
  \item The transformation into a two-dimensional printed
    representation is rather simple.
  \item It is possible to use a mouse or similar device to select parts
    of an algebraic expression interactively.
  \item There are several ways to break an expression across lines:
    a simple look-ahead mechanism is possible, but also a global
    optimization scheme \`a la \TeX.
\end{itemize}
      
\section{Implementation considerations}

A prototype of such a system was implemented.
A display box is a data type with the following components:
\begin{itemize}
  \item height, depth, and width,
  \item the expression it represents,
  \item the operator of the expression,
  \item a flag indicating that the expression has to be printed
    with braces around it,
  \item for each argument of the operator: the position of its display
    box relative to the reference point of the enclosing box,
  \item a penalty to be used if this box is to be broken across several
    lines (for certain line-breaking schemes only),
  \item a pointer to the enclosing box (to facilitate use of a mouse).
\end{itemize}
Transformation to the printed representation is done by
allocating a two-dimensional array and filling it.
(Array slices would be useful here.)
As for breaking across lines, we implemented a simple mechanism that
checks whether the next box to be printed fits on the line and
determines the break point if necessary.
It seems that this is sufficient to get reasonable line breaks.
However, the prototype implementation is neither very efficient nor
does it handle all possible cases.

\end{document}
