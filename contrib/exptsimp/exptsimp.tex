\documentstyle[11pt,reduce]{article}
\newcommand{\MACSYMA}{{\sf MACSYMA}}
\newcommand{\MAPLE}{{\sf MAPLE}}
\newcommand{\Mathematica}{{\sf Mathematica}}
\newcommand{\PSL}{{\sf PSL}}
\title{Simplification of expressions \\
       involving exponentials and surds}
\date{}
\author{Rainer Sch\"opf\\
        Konrad-Zuse-Zentrum f\"ur Informationstechnik Berlin\\
        Heilbronner Str.\ 10\\
        W-1000 Berlin 31\\
        Federal Republic of Germany\\
        Email: Schoepf@sc.ZIB-Berlin.de}
\begin{document}
\maketitle

This short note describes a new \REDUCE{} operator that tries to
simplify expressions that contain exponentials and surds. It does this
by applying algebraic relations between these.

The operator is `\verb+EXPTSIMP+'\index{EXPTSIMP operator}. Usage is

\noindent {\tt EXPTSIMP}(EXP:{\em exprn}):{\em exprn}

The result is an equivalent expression.

\noindent {\em Method.} The basic idea is to replace every surd or
exponential by a variable and add a equations defining the variable.
For the resulting polynomial system is the Groebner basis is then
calculated, and the surds and exponentials are substituted back into
the result, leading to a possibly simpler expression.  Due to this
method, no assumptions about a specific choice of branches are made,
with the exception that multiple occurences of the same surd always
denote the same branch.

\end{document}
