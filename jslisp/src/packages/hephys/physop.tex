\documentstyle[11pt,reduce,makeidx]{article}
\makeindex
%%%%%%%%%%%%%%%%%%%%%%%%%%%%%%%%%%%%%%%%%%%%%%%%%%%%%%%%
% The following definitions should be commented out by those
% wishing to use MakeIndeX
\def\indexentry#1#2{{\tt #1} \dotfill\ #2\newline}
\renewcommand{\printindex}{\noindent \@input{\jobname.idx}}
%%%%%%% end of definitions %%%%%%%%%%%%%%%%%%%%%

\title{PHYSOP \\ A Package for Operator Calculus in Quantum Theory}
\author{User's Manual \\  Version 1.5  \\ January 1992}
\date{Mathias Warns \\
Physikalisches Institut der Universit\"at Bonn  \\
Endenicher Allee 11--13 \\
D--5300 BONN 1 \\
Germany \\*[2\parskip]
Tel: (++49) 228 733724 \\
Fax: (++49) 228 737869 \\
e--mail: UNP008@DBNRHRZ1.bitnet}

\begin{document}
\maketitle
\section{Introduction}
The package PHYSOP has been designed to meet the requirements of
theoretical physicists looking for a
computer algebra tool to perform complicated calculations
in quantum theory
with expressions containing operators. These operations
consist mainly in the calculation of commutators between operator
expressions and in the evaluations of operator matrix elements
in some abstract space. Since the capabilities
of the current \REDUCE\ release  to deal with complex
expressions containing noncommutative operators are rather restricted,
the first step was to enhance these possibilities in order to
achieve a better  usability of \REDUCE\ for these kind of calculations.
This has led to the development of a first package called
NONCOM2 which is described in section 2. For more complicated
expressions involving both scalar quantities and operators
the need for an additional data type has emerged in order to make a
clear separation between the various objects present in the calculation.
The implementation of this  new \REDUCE\ data type is realized by the
PHYSOP (for PHYSical OPerator) package described in section 3.

\section{The NONCOM2 Package}

The package NONCOM2  redefines some standard \REDUCE\ routines
in order to modify the way noncommutative operators are handled by the
system. In standard \REDUCE\ declaring an operator to be noncommutative
using the \f{NONCOM} statement puts a global flag on the
operator. This flag is checked when the system has to decide
whether or not two operators commute during the manipulation of an
expression.

The NONCOM2 package redefines the \f{NONCOM} \index{NONCOM} statement in
a way more suitable for calculations in physics. Operators have now to
be declared noncommutative pairwise, i.e. coding: \\

\begin{framedverbatim}
NONCOM A,B;
\end{framedverbatim}
declares the operators \f{A} and \f{B} to be noncommutative but allows them
to commute with any other (noncommutative or not) operator present in
the expression. In a similar way if one wants e.g.\ \f{A(X)} and
 \f{A(Y)} not to commute, one has now to code: \\

\begin{framedverbatim}
 NONCOM A,A;
\end{framedverbatim}
Each operator gets a new property list containing the
operators with which it does not commute.
A final example should make
the use of the redefined \f{NONCOM} statement clear: \\

\begin{framedverbatim}
NONCOM A,B,C;
\end{framedverbatim}
declares \f{A}  to be noncommutative with \f{B} and \f{C},
\f{B} to be noncommutative
with \f{A} and \f{C} and \f{C} to be noncommutative
with \f{A} and \f{B}.
Note that after these declaration
e.g.\ \f{A(X)} and \f{A(Y)}
are still commuting kernels.

Finally to keep the compatibility with standard \REDUCE\, declaring a
\underline{single} identifier using the \f{NONCOM} statement has the same
effect as in
standard \REDUCE\, i.e., the identifier is flagged with the \f{NONCOM} tag.

From the user's point of view there are no other
new commands implemented by the package. Commutation
relations have to be declared in the standard way as described in
the manual i.e.\ using
\f{LET} statements. The package itself consists of several redefined
standard
\REDUCE\ routines to handle the new definition of noncommutativity in
multiplications and pattern matching processes.

{\bf CAVEAT: } Due to its nature, the package is highly version
dependent. The current version has been designed for the 3.3  and 3.4
releases
of \REDUCE\ and may not work with previous versions. Some different
(but still correct) results may occur by using this package in
conjunction with
LET statements since part of the pattern matching routines have been
redesigned. The package has been designed to bridge a deficiency of the
current \REDUCE\ version concerning the notion of noncommutativity
 and it is the author's hope that it will be made  obsolete
by a future release of \REDUCE.

\section{The PHYSOP package}

The package PHYSOP implements a new \REDUCE\ data type to perform
calculations with physical operators. The noncommutativity of
operators is
implemented using the NONCOM2 package so this file should be loaded
prior to the use of PHYSOP\footnote{To build a fast
loading version of PHYSOP the NONCOM2
source code should be read in prior to the PHYSOP
code}.
In the following the new commands  implemented by the package
are described.  Beside these additional commands,
the full set of standard \REDUCE\ instructions remains
available for performing  any other calculation.

\subsection{Type declaration commands}

The new \REDUCE\ data type PHYSOP implemented by the package allows the
definition of a new kind of operators (i.e. kernels carrying
an arbitrary
number of arguments). Throughout this manual, the name
``operator''
will refer, unless explicitly stated otherwise, to this new data type.
This data type is in turn
divided into 5 subtypes. For each of this subtype, a declaration command
has been defined:
\begin{description}
\item[\f{SCALOP A;} ] \index{SCALOP} declares \f{A} to be a scalar
operator. This operator may
carry an arbitrary number of arguments i.e.\ after the
declaration: \f{ SCALOP A; }
all kernels of the form e.g.\
\f{A(J), A(1,N), A(N,L,M)}
are recognized by the system as being scalar operators.

\item[\f{VECOP V;} ] \index{VECOP} declares \f{V} to be a vector operator.
As for scalar operators, the vector operators may carry an arbitrary
number of arguments. For example \f{V(3)} can be used to represent
the vector operator $\vec{V}_{3}$. Note that the dimension of space
in which this operator lives is \underline{arbitrary}.
One can however address a specific component of the
vector operator by using a special index declared as \f{PHYSINDEX} (see
below). This index must then be the first in the argument list
of the vector operator.

\item[\f{TENSOP C(3);} ]  \index{TENSOP}
declares \f{C} to be a tensor operator of rank 3. Tensor operators
of any fixed integer rank larger than 1 can be declared.
Again this operator may carry an arbitrary number of arguments
and the space dimension is not fixed.
The tensor
components can be addressed by using  special \f{PHYSINDEX} indices
(see below) which have to be placed in front of all other
arguments in the argument list.


\item[\f{STATE U;} ] \index{STATE} declares \f{U} to be a state, i.e.\ an
object on
which operators have a certain action.  The state  U can also carry an
arbitrary number of arguments.

\item[\f{PHYSINDEX X;} ] \index{PHYSINDEX} declares \f{X} to be a special
index which will be used
to address components of vector and tensor operators.
\end{description}

It is very important to understand precisely the way how the type
declaration commands work in order to avoid type mismatch errors when
using the PHYSOP package. The following examples should illustrate the
way the program interprets type declarations.
Assume that the declarations listed above have
been typed in by the user, then:
\begin{description}
\item[$\bullet$] \f{A,A(1,N),A(N,M,K)} are SCALAR operators.
\item[$\bullet$] \f{V,V(3),V(N,M)} are VECTOR operators.
\item[$\bullet$] \f{C, C(5),C(Y,Z)} are TENSOR operators of
rank 3.
\item[$\bullet$] \f{U,U(P),U(N,L,M)} are  STATES.
\item[BUT:] \f{V(X),V(X,3),V(X,N,M)} are all \underline{scalar}
operators
since the \underline{special index} \f{X}  addresses a
specific component
of the vector operator (which is a scalar operator). Accordingly,
\f{C(X,X,X)} is  also a \underline{scalar} operator because
the diagonal component $C_{xxx}$
of the tensor operator \f{C} is meant here
(C has rank 3 so 3 special indices must be used for the components).
\end{description}

In view of these examples, every time the following text
refers to \underline{scalar} operators,
it should be understood that this means not only operators defined by
the
\f{SCALOP} statement but also components of vector and tensor operators.
Depending on the situation, in some case when dealing only with the
components of vector or tensor operators it may be preferable to use
an operator declared with \f{SCALOP} rather than addressing  the
components using several special indices (throughout the
manual,
indices declared with the \f{PHYSINDEX} command are referred to as special
indices).

Another important feature of the system  is that
for each operator declared using the statements described above, the
system generates 2 additional operators of the same type:
the \underline{adjoint} and the \underline{inverse} operator.
These operators are accessible to the user for subsequent calculations
without any new declaration.  The syntax is as following:

If \f{A} has been declared to be an operator (scalar, vector or tensor)
the \underline{adjoint} operator is denoted \f{A!+} and the
\underline{inverse}
operator is denoted \f{A!-1} (an inverse adjoint operator \f{A!+!-1}
is also generated).
The exclamation marks do not appear
when these operators are printed out by \REDUCE\ (except when the switch
\f{NAT} is set to off)
but have to be typed in when these operators are used in an input
expression.
An adjoint (but \underline{no} inverse)  state  is also
generated  for every state defined by the user.
One may consider these generated operators as ''placeholders'' which
means that these operators are considered by default as
being completely independent of the original operator.
Especially if some value is assigned to the original operator,
this value is \underline{not} automatically assigned to the
generated operators. The user must code additional assignement
statements in order to get the corresponding values.

Exceptions from these rules are (i) that inverse operators are
\underline{always} ordered at the same place as the original operators
and (ii) that  the expressions \f{A!-1*A}
and \f{A*A!-1} are replaced\footnote{This may not always occur in
intermediate steps of a calculation  due to efficiency reasons.}
by the unit operator \f{UNIT} \index{UNIT}.
This operator is defined
as a scalar operator during the initialization of the PHYSOP package.
It should be used to indicate
the type of an operator expression whenever no other PHYSOP
occur in it. For example, the following sequence: \\

\begin{framedverbatim}
SCALOP A;
A:= 5;
\end{framedverbatim}
leads to a type mismatch error and should be replaced by: \\

\begin{framedverbatim}
SCALOP A;
A:=5*UNIT;
\end{framedverbatim}
The operator \f{UNIT} is a reserved variable of the system and should
not be used for other purposes.

All other kernels  (including standard \REDUCE\ operators)
occurring in expressions are treated as ordinary scalar variables
without any PHYSOP type (referred to as \underline{scalars} in the
following).
Assignement statements are checked to ensure correct operator
type assignement on both sides leading to an error if a type
mismatch occurs. However an assignement statement of the form
\f{A:= 0} or \f{LET A = 0} is \underline{always} valid regardless of the
type of \f{A}.

Finally a command \f{CLEARPHYSOP} \index{CLEARPHYSOP}
has been defined to remove
the PHYSOP type from an identifier in order to use it for subsequent
calculations (e.g. as an ordinary \REDUCE\ operator). However it should be
remembered that \underline{no}
substitution rule is cleared by this function. It
is therefore left to the user's responsibility to clear previously all
substitution rules involving the identifier from which the PHYSOP type
is removed.

Users should be very careful when defining procedures or statements of
the type \f{FOR ALL ... LET ...} that the PHYSOP type of all identifiers
occurring in such expressions is unambigously fixed. The type analysing
procedure is rather restrictive and will print out a ''PHYSOP type
conflict'' error message if such ambiguities occur.

\subsection{Ordering of operators in an expression}

The ordering of kernels in an expression is performed according to
the following rules: \\
1. \underline{Scalars} are always ordered ahead of
PHYSOP \underline{operators} in an expression.
The \REDUCE\ statement \f{KORDER} \index{KORDER} can be used to control the
ordering of scalars but has \underline{no}
effect on the ordering of operators.

2. The default ordering of \underline{operators} follows the
order in which they have been declared (and \underline{not}
the alphabetical one).
This ordering scheme can be changed using the command \f{OPORDER}.
\index{OPORDER}
Its syntax is similar to the \f{KORDER} statement, i.e.\ coding:
\f{OPORDER A,V,F;}
means that all occurrences of the operator \f{A} are ordered ahead of
those of \f{V} etc. It is also possible to include operators
carrying
indices (both normal and special ones) in the argument list of
\f{OPORDER}. However including objects  \underline{not}
defined as operators (i.e. scalars or indices) in the argument list
of the \f{OPORDER} command leads to an error.

3. Adjoint operators are placed by the declaration commands just
after the original operators on the \f{OPORDER} list. Changing the
place of an operator on this list means \underline{not} that the
adjoint operator is moved accordingly. This adjoint operator can
be moved freely  by including it in the argument list of the
\f{OPORDER} command.

\subsection{Arithmetic operations on operators}

The following arithmetic operations are possible with
operator expressions: \\

1. Multiplication or division of an operator by a scalar.

2. Addition and subtraction of operators of the \underline{same} type.

3. Multiplication of operators is only defined between two
\underline{scalar} operators.

4. The scalar product of two VECTOR operators is implemented with
a new function \f{DOT} \index{DOT}. The system expands the product of
two vector operators into an ordinary product of the components of these
operators by inserting a special index generated by the program.
To give an example, if one codes: \\

\begin{framedverbatim}
VECOP V,W;
V DOT W;
\end{framedverbatim}
the system will transform the product into: \\

\begin{framedverbatim}
V(IDX1) * W(IDX1)
\end{framedverbatim}
where \f{IDX1} is a \f{PHYSINDEX} generated by the system (called a DUMMY
INDEX in the following) to express the summation over the components.
The identifiers \f{IDXn} (\f{n} is
a nonzero integer) are
reserved variables for this purpose and should not be used for other
applications. The arithmetic operator
\f{DOT} can be used both in infix and prefix form with two arguments.

5. Operators (but not states) can only be raised to an
\underline{integer} power. The system expands this power
expression into a product of the corresponding number of terms
inserting dummy indices if necessary. The following examples explain
the transformations occurring on power expressions (system output
is indicated with an \f{-->}): \\

\begin{framedverbatim}
SCALOP A; A**2;
- --> A*A
VECOP V; V**4;
- --> V(IDX1)*V(IDX1)*V(IDX2)*V(IDX2)
TENSOP C(2); C**2;
- --> C(IDX3,IDX4)*C(IDX3,IDX4)
\end{framedverbatim}
Note in particular the way how the system interprets powers of
tensor operators which is different from the notation used in matrix
algebra.

6. Quotients of operators are only defined between
\underline{scalar} operator expressions.
The system transforms the quotient of 2 scalar operators into the
product of the first operator times the inverse of the second one.
Example\footnote{This shows how inverse operators are printed out when
the switch \f{NAT} is on}: \\

\begin{framedverbatim}
SCALOP A,B;   A / B;
       -1
 --> (B  )*A
\end{framedverbatim}

7. Combining the  last 2 rules explains the way how the system
handles negative powers of operators: \\

\noindent
\begin{framedverbatim}
SCALOP B;
B**(-3);
       -1    -1    -1
 --> (B  )*(B  )*(B  )
\end{framedverbatim}


The method of inserting dummy indices and expanding powers of
operators has been chosen to facilitate the handling of
complicated operator
expressions and particularly their application  on states
(see section 3.4.3). However it may be useful to get rid of these
dummy indices in order to enhance the readability of the
system's final output.
For this purpose the switch \f{CONTRACT} \index{CONTRACT} has to
be turned on (\f{CONTRACT} is normally set to \f{OFF}).
The system in this case contracts over dummy indices reinserting the
\f{DOT} operator and reassembling the expanded powers.  However due to
the predefined operator ordering the system may not remove all the
dummy indices introduced previously.

\subsection{Special functions}

\subsubsection{Commutation relations}

If 2 PHYSOPs have been declared noncommutative using the (redefined)
\f{NONCOM} statement, it is possible to introduce in the environment
\underline{elementary} (anti-) commutation relations between them. For
this purpose,
2 \underline{scalar} operators \f{COMM} \index{COMM} and
\f{ANTICOMM} \index{ANTICOMM} are available.
These operators are used in conjunction with \f{LET} statements.
Example: \\

\begin{framedverbatim}
SCALOP A,B,C,D;
LET COMM(A,B)=C;
FOR ALL N,M LET ANTICOMM(A(N),B(M))=D;
VECOP U,V,W; PHYSINDEX X,Y,Z;
FOR ALL X,Y LET COMM(V(X),W(Y))=U(Z);
\end{framedverbatim}

Note that if special indices are used as dummy variables in
\f{FOR ALL ... LET} constructs then these indices should  have been
declared previously using the \f{PHYSINDEX} command.

Every time the system
encounters a product term involving 2
noncommutative operators which have to be reordered on account of the
given operator ordering, the list of available (anti-) commutators is
checked in the following way: First the system looks for a
\underline{commutation} relation which matches the  product term. If it
fails then the defined \underline{anticommutation} relations are
checked. If there is no successful match the product term
 \f{A*B} is replaced by: \\

\begin{framedverbatim}
A*B;
 --> COMM(A,B) + B*A
\end{framedverbatim}
so that the user may introduce the commutation relation later on.

The user may want to force the system to look for
\underline{anticommutators} only; for this purpose a switch \f{ANTICOM}
\index{ANTICOM}
is defined which has to be turned on ( \f{ANTICOM} is normally set to
\f{OFF}). In this case, the above example is replaced by: \\

\begin{framedverbatim}
ON ANTICOM;
A*B;
 -->  ANTICOMM(A,B) - B*A
\end{framedverbatim}

Once the operator ordering has been fixed (in the example above \f{B}
has to be ordered ahead of \f{A}),
there is \underline{no way} to prevent the
system from introducing (anti-)commutators  every time it encounters
a product whose terms are not in the right order. On the other hand,
simply by changing the \f{OPORDER} statement  and reevaluating the
expression one can change  the operator ordering
\underline{without}
the need to introduce new commutation relations.
Consider the following example: \\

\begin{framedverbatim}
SCALOP A,B,C;   NONCOM A,B;   OPORDER B,A;
LET COMM(A,B)=C;
A*B;
- -->  B*A + C;
OPORDER A,B;
 B*A;
- --> A*B - C;
\end{framedverbatim}

The functions \f{COMM} and \f{ANTICOMM} should only be used  to
define
elementary (anti-) commutation relations between single operators.
For the calculation of (anti-) commutators between complex
operator
expressions, the functions \f{COMMUTE} \index{COMMUTE} and
\f{ANTICOMMUTE} \index{ANTICOMMUTE} have been defined.
Example (is included as example 1 in the test file): \\

\begin{framedverbatim}
VECOP P,A,K;
PHYSINDEX X,Y;
FOR ALL X,Y LET COMM(P(X),A(Y))=K(X)*A(Y);
COMMUTE(P**2,P DOT A);
\end{framedverbatim}

\subsubsection{Adjoint expressions}

As has been already mentioned, for each operator and state defined
using the declaration commands quoted in section 3.1, the system
generates automatically the corresponding adjoint operator. For the
calculation of the adjoint representation of a complicated
operator expression, a function  \f{ADJ} \index{ADJ} has been defined.
Example\footnote{This shows how adjoint operators are printed out
when the switch \f{NAT} is on}: \\

\begin{framedverbatim}
SCALOP A,B;
ADJ(A*B);
       +    +
 --> (B )*(A )
\end{framedverbatim}

\subsubsection{Application of operators on states}

For this purpose, a function \f{OPAPPLY} \index{OPAPPLY} has been
defined.
It has 2 arguments and is used in the following combinations: \\

{\bf (i)}  \f{LET OPAPPLY(}{\it operator, state}\f{) =} {\it state};
This is to define a elementary
action of an operator on a state in analogy to the way
elementary commutation relations are introduced to the system.
Example: \\

\begin{framedverbatim}
SCALOP A; STATE U;
FOR ALL N,P LET OPAPPLY((A(N),U(P))= EXP(I*N*P)*U(P);
\end{framedverbatim}

{\bf (ii)} \f{LET OPAPPLY(}{\it state, state}\f{) =} {\it scalar exp.};
This form is to define scalar products between states and normalization
conditions.
Example:  \\

\begin{framedverbatim}
STATE U;
FOR ALL N,M LET OPAPPLY(U(N),U(M)) = IF N=M THEN 1 ELSE 0;
\end{framedverbatim}

{\bf (iii)} {\it state} \f{:= OPAPPLY(}{\it operator expression, state});
 In this way, the action of an operator expression on a given state
is calculated using elementary relations defined as explained in {\bf
(i)}. The result may be assigned to a different state vector.

{\bf (iv)} \f{OPAPPLY(}{\it state}\f{, OPAPPLY(}{\it operator expression,
state}\f{))}; This is the way how to calculate matrix elements of
operator
expressions. The system proceeds in the following way: first the
rightmost operator is applied on the right state, which means that the
system tries
to find an elementary relation which match the application of the
operator on the state. If it fails
the system tries to apply the leftmost operator of the expression on the
left state using the adjoint representations. If this fails also,
the system prints out a warning message and stops the evaluation.
Otherwise the next operator occuring in the expression is
taken and so on until the complete expression is applied.  Then the
system
looks for a relation expressing the scalar product of the two
resulting states and prints out the final result. An example of such
a calculation is given in the test file.

The infix version of the \f{OPAPPLY} function is the vertical bar $\mid$
. It is \underline{right} associative and placed in the precedence
list just above the minus ($-$) operator.
Some of the \REDUCE\ implementation may not work with this character,
the prefix form should then be used instead\footnote{The source code
can also be modified to choose another special character for the
function}.

\section{Known problems in the current release of PHYSOP}

\indent {\bf (i)} Some spurious negative powers  of operators
may appear
in the result of a calculation using the PHYSOP package. This is a
purely ''cosmetic'' effect which is due to an additional
factorization of the expression in the output printing routines of
\REDUCE. Setting off the \REDUCE\ switch \f{ALLFAC}  (\f{ALLFAC} is normally
on)
should make these
terms disappear and print out the correct result (see example 1
in the test file).

{\bf (ii)} The current release of the PHYSOP package is not optimized
w.r.t. computation speed. Users should be aware that the evaluation
of complicated expressions involving a lot of commutation relations
requires a significant amount of CPU time \underline{and} memory.
Therefore the use of PHYSOP on small machines is rather limited. A
minimal hardware configuration should include at least 4 MB of
memory and a reasonably fast CPU (type Intel 80386 or equiv.).

{\bf (iii)} Slightly different ordering of operators (especially with
multiple occurrences of the same operator with different indices)
may appear in some calculations
due to the internal ordering of atoms in the underlying LISP system
(see last example in the test file). This cannot be entirely avoided
by the package but does not affect the correctness  of the results.

\section{Compilation of the packages}
To build a fast loading module of the NONCOM2 package, enter the
following commands after starting the \REDUCE\ system: \\

\begin{framedverbatim}
  faslout "noncom2";
   in "noncom2.red";
  faslend;
\end{framedverbatim}
To build a fast loading module of the PHYSOP package, enter the
following commands after starting the \REDUCE\ system: \\

\begin{framedverbatim}
  faslout "physop";
   in "noncom2.red";
   in "physop.red";
  faslend;
\end{framedverbatim}
Input and output file specifications may change according to the
underlying operating system. \\
On PSL--based systems, a spurious message: \\

\begin{framedverbatim}
*** unknown function PHYSOP!*SQ called from compiled code
\end{framedverbatim}
may appear during the compilation of the PHYSOP package. This warning
has no effect on the functionality of the package.

\section{Final remarks}
The package PHYSOP has been presented by
the author at the IV inter. Conference on Computer Algebra in Physical
Research, Dubna (USSR) 1990 (see M. Warns, {\it
Software Extensions of \REDUCE\ for Operator Calculus in Quantum Theory},
Proc.\ of the IV inter.\ Conf.\ on Computer Algebra in Physical
Research, Dubna 1990, to appear). It has been developed with the aim in
mind to perform calculations of the type exemplified in the test file
included in the distribution of this package.
However it should
also be useful in some other domains like e.g.\ the calculations of
complicated Feynman diagrams in QCD  which could not be  performed using
the HEPHYS package. The author is  therefore grateful for any
suggestion
to improve or extend the usability of the package. Users should not
hesitate to contact the author for additional help and explanations on
how to use
this package. Some bugs may also
appear which have not been discovered during the tests performed
prior to the release of this version. Please send in this case to the
author  a short
input and output listing  displaying the encountered problem.

\section*{Acknowledgements}
The main ideas for the implementation of a new data type in the \REDUCE\
environnement have been taken from the VECTOR package developed by
Dr.\ David Harper (D. Harper, Comp.\ Phys.\ Comm.\ {\bf 54} (1989)
295).
Useful discussions with  Dr.\ Eberhard Schr\"ufer  and
Prof.\ John Fitch are also gratefully acknowledged.

\appendix

\section{List of error and warning messages}
In the following the error (E) and warning (W) messages specific to the
PHYSOP package are listed.
\begin{description}
\item[\f{cannot declare} {\it x}\f{ as }{\it data type}] (W):
 An attempt has been made to declare an
object {\it x} which cannot be used as a PHYSOP operator of the
required type. The declaration command is ignored.

\item [{\it x} \f{already defined as} {\it data type}] (W): The object
{\it x} has already been declared using a \REDUCE\ type declaration
command and can therefore not be used as a PHYSOP operator.
The declaration command is ignored.

\item [{\it x} \f{already declared as} {\it data type}] (W): The object
\f{x} has already been declared with a PHYSOP declaration command.
The declaration command is ignored.

\item[{\it x} \f{is not a PHYSOP}] (E): An invalid argument has been
included in an \f{OPORDER} command. Check the arguments.

\item[\f{invalid argument(s) to }{\it function}] (E): A
function implemented by the PHYSOP package has been called with an
invalid argument. Check type of arguments.


\item[\f{Type conflict in }{\it operation}] (E): A PHYSOP type conflict
has occured during an arithmetic operation. Check the arguments.

\item [\f{invalid call of }{\it function} \f{with args:} {\it arguments}]
(E): A function
of the PHYSOP package has been declared with invalid argument(s). Check
the argument list.

\item[\f{type mismatch in} {\it expression}] (E): A type mismatch has
been detected in an expression.  Check the corresponding expression.

\item[\f{type mismatch in} {\it assignement}] (E): A type
mismatch has been detected in an assignment or in a \f{LET}
statement. Check the listed statement.

\item[\f{PHYSOP type conflict in} {\it expr}] (E): A ambiguity has been
detected during the type analysis of the expression. Check the
expression.

\item[\f{operators in exponent cannot be handled}] (E): An operator has
occurred in the exponent of an expression.

\item[\f{cannot raise a state to a power}] (E): states cannot be
exponentiated by the system.

\item[\f{invalid quotient}] (E): An invalid denominator has occurred in a
quotient. Check the expression.

\item[\f{physops of different types cannot be commuted}] (E): An invalid
operator has occurred in a call of the \f{COMMUTE}/\f{ANTICOMMUTE} function.

\item[\f{commutators only implemented between scalar operators}] (E):
An invalid operator has occurred in the call of the
\f{COMMUTE}/\f{ANTICOMMUTE} function.

\item[\f{evaluation incomplete due to missing elementary relations}] (W):
\\
The system has not found all
the elementary commutators or application relations necessary to
calculate or reorder the input expression. The result may however be
used for further calculations.
\end{description}

\section{List of available commands}
\printindex
\end{document}
