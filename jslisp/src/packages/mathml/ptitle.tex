\begin{center}
\large\sc
MATH0082 Double Unit Project			\\ [4ex]
\large\bf
An OpenMath to MathML translator.
\end{center}

\begin{center}\begin{tabular}{ll}
  Candidate: 		 &Alvarez,L.		\\	% SURNAME, INITALS
  Supervisor:		 &JHD			\\	% TITLE INITALS SURNAME
  Checker:		 & 			\\	% LEAVE BLANK
  Review date:		 &	3 March 2000	\\
  Final submission date: &	2 May 2000	\\
  Equipment required:	 &own Linux Computer; BUCS	\\	% SPECIFY
\end{tabular}\end{center}


\subsubsection{ Description}					% DESCRIPTION (in the range of 100--200 words)

OpenMath and MathML are two ways of representing mathematical objects. Semantically, OpenMath is a superset of (content) MathML. The aim is to
build a translator from OpenMath to content MathML, using presentation MathML where necessary as in the example of rank in Section 5.3 on MathML
{\tt http://www.w3c.org/TR/REC-MathML/chapter5.html}. Since OpenMath is extensible, the translator will need to be.  There is no a priori choice of
implementation language. A {\em viva voce} examination will be held.  % DELETE IF INAPPLICABLE

The project report should be no more than 40 pages.


\vspace{2ex}
 \subsubsection{Marking Scheme}					

%
%	THIS DOUBLE UNIT CARRIES 12 CREDITS AND 6 ALPHAS
%
%	THERE ARE 100 AVAILABLE MARKS
%
%
\begin{center}\begin{tabular}{lrr}
  Background research           	  &	 15 &	 $\alpha$	\\
  Analysis of MathML/OpenMath translation &      10 &    $\alpha$        \\ 
  Design				  &	 20 &	2$\alpha$	\\
  Implementation	        	  &	 25 &	 $\alpha$	\\
  Testing       			  &	 10 &			\\
  Report and Documentation		  &	 15 &	 $\alpha$	\\
  Viva and Demonstration		  &	  5 &			\\ \cline{2-3}
								\\
  Total					  &	100 &	6$\alpha$	\\
\end{tabular}\end{center}




