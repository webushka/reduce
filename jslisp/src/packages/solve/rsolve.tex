\documentstyle[11pt]{article}

\title{R/I\_SOLVE: Rational/Integer Polynomial Solvers}

\author{Francis J. Wright \\
School of Mathematical Sciences \\
Queen Mary and Westfield College \\
University of London \\
Mile End Road, London E1 4NS, UK. \\
E-mail: {\tt F.J.Wright@QMW.ac.uk}}

\date{27 January 1995}

\begin{document}
\maketitle

\begin{abstract}
  This package provides the operators \verb|r/i_solve| that compute
  respectively the exact rational or integer zeros of a single
  univariate polynomial using fast modular methods.
\end{abstract}


\section{Introduction}

This package provides operators that compute the exact rational zeros
of a single univariate polynomial using fast modular methods.  The
algorithm used is that described by R. Loos (1983): Computing rational
zeros of integral polynomials by $p$-adic expansion, {\it SIAM J.
Computing}, {\bf 12}, 286--293.  The operator \verb|r_solve| computes
all rational zeros whereas the operator \verb|i_solve| computes only
integer zeros in a way that is slightly more efficient than extracting
them from the rational zeros.  The \verb|r_solve| and \verb|i_solve|
interfaces are almost identical, and are intended to be completely
compatible with that of the general \verb|solve| operator, although
\verb|r_solve| and \verb|i_solve| give more convenient output when
only rational or integer zeros respectively are required.  The current
implementation appears to be faster than \verb|solve| by a factor that
depends on the example, but is typically up to about 2.
  
I plan to extend this package to compute Gaussian integer and rational
zeros and zeros of polynomial systems.


\section{The user interface}

The first argument is required and must simplify to either a
univariate polynomial expression or equation with integer, rational or
rounded coefficients.  Symbolic coefficients are not allowed (and
currently complex coefficients are not allowed either.)  The argument
is simplified to a quotient of integer polynomials and the denominator
is silently ignored.
  
Subsequent arguments are optional.  If the polynomial variable is to
be specified then it must be the first optional argument, and if the
first optional argument is not a valid option (see below) then it is
(mis-)interpreted as the polynomial variable.  However, since the
variable in a non-constant univariate polynomial can be deduced from
the polynomial it is unnecessary to specify it separately, except in
the degenerate case that the first argument simplifies to either 0 or
$0 = 0$.  In this case the result is returned by \verb|i_solve| in
terms of the operator \verb|arbint| and by \verb|r_solve| in terms of
the (new) analogous operator \verb|arbrat|.  The operator
\verb|i_solve| will generally run slightly faster than \verb|r_solve|.
  
The (rational or integer) zeros of the first argument are returned as
a list and the default output format is the same as that used by
\verb|solve|.  Each distinct zero is returned in the form of an
equation with the variable on the left and the multiplicities of the
zeros are assigned to the variable \verb|root_multiplicities| as a
list.  However, if the switch \verb|multiplicities| is turned on then
each zero is explicitly included in the solution list the appropriate
number of times (and \verb|root_multiplicities| has no value).
  
\begin{sloppypar}
Optional keyword arguments acting as local switches allow other output
formats.  They have the following meanings:
\begin{description}
\item[\verb|separate|:] assign the multiplicity list to the global
  variable \verb|root_multiplicities| (the default);
\item[\verb|expand| or \verb|multiplicities|:] expand the solution
  list to include multiple zeros multiple times (the default if the
  \verb|multiplicities| switch is on);
\item[\verb|together|:] return each solution as a list whose second
  element is the multiplicity;
\item[\verb|nomul|:] do not compute multiplicities (thereby saving
  some time);
\item[\verb|noeqs|:] do not return univariate zeros as equations but
  just as values.
\end{description}
\end{sloppypar}


\section{Examples}

\begin{verbatim}
r_solve((9x^2 - 16)*(x^2 - 9), x);
\end{verbatim}
\[
  \left\{x=\frac{-4}{3},x=3,x=-3,x=\frac{4}{3}\right\}
\]
\begin{verbatim}
i_solve((9x^2 - 16)*(x^2 - 9), x);
\end{verbatim}
\[
  \{x=3,x=-3\}
\]
See the test/demonstration file \verb|rsolve.tst| for more examples.


\section{Tracing}

The switch {\tt trsolve} turns on tracing of the algorithm.  It is off
by default.

\end{document}
