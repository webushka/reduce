

%\newcommand{\xr}{\texttt{XR}}

%\setlength{\oddsidemargin}{5mm}
%\setlength{\evensidemargin}{-5mm}
%\setlength{\textwidth}{159.2mm}
%\setlength{\textheight}{235mm}
%\addtolength{\topmargin}{-18mm}

\newcommand{\Gnuplot}{\textsc{GnuPlot}}

\iffalse
\newpage
  Gnuplot
\newpage
\title{{\Gnuplot} Interface for REDUCE\\Version 4}
\author{Herbert Melenk \\ 
Konrad--Zuse--Zentrum f\"ur Informationstechnik Berlin \\
E--mail: Melenk@zib.de}
\maketitle

\fi

\index{GNUPLOT package}
%\markboth{APPENDIX}{GNUPLOT Interface for REDUCE}

%\section{APPENDIX: GNUPLOT Interface for REDUCE}

\subsection{Introduction}

The {\Gnuplot} system provides easy to use graphics output for curves or surfaces
which are defined by formulas and/or data sets.  {\Gnuplot} supports a variety of
output devices such as
\verb+VGA screen+, \verb+postscript+, \verb+pic+\TeX, \verb+MS Windows+.
The {\REDUCE} {\Gnuplot} package lets one use the {\Gnuplot} graphical output
directly from inside {\REDUCE}, either for the interactive display of
curves/surfaces or for the production of pictures on paper.

\iffalse
{\REDUCE} supports {\Gnuplot} 3.4 (or higher).  For DOS, Windows, Windows 95,
Windows NT, OS/2 and Unix versions of {\REDUCE} {\Gnuplot} binaries are delivered
together with {\REDUCE}\footnote{The {\Gnuplot} developers have agreed that {\Gnuplot}
binaries can be distributed together with {\REDUCE}.  As {\Gnuplot} is a package
distributed without cost, the {\Gnuplot} support of {\REDUCE} also is an add-on to
the {\REDUCE} kernel system without charge.  We recommend fetching the full
{\Gnuplot} system by anonymous FTP from a file server.  } % end of footnote
. However, this is a basic set only.  If you intend to use more facilities of
the {\Gnuplot} system you should pick up the full {\Gnuplot} file tree
from \url{http://sourceforge.net/projects/gnuplot}.
\fi

\subsection{Command \texttt{plot}}
\ttindex{PLOT}

Under {\REDUCE} {\Gnuplot} is used as graphical output server, invoked by the
command \texttt{plot(...)}.  This command can have a variable number of
parameters:
\begin{itemize}
\item A function to plot; a function can be
  \begin{itemize}
    \item an expression with one unknown, e.g. \texttt{u*sin(u)\^{}2}.
    \item a list of expressions with one (identical) unknown,
      e.g. \texttt{\{sin(u),\allowbreak cos(u)\}}.
    \item an expression with two unknowns, e.g. \texttt{u*sin(u)\^{}2+sqrt(v)}.
    \item a list of expressions with two (identical) unknowns,
      e.g. \linebreak
      \texttt{\{x\textasciicircum{2}+y\textasciicircum{2},\allowbreak x\textasciicircum{2}-y\textasciicircum{2}\}}.
    \item a parametic expression of the form \texttt{point(<u>,<v>)} or
      \texttt{point(<u>,\allowbreak <v>,<w>)} where \texttt{u,v,w} are
      expressions which depend of one or two parameters; if there is one
      parameter, the object describes a curve in the plane (only \texttt{u} and
      \texttt{v}) or in 3D space; if there are two parameters, the object
      describes a surface in 3D. The parameters are treated as independent
      variables.  Example: \texttt{point(sin t,cos t,t/10)}.
    \item an equation with a symbol on the left-hand side and an expression with
      one or two unknowns on the right-hand side, e.g.\linebreak[3]
      \texttt{dome=\allowbreak
        1/(x\^{}2+y\^{}2)}.
    \item an equation with an expression on the left-hand side and a zero on
      right-hand side describing implicitly a one dimensional variety in the
      plane (implicitly given curve), e.g. 
      \texttt{x\^{}3 + x*y\^{}2-9x = 0}, or a
      two-dimensional surface in 3-dimensional Euclidean space,
    \item an equation with an expression in two variables on the left-hand side
      and a list of numbers on the right-hand side; the contour lines
      corresponding to the given values are drawn, e.g. \\
      \texttt{x\^{}3 - y\^{}2 + x*y = \{-2,-1,0,1,2\}}.
    \item a list of points in 2 or 3 dimensions, e.g.
      \texttt{\{\{0,0\},\{0,1\},\{1,1\}\}} representing a curve,
    \item a list of lists of points in 2 or 3 dimensions
      e.g. \texttt{\{\{\{0,0\},\{0,1\},\{1,1\}\}, \{\{0,0\},\{0,1\},\{1,1\}\}\}}
      representing a family of curves.
  \end{itemize}
  \hypertarget{reserved:intervalop}
\item A range for a variable; this has the form
  \texttt{variable=(lower\_bound,..,\allowbreak upper\_bound)} where
  \texttt{lower\_bound} and \texttt{upper\_bound} must be expressions which
  evaluate to numbers. If no range is specified the default ranges for
  independent variables are $(-10\,\,..\,\,10)$ and the range for the dependent
  variable is set to maximum number of the {\Gnuplot} executable (using double
  floats on most IEEE machines).  Additionally the
  number of interval subdivisions can be assigned as a formal quotient
  \texttt{variable=(lower\_bound\,..\,upper\_bound)\allowbreak /<it>} where
  \texttt{it} is a positive integer. E.g. \texttt{(1 .. 5)/30} means the
  interval from $1$ to $5$ subdivided into $30$ pieces of equal size. A
  subdivision parameter overrides the value of the variable \texttt{points} for
  this variable.
\item A plot option, either as fixed keyword, e.g. \texttt{hidden3d} or as
  equation e.g. \texttt{term=pictex}; free texts such as titles and labels
  should be enclosed in string quotes.
\end{itemize}
Please note that a blank has to be inserted between a number and a dot,
otherwise the \REDUCE{} translator will be misled.
 
If a function is given as an equation the left-hand side is mainly used as a
label for the axis of the dependent variable.

In two dimensions, \texttt{plot} can be called with more than one explicit
function; all curves are drawn in one picture. However, all these must use the
same independent variable name.  One of the functions can be a point set or a
point set list.  Normally all functions and point sets are plotted by lines. A
point set is drawn by points only if functions and the point set are drawn in
one picture.

The same applies to three dimensions with explicit functions. However, an
implicitly given curve must be the sole object for one picture.

The functional expressions are evaluated in \texttt{rounded} mode.  This is done
automatically, it is not necessary to turn on rounded mode explicitly.

\textbf{Examples:}
\begin{verbatim}
plot(cos x);
plot(s=sin phi, phi=(-3 .. 3));
plot(sin phi, cos phi, phi=(-3 .. 3));
plot (cos sqrt(x^2 + y^2), x=(-3 .. 3), y=(-3 .. 3), hidden3d);
plot {{0,0},{0,1},{1,1},{0,0},{1,0},{0,1},{0.5,1.5},{1,1},{1,0}};

% parametric: screw

on rounded;
w := for j := 1:200 collect {1/j*sin j, 1/j*cos j, j/200}$
plot w;

% parametric: globe
dd := pi/15$
w := for u := dd step dd until pi-dd collect
    for v := 0 step dd until 2pi collect
      {sin(u)*cos(v), sin(u)*sin(v), cos(u)}$
plot w;

% implicit: superposition of polynomials
plot((x^2+y^2-9)*x*y = 0);
\end{verbatim}
 

\subsubsection{Piecewise-defined functions}
A composed graph can be defined by a rule-based operator.  In that case each
rule must contain a clause which restricts the rule application to numeric
arguments, e.g.
\begin{verbatim}
   operator my_step1;
   let {my_step1(~x) => -1 when numberp x and x<-pi/2, 
        my_step1(~x) =>  1 when numberp x and x>pi/2,
        my_step1(~x) => sin x
            when numberp x and -pi/2<=x and x<=pi/2};
   plot(my_step2(x));
\end{verbatim}
Of course, such a rule may call a procedure:
\begin{verbatim}
   procedure my_step3(x);
      if x<-1 then -1 else if x>1 then 1 else x;
   operator my_step2;
   let my_step2(~x) => my_step3(x) when numberp x; 
   plot(my_step2(x));
\end{verbatim}
The direct use of a produre with a numeric \texttt{if} clause is impossible. 

\subsubsection{Plot options}
The following plot options are supported in the \texttt{plot} command:
\begin{itemize}
   \item \texttt{points=<integer>}: the number of unconditionally computed data
     points; for a grid \texttt{points\^{}2} grid points are used.  The default
     value is 20. The value of \texttt{points} is used only for variables for
     which no individual interval subdivision has been specified in the range
     specification.
   \item \texttt{refine=<integer>}: the maximum depth of adaptive interval
     intersections. The default is 8. A value 0 switches any refinement
     off. Note that a high value may increase the computing time significantly.
\end{itemize}


\subsubsection{Additional options}

The following additional {\Gnuplot} options are supported in the \texttt{plot} command:
\begin{itemize}
  \item \texttt{title=name}: the title (string) is put at the top of the picture. 
  \item axes labels: \texttt{xlabel="text1", ylabel="text2"}, and for surfaces
    \texttt{zlabel="text3"}. If omitted the axes are labeled by the independent
    and dependent variable names from the expression. Note that \texttt{xlabel},
    \texttt{ylabel}, and \texttt{zlabel} here are used in the usual sense, $x$
    for the horizontal and $y$ for the vertical axis in 2-d and $z$ for the
    perpendicular axis under 3-d -- these names do not refer to the variable
    names used in the expressions.
 \begin{verbatim}
 plot(1,x,(4*x^2-1)/2,(x*(12*x^2-5))/3, x=(-1 .. 1),
      ylabel="L(x,n)", title="Legendre Polynomials");
 \end{verbatim}
  \item \texttt{terminal=name}: prepare output for device type \texttt{name}.
    Every installation uses a default terminal as output device; some
    installations support additional devices such as printers; consult the
    original {\Gnuplot} documentation or the {\Gnuplot} Help for details.
  \item \texttt{output="filename"}: redirect the output to a file.
  \item \texttt{size="s\_x,s\_y"}: rescale the graph (not the window) where
    $s_x$ and $s_y$ are scaling factors for the $x$- and $y$-sizes.  Defaults
    are $s_x=1,x_z=1$.  Note that scaling factors greater than 1 will often
    cause the picture to be too big for the window.
\begin{verbatim}
plot(1/(x^2+y^2), x=(0.1 .. 5), y=(0.1 .. 5), size="0.7,1");
\end{verbatim}
  \item \texttt{view="r\_x,r\_z"}: set the viewpoint in 3 dimensions by turning
    the object around the $x$ or $z$ axis; the values are degrees (integers).
    Defaults are $r_x=60,r_z=30$.
\begin{verbatim}
plot(1/(x^2+y^2), x=(0.1 .. 5), y=(0.1 .. 5), view="30,130");
\end{verbatim}
  \item \texttt{contour} resp. \texttt{nocontour}: in 3 dimensions an additional
    contour map is drawn (default: \texttt{nocontour}).  Note that
    \texttt{contour} is an option which is executed by {\Gnuplot} by
    interpolating the precomputed function values. If you want to draw contour
    lines of a delicate formula, you had better use the contour form of the
    REDUCE \texttt{plot} command.
  \item \texttt{surface} resp. \texttt{nosurface}: in 3 dimensions the surface
    is drawn, resp.  suppressed (default: \texttt{surface}).
  \item \texttt{hidden3d}: hidden line removal in 3 dimensions.
\end{itemize}


\subsection{Paper output}

The following example works for a PostScript printer.  If your printer uses a
different communication, please find the correct setting for the \texttt{terminal}
variable in the {\Gnuplot} documentation.

For a PostScript printer, add the options \texttt{terminal=postscript} and \allowbreak
\texttt{output="filename"} to your plot command, e.g.
\begin{verbatim}
plot(sin x, x=(0 .. 10), terminal=postscript, output="sin.ps");
\end{verbatim}


\subsection{Mesh generation for implicit curves}

The basic mesh for finding an implicitly-given curve, the $x,y$ plane is
subdivided into an initial set of triangles.  Those triangles which have an
explicit zero point or which have two points with different signs are refined by
subdivision.  A further refinement is performed for triangles which do not have
exactly two zero neighbours because such places may represent crossings,
bifurcations, turning points or other difficulties.  The initial subdivision and
the refinements are controlled by the option \texttt{points} which is initially
set to 20: the initial grid is refined unconditionally until
approximately \texttt{points * points} equally-distributed points in the $x,y$
plane have been generated.

The final mesh can be visualized in the picture by setting
\begin{verbatim}
    on show_grid;
\end{verbatim}


\subsection{Mesh generation for surfaces}

By default the functions are computed at predefined mesh points: the ranges are
divided by the number associated with the option \texttt{points} in both
directions.

For two dimensions the given mesh is adaptively smoothed when the curves are too
coarse, especially if singularities are present. On the other hand refinement
can be rather time-consuming if used with complicated expressions. You can
control it with the option \texttt{refine}.  At singularities the graph is
interrupted.

In three dimensions no refinement is possible as {\Gnuplot} supports surfaces only
with a fixed regular grid. In the case of a singularity the near neighborhood is
tested; if a point there allows a function evaluation, its clipped value is used
instead, otherwise a zero is inserted.

When plotting surfaces in three dimensions you have the option of hidden line
removal. Because of an error in Gnuplot 3.2 the axes cannot be labeled correctly
when hidden3d is used ; therefore they aren't labelled at all.  Hidden line
removal is not available with point lists.


\subsection{{\Gnuplot} operation}
\ttindex{PLOTRESET}

The command \texttt{plotreset;} deletes the current {\Gnuplot} output
window. The next call to \texttt{plot} will then open a new one.

If {\Gnuplot} is invoked directly by an output pipe (UNIX and Windows), an eventual
error in the {\Gnuplot} data transmission might cause {\Gnuplot} to quit. As {\REDUCE}
is unable to detect the broken pipe, you have to reset the plot system by
calling the command \texttt{plotreset;} explicitly. Afterwards new graphics
output can be produced.

Under Windows 3.1 and Windows NT, {\Gnuplot} has a text and a graph window.  If you
don't want to see the text window, iconify it and activate the
option \texttt{update wgnuplot.ini} from the graph window system menu - then the
present screen layout (including the graph window size) will be saved and the
text windows will come up iconified in future.  You can also select some more
features there and so tailor the graphic output.  Before you terminate {\REDUCE}
you should terminate the graphic window by calling \texttt{plotreset;}.  If you
terminate {\REDUCE} without deleting the {\Gnuplot} windows, use the command button
from the {\Gnuplot} text window - it offers an exit function.


\subsection{Saving {\Gnuplot} command sequences}
\ttindex{TRPLOT}\ttindex{PLOTKEEP}

If you want to use the internal {\Gnuplot} command sequence more than once
(e.g. for producing a picture for a publication), you may set
\begin{verbatim}
on trplot, plotkeep;
\end{verbatim}
\texttt{trplot} causes all {\Gnuplot} commands to be written additionally to the
actual {\REDUCE} output.  Normally the data files are erased after calling
{\Gnuplot}, however with \texttt{plotkeep} on the files are not erased.


\subsection{Direct Call of {\Gnuplot}}
\ttindextype{{\Gnuplot}}{command}

{\Gnuplot} has a lot of facilities which are not accessed by the operators and
parameters described above. Therefore genuine {\Gnuplot} commands can be sent by
{\REDUCE}.  Please consult the {\Gnuplot} manual for the available commands and
parameters. The general syntax for a {\Gnuplot} call inside {\REDUCE} is
\begin{verbatim}
    gnuplot(<cmd>,<p_1>,<p_2> ...)
\end{verbatim}
where \texttt{cmd} is a command name and $p_1,p_2, \ldots$ are the parameters,
inside {\REDUCE} separated by commas. The parameters are evaluated by {\REDUCE}
and then transmitted to {\Gnuplot} in {\Gnuplot} syntax. Usually a drawing is built by
a sequence of commands which are buffered by {\REDUCE} or the operating
system. For terminating and activating them use the {\REDUCE}
command \texttt{plotshow}.  Example:
\begin{verbatim}
     gnuplot(set,polar);
     gnuplot(set,noparametric);
     gnuplot(plot, x*sin x);
     plotshow;
\end{verbatim}
In this example the function expression is transferred literally to {\Gnuplot},
while {\REDUCE} is responsible for computing the function values
when \texttt{plot} is called.  Note that {\Gnuplot} restrictions with respect to
variable and function names have to be taken into account when using this type
of operation. \textbf{Important}: String quotes are not transferred to the {\Gnuplot}
executable; if the {\Gnuplot} syntax needs string quotes, you must add doubled
stringquotes \emph{inside} the argument string, e.g.
\begin{verbatim}
     gnuplot(plot, """mydata""", "using 2:1");
\end{verbatim}


\subsection{Examples}

The following are taken from a collection of sample plots
(\texttt{gnuplot.tst}) and a set of tests for plotting special
functions. The pictures are made using the \texttt{qt} {\Gnuplot}
device and using the menu of the graphics window to export to PDF or
PNG.

A simple plot for $\sin(1/x)$:
\begin{verbatim}
plot(sin(1/x), x=(-1 .. 1), y=(-3 .. 3));
\end{verbatim}

\unitlength=1cm
\begin{picture}(12,8)(0,0)
%\put(0,0){\Includegraphics[bb=128 93 430 315]{gnuplotex1}}
\put(0,0){\Includegraphics[scale=0.5, natwidth=640, natheight=480]{gnuplotex1}}
\end{picture}

Some implicitly-defined curves:
\begin{verbatim}
plot(x^3 + y^3 - 3*x*y = {0,1,2,3}, x=(-2.5 .. 2), y=(-5 .. 5));
\end{verbatim}
\unitlength=1cm
\begin{picture}(10,8)(0,0)
%\put(-1,-1){\Includegraphics[bb=0 0 360 270]{bild1}}
\put(0,0){\Includegraphics[scale=0.5, natwidth=640, natheight=480]{bild1}}
\end{picture}

\newpage
A test for hidden surfaces:
\begin{verbatim}
plot(cos sqrt(x^2 + y^2), x=(-3 .. 3), y=(-3 .. 3), hidden3d);
\end{verbatim}

\begin{picture}(12,8)(0,0)
%\put(0,0){\Includegraphics[bb=50 0 350 220]{gnuplotex2}}
\put(0,0){\Includegraphics[scale=0.5, natwidth=640, natheight=480]{gnuplotex2}}
\end{picture}

This may be slow on some machines because of a delicate evaluation context:
\begin{verbatim}
plot(sinh(x*y)/sinh(2*x*y), hidden3d);
\end{verbatim}

\begin{picture}(12,8)(0,0)
%\put(0,0){\Includegraphics[bb=128 93 430 315]{gnuplotex3}}
\put(0,0){\Includegraphics[scale=0.5, natwidth=640, natheight=480]{gnuplotex3}}
\end{picture}

\newpage

\begin{verbatim}
on rounded;
w:= {for j:=1 step 0.1 until 20 collect {1/j*sin j, 1/j*cos j, j},
     for j:=1 step 0.1 until 20 collect
       {(0.1+1/j)*sin j, (0.1+1/j)*cos j, j} }$
plot w;
\end{verbatim}
\begin{picture}(12,8)(0,0)
%\put(0,0){\Includegraphics[bb=127 93 429 314]{gnuplotex4}}
\put(0,0){\Includegraphics[scale=0.5, natwidth=640, natheight=480]{gnuplotex4}}
\end{picture}

An example taken from: Cox, Little, O'Shea, \emph{Ideals, Varieties and Algorithms}:
\begin{verbatim}
plot(point(3u+3u*v^2-u^3, 3v+3u^2*v-v^3, 3u^2-3v^2), hidden3d, 
     title="Enneper Surface");
\end{verbatim}

\begin{picture}(10,7)(0,1)
%\put(1,1){\Includegraphics[bb=69 69 319 246]{bild2}}
\put(0,0){\Includegraphics[scale=0.5, natwidth=640, natheight=480]{bild2}}
\end{picture}

\newpage

The following examples use the \texttt{specfn} package to draw a collection of
Chebyshev T polynomials and Bessel Y functions.
The special function package has to be loaded explicitely
to make the operator ChebyshevT and BesselY available.

%\newpage
\begin{verbatim}
load_package specfn;
plot(chebyshevt(1,x), chebyshevt(2,x), chebyshevt(3,x),
     chebyshevt(4,x), chebyshevt(5,x),
     x=(-1 .. 1), title="Chebyshev t Polynomials");
\end{verbatim}

\begin{picture}(12,7.5)(0,0)
%\put(0,0){\Includegraphics[bb=128 93 430 315]{gnuplotex5}}
\put(0,0){\Includegraphics[scale=0.45, natwidth=640, natheight=480]{gnuplotex5}}
\end{picture}
\enlargethispage{1cm}
\begin{verbatim}
plot(bessely(0,x), bessely(1,x), bessely(2,x), x=(0.1 .. 10),
     y=(-1 .. 1), title="Bessel functions of 2nd kind");
\end{verbatim}

\begin{picture}(12,7.5)(0,0)
%\put(0,0){\Includegraphics[bb=128 93 430 315]{gnuplotex6}}
\put(0,0){\Includegraphics[scale=0.45, natwidth=640, natheight=480]{gnuplotex6}}
\end{picture}
