\index{derivatives} 
\index{partial derivatives}
\index{generic function}

The package {\tt DFPART} supports computations with total and partial
derivatives of formal function objects. Such computations can
be useful in the context of differential equations or
power series expansions.

\subsection{Generic Functions}

\ttindex{GENERIC\_FUNCTION}
A generic function is a symbol which represents a mathematical
function. The minimal information about a generic function 
function is the number of its arguments. In order to facilitate
the programming and for a better readable output this package
assumes that the arguments of a generic function have default
names such as $f(x,y)$,$q(rho,phi)$. 
A generic function is declared by prototype form in a statement

\vspace{.1in}
 \f{GENERIC\_FUNCTION} $fname(arg_1,arg_2\cdots arg_n)$;
\vspace{.1in}

\noindent
where $fname$ is the (new) name of a function and $arg_i$ are
symbols for its formal arguments. 
In the following $fname$ is referred to as ``generic function",
$arg_1,arg_2\cdots arg_n$ as ``generic arguments" and
$fname(arg_1,arg_2\cdots arg_n)$ as ``generic form".
Examples:

\begin{verbatim}
   generic_function f(x,y);
   generic_function g(z);
\end{verbatim}


After this declaration {\REDUCE} knows that 
\begin{itemize}
\item there are formal partial derivatives $\frac{\partial f}{\partial x}$,
$\frac{\partial f}{\partial y}$ $\frac{\partial g}{\partial z}$
and higher ones, while partial derivatives of $f$ and $g$
with respect to other variables are assumed as zero,
\item expressions of the type $f()$, $g()$ are abbreviations for
$f(x,y)$, $g(z)$,
\item expressions of the type $f(u,v)$ are abbreviations for\\
$sub(x=u,y=v,f(x,y))$
\item a total derivative $\frac{d f(u,v)}{d w}$ has to be computed
as $\frac{\partial f}{\partial x} \frac{d u}{d w} +
    \frac{\partial f}{\partial y} \frac{d v}{d w}$
\end{itemize}

\subsection{Partial Derivatives}

\ttindex{DFP}
The operator {\tt DFP} represents a partial derivative:

\vspace{.1in}
 {\tt DFP}($expr,{dfarg_1,dfarg_2\cdots dfarg_n}$);
\vspace{.1in}

\noindent
where $expr$ is a function expression and $dfarg_i$ are
the differentiation variables. Examples:

\begin{verbatim}
    dfp(f(),{x,y});
\end{verbatim}
means $\frac{\partial ^2 f}{\partial x \partial y}$ and
\begin{verbatim}
    dfp(f(u,v),{x,y});
\end{verbatim}
stands for $\frac{\partial ^2 f}{\partial x \partial y} (u,v)$.
For compatibility with the $DF$ operator the differentiation
variables need not be entered in list form; instead the syntax
of {\tt DF} can be used, where the function expression is followed
by the differentiation variables, eventually with repetition
numbers. Such forms are interenally converted to the above
form with a list as second parameter.

The expression $expr$ can be a generic function
with or without arguments, or an arithmetic expression built
from generic functions and other algebraic parts. In the
second case the standard differentiation rules are applied
in order to reduce each derivative expressions to a minimal
form.

When the switch {\tt NAT} is on partial derivatives of generic
functions are printed in standard index notation, that is
$f_{xy}$ for $\frac{\partial ^2 f}{\partial x \partial y}$
and $f_{xy}(u,v)$ for $\frac{\partial ^2 f}{\partial x \partial y}(u,v)$.
Therefore single characters should be used for the arguments
whenever possible. Examples:

\begin{verbatim}

  generic_function f(x,y);
  generic_function g(y);
  dfp(f(),x,2);

  F
   XX

  dfp(f()*g(),x,2);

  F  *G()
   XX

  dfp(f()*g(),x,y);

  F  *G() + F *G
   XY        X  Y

\end{verbatim}

The difference between partial and total derivatives is
illustrated by the following example:

\begin{verbatim}
  generic_function h(x);
  dfp(f(x,h(x))*g(h(x)),x);

  F (X,H(X))*G(H(X))
   X

  df(f(x,h(x))*g(h(x)),x);

  F (X,H(X))*G(H(X)) + F (X,H(X))*H (X)*G(H(X))
   X                    Y          X

   + G (H(X))*H (X)*F(X,H(X))
      Y        X
\end{verbatim}

Cooperation of partial derivatives and Taylor series under
a differential side relation $\frac{dq}{dx}=f(x,q)$:

\begin{verbatim}
  load_package taylor;
  operator q; 
  let df(q(~x),x) => f(x,q(x));
  taylor(q(x0+h),h,0,3);

                         F (X0,Q(X0)) + F (X0,Q(X0))*F(X0,Q(X0))
                          X              Y                         2
Q(X0) + F(X0,Q(X0))*H + -----------------------------------------*H
                                            2

+ (F  (X0,Q(X0)) + F  (X0,Q(X0))*F(X0,Q(X0))
    XX              XY

    + F (X0,Q(X0))*F (X0,Q(X0)) + F  (X0,Q(X0))*F(X0,Q(X0))
       X            Y              YX

                               2               2                 3
    + F  (X0,Q(X0))*F(X0,Q(X0))  + F (X0,Q(X0)) *F(X0,Q(X0)))/6*H
       YY                           Y

      4
 + O(H )

\end{verbatim}

Normally partial differentials are assumed as non-commutative

\begin{verbatim}
  dfp(f(),x,y)-dfp(f(),y,x);

  F   - F
   XY    YX
\end{verbatim}

However, a generic function can be declared to have globally
interchangeable partial derivatives using the declaration 
{\tt DFP\_COMMUTE}
which takes the name of a generic function or a generic function
form as argument. For such a function differentiation variables are
rearranged corresponding to the sequence of the generic variables.

\begin{verbatim}
  generic_function q(x,y);
  dfp_commute q(x,y);
  dfp(q(),{x,y,y}) + dfp(q(),{y,x,y}) + dfp(q(),{y,y,x});

  3*Q
     XYY
\end{verbatim}

If only a part of the derivatives commute, this has to be
declared using the standard {\REDUCE} rule mechanism. Please
note that then the derivative variables must be written as
list.

\subsection{Substitutions}

When a generic form or a {\tt DFP} expression takes part in a 
substitution the following steps are performed:
\begin{enumerate}
\item The substitutions are performed for the arguments. If the
argument list is empty the substitution is applied to the
generic arguments of the function; if these change, the resulting
forms are used as new actual arguments.
If the generic function itself is not affected by the substitution,
the process stops here.
\item If the function name or the generic function
form occurs as a left hand side in the substitution list,
it is replaced by the corresponding right hand side.
\item The new form is partially differentiated according to the
list of partial derivative variables.
\item The (eventually modified) actual parameters are substituted
into the form for their corresponding generic variables.
This substitution is done by name.
\end{enumerate}

Examples:
\begin{verbatim}
  generic_function f(x,y);
  sub(y=10,f());
 
  F(X,10)

  sub(y=10,dfp(f(),x,2));


  F  (X,10)
   XX

  sub(y=10,dfp(f(y,y),x,2));

  F  (10,10)
   XX

  sub(f=x**3*y**3,dfp(f(),x,2));

       3
  6*X*Y

  generic_function ff(y,z);
  sub(f=ff,f(a,b));

  FF(B,Z)
\end{verbatim}

The dataset \texttt{dfpart.tst} contains more examples,
including a complete application for computing the coefficient
equations for Runge-Kutta ODE solvers.

