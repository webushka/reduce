
{\bf LIE} is a package of functions for the classification of real n-dimensional
Lie algebras. It consists of two modules: {\bf liendmc1} and {\bf lie1234}.
\\[0.3cm]{\large\bf liendmc1}\\[0.1cm]
With the help of the functions in this module real n-dimensional Lie algebras
$L$ with a derived algebra $L^{(1)}$ of dimension 1 can be classified. $L$ has
to be defined by its structure constants $c_{ij}^k$ in the basis
$\{X_1,\ldots,X_n\}$ with $[X_i,X_j]=c_{ij}^k X_k$. The user must define an
ARRAY LIENSTRUCIN($n,n,n$) with n being the dimension of the Lie algebra $L$.
The structure constants LIENSTRUCIN($i,j,k$):=$c_{ij}^k$ for $i<j$ should be
given. Then the procedure LIENDIMCOM1 can be called. Its syntax is:
\begin{verbatim}
   LIENDIMCOM1(<number>).
\end{verbatim}
{\tt <number>} corresponds to the dimension $n$. The procedure simplifies
the structure of $L$ performing real linear transformations. The returned
value is a list of the form
\begin{verbatim}
   (i) {LIE_ALGEBRA(2),COMMUTATIVE(n-2)} or
   (ii) {HEISENBERG(k),COMMUTATIVE(n-k)}
\end{verbatim}
with $3\leq k\leq n$, $k$ odd.\\
The concepts correspond to the following theorem ({\tt LIE\_ALGEBRA(2)}
$\rightarrow L_2$, {\tt HEISENBERG(k)} $\rightarrow H_k$ and
{\tt COMMUTATIVE(n-k)} $\rightarrow C_{n-k}$):\\[0.2cm]
{\bf Theorem.} Every real $n$-dimensional Lie algebra $L$ with a 1-dimensional
derived algebra can be decomposed into one of the following forms:\\[0.1cm]
\hspace*{0.3cm} (i) $C(L)\cap L^{(1)}=\{0\}\, :\; L_2\oplus C_{n-2}$
or\\[0.05cm]
\hspace*{0.3cm} (ii) $C(L)\cap L^{(1)}=L^{(1)}\, :\; H_k\oplus C_{n-k}\quad
(k=2r-1,\, r\geq 2)$, with\newpage
\hspace*{0.3cm} 1. $C(L)=C_j\oplus (L^{(1)}\cap C(L))$
and dim$\,C_j=j$ ,\\[0.05cm]
\hspace*{0.3cm} 2. $L_2$ is generated by
$Y_1,Y_2$ with $[Y_1,Y_2]=Y_1$ ,\\[0.05cm]
\hspace*{0.3cm} 3. $H_k$ is generated by $\{Y_1,\ldots,Y_k\}$ with\\
\hspace*{0.7cm} $[Y_2,Y_3]=\cdots =[Y_{k-1},Y_k]=Y_1$.\\[0.2cm]
(cf. \cite{cssmp92})\\[0.2cm]
The returned list is also stored as LIE\_LIST. The matrix LIENTRANS gives the
transformation from the given basis $\{X_1,\ldots ,X_n\}$ into the standard
basis $\{Y_1,\ldots ,Y_n\}$: $Y_j=($LIENTRANS$)_j^k X_k$.\\[0.1cm]
A more detailed output can be obtained by turning on the switch TR\_LIE:
\begin{verbatim}
   ON TR_LIE;
\end{verbatim}
before the procedure LIENDIMCOM1 is called.\\[0.1cm]
The returned list could be an input for a data bank in which mathematical
relevant properties of the obtained Lie algebras are stored.\\[0.3cm]
{\large\bf lie1234}\\[0.1cm]
This part of the package classifies real low-dimensional Lie algebras $L$
of the dimension
$n:=$dim$\,L=1,2,3,4$. $L$ is also given by its structure constants $c_{ij}^k$
in the basis $\{X_1,\ldots,X_n\}$ with $[X_i,X_j]=c_{ij}^k X_k$. An ARRAY
LIESTRIN($n,n,n$) has to be defined and LIESTRIN($i,j,k$):=$c_{ij}^k$ for
$i<j$ should be given. Then the procedure LIECLASS can be performed
whose syntax is:
\begin{verbatim}
   LIECLASS(<number>).
\end{verbatim}
{\tt <number>} should be the dimension of the Lie algebra $L$. The procedure
stepwise simplifies the commutator relations of $L$ using properties of
invariance like the dimension of the centre, of the derived algebra,
unimodularity etc.  The returned value has the form:
\begin{verbatim}
   {LIEALG(n),COMTAB(m)},
\end{verbatim}
where $m$ corresponds to the number of the standard form (basis:
$\{Y_1,\ldots,Y_n\}$) in an enumeration scheme. The corresponding enumeration
schemes are listed below (cf. \cite{ntz-preprint27/92},\cite{mmpreprint1979}).
In case that the standard form in the enumeration scheme depends on one (or two)
parameter(s) $p_1$ (and $p_2$) the list is expanded to:
\begin{verbatim}
   {LIEALG(n),COMTAB(m),p1,p2}.
\end{verbatim}
This returned value is also stored as LIE\_CLASS. The linear transformation from
the basis $\{X_1,\ldots,X_n\}$ into the basis of the standard form
$\{Y_1,\ldots,Y_n\}$ is given by the matrix LIEMAT:
$Y_j=($LIEMAT$)_j^k X_k$.\newpage
By turning on the switch TR\_LIE:
\begin{verbatim}
   ON TR_LIE;
\end{verbatim}
before the procedure LIECLASS is called the output contains not only the
list LIE\_CLASS but also the non-vanishing commutator relations in the
standard form.\\[0.1cm]
By the value $m$ and the parameters further examinations of the Lie algebra
are possible, especially if in a data bank mathematical relevant properties
of the enumerated standard forms are stored.\\[0.3cm]
{\large\bf Enumeration schemes for lie1234}\\[0.2cm]
\hspace*{0.3cm}\begin{tabular}{l|l}returned list LIE\_CLASS&
the corresponding commutator relations\\[0.1cm]\hline
{LIEALG(1),COMTAB(0)}&commutative case\\[0.1cm]\hline
{LIEALG(2),COMTAB(0)}&commutative case\\[0.1cm]
{LIEALG(2),COMTAB(1)}&$[Y_1,Y_2]=Y_2$\\[0.1cm]\hline
{LIEALG(3),COMTAB(0)}&commutative case\\[0.1cm]
{LIEALG(3),COMTAB(1)}&$[Y_1,Y_2]=Y_3$\\[0.1cm]
{LIEALG(3),COMTAB(2)}&$[Y_1,Y_3]=Y_3$\\[0.1cm]
{LIEALG(3),COMTAB(3)}&$[Y_1,Y_3]=Y_1,[Y_2,Y_3]=Y_2$\\[0.1cm]
{LIEALG(3),COMTAB(4)}&$[Y_1,Y_3]=Y_2,[Y_2,Y_3]=Y_1$\\[0.1cm]
{LIEALG(3),COMTAB(5)}&$[Y_1,Y_3]=-Y_2,[Y_2,Y_3]=Y_1$\\[0.1cm]
{LIEALG(3),COMTAB(6)}&$[Y_1,Y_3]=-Y_1+p_1 Y_2,[Y_2,Y_3]=Y_1,p_1\neq 0$\\[0.1cm]
{LIEALG(3),COMTAB(7)}&$[Y_1,Y_2]=Y_3,[Y_1,Y_3]=-Y_2,[Y_2,Y_3]=Y_1$\\[0.1cm]
{LIEALG(3),COMTAB(8)}&$[Y_1,Y_2]=Y_3,[Y_1,Y_3]=Y_2,[Y_2,Y_3]=Y_1$\\[0.1cm]\hline
{LIEALG(4),COMTAB(0)}&commutative case\\[0.1cm]
{LIEALG(4),COMTAB(1)}&$[Y_1,Y_4]=Y_1$\\[0.1cm]
{LIEALG(4),COMTAB(2)}&$[Y_2,Y_4]=Y_1$\\[0,1cm]
{LIEALG(4),COMTAB(3)}&$[Y_1,Y_3]=Y_1,[Y_2,Y_4]=Y_2$\\[0.1cm]
{LIEALG(4),COMTAB(4)}&$[Y_1,Y_3]=-Y_2,[Y_2,Y_4]=Y_2,$\\
                     &$[Y_1,Y_4]=[Y_2,Y_3]=Y_1$\\[0.1cm]
{LIEALG(4),COMTAB(5)}&$[Y_2,Y_4]=Y_2,[Y_1,Y_4]=[Y_2,Y_3]=Y_1$\\[0.1cm]
{LIEALG(4),COMTAB(6)}&$[Y_2,Y_4]=Y_1,[Y_3,Y_4]=Y_2$\\[0.1cm]
{LIEALG(4),COMTAB(7)}&$[Y_2,Y_4]=Y_2,[Y_3,Y_4]=Y_1$\\[0.1cm]
{LIEALG(4),COMTAB(8)}&$[Y_1,Y_4]=-Y_2,[Y_2,Y_4]=Y_1$\\[0.1cm]
{LIEALG(4),COMTAB(9)}&$[Y_1,Y_4]=-Y_1+p_1 Y_2,[Y_2,Y_4]=Y_1,p_1\neq 0$\\[0.1cm]
{LIEALG(4),COMTAB(10)}&$[Y_1,Y_4]=Y_1,[Y_2,Y_4]=Y_2$\\[0.1cm]
{LIEALG(4),COMTAB(11)}&$[Y_1,Y_4]=Y_2,[Y_2,Y_4]=Y_1$
\end{tabular}\\
\hspace*{0.3cm}\begin{tabular}{l|l}returned list LIE\_CLASS&
the corresponding commutator relations\\[0.1cm]\hline
{LIEALG(4),COMTAB(12)}&$[Y_1,Y_4]=Y_1+Y_2,[Y_2,Y_4]=Y_2+Y_3,$\\
                      &$[Y_3,Y_4]=Y_3$\\[0.1cm]
{LIEALG(4),COMTAB(13)}&$[Y_1,Y_4]=Y_1,[Y_2,Y_4]=p_1 Y_2,[Y_3,Y_4]=p_2 Y_3,$\\
                      &$p_1,p_2\neq 0$\\[0.1cm]
{LIEALG(4),COMTAB(14)}&$[Y_1,Y_4]=p_1 Y_1+Y_2,[Y_2,Y_4]=-Y_1+p_1 Y_2,$\\
                      &$[Y_3,Y_4]=p_2 Y_3,p_2\neq 0$\\[0.1cm]
{LIEALG(4),COMTAB(15)}&$[Y_1,Y_4]=p_1 Y_1+Y_2,[Y_2,Y_4]=p_1 Y_2,$\\
                      &$[Y_3,Y_4]=Y_3,p_1\neq 0$\\[0.1cm]
{LIEALG(4),COMTAB(16)}&$[Y_1,Y_4]=2 Y_1,[Y_2,Y_3]=Y_1,$\\
                      &$[Y_2,Y_4]=(1+p_1) Y_2,[Y_3,Y_4]=(1-p_1) Y_3,$\\
                      &$p_1\geq 0$\\[0.1cm]
{LIEALG(4),COMTAB(17)}&$[Y_1,Y_4]=2 Y_1,[Y_2,Y_3]=Y_1,$\\
                      &$[Y_2,Y_4]=Y_2-p_1 Y_3,[Y_3,Y_4]=p_1 Y_2+Y_3,$\\
                      &$p_1\neq 0$\\[0.1cm]
{LIEALG(4),COMTAB(18)}&$[Y_1,Y_4]=2 Y_1,[Y_2,Y_3]=Y_1,$\\
                      &$[Y_2,Y_4]=Y_2+Y_3,[Y_3,Y_4]=Y_3$\\[0.1cm]
{LIEALG(4),COMTAB(19)}&$[Y_2,Y_3]=Y_1,[Y_2,Y_4]=Y_3,[Y_3,Y_4]=Y_2$\\[0.1cm]
{LIEALG(4),COMTAB(20)}&$[Y_2,Y_3]=Y_1,[Y_2,Y_4]=-Y_3,[Y_3,Y_4]=Y_2$\\[0.1cm]
{LIEALG(4),COMTAB(21)}&$[Y_1,Y_2]=Y_3,[Y_1,Y_3]=-Y_2,[Y_2,Y_3]=Y_1$\\[0.1cm]
{LIEALG(4),COMTAB(22)}&$[Y_1,Y_2]=Y_3,[Y_1,Y_3]=Y_2,[Y_2,Y_3]=Y_1$
\end{tabular}


\begin{thebibliography}{1}

\bibitem{mmpreprint1979}
M.A.H. MacCallum.
\newblock On the classification of the real four-dimensional lie algebras.
\newblock 1979.

\bibitem{cssmp92}
C.~Schoebel.
\newblock Classification of real n-dimensional lie algebras with a
  low-dimensional derived algebra.
\newblock In {\em Proc. {Symposium on Mathematical Physics} '92}, 1993.

\bibitem{ntz-preprint27/92}
F.~Schoebel.
\newblock The symbolic classification of real four-dimensional lie algebras.
\newblock 1992.

\end{thebibliography}

