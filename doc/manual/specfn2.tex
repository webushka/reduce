
The (generalised) hypergeometric functions  
\begin{displaymath}
_pF_q \left( {{a_1, \ldots , a_p} \atop {b_1, \ldots ,b_q}} \Bigg\vert z \right)
\end{displaymath}
are defined in textbooks on special functions as
\begin{displaymath}
_pF_q \left( {{a_1, \ldots , a_p} \atop {b_1, \ldots ,b_q}} \Bigg\vert z \right)
  = \sum_{n=0}^{\infty}\frac{(a_{1})_{n}\dots(a_{p})_{n}}{(b_{1})_{n}\dots(b_{q})_{n}}\frac{z^{n}}{n!}
\end{displaymath}w
where $(a)_{n}$ is the Pochhammer symbol
\begin{displaymath}
 (a)_{n} = \prod_{k=0}^{n-1} (a+k)
\end{displaymath}


\subsection{\REDUCE{} operator HYPERGEOMETRIC}
\hypertarget{operator:HYPERGEOMETRIC}{}

The operator {\tt hypergeometric} expects 3 arguments, namely the 
list of upper parameters (which may be empty), the list of lower
parameters (which may be empty too), and the argument, e.g the input:
\begin{verbatim}
hypergeometric ({},{},z);
\end{verbatim}
yields the output
\begin{verbatim}
 z
e
\end{verbatim}
and the input
\begin{verbatim}
hypergeometric ({1/2,1},{3/2},-x^2);
\end{verbatim}
gives
\begin{verbatim}
 atan(abs(x))
--------------
    abs(x)
\end{verbatim}


\subsection{Extending the HYPERGEOMETRIC operator}

Since hundreds of particular cases for the generalised hypergeometric
functions can be found in the literature, one cannot expect that all
cases are known to the \texttt{hypergeometric} operator.
Nevertheless the set of special cases can be augmented by adding
rules to the \REDUCE{} system, {\em e.g.}
\begin{verbatim}
let {hypergeometric({1/2,1/2},{3/2},-(~x)^2) => asinh(x)/x};
\end{verbatim}
