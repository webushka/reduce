\documentclass[a4paper,11pt]{article}
\title{CSL reference}
\author{A C Norman}
\begin{document}
\maketitle


\section{Introduction}
This is reference material for CSL. The Lisp identifiers mentioned here are
the ones that are initially present in a raw CSL image. Some proportion of them
are not really intended to be used by end-users but are merely the
internal components of some feature.

\section{Command-line options}

The items shown here are the ones that are recognized on the CSL command
line. In general an option that requires an argument can be written as either
{\ttfamily -x yyy} or as {\ttfamily -xyyy}. Arguments should be case
insensitive. 

\subsection{\ttfamily -a}
{\ttfamily -a} is a curious option, not intended for general or casual use.
If given it causes the {\ttfamily (batchp)} function to return the opposite
result from normal!  Without ``{attfamily -a}'' {\ttfamily (batchp)} returns
{\ttfamily T} either if at least one file was specified on the command line,
or if the standard input is ``not a tty'' (under some operating systems this
makes sense -- for instance the standard input might not be a ``tty'' if it
is provided via file redirection).  Otherwise (ie primary input is directly
from a keyboard) {\ttfamily (batchp)} returns {\ttfamily nil}.  Sometimes
this judgement about how ``batch'' the current run is will be wrong or
unhelpful, so {\ttfamily -a} allows the user to coax the system into better
behaviour.  I hope that this is never used!

\subsection{\ttfamily -b}
{\ttfamily -b} tells the system to avoid any attempt to recolour prompts and
input text. It will mainly be needed on X terminals that have been set up
so that they use colours that make the defaults here unhelpful.
Specifically white-on-black and so on.

{\ttfamily -b} can be followed by colour specifications to make things yet
more specific. It is supposed to be the idea that three colours can be
specified after it for output, input and prompts, with the letters KRGYbMCW
standing for blacK, Red, Green, Yellow, blue, Magenta, Cyan and White. This may
not fully work yet!

\subsection{\ttfamily -c}
Displays a notice retalting to the authorship of CSL.
\subsection{\ttfamily -d}
A command line entry {\ttfamily -Dname=value} or {\ttfamily -D name=value}
sets the value of the named lisp variable to the value (as a string).
\subsection{\ttfamily -e}
A ``spare'' option used from tim eto time to activate experiments within CSL.
\subsection{\ttfamily -f}
At one stage CSL could run as a socket server, and {\ttfamily -f portnumber}
activated that mode. {\ttfamily -f-} used a default port, 1206 (a number
inspired by an account number on Titan that I used in the 1960s). The code
that supports this may be a useful foundation to others who want to make a
network service out of this code-base.
\subsection{\ttfamily -g}
In line with the implication of this option for C compilers, this enables
a debugging mode. It sets a lisp variable {\ttfamily !*backtrace} and arranges
that all backtraces are displayed notwithstanding use of {\ttfamily errorset}.
\subsection{\ttfamily -h}
This option is a left-over. When the X-windows version of the code first
started to use Xft it viewed that as optional and could allow a build even when
it was not available. And then even if Xft was detected and liable to be used
by default it provided this option to disable its use. The remnants of the
switch that disabled use of Xft (relating to fonts living on the Host or
the Server) used this switch, but it now has no effect.
\subsection{\ttfamily -i}
CSL and Reduce use image files to keep both initial heap images and
``fasl'' loadable modules. By default if the executable launched has some name,
say xxx, then an image file xxx.img is used. But to support greater
generality {\ttfamily -i} introduces a new image, {\ttfamily -i-} indicates
the default one and a sequence of such directives list image files that are
searche din the order given. These are read-only. The similar option
{\ttfamily -o} equally introduces image files that are scanned for input, but
that can also be used for output. Normally there would only be one
{\ttfamily -o} directive.
\subsection{\ttfamily -j}
Follow this directive with a file-name, and a record of all the files read
during the Lisp run will be dumped there with a view that it can be included
in a Makefile to document dependencies.
\subsection{\ttfamily -k}
{\ttfamily -K nnn} sets the size of heap to be used.  If it is given then that much
memory will be allocated and the heap will never expand.  Without this
option a default amount is used, and (on many machines) it will grow
if space seems tight.

The extended version of this option is {\ttfamily -K nnn/ss} and then ss is the
number of ``CSL pages'' to be allocated to the Lisp stack. The default
value (which is 1) should suffice for almost all users, and it should
be noted that the C stack is separate from and independent of this one and
it too could overflow.

A suffix K, M or G on the number indicates units of kilobytes, megabytes or
gigabytes, with megabytes being the default. So {\ttfamily -K200M} might
represent typical usage.
\subsection{\ttfamily -l}
This is to send a copy of the standard output to a named log file. It is
very much as if the Lisp function {\ttfamily (spool ``logfile'')} had been
invoked at the start of the run.
\subsection{\ttfamily -m}
Memory trace mode. An option that represents an experiment from the past,
and no longer reliably in use.
\subsection{\ttfamily -n}
Normally when the system is started it will run a ``restart function'' as
indicated in its heap image. There can be cases where a heap image has been
created in a bad way such that the saved restart function always fails
abruptly, and hence working out what was wrong becomes hard. In such cases
it may be useful to give the {\ttfamily -n} option that forces CSL to
ignore any startup function and merely always begin in a minimal Lisp-style
read-eval-print loop.
\subsection{\ttfamily -o}
See {\ttfamily -i}.
\subsection{\ttfamily -p}
If a suitable profile option gets implemented one day this will activate it,
but for now it has no effect.
\subsection{\ttfamily -q}
This option sets {\ttfamily !*echo} to {\ttfamily nil} and switches off
garbage collector messages to give a slightly quieter run.
\subsection{\ttfamily -r}
The random-number generator in CSL is normally initialised to a value
based on the time of day and is hence not reproducible from run to run.
In many cases that behavious is desirable, but for debugging it can be useful
to force a seed. The directive {\ttfamily -r nnn,mmm} sets the seed to
up to 64 bits taken from the values nnn and mmm. The second value if optional,
and specifying {\ttfamily -r0}  explicitly asks for the non-reproducible
behaviour (I hope). Note that the main Reduce-level random number source is
coded at a higher level and does not get reset this way -- this is the
lower level CSL generator.
\subsection{\ttfamily -s}
Sets the Lisp variable {\ttfamily !*plap} and hence the compiler generates
an assembly listing.
\subsection{\ttfamily -t}
{\ttfamily -t name} reports the time-stamp on the named module, and then
exits. This is for use in perl scripts and the like, and is
needed because the stamps on modules within an image or
library file are not otherwise instantly available.

Note that especially on windowed systems it may be necessary to use this
with {\ttfamily -- filename} since the information generated here goes to
the default output, which in some cases is just the screen.
\subsection{\ttfamily -u}
See {\ttfamily -d}, but this forcibly undefines a symbol. There are probably
very very few cases where it is useful since I do not have a large
number of system-specific predefined names.
\subsection{\ttfamily -v}
An option to make things mildly more verbose. It displays more of a banner
at startup and switches garbage collection messages on.
\subsection{\ttfamily -w}
On a typical system if the system is launched it creates a new window and uses
its own windowed intarface in that. If it is run such that at startup the
standard input or output are associated with a file or pipe, or under X the
variable {\ttfamily DISPLAY} is not set it will try to start up in console
mode. The flag {\ttfamily -w} indicates that the system should run in console
more regadless, while {\ttfamily -w+} attempts a window even if that seems
doomed to failure. When running the system to obey a script it will often make
sense to use the {\ttfamily -w} option. Note that on Windows the system is
provided as two separate (but almost identical) binaries. For example the
file {\ttfamily csl.exe} is linked in windows mode. A result is that if
launched from the command line it detaches from its console, and if launched
by double-clicking it does not create a console. It is in fact very ugly when
double clicking on an application causes an unwanted console window to appear.
In contrast {\ttfamily csl.com} is a console mode version of just the same
program, so when launched from a command line it can communicate with the
console in the ordinary expected manner.
\subsection{\ttfamily -x}
{\ttfamily -x} is an option intended for use only by system
support experts -- it disables trapping if segment violations by
errorset and so makes it easier to track down low level disasters --
maybe!  This can be valuable when running under a debugger since if the
code traps signals in its usual way and tries to recover it can make it a lot
harder to find out just what was going wrong.
\subsection{\ttfamily -y}
{\ttfamily -y } sets the variable {\ttfamily !*hankaku}, which causes the
lisp reader convert a Zenkaku code to Hankaku one when read. I leave this
option decoded on the command line even if the Kanji support code is not
otherwise compiled into CSL just so I can reduce conditional compilation.
This was part of the Internationalisation effort for CSL bu this is no longer
supported. 
\subsection{\ttfamily -z}
When bootstrapping it is necessary to start up the system for one initial time
without the benefit of any image file at all. The option {\ttfamily -z} makes
this happen, so when it is specified the system starts up with a minimal
environment and only those capabilities that are present in the CSL
kernel. It will normally make sense to start loading some basic Lisp
definitions rather rapidly. The files {\ttfamily compat.lsp},
{\ttfamily extras.lsp} and {\ttfamily compiler.lsp} have Lisp source for the
main things I use, and once they are loaded the Lisp compiler can be used
to compile itself.
\subsection{\ttfamily --dump-source}
The linked executable of CSL may include some libraries that were licensed
under the Lesser GNU Public License version 2.1. Even if the linking had
been dynamic it would be necessary for the binary package that a user would
need in order to use CSL to contain a binary copy of the relevant library, and
when a copy of such a binary is provided to another it must be accompanied with
at least an offer of provision of the associated source code. The LGPL
asserts that if is not necessary to force the source code onto any recipient,
but there are two reasons why that seems to me to be the only viable way to
ensure license compliance in a manner that is at all convenient:
\begin{itemize}
\item If a binary is distributed without accompanying source then you have to
commit to being able to provide the source corresponding to that exact version
(neither earlier nor later) to the person who receieved the file if they ask for
it at a time up to some years in the future. Maintaining confidence that I can
track alterations and guarantee to deliver an exact version as required seems
to me in conflict with wishing to ship snapshots, trial versions and generally
behave in a flexible manner.
\item If an individual fetches any material, binary or source, associated with
this project I would like them to be able to pass it on to others. If they
fetch just a binary then they can not pass that on unless they commit to
providing its associated source. For LGPL material they can not rely on the
source being available from me as an upstream supplier. Now some will say that
the issue of what that individual does is their own responsibility, but is is
mine to ensure that they know what they may and may not do, and I do not want
to have to explain that stuff they receive from me can not be passed on to
their friends without them taking special extra steps ``to preserve freedom''.
\end{itemize}

Because of these constraints and to satisfy my own view about what ``freedom''
is I therefore arrange that the ``minimal sources'' as envisaged in the LGPL
are formed into an archive which is included as a resource within the CSL or
Reduce executable. If the executable is run from a command line with the
{\ttfamily --dump-source} option (which can be followed by a file-name where
things should be places) it writes this archive to a file. Anybody can then
unpack it using standard tools and find both detailed license terms and all
the C-coded source they could possibly require.

This adds about 3 megabytes to the size of each executable, but is the best
way I have identified of meeting LGPL requirements while preserving the freedom
of all users to distribute and redistribute unmodified binaries without any
constrains that they might fail to understand or adhere to.
\subsection{\ttfamily --help}
It is probably obvious what this option does! But in particular it displays
and explantion of the {\ttfamily --dump-source} option, and hence should
count as a prominent and easy-to-find way of alerting people to their rights
and obligations. Note that on Windows of the application was linked as a
windows binary it carefully creates a console to display the help text
in, and organizes a delay to give people a chance to read it.
\subsection{\ttfamily --my-path}
At some time I had felt the need for this option, but I now forget what I
expected to use it for! It leads the executable to display the fully
rooted name of the directory it was in and then terminate. It may be useful
in some script?
\subsection{\ttfamily --texmacs}
If CSL/Reduce is launched from texmacs this command-line flag should be
used to arrange that the {\ttfamily texmacs} flag is set in
{\ttfamily lispsystem!*}, and the code may then do special things.
\subsection{\ttfamily --}
If the application is run in console mode then its standard output could
be redirected to a file using shell facilities. But the {\ttfamily --}
directive (followed by a file name) redirects output within the Lisp rather
than outside it. If this is done a very limited capability for sending
progress or status reports to stderr (or the title-bar when running in windowed
mode) remains via the {\ttfamily report!-right} function.

The {\ttfamily -w} option may frequently make sense in such cases, but if that
is not used and the system tries to run in a window it will create it
starting off minimised.

\section{Predefined variables}

\subsection{\ttfamily !!fleps1}
There is a function safe!-fp!-plus that performs floating point
arithmetic but guarantees never to raise an exception. This value was
at one stage related to when small values created there got truncated to zero,
but the current code does not use the Lisp variable at all and instead does
things based on the bitwise representation of the numbers.
\subsection{\ttfamily !\$eof!\$}
The value of this variable is a special ``character'' used to denote an
end-of-file condition.

\subsection{\ttfamily !\$eol!\$}
The value of this variable is an end-of-line character.
\subsection{\ttfamily !*applyhook!*}
If this is set it might be supposed to be the name of a function used
by the interpreter as a callbackm but at presnet it does not actually do
anything!
\subsection{\ttfamily !*break!-loop!*}
If the value of this is a symbol that is defined as a function of one
argument then it is called during the processing on an error. This has not
been used in anger and so its whole status may be dubious!
\subsection{\ttfamily !*carcheckflag}
In general CSL arranges that every {\ttfamily car} or {\ttfamily cdr} access
is checked for validity. Once upon a time setting this variable to nil
turned such checks off in the hope of gaining a little speed. But it no
longer does that. It may have a minor effect on array access primitives.
\subsection{\ttfamily !*comp}
When set each function is compiled (into bytecodes) as it gets defined.
\subsection{\ttfamily !*debug!-io!*}
An I/O channel intended to be used for diagnostic interactions.
\subsection{\ttfamily !*echo}
When this is non-nil characters that are read from an input file are
echoed to the standard output. This gives a more comlete transcript in
a log file, but can sometimes amount to over-verbose output.
\subsection{\ttfamily !*error!-messages!*}
Has the value nil and does not do anything!
\subsection{\ttfamily !*error!-output!*}
An I/O channel intended for diagnostic output.
\subsection{\ttfamily !*evalhook!*}
See {\ttfamily !*applyhool!*}. This also does not do anything at present.
\subsection{\ttfamily !*gc!-hook!*}
If this is set to have as its value that is a function of one argument then
that function is called with {\ttfamily nil} on every minor entry to the
garbage collection, and with argument {\ttfamily t} at the end of a ``genuine''
full garbage collection.
\subsection{\ttfamily !*hankaku}
This was concerned with internationalisation to support a Japanese
locale but has not been activated for some while.
\subsection{\ttfamily !*loop!-print!*}
Probably not used at present.
\subsection{\ttfamily !*lower}
Not yet written

\subsection{\ttfamily !*macroexpand!-hook!*}
Not yet written

\subsection{\ttfamily !*math!-output!*}
Not yet written

\subsection{\ttfamily !*native\_code}
Not yet written

\subsection{\ttfamily !*notailcall}
Not yet written

\subsection{\ttfamily !*package!*}
Not yet written

\subsection{\ttfamily !*pgwd}
Not yet written

\subsection{\ttfamily !*plap}
Not yet written

\subsection{\ttfamily !*pretty!-symmetric}
Not yet written

\subsection{\ttfamily !*prinl!-fn!*}
Not yet written

\subsection{\ttfamily !*prinl!-index!*}
Not yet written

\subsection{\ttfamily !*prinl!-visited!-nodes!*}
Not yet written

\subsection{\ttfamily !*print!-array!*}
Not yet written

\subsection{\ttfamily !*print!-length!*}
Not yet written

\subsection{\ttfamily !*print!-level!*}
Not yet written

\subsection{\ttfamily !*pwrds}
Not yet written

\subsection{\ttfamily !*query!-io!*}
Not yet written

\subsection{\ttfamily !*quotes}
Not yet written

\subsection{\ttfamily !*raise}
Not yet written

\subsection{\ttfamily !*redefmsg}
Not yet written

\subsection{\ttfamily !*resources!*}
Not yet written

\subsection{\ttfamily !*savedef}
Not yet written

\subsection{\ttfamily !*spool!-output!*}
Not yet written

\subsection{\ttfamily !*standard!-input!*}
Not yet written

\subsection{\ttfamily !*standard!-output!*}
Not yet written

\subsection{\ttfamily !*terminal!-io!*}
Not yet written

\subsection{\ttfamily !*trace!-output!*}
Not yet written

\subsection{\ttfamily !@cslbase}
Not yet written

\subsection{\ttfamily blank}
Not yet written

\subsection{\ttfamily bn}
Not yet written

\subsection{\ttfamily bufferi}
Not yet written

\subsection{\ttfamily buffero}
Not yet written

\subsection{\ttfamily common!-lisp!-mode}
Not yet written

\subsection{\ttfamily crbuf!*}
Not yet written

\subsection{\ttfamily emsg!*}
Not yet written

\subsection{\ttfamily eof!*}
Not yet written

\subsection{\ttfamily esc!*}
Not yet written

\subsection{\ttfamily indblanks}
Not yet written

\subsection{\ttfamily indentlevel}
Not yet written

\subsection{\ttfamily initialblanks}
Not yet written

\subsection{\ttfamily lispsystem!*}
Not yet written

\subsection{\ttfamily lmar}
Not yet written

\subsection{\ttfamily load!-source}
Not yet written

\subsection{\ttfamily nil}
Not yet written

\subsection{\ttfamily ofl!*}
Not yet written

\subsection{\ttfamily pendingrpars}
Not yet written

\subsection{\ttfamily program!*}
Not yet written

\subsection{\ttfamily rmar}
Not yet written

\subsection{\ttfamily rparcount}
Not yet written

\subsection{\ttfamily s!:gensym!-serial}
Not yet written

\subsection{\ttfamily stack}
Not yet written

\subsection{\ttfamily t}
Not yet written

\subsection{\ttfamily tab}
Not yet written

\subsection{\ttfamily thin!*}
Not yet written

\subsection{\ttfamily ttype!*}
Not yet written




\section{Items that can appear in {\ttfamily lispsystem!*}}

There is a global variable called {\ttfamily lispsystem!*} whose value is
reset in the process of CSL starting up. An effect of this is that if the
user changes its value those changes do not survice a preserving and
re-loading a heap image: this is deliberate since the heap image may be
re-loaded on a different instance of CSL possibly on a quite different
computer of with a different configuration. The value of {\ttfamily
lispsystem!*} is a list of items, where each item is either an atomic tag
of a pair whose first component is a key. In general it would be unwise
to rely on exactly what information is present without review of the code
that sets it up. The information may be of interest to anybody but some tags
and keys are reflections of experiments rather than fullt stable facilities.

\subsection{\ttfamily (c!-code . count)}
This will be present if code has been optimised into C through the source
files u01.c to u12.c, and in that case the value tells you how many functions
have been optimised in this manner.

\subsection{\ttfamily common!-lisp}
For a project some while ago a limited Common Lisp compatibility mode was
being developed, and this tag indicated that it was active. In that case all
entries are in upper case and the variable is called {\ttfamily *FEATURES*}
rather than {\ttfamily lispsystem!*}. But note that this Lisp has never even
aspired to be a full Common Lisp, since its author considers Common Lisp to
have been a sad mistake that must bear significant responsibility for the
fact that interest in Lisp has faded dramatically since its introduction.

\subsection{\ttfamily (compiler!-command . command)}
The value associated with this key is a string that was used to compile the
files of C code making up CSL. It should contain directives to set up
search paths and predefined symbols. It is intended to be used in an
experiment that generates C code synamically, uses a command based on this
string to compile it and then dynamically links the resulting code in with
the running system.
\subsection{\ttfamily csl}
A simple tag intended to indicate that this Lisp system is CSL and not any
other. This can of course only work properly if all other Lisp systems
agree not to set this tag! In the context of Reduce I note that the PSL
Lisp system sets a tag {\ttfamily psl} on {\ttfamily lispsystem!*} and
the realistic use of this is to discriminate between CSL and PSL hosted
copies of Reduce.
\subsection{\ttfamily debug}
If CSL was compiled with debugging options this is present, and one can imagine
various bits of code being more cautious or more verbose if it is detected.
\subsection{\ttfamily (executable . name)}
The value is the fully rooted name of the executable file that was launched.
\subsection{\ttfamily fox}
Used to be present if the FOX GUI toolkit was detected and incorporated as
part of CSL, but now probably never used!
\subsection{\ttfamily (linker . type)}
Intended for use in association with {\ttfamily compiler!-command}, the value
is {\ttfamily win32} on Windows, {\ttfamily x86\_64} on 64-bit Linux and
other things on other systems, as detected using the program {\ttfamily
objtype.c}.
\subsection{\ttfamily (name . name)}
Some indication of the platform. For instance on one system I use it
is {\ttfamily linux-gnu:x86\_64} and on anther it is just {\ttfamily win32}.
\subsection{\ttfamily (native . tag)}
One of the many experiments within CSL that were active at one stage but are
not current involved compilation directly into machine code. The strong
desire to ensure that image files coudl be used on a cross-platform basis
led to saved compiled code being tagged with a numeric ``native code tag'',
and this key/value pair identified the value to be used on the current
machine.
\subsection{\ttfamily (opsys . operating-system)}
Some crude indication of the host operating system.
\subsection{\ttfamily pipes}
In the earlier days of CSL there were computers where pipes were not
supported, so this tag notes when they are present and hance the facility
to create sub-tasks through them can be used. 
\subsection{\ttfamily record\_get}
An an extension to the CSL profiling scheme it it possible to compile
a special version that tracks and counts each use of property-list access
functions. This can be useful because there are ways to give special
treatment to a small number of flags and a small number of properties. The
special-case flage end up stored as a bitmap in the symbol-header so avoid
need for property-list searching. But of course recording this extra
information slows things down. This tag notes when the slow version is
in use. It might be used to trigger a display of statistics at the end of
a calculation.
\subsection{\ttfamily reduce}
This is intended to report if the initial heap image is for Reduce rather than
merely for Lisp.
\subsection{\ttfamily (shortname . name)}
Gives the short name of the current executable, without its full path.
\subsection{\ttfamily showmath}
If the ``showmath'' capability has been compiled into CSL this will be present
so that Lisp code can know it is reasonable to try to use it.
\subsection{\ttfamily sixty!-four}
Present if the Lisp was compiled for a 64-bit computer.
\subsection{\ttfamily termed}
Present if a cursor-addressable console was detected.
\subsection{\ttfamily texmacs}
Present if the system was launched with the {\ttfamily --texmacs} flag.
The intent is that this should only be done when it has been launched with
texmacs as a front-end.
\subsection{\ttfamily (version . ver)}
The CSL version number.
\subsection{\ttfamily win32}
Present on Windows platforms, both the 32 and 64-bit variants!
\subsection{\ttfamily windowed}
Present if CSL is running in its own window rather than in console mode.




\section{Flags and Properties}

The tags here are probably not much use to end-users, but I am noting them
as a matter of completeness.

\subsection{\ttfamily s!:ppchar and \ttfamily s!:ppformat}
These are used in the prettyprint code found in {\ttfamily extras.red}. A
name is given a property {\ttfamily s!:ppformat} if in prettyprinted display
its first few arguments should appear on the same line as it if at all
possible. The {\ttfamily s!:ppchar} property is used to make the display of
bracket characters a little more tide in the source code.
\subsection{\ttfamily switch}
In the Reduce parser some names are ``switches'', and then directives such
as {\ttfamily on xxx} and {\ttfamily off xx} have the effect of setting or
clearing the value of a variable {\ttfamily !*xxx}. This is managed by
setting the {\ttfamily switch} flag om {\ttfamily xxx}. CSL sets some
things as switches ready for when they may be used by the Reduce parser.
\subsection{\ttfamily lose}
If a name is flagged as {ttfamily lose} then a subsequent attempt to
define or redefine it will be ignored.
\subsection{\ttfamily !$\sim$magic!-internal!-symbol!$\sim$}
CSL does not have a clear representation for functions that is separated from
the representation of an identifier, and so when you ask to get the value
of a raw function you get an identifier (probably a gensym) and this
tag is used to link such values with the symbols they were originally
extracted from.

\section{Functions and Special Forms}

Each line here shows a name and then one of the words {\ttfamily expr},
{\ttfamily fexpr} or {\ttfamily macro}. In some cases there can also be special
treatment of functions by the compiler so that they get compiled in-line.

\subsection{\ttfamily abs expr}
Not yet written

\subsection{\ttfamily acons expr}
Not yet written

\subsection{\ttfamily acos expr}
Not yet written

\subsection{\ttfamily acosd expr}
Not yet written

\subsection{\ttfamily acosh expr}
Not yet written

\subsection{\ttfamily acot expr}
Not yet written

\subsection{\ttfamily acotd expr}
Not yet written

\subsection{\ttfamily acoth expr}
Not yet written

\subsection{\ttfamily acsc expr}
Not yet written

\subsection{\ttfamily acscd expr}
Not yet written

\subsection{\ttfamily acsch expr}
Not yet written

\subsection{\ttfamily add1 expr}
Not yet written

\subsection{\ttfamily and fexpr}
Not yet written

\subsection{\ttfamily append expr}
Not yet written

\subsection{\ttfamily apply expr}
Not yet written

\subsection{\ttfamily apply0 expr}
Not yet written

\subsection{\ttfamily apply1 expr}
Not yet written

\subsection{\ttfamily apply2 expr}
Not yet written

\subsection{\ttfamily apply3 expr}
Not yet written

\subsection{\ttfamily asec expr}
Not yet written

\subsection{\ttfamily asecd expr}
Not yet written

\subsection{\ttfamily asech expr}
Not yet written

\subsection{\ttfamily ash expr}
Not yet written

\subsection{\ttfamily ash1 expr}
Not yet written

\subsection{\ttfamily asin expr}
Not yet written

\subsection{\ttfamily asind expr}
Not yet written

\subsection{\ttfamily asinh expr}
Not yet written

\subsection{\ttfamily assoc expr}
Not yet written

\subsection{\ttfamily assoc!*!* expr}
Not yet written

\subsection{\ttfamily atan expr}
Not yet written

\subsection{\ttfamily atan2 expr}
Not yet written

\subsection{\ttfamily atan2d expr}
Not yet written

\subsection{\ttfamily atand expr}
Not yet written

\subsection{\ttfamily atanh expr}
Not yet written

\subsection{\ttfamily atom expr}
Not yet written

\subsection{\ttfamily atsoc expr}
Not yet written

\subsection{\ttfamily batchp expr}
Not yet written

\subsection{\ttfamily binary\_close\_input expr}
Not yet written

\subsection{\ttfamily binary\_close\_output expr}
Not yet written

\subsection{\ttfamily binary\_open\_input expr}
Not yet written

\subsection{\ttfamily binary\_open\_output expr}
Not yet written

\subsection{\ttfamily binary\_prin1 expr}
Not yet written

\subsection{\ttfamily binary\_prin2 expr}
Not yet written

\subsection{\ttfamily binary\_prin3 expr}
Not yet written

\subsection{\ttfamily binary\_prinbyte expr}
Not yet written

\subsection{\ttfamily binary\_princ expr}
Not yet written

\subsection{\ttfamily binary\_prinfloat expr}
Not yet written

\subsection{\ttfamily binary\_read2 expr}
Not yet written

\subsection{\ttfamily binary\_read3 expr}
Not yet written

\subsection{\ttfamily binary\_read4 expr}
Not yet written

\subsection{\ttfamily binary\_readbyte expr}
Not yet written

\subsection{\ttfamily binary\_readfloat expr}
Not yet written

\subsection{\ttfamily binary\_select\_input expr}
Not yet written

\subsection{\ttfamily binary\_terpri expr}
Not yet written

\subsection{\ttfamily binopen expr}
Not yet written

\subsection{\ttfamily boundp expr}
Not yet written

\subsection{\ttfamily bps!-getv expr}
Not yet written

\subsection{\ttfamily bps!-putv expr}
Not yet written

\subsection{\ttfamily bps!-upbv expr}
Not yet written

\subsection{\ttfamily bpsp expr}
Not yet written

\subsection{\ttfamily break!-loop expr}
Not yet written

\subsection{\ttfamily byte!-getv expr}
Not yet written

\subsection{\ttfamily bytecounts expr}
Not yet written

\subsection{\ttfamily c\_out expr}
Not yet written

\subsection{\ttfamily caaaar expr}
Not yet written

\subsection{\ttfamily caaadr expr}
Not yet written

\subsection{\ttfamily caaar expr}
Not yet written

\subsection{\ttfamily caadar expr}
Not yet written

\subsection{\ttfamily caaddr expr}
Not yet written

\subsection{\ttfamily caadr expr}
Not yet written

\subsection{\ttfamily caar expr}
Not yet written

\subsection{\ttfamily cadaar expr}
Not yet written

\subsection{\ttfamily cadadr expr}
Not yet written

\subsection{\ttfamily cadar expr}
Not yet written

\subsection{\ttfamily caddar expr}
Not yet written

\subsection{\ttfamily cadddr expr}
Not yet written

\subsection{\ttfamily caddr expr}
Not yet written

\subsection{\ttfamily cadr expr}
Not yet written

\subsection{\ttfamily car expr}
Not yet written

\subsection{\ttfamily car!* expr}
Not yet written

\subsection{\ttfamily carcheck expr}
Not yet written

\subsection{\ttfamily catch fexpr}
Not yet written

\subsection{\ttfamily cbrt expr}
Not yet written

\subsection{\ttfamily cdaaar expr}
Not yet written

\subsection{\ttfamily cdaadr expr}
Not yet written

\subsection{\ttfamily cdaar expr}
Not yet written

\subsection{\ttfamily cdadar expr}
Not yet written

\subsection{\ttfamily cdaddr expr}
Not yet written

\subsection{\ttfamily cdadr expr}
Not yet written

\subsection{\ttfamily cdar expr}
Not yet written

\subsection{\ttfamily cddaar expr}
Not yet written

\subsection{\ttfamily cddadr expr}
Not yet written

\subsection{\ttfamily cddar expr}
Not yet written

\subsection{\ttfamily cdddar expr}
Not yet written

\subsection{\ttfamily cddddr expr}
Not yet written

\subsection{\ttfamily cdddr expr}
Not yet written

\subsection{\ttfamily cddr expr}
Not yet written

\subsection{\ttfamily cdr expr}
Not yet written

\subsection{\ttfamily ceiling expr}
Not yet written

\subsection{\ttfamily char!-code expr}
Not yet written

\subsection{\ttfamily char!-downcase expr}
Not yet written

\subsection{\ttfamily char!-upcase expr}
Not yet written

\subsection{\ttfamily chdir expr}
Not yet written

\subsection{\ttfamily check!-c!-code expr}
Not yet written

\subsection{\ttfamily checkpoint expr}
Not yet written

\subsection{\ttfamily cl!-equal expr}
Not yet written

\subsection{\ttfamily close expr}
Not yet written

\subsection{\ttfamily close!-library expr}
Not yet written

\subsection{\ttfamily clrhash expr}
Not yet written

\subsection{\ttfamily code!-char expr}
Not yet written

\subsection{\ttfamily codep expr}
Not yet written

\subsection{\ttfamily compile expr}
Not yet written

\subsection{\ttfamily compile!-all expr}
Not yet written

\subsection{\ttfamily compress expr}
Not yet written

\subsection{\ttfamily cond fexpr}
Not yet written

\subsection{\ttfamily cons expr}
Not yet written

\subsection{\ttfamily consp expr}
Not yet written

\subsection{\ttfamily constantp expr}
Not yet written

\subsection{\ttfamily contained expr}
Not yet written

\subsection{\ttfamily convert!-to!-evector expr}
Not yet written

\subsection{\ttfamily copy expr}
Not yet written

\subsection{\ttfamily copy!-module expr}
Not yet written

\subsection{\ttfamily copy!-native expr}
Not yet written

\subsection{\ttfamily cos expr}
Not yet written

\subsection{\ttfamily cosd expr}
Not yet written

\subsection{\ttfamily cosh expr}
Not yet written

\subsection{\ttfamily cot expr}
Not yet written

\subsection{\ttfamily cotd expr}
Not yet written

\subsection{\ttfamily coth expr}
Not yet written

\subsection{\ttfamily create!-directory expr}
Not yet written

\subsection{\ttfamily csc expr}
Not yet written

\subsection{\ttfamily cscd expr}
Not yet written

\subsection{\ttfamily csch expr}
Not yet written

\subsection{\ttfamily date expr}
Not yet written

\subsection{\ttfamily dated!-name expr}
Not yet written

\subsection{\ttfamily datelessp expr}
Not yet written

\subsection{\ttfamily datestamp expr}
Not yet written

\subsection{\ttfamily de fexpr}
Not yet written

\subsection{\ttfamily define!-in!-module expr}
Not yet written

\subsection{\ttfamily deflist expr}
Not yet written

\subsection{\ttfamily deleq expr}
Not yet written

\subsection{\ttfamily delete expr}
Not yet written

\subsection{\ttfamily delete!-file expr}
Not yet written

\subsection{\ttfamily delete!-module expr}
Not yet written

\subsection{\ttfamily difference expr}
Not yet written

\subsection{\ttfamily digit expr}
Not yet written

\subsection{\ttfamily directoryp expr}
Not yet written

\subsection{\ttfamily divide expr}
Not yet written

\subsection{\ttfamily dm fexpr}
Not yet written

\subsection{\ttfamily do macro}
Not yet written

\subsection{\ttfamily do!* macro}
Not yet written

\subsection{\ttfamily do!*\_z2tw2evoft83 expr}
Not yet written

\subsection{\ttfamily do\_tys294e5sboe expr}
Not yet written

\subsection{\ttfamily dolist macro}
Not yet written

\subsection{\ttfamily dolist\_2oc4v2mwnrv2 expr}
Not yet written

\subsection{\ttfamily dotimes macro}
Not yet written

\subsection{\ttfamily dotimes\_cm3wu6zfgv79 expr}
Not yet written

\subsection{\ttfamily double!-execute expr}
Not yet written

\subsection{\ttfamily egetv expr}
Not yet written

\subsection{\ttfamily eject expr}
Not yet written

\subsection{\ttfamily enable!-backtrace expr}
Not yet written

\subsection{\ttfamily encapsulatedp expr}
Not yet written

\subsection{\ttfamily endp expr}
Not yet written

\subsection{\ttfamily eputv expr}
Not yet written

\subsection{\ttfamily eq expr}
Not yet written

\subsection{\ttfamily eq!-safe expr}
Not yet written

\subsection{\ttfamily eqcar expr}
Not yet written

\subsection{\ttfamily eql expr}
Not yet written

\subsection{\ttfamily eqlhash expr}
Not yet written

\subsection{\ttfamily eqn expr}
Not yet written

\subsection{\ttfamily equal expr}
Not yet written

\subsection{\ttfamily equalcar expr}
Not yet written

\subsection{\ttfamily equalp expr}
Not yet written

\subsection{\ttfamily error expr}
Not yet written

\subsection{\ttfamily error1 expr}
Not yet written

\subsection{\ttfamily errorset expr}
Not yet written

\subsection{\ttfamily eupbv expr}
Not yet written

\subsection{\ttfamily eval expr}
Not yet written

\subsection{\ttfamily eval!-when fexpr}
Not yet written

\subsection{\ttfamily evectorp expr}
Not yet written

\subsection{\ttfamily evenp expr}
Not yet written

\subsection{\ttfamily evlis expr}
Not yet written

\subsection{\ttfamily exp expr}
Not yet written

\subsection{\ttfamily expand expr}
Not yet written

\subsection{\ttfamily explode expr}
Not yet written

\subsection{\ttfamily explode2 expr}
Not yet written

\subsection{\ttfamily explode2lc expr}
Not yet written

\subsection{\ttfamily explode2lcn expr}
Not yet written

\subsection{\ttfamily explode2n expr}
Not yet written

\subsection{\ttfamily explode2uc expr}
Not yet written

\subsection{\ttfamily explode2ucn expr}
Not yet written

\subsection{\ttfamily explodebinary expr}
Not yet written

\subsection{\ttfamily explodec expr}
Not yet written

\subsection{\ttfamily explodecn expr}
Not yet written

\subsection{\ttfamily explodehex expr}
Not yet written

\subsection{\ttfamily exploden expr}
Not yet written

\subsection{\ttfamily explodeoctal expr}
Not yet written

\subsection{\ttfamily expt expr}
Not yet written

\subsection{\ttfamily faslout expr}
Not yet written

\subsection{\ttfamily fetch!-url expr}
Not yet written

\subsection{\ttfamily fgetv32 expr}
Not yet written

\subsection{\ttfamily fgetv64 expr}
Not yet written

\subsection{\ttfamily file!-length expr}
Not yet written

\subsection{\ttfamily file!-readablep expr}
Not yet written

\subsection{\ttfamily file!-writeablep expr}
Not yet written

\subsection{\ttfamily filedate expr}
Not yet written

\subsection{\ttfamily filep expr}
Not yet written

\subsection{\ttfamily fix expr}
Not yet written

\subsection{\ttfamily fixp expr}
Not yet written

\subsection{\ttfamily flag expr}
Not yet written

\subsection{\ttfamily flagp expr}
Not yet written

\subsection{\ttfamily flagp!*!* expr}
Not yet written

\subsection{\ttfamily flagpcar expr}
Not yet written

\subsection{\ttfamily float expr}
Not yet written

\subsection{\ttfamily floatp expr}
Not yet written

\subsection{\ttfamily floor expr}
Not yet written

\subsection{\ttfamily fluid expr}
Not yet written

\subsection{\ttfamily fluidp expr}
Not yet written

\subsection{\ttfamily flush expr}
Not yet written

\subsection{\ttfamily format macro}
Not yet written

\subsection{\ttfamily format\_vqx39lgqssd1 expr}
Not yet written

\subsection{\ttfamily fp!-evaluate expr}
Not yet written

\subsection{\ttfamily fputv32 expr}
Not yet written

\subsection{\ttfamily fputv64 expr}
Not yet written

\subsection{\ttfamily frexp expr}
Not yet written

\subsection{\ttfamily funcall expr}
Not yet written

\subsection{\ttfamily funcall!* expr}
Not yet written

\subsection{\ttfamily function fexpr}
Not yet written

\subsection{\ttfamily gcdn expr}
Not yet written

\subsection{\ttfamily gctime expr}
Not yet written

\subsection{\ttfamily gensym expr}
Not yet written

\subsection{\ttfamily gensym1 expr}
Not yet written

\subsection{\ttfamily gensym2 expr}
Not yet written

\subsection{\ttfamily gensymp expr}
Not yet written

\subsection{\ttfamily geq expr}
Not yet written

\subsection{\ttfamily get expr}
Not yet written

\subsection{\ttfamily get!* expr}
Not yet written

\subsection{\ttfamily get!-current!-directory expr}
Not yet written

\subsection{\ttfamily get!-lisp!-directory expr}
Not yet written

\subsection{\ttfamily getd expr}
Not yet written

\subsection{\ttfamily getenv expr}
Not yet written

\subsection{\ttfamily gethash expr}
Not yet written

\subsection{\ttfamily getv expr}
Not yet written

\subsection{\ttfamily getv16 expr}
Not yet written

\subsection{\ttfamily getv32 expr}
Not yet written

\subsection{\ttfamily getv8 expr}
Not yet written

\subsection{\ttfamily global expr}
Not yet written

\subsection{\ttfamily globalp expr}
Not yet written

\subsection{\ttfamily go fexpr}
Not yet written

\subsection{\ttfamily greaterp expr}
Not yet written

\subsection{\ttfamily hash!-table!-p expr}
Not yet written

\subsection{\ttfamily hashcontents expr}
Not yet written

\subsection{\ttfamily hashtagged!-name expr}
Not yet written

\subsection{\ttfamily hypot expr}
Not yet written

\subsection{\ttfamily iadd1 expr}
Not yet written

\subsection{\ttfamily idapply expr}
Not yet written

\subsection{\ttfamily idifference expr}
Not yet written

\subsection{\ttfamily idp expr}
Not yet written

\subsection{\ttfamily iequal expr}
Not yet written

\subsection{\ttfamily if fexpr}
Not yet written

\subsection{\ttfamily igeq expr}
Not yet written

\subsection{\ttfamily igreaterp expr}
Not yet written

\subsection{\ttfamily ileq expr}
Not yet written

\subsection{\ttfamily ilessp expr}
Not yet written

\subsection{\ttfamily ilogand expr}
Not yet written

\subsection{\ttfamily ilogor expr}
Not yet written

\subsection{\ttfamily ilogxor expr}
Not yet written

\subsection{\ttfamily imax expr}
Not yet written

\subsection{\ttfamily imin expr}
Not yet written

\subsection{\ttfamily iminus expr}
Not yet written

\subsection{\ttfamily iminusp expr}
Not yet written

\subsection{\ttfamily indirect expr}
Not yet written

\subsection{\ttfamily inorm expr}
Not yet written

\subsection{\ttfamily input!-libraries fexpr}
Not yet written

\subsection{\ttfamily instate!-c!-code expr}
Not yet written

\subsection{\ttfamily integerp expr}
Not yet written

\subsection{\ttfamily intern expr}
Not yet written

\subsection{\ttfamily internal!-open expr}
Not yet written

\subsection{\ttfamily intersection expr}
Not yet written

\subsection{\ttfamily ionep expr}
Not yet written

\subsection{\ttfamily iplus expr}
Not yet written

\subsection{\ttfamily iplus2 expr}
Not yet written

\subsection{\ttfamily iquotient expr}
Not yet written

\subsection{\ttfamily iremainder expr}
Not yet written

\subsection{\ttfamily irightshift expr}
Not yet written

\subsection{\ttfamily is!-console expr}
Not yet written

\subsection{\ttfamily isub1 expr}
Not yet written

\subsection{\ttfamily itimes expr}
Not yet written

\subsection{\ttfamily itimes2 expr}
Not yet written

\subsection{\ttfamily izerop expr}
Not yet written

\subsection{\ttfamily last expr}
Not yet written

\subsection{\ttfamily lastcar expr}
Not yet written

\subsection{\ttfamily lastpair expr}
Not yet written

\subsection{\ttfamily lcmn expr}
Not yet written

\subsection{\ttfamily length expr}
Not yet written

\subsection{\ttfamily lengthc expr}
Not yet written

\subsection{\ttfamily leq expr}
Not yet written

\subsection{\ttfamily lessp expr}
Not yet written

\subsection{\ttfamily let!* fexpr}
Not yet written

\subsection{\ttfamily library!-members expr}
Returns a list of all the modules that could potentially be loaded using
{\ttfamily load!-module}. See {\ttfamily list!-modules} to get a human
readable display that looks more like the result of listing a directory, or
{\ttfamily modulep} for checking the state of a particular named module.

\subsection{\ttfamily library!-name expr}
Not yet written

\subsection{\ttfamily linelength expr}
Not yet written

\subsection{\ttfamily list fexpr}
Not yet written

\subsection{\ttfamily list!* fexpr}
Not yet written

\subsection{\ttfamily list!-directory expr}
Not yet written


\subsection{\ttfamily list!-modules expr}
This prints a human-readable display of the modules present in the current
image files. This will include ``InitialImage'' which is the heap-image
loaded at system startup. For example
\begin{verbatim}
> (list!-modules)

File d:\csl\csl.img (dirsize 8  length 155016, Writable):
  compat       Sat Jul 26 10:20:08 2008  position 556   size: 9320
  compiler     Sat Jul 26 10:20:08 2008  position 9880  size: 81088
  InitialImage Sat Jul 26 10:20:09 2008  position 90972 size: 64040

nil
\end{verbatim}

See {\ttfamily library!-members} and {\ttfamily modulep} for functions that
make it possible for Lisp code to discover about the loadable modules that are
available.
\subsection{\ttfamily list!-to!-string expr}
Not yet written

\subsection{\ttfamily list!-to!-symbol expr}
Not yet written

\subsection{\ttfamily list!-to!-vector expr}
Not yet written

\subsection{\ttfamily list2 expr}
Not yet written

\subsection{\ttfamily list2!* expr}
Not yet written

\subsection{\ttfamily list3 expr}
Not yet written

\subsection{\ttfamily list3!* expr}
Not yet written

\subsection{\ttfamily list4 expr}
Not yet written

\subsection{\ttfamily liter expr}
Not yet written

\subsection{\ttfamily ln expr}
Not yet written

\subsection{\ttfamily load!-module expr}
Not yet written

\subsection{\ttfamily load!-source expr}
Not yet written

\subsection{\ttfamily log expr}
Not yet written

\subsection{\ttfamily log10 expr}
Not yet written

\subsection{\ttfamily logand expr}
Not yet written

\subsection{\ttfamily logb expr}
Not yet written

\subsection{\ttfamily logeqv expr}
Not yet written

\subsection{\ttfamily lognot expr}
Not yet written

\subsection{\ttfamily logor expr}
Not yet written

\subsection{\ttfamily logxor expr}
Not yet written

\subsection{\ttfamily lose!-precision expr}
Not yet written

\subsection{\ttfamily lposn expr}
Not yet written

\subsection{\ttfamily lsd expr}
Not yet written

\subsection{\ttfamily macro!-function expr}
Not yet written

\subsection{\ttfamily macroexpand expr}
Not yet written

\subsection{\ttfamily macroexpand!-1 expr}
Not yet written

\subsection{\ttfamily make!-bps expr}
Not yet written

\subsection{\ttfamily make!-function!-stream expr}
Not yet written

\subsection{\ttfamily make!-global expr}
Not yet written

\subsection{\ttfamily make!-native expr}
Not yet written

\subsection{\ttfamily make!-random!-state expr}
Not yet written

\subsection{\ttfamily make!-simple!-string expr}
Not yet written

\subsection{\ttfamily make!-special expr}
Not yet written

\subsection{\ttfamily map expr}
Not yet written

\subsection{\ttfamily mapc expr}
Not yet written

\subsection{\ttfamily mapcan expr}
Not yet written

\subsection{\ttfamily mapcar expr}
Not yet written

\subsection{\ttfamily mapcon expr}
Not yet written

\subsection{\ttfamily maphash expr}
Not yet written

\subsection{\ttfamily maple\_atomic\_value expr}
Not yet written

\subsection{\ttfamily maple\_component expr}
Not yet written

\subsection{\ttfamily maple\_integer expr}
Not yet written

\subsection{\ttfamily maple\_length expr}
Not yet written

\subsection{\ttfamily maple\_string\_data expr}
Not yet written

\subsection{\ttfamily maple\_tag expr}
Not yet written

\subsection{\ttfamily maplist expr}
Not yet written

\subsection{\ttfamily mapstore expr}
Not yet written

\subsection{\ttfamily math!-display expr}
Not yet written

\subsection{\ttfamily max expr}
Not yet written

\subsection{\ttfamily max2 expr}
Not yet written

\subsection{\ttfamily md5 expr}
Not yet written

\subsection{\ttfamily md60 expr}
Not yet written

\subsection{\ttfamily member expr}
Not yet written

\subsection{\ttfamily member!*!* expr}
Not yet written

\subsection{\ttfamily memq expr}
Not yet written

\subsection{\ttfamily min expr}
Not yet written

\subsection{\ttfamily min2 expr}
Not yet written

\subsection{\ttfamily minus expr}
Not yet written

\subsection{\ttfamily minusp expr}
Not yet written

\subsection{\ttfamily mkevect expr}
Not yet written

\subsection{\ttfamily mkfvect32 expr}
Not yet written

\subsection{\ttfamily mkfvect64 expr}
Not yet written

\subsection{\ttfamily mkhash expr}
Not yet written

\subsection{\ttfamily mkquote expr}
Not yet written

\subsection{\ttfamily mkvect expr}
Not yet written

\subsection{\ttfamily mkvect16 expr}
Not yet written

\subsection{\ttfamily mkvect32 expr}
Not yet written

\subsection{\ttfamily mkvect8 expr}
Not yet written

\subsection{\ttfamily mkxvect expr}
Not yet written

\subsection{\ttfamily mod expr}
Not yet written

\subsection{\ttfamily modular!-difference expr}
Not yet written

\subsection{\ttfamily modular!-expt expr}
Not yet written

\subsection{\ttfamily modular!-minus expr}
Not yet written

\subsection{\ttfamily modular!-number expr}
Not yet written

\subsection{\ttfamily modular!-plus expr}
Not yet written

\subsection{\ttfamily modular!-quotient expr}
Not yet written

\subsection{\ttfamily modular!-reciprocal expr}
Not yet written

\subsection{\ttfamily modular!-times expr}
Not yet written

\subsection{\ttfamily modulep expr}
This takes a single argument and checks whether there is a loadable module
of that name. If there is not then {\ttfamily nil} is returned, otherwise a
string that indicates the date-stamp on the module is given. See
{\ttfamily datelessp} for working with such dates, and {\ttfamily
library!-members} for finding a list of all modules that are available.

\subsection{\ttfamily mpi\_allgather expr}
Not yet written

\subsection{\ttfamily mpi\_alltoall expr}
Not yet written

\subsection{\ttfamily mpi\_barrier expr}
Not yet written

\subsection{\ttfamily mpi\_bcast expr}
Not yet written

\subsection{\ttfamily mpi\_comm\_rank expr}
Not yet written

\subsection{\ttfamily mpi\_comm\_size expr}
Not yet written

\subsection{\ttfamily mpi\_gather expr}
Not yet written

\subsection{\ttfamily mpi\_iprobe expr}
Not yet written

\subsection{\ttfamily mpi\_irecv expr}
Not yet written

\subsection{\ttfamily mpi\_isend expr}
Not yet written

\subsection{\ttfamily mpi\_probe expr}
Not yet written

\subsection{\ttfamily mpi\_recv expr}
Not yet written

\subsection{\ttfamily mpi\_scatter expr}
Not yet written

\subsection{\ttfamily mpi\_send expr}
Not yet written

\subsection{\ttfamily mpi\_sendrecv expr}
Not yet written

\subsection{\ttfamily mpi\_test expr}
Not yet written

\subsection{\ttfamily mpi\_wait expr}
Not yet written

\subsection{\ttfamily msd expr}
Not yet written

\subsection{\ttfamily native!-address expr}
Not yet written

\subsection{\ttfamily native!-getv expr}
Not yet written

\subsection{\ttfamily native!-putv expr}
Not yet written

\subsection{\ttfamily native!-type expr}
Not yet written

\subsection{\ttfamily nconc expr}
Not yet written

\subsection{\ttfamily ncons expr}
Not yet written

\subsection{\ttfamily neq expr}
Not yet written

\subsection{\ttfamily noisy!-setq fexpr}
Not yet written

\subsection{\ttfamily not expr}
Not yet written

\subsection{\ttfamily nreverse expr}
Not yet written

\subsection{\ttfamily null expr}
Not yet written

\subsection{\ttfamily numberp expr}
Not yet written

\subsection{\ttfamily oblist expr}
Not yet written

\subsection{\ttfamily oddp expr}
Not yet written

\subsection{\ttfamily oem!-supervisor expr}
Not yet written

\subsection{\ttfamily onep expr}
Not yet written

\subsection{\ttfamily open expr}
Not yet written

\subsection{\ttfamily open!-library expr}
Not yet written

\subsection{\ttfamily open!-url expr}
Not yet written

\subsection{\ttfamily or fexpr}
Not yet written

\subsection{\ttfamily orderp expr}
Not yet written

\subsection{\ttfamily ordp expr}
Not yet written

\subsection{\ttfamily output!-library fexpr}
Not yet written

\subsection{\ttfamily pagelength expr}
Not yet written

\subsection{\ttfamily pair expr}
Not yet written

\subsection{\ttfamily pairp expr}
Not yet written

\subsection{\ttfamily parallel expr}
Not yet written

\subsection{\ttfamily peekch expr}
Not yet written

\subsection{\ttfamily pipe!-open expr}
Not yet written

\subsection{\ttfamily plist expr}
Not yet written

\subsection{\ttfamily plus fexpr}
Not yet written

\subsection{\ttfamily plus2 expr}
Not yet written

\subsection{\ttfamily plus\_4lcok6r6bp3g expr}
Not yet written

\subsection{\ttfamily plusp expr}
Not yet written

\subsection{\ttfamily posn expr}
Not yet written

\subsection{\ttfamily preserve expr}
Not yet written

\subsection{\ttfamily prettyprint expr}
Not yet written

\subsection{\ttfamily prin expr}
Not yet written

\subsection{\ttfamily prin1 expr}
Not yet written

\subsection{\ttfamily prin2 expr}
Not yet written

\subsection{\ttfamily prin2a expr}
Not yet written

\subsection{\ttfamily prinbinary expr}
Not yet written

\subsection{\ttfamily princ expr}
Not yet written

\subsection{\ttfamily princ!-downcase expr}
Not yet written

\subsection{\ttfamily princ!-upcase expr}
Not yet written

\subsection{\ttfamily princl expr}
Not yet written

\subsection{\ttfamily prinhex expr}
Not yet written

\subsection{\ttfamily prinl expr}
Not yet written

\subsection{\ttfamily prinoctal expr}
Not yet written

\subsection{\ttfamily prinraw expr}
Not yet written

\subsection{\ttfamily print expr}
Not yet written

\subsection{\ttfamily print!-config!-header expr}
Not yet written

\subsection{\ttfamily print!-csl!-headers expr}
Not yet written

\subsection{\ttfamily print!-imports expr}
Not yet written

\subsection{\ttfamily printc expr}
Not yet written

\subsection{\ttfamily printcl expr}
Not yet written

\subsection{\ttfamily printl expr}
Not yet written

\subsection{\ttfamily printprompt expr}
Not yet written

\subsection{\ttfamily prog fexpr}
Not yet written

\subsection{\ttfamily prog1 fexpr}
Not yet written

\subsection{\ttfamily prog2 fexpr}
Not yet written

\subsection{\ttfamily progn fexpr}
Not yet written

\subsection{\ttfamily protect!-symbols expr}
Not yet written

\subsection{\ttfamily protected!-symbol!-warn expr}
Not yet written

\subsection{\ttfamily psetq macro}
Not yet written

\subsection{\ttfamily psetq\_vg20v16gc5na expr}
Not yet written

\subsection{\ttfamily put expr}
Not yet written

\subsection{\ttfamily putc expr}
Not yet written

\subsection{\ttfamily putd expr}
Not yet written

\subsection{\ttfamily puthash expr}
Not yet written

\subsection{\ttfamily putv expr}
Not yet written

\subsection{\ttfamily putv!-char expr}
Not yet written

\subsection{\ttfamily putv16 expr}
Not yet written

\subsection{\ttfamily putv32 expr}
Not yet written

\subsection{\ttfamily putv8 expr}
Not yet written

\subsection{\ttfamily qcaar expr}
Not yet written

\subsection{\ttfamily qcadr expr}
Not yet written

\subsection{\ttfamily qcar expr}
Not yet written

\subsection{\ttfamily qcdar expr}
Not yet written

\subsection{\ttfamily qcddr expr}
Not yet written

\subsection{\ttfamily qcdr expr}
Not yet written

\subsection{\ttfamily qgetv expr}
Not yet written

\subsection{\ttfamily qputv expr}
Not yet written

\subsection{\ttfamily quote fexpr}
Not yet written

\subsection{\ttfamily quotient expr}
Not yet written

\subsection{\ttfamily random!-fixnum expr}
Not yet written

\subsection{\ttfamily random!-number expr}
Not yet written

\subsection{\ttfamily rassoc expr}
Not yet written

\subsection{\ttfamily rational expr}
Not yet written

\subsection{\ttfamily rdf expr}
Not yet written

\subsection{\ttfamily rds expr}
Not yet written

\subsection{\ttfamily read expr}
Not yet written

\subsection{\ttfamily readb expr}
Not yet written

\subsection{\ttfamily readch expr}
Not yet written

\subsection{\ttfamily readline expr}
Not yet written

\subsection{\ttfamily reclaim expr}
Not yet written

\subsection{\ttfamily remainder expr}
Not yet written

\subsection{\ttfamily remd expr}
Not yet written

\subsection{\ttfamily remflag expr}
Not yet written

\subsection{\ttfamily remhash expr}
Not yet written

\subsection{\ttfamily remob expr}
Not yet written

\subsection{\ttfamily remprop expr}
Not yet written

\subsection{\ttfamily rename!-file expr}
Not yet written

\subsection{\ttfamily representation expr}
Not yet written

\subsection{\ttfamily resource!-exceeded expr}
Not yet written

\subsection{\ttfamily resource!-limit expr}
Not yet written

\subsection{\ttfamily restart!-csl expr}
Not yet written

\subsection{\ttfamily restore!-c!-code expr}
Not yet written

\subsection{\ttfamily return fexpr}
Not yet written

\subsection{\ttfamily reverse expr}
Not yet written

\subsection{\ttfamily reversip expr}
Not yet written

\subsection{\ttfamily round expr}
Not yet written

\subsection{\ttfamily rplaca expr}
Not yet written

\subsection{\ttfamily rplacd expr}
Not yet written

\subsection{\ttfamily rplacw expr}
Not yet written

\subsection{\ttfamily rseek expr}
Not yet written

\subsection{\ttfamily rtell expr}
Not yet written

\subsection{\ttfamily s!:blankcount macro}
Not yet written

\subsection{\ttfamily s!:blankcount\_di4u8tiv3pra expr}
Not yet written

\subsection{\ttfamily s!:blanklist macro}
Not yet written

\subsection{\ttfamily s!:blanklist\_3grr8hhc8kse expr}
Not yet written

\subsection{\ttfamily s!:blankp macro}
Not yet written

\subsection{\ttfamily s!:blankp\_q4md8q4t32hd expr}
Not yet written

\subsection{\ttfamily s!:depth macro}
Not yet written

\subsection{\ttfamily s!:depth\_nywe93u7asd2 expr}
Not yet written

\subsection{\ttfamily s!:do!-bindings expr}
Not yet written

\subsection{\ttfamily s!:do!-endtest expr}
Not yet written

\subsection{\ttfamily s!:do!-result expr}
Not yet written

\subsection{\ttfamily s!:do!-updates expr}
Not yet written

\subsection{\ttfamily s!:endlist expr}
Not yet written

\subsection{\ttfamily s!:expand!-do expr}
Not yet written

\subsection{\ttfamily s!:expand!-dolist expr}
Not yet written

\subsection{\ttfamily s!:expand!-dotimes expr}
Not yet written

\subsection{\ttfamily s!:explodes expr}
Not yet written

\subsection{\ttfamily s!:finishpending expr}
Not yet written

\subsection{\ttfamily s!:format expr}
Not yet written

\subsection{\ttfamily s!:indenting macro}
Not yet written

\subsection{\ttfamily s!:indenting\_uugpn161oe9g expr}
Not yet written

\subsection{\ttfamily s!:make!-psetq!-assignments expr}
Not yet written

\subsection{\ttfamily s!:make!-psetq!-bindings expr}
Not yet written

\subsection{\ttfamily s!:make!-psetq!-vars expr}
Not yet written

\subsection{\ttfamily s!:newframe macro}
Not yet written

\subsection{\ttfamily s!:newframe\_jj3e2mkec583 expr}
Not yet written

\subsection{\ttfamily s!:oblist expr}
Not yet written

\subsection{\ttfamily s!:oblist1 expr}
Not yet written

\subsection{\ttfamily s!:overflow expr}
Not yet written

\subsection{\ttfamily s!:prindent expr}
Not yet written

\subsection{\ttfamily s!:prinl0 expr}
Not yet written

\subsection{\ttfamily s!:prinl1 expr}
Not yet written

\subsection{\ttfamily s!:prinl2 expr}
Not yet written

\subsection{\ttfamily s!:prvector expr}
Not yet written

\subsection{\ttfamily s!:putblank expr}
Not yet written

\subsection{\ttfamily s!:putch expr}
Not yet written

\subsection{\ttfamily s!:quotep expr}
Not yet written

\subsection{\ttfamily s!:setblankcount macro}
Not yet written

\subsection{\ttfamily s!:setblankcount\_wqtabtq2ayhf expr}
Not yet written

\subsection{\ttfamily s!:setblanklist macro}
Not yet written

\subsection{\ttfamily s!:setblanklist\_yx45qh074fy7 expr}
Not yet written

\subsection{\ttfamily s!:setindenting macro}
Not yet written

\subsection{\ttfamily s!:setindenting\_wlwn13x1f3y expr}
Not yet written

\subsection{\ttfamily s!:stamp expr}
Not yet written

\subsection{\ttfamily s!:top macro}
Not yet written

\subsection{\ttfamily s!:top\_su2dv6yphmp9 expr}
Not yet written

\subsection{\ttfamily safe!-fp!-pl expr}
Not yet written

\subsection{\ttfamily safe!-fp!-pl0 expr}
Not yet written

\subsection{\ttfamily safe!-fp!-plus expr}
Not yet written

\subsection{\ttfamily safe!-fp!-quot expr}
Not yet written

\subsection{\ttfamily safe!-fp!-times expr}
Not yet written

\subsection{\ttfamily sample expr}
Not yet written

\subsection{\ttfamily sassoc expr}
Not yet written

\subsection{\ttfamily schar expr}
Not yet written

\subsection{\ttfamily scharn expr}
Not yet written

\subsection{\ttfamily sec expr}
Not yet written

\subsection{\ttfamily secd expr}
Not yet written

\subsection{\ttfamily sech expr}
Not yet written

\subsection{\ttfamily seprp expr}
Not yet written

\subsection{\ttfamily set expr}
Not yet written

\subsection{\ttfamily set!-autoload expr}
Not yet written

\subsection{\ttfamily set!-help!-file expr}
Not yet written

\subsection{\ttfamily set!-print!-precision expr}
Not yet written

\subsection{\ttfamily set!-small!-modulus expr}
Not yet written

\subsection{\ttfamily setpchar expr}
Not yet written

\subsection{\ttfamily setq fexpr}
Not yet written

\subsection{\ttfamily silent!-system expr}
Not yet written

\subsection{\ttfamily simple!-string!-p expr}
Not yet written

\subsection{\ttfamily simple!-vector!-p expr}
Not yet written

\subsection{\ttfamily sin expr}
Not yet written

\subsection{\ttfamily sind expr}
Not yet written

\subsection{\ttfamily sinh expr}
Not yet written

\subsection{\ttfamily smemq expr}
Not yet written

\subsection{\ttfamily sort expr}
Not yet written

\subsection{\ttfamily sortip expr}
Not yet written

\subsection{\ttfamily spaces expr}
Not yet written

\subsection{\ttfamily special!-char expr}
Not yet written

\subsection{\ttfamily special!-form!-p expr}
Not yet written

\subsection{\ttfamily spool expr}
Not yet written

\subsection{\ttfamily sqrt expr}
Not yet written

\subsection{\ttfamily stable!-sort expr}
Not yet written

\subsection{\ttfamily stable!-sortip expr}
Not yet written

\subsection{\ttfamily start!-module expr}
Not yet written

\subsection{\ttfamily startup!-banner expr}
Not yet written

\subsection{\ttfamily stop expr}
Not yet written

\subsection{\ttfamily streamp expr}
Not yet written

\subsection{\ttfamily stringp expr}
Not yet written

\subsection{\ttfamily sub1 expr}
Not yet written

\subsection{\ttfamily subla expr}
Not yet written

\subsection{\ttfamily sublis expr}
Not yet written

\subsection{\ttfamily subst expr}
Not yet written

\subsection{\ttfamily superprinm expr}
Not yet written

\subsection{\ttfamily superprintm expr}
Not yet written

\subsection{\ttfamily sxhash expr}
Not yet written

\subsection{\ttfamily symbol!-argcode expr}
Not yet written

\subsection{\ttfamily symbol!-argcount expr}
Not yet written

\subsection{\ttfamily symbol!-env expr}
Not yet written

\subsection{\ttfamily symbol!-fastgets expr}
Not yet written

\subsection{\ttfamily symbol!-fn!-cell expr}
Not yet written

\subsection{\ttfamily symbol!-function expr}
Not yet written

\subsection{\ttfamily symbol!-make!-fastget expr}
Not yet written

\subsection{\ttfamily symbol!-name expr}
Not yet written

\subsection{\ttfamily symbol!-protect expr}
Not yet written

\subsection{\ttfamily symbol!-restore!-fns expr}
Not yet written

\subsection{\ttfamily symbol!-set!-definition expr}
Not yet written

\subsection{\ttfamily symbol!-set!-env expr}
Not yet written

\subsection{\ttfamily symbol!-set!-native expr}
Not yet written

\subsection{\ttfamily symbol!-value expr}
Not yet written

\subsection{\ttfamily symbolp expr}
Not yet written

\subsection{\ttfamily symerr expr}
Not yet written

\subsection{\ttfamily system expr}
Not yet written

\subsection{\ttfamily tagbody fexpr}
Not yet written

\subsection{\ttfamily tan expr}
Not yet written

\subsection{\ttfamily tand expr}
Not yet written

\subsection{\ttfamily tanh expr}
Not yet written

\subsection{\ttfamily terpri expr}
Not yet written

\subsection{\ttfamily threevectorp expr}
Not yet written

\subsection{\ttfamily throw fexpr}
Not yet written

\subsection{\ttfamily time expr}
Not yet written

\subsection{\ttfamily times fexpr}
Not yet written

\subsection{\ttfamily times2 expr}
Not yet written

\subsection{\ttfamily times\_z6u5f3t8dwo4 expr}
Not yet written

\subsection{\ttfamily tmpnam expr}
Not yet written

\subsection{\ttfamily trace expr}
Not yet written

\subsection{\ttfamily trace!-all expr}
Not yet written

\subsection{\ttfamily traceset expr}
Not yet written

\subsection{\ttfamily traceset1 expr}
Not yet written

\subsection{\ttfamily truename expr}
Not yet written

\subsection{\ttfamily truncate expr}
Not yet written

\subsection{\ttfamily ttab expr}
Not yet written

\subsection{\ttfamily tyo expr}
Not yet written

\subsection{\ttfamily undouble!-execute expr}
Not yet written

\subsection{\ttfamily unfluid expr}
Not yet written

\subsection{\ttfamily unglobal expr}
Not yet written

\subsection{\ttfamily union expr}
Not yet written

\subsection{\ttfamily unless fexpr}
Not yet written

\subsection{\ttfamily unmake!-global expr}
Not yet written

\subsection{\ttfamily unmake!-special expr}
Not yet written

\subsection{\ttfamily unreadch expr}
Not yet written

\subsection{\ttfamily untrace expr}
Not yet written

\subsection{\ttfamily untraceset expr}
Not yet written

\subsection{\ttfamily untraceset1 expr}
Not yet written

\subsection{\ttfamily unwind!-protect fexpr}
Not yet written

\subsection{\ttfamily upbv expr}
Not yet written

\subsection{\ttfamily user!-homedir!-pathname expr}
Not yet written

\subsection{\ttfamily vectorp expr}
Not yet written

\subsection{\ttfamily verbos expr}
Not yet written

\subsection{\ttfamily when fexpr}
Not yet written

\subsection{\ttfamily where!-was!-that expr}
Not yet written

\subsection{\ttfamily window!-heading expr}
Not yet written

\subsection{\ttfamily writable!-libraryp expr}
Not yet written

\subsection{\ttfamily write!-module expr}
Not yet written

\subsection{\ttfamily wrs expr}
Not yet written

\subsection{\ttfamily xassoc expr}
Not yet written

\subsection{\ttfamily xcons expr}
Not yet written

\subsection{\ttfamily xdifference expr}
Not yet written

\subsection{\ttfamily xtab expr}
Not yet written

\subsection{\ttfamily zerop expr}
Not yet written

\subsection{\ttfamily !$\sim$block fexpr}
Not yet written

\subsection{\ttfamily !$\sim$let fexpr}
Not yet written

\subsection{\ttfamily !$\sim$tyi expr}
Not yet written




\end{document}
