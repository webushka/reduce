\section{Groebner package}
\begin{Introduction}{Groebner bases}
The GROEBNER package calculates \nameindex{Groebner bases} using the
\nameindex{Buchberger algorithm} and provides related algorithms
for arithmetic with ideal bases, such as ideal quotients,
Hilbert polynomials (\nameindex{Hollmann algorithm}), 
basis conversion (
\nameindex{Faugere-Gianni-Lazard-Mora algorithm}), independent  
variable set (\nameindex{Kredel-Weispfenning algorithm}).



Some routines of the Groebner package are used by \nameref{solve} - in
that context the package is loaded automatically. However, if you
want to use the package by explicit calls you must load it by
\begin{verbatim}
    load_package groebner;
\end{verbatim}

For the common parameter setting of most operators in this package 
see \nameref{ideal parameters}.
\end{Introduction}



\begin{Concept}{Ideal Parameters}
\index{polynomial}
Most operators of the \name{Groebner} package compute expressions in a
polynomial ring which given as \meta{R}[\meta{var},\meta{var},...] where
\meta{R} is the current REDUCE coefficient domain.  All algebraically
exact domains of REDUCE are supported.  The package can operate over rings
and fields.  The operation mode is distinguished automatically.  In
general the ring mode is a bit faster than the field mode.  The factoring
variant can be applied only over domains which allow you factoring of
multivariate polynomials.

The variable sequence \meta{var} is either declared explicitly as argument
in form of a \nameref{list} in \nameref{torder}, or it is extracted
automatically from the expressions.  In the second case the current REDUCE
system order is used (see \nameref{korder}) for arranging the variables.
If some kernels should play the role of formal parameters (the ground
domain \meta{R} then is the polynomial ring over these), the variable
sequences must be given explicitly.

All REDUCE \nameref{kernel}s can be used as variables.  But please note,
that all variables are considered as independent.  E.g. when using
\name{sin(a)} and \name{cos(a)} as variables, the basic relation
\name{sin(a)^2+cos(a)^2-1=0} must be explicitly added to an equation set
because the Groebner operators don't include such knowledge automatically.

The terms (monomials) in polynomials are arranged according to the current
\nameref{term order}.  Note that the algebraic properties of the computed
results only are valid as long as neither the ordering nor the variable
sequence changes.

The input expressions \meta{exp} can be polynomials \meta{p}, rational
functions \meta{n}/\meta{d} or equations \meta{lh}=\meta{rh} built from
polynomials or rational functions.  Apart from the \name{tracing}
algorithms \nameref{groebnert} and \nameref{preducet}, where the equations
have a specific meaning, equations are converted to simple expressions by
taking the difference of the left-hand and right-hand sides
\meta{lh}-\meta{rh}=>\meta{p}.  Rational functions are converted to
polynomials by converting the expression to a common denominator form
first, and then using the numerator only \meta{n}=>\meta{p}.  So eventual
zeros of the denominators are ignored.

A basis on input or output of an algorithm is coded as \nameref{list} of
expressions \{\meta{exp},\meta{exp},...\} . \end{Concept}

%-----------------------------------------------------------------
\subsection{Term order}
%-----------------------------------------------------------------
\begin{Introduction}{Term order}
\index{distributive polynomials}
For all \name{Groebner} operations the polynomials are 
represented in distributive form: a sum of terms (monomials).
The terms are ordered corresponding to the actual \name{term order}
which is set by the \nameref{torder} operator, and to the
actual variable sequence which is either given as explicit
parameter or by the system \nameref{kernel} order. 
\end{Introduction}

\begin{Operator}{torder}
The operator \name{torder} sets the actual variable sequence and term order.

1. simple term order:
\begin{Syntax}
  \name{torder}\(\meta{vl}, \meta{m}\)
\end{Syntax}

where  \meta{vl} is a \nameref{list} of variables (\nameref{kernel}s) and
\meta{m} is the name of a simple \nameref{term order} mode 
\ref{lex term order}, \ref{gradlex term order}, 
\ref{revgradlex term order} or another implemented parameterless mode.

2. stepped term order:
\begin{Syntax}
  
  \name{torder} \(\meta{vl},\meta{m},\meta{n}\)

\end{Syntax}
  
where \meta{m} is the name of a two step term order, one of
\nameref{gradlexgradlex term order}, \nameref{gradlexrevgradlex term order},
\nameref{lexgradlex term order} or \nameref{lexrevgradlex term order}, and
\meta{n} is a positive integer.

3. weighted term order
\begin{Syntax}
 \name{torder} \(\meta{vl}, \name{weighted}, \meta{n},\meta{n},...\); 
\end{Syntax}

where the \meta{n} are positive integers, see \nameref{weighted term order}.

4. matrix term order
\begin{Syntax}
 \name{torder} \(\meta{vl}, \name{matrix}, \meta{m}\); 
\end{Syntax}

where \meta{m} is a matrix with integer elements, see 
\nameref{torder_compile}.

5. compiled term order
\begin{Syntax}
 \name{torder} \(\meta{vl}, \name{co}\); 
\end{Syntax}

where \meta{co} is the name of a routine generated by 
\nameref{torder_compile}.

\name{torder} sets the variable sequence and the term order mode. If the
an empty list is used as variable sequence, the automatic variable extraction
is activated. The defaults are the empty variable list an the 
\nameref{lex term order}. 
The previous setting is returned as a list. 

Alternatively to the above syntax the arguments of \name{torder} may be 
collected in a \nameref{list} and passed as one argument to 
\name{torder}.

\end{Operator}
%------------------------------------------------------------
\begin{Operator}{torder_compile}
\index{term order}
A matrix can be converted into
a compilable LISP program for faster execution by using
\begin{Syntax}
    \name{torder\_compile}\(\meta{name},\meta{mat}\)
\end{Syntax}
where \meta{name} is an identifier for the new term order and \meta{mat}
is an integer matrix to be used as \nameref{matrix term order}. Afterwards
the term order can be activated by using \meta{name} in a \nameref{torder}
expression. The resulting program is compiled if the switch \nameref{comp}
is on, or if the  \name{torder\_compile} expression is part of a compiled
module.
\end{Operator}
%------------------------------------------------------------
\begin{Concept}{lex term order}
\index{term order}\index{variable elimination}
The terms are ordered lexicographically: two terms t1 t2 
are compared for their degrees 
along the fixed variable sequence: t1 is higher than t2
if the first different degree is higher in t1.
This order has the \name{elimination property}
for \name{groebner basis} calculations.
If the ideal has a univariate polynomial in the last
variable the groebner basis will contain
such polynomial. \name{Lex} is best
suited for solving of polynomial equation systems.

\end{Concept}

%------------------------------------------------------------
\begin{Concept}{gradlex term order}
\index{term order}
The terms are ordered first with their total
degree, and if the total degree is identical
the comparison is \nameref{lex term order}.
With \name{groebner} basis calculations this term order
produces polynomials of lowest degree.
\end{Concept}

%------------------------------------------------------------
\begin{Concept}{revgradlex term order}
\index{term order}
The terms are ordered first with their total
degree (degree sum), and if the total degree is identical
the comparison is the inverse of \nameref{lex term order}.
With \nameref{groebner} and \nameref{groebnerf} 
calculations this term order
is similar to \nameref{gradlex term order}; it is known
as most efficient ordering with respect to computing time.
\end{Concept}

%------------------------------------------------------------
\begin{Concept}{gradlexgradlex term order}
\index{term order}
The terms are separated into two groups where the
second parameter of the \nameref{torder} call determines
the length of the first group. For a comparison first
the total degrees of both variable groups are compared.
If both are equal 
\nameref{gradlex term order} comparison is applied to the first
group, and if that does not decide \nameref{gradlex term order}
is applied for the second group. This order has the elimination
property for the variable groups. It can be used e.g. for
separating variables from parameters.
\end{Concept}
%------------------------------------------------------------
\begin{Concept}{gradlexrevgradlex term order}
\index{term order}
Similar to \nameref{gradlexgradlex term order}, but using
\nameref{revgradlex term order} for the second group.
\end{Concept}
%------------------------------------------------------------
\begin{Concept}{lexgradlex term order}
\index{term order}
Similar to \nameref{gradlexgradlex term order}, but using
\nameref{lex term order} for the first group.
\end{Concept}
%------------------------------------------------------------
\begin{Concept}{lexrevgradlex term order}
\index{term order}
Similar to \nameref{gradlexgradlex term order}, but using
\nameref{lex term order} for the first group
\nameref{revgradlex term order} for the second group.
\end{Concept}
%------------------------------------------------------------
\begin{Concept}{weighted term order}
\index{term order}
establishes a graduated ordering
similar to \nameref{gradlex term order}, where the exponents first are
multiplied by the given weights. If there are less weight values than
variables, the weight list is extended by ones. If the weighted degree
comparison is not decidable, the 
\nameref{lex term order} is used.
\end{Concept}
%------------------------------------------------------------
\begin{Concept}{graded term order}
\index{term order}
establishes a cascaded term ordering:  first a graduated ordering
similar to \nameref{gradlex term order} is used, where the exponents first are
multiplied by the given weights. If there are less weight values than
variables, the weight list is extended by ones. If the weighted degree
comparison is not decidable, the term ordering described in the following
parameters of the \nameref{torder} command is used.
\end{Concept}
%------------------------------------------------------------
\begin{Concept}{matrix term order}
\index{term order}
Any arbitrary term order mode can be installed by a matrix with
integer elements where the row length corresponds to the variable
number. The matrix must have at least as many rows as columns.
It must have full rank, and the top nonzero element of each column
must be positive.

The matrix \name{term order mode}
defines a term order where the exponent vectors of the monomials are
first multiplied by the matrix and the resulting vectors are compared
lexicographically.

If the switch \nameref{comp} is on, the matrix is converted into
a compiled LISP program for faster execution. A matrix can also be
compiled explicitly, see \nameref{torder_compile}.
\end{Concept}
%--------------------------------------------------------------- 
%------------------------------------------------------------
\subsection{Basic Groebner operators}
%-------------------------------------------------------------
\begin{Operator}{gvars}
\begin{Syntax}

  \name{gvars}\(\{\meta{exp},\meta{exp},... \}\)

\end{Syntax}
 where \meta{exp} are expressions or \nameref{equation}s.

\name{gvars} extracts from the expressions the \nameref{kernel}\name{s} 
which can 
play the role of variables for a \nameref{groebner} or \nameref{groebnerf} 
calculation. 
\end{Operator}

%---------------------------------------------------------------

\begin{Operator}{groebner}
\index{Buchberger algorithm}
\begin{Syntax}

  \name{groebner}\(\{\name{exp}, ...\}\)

\end{Syntax}
where \{\name{exp}, ... \} is a list of
expressions or equations.


The operator \name{groebner} implements the Buchberger algorithm
for computing Groebner bases for a given set of
expressions with respect to the given set of variables in the order
given.  As a side effect, the sequence of variables is stored as a REDUCE list
in the shared variable \nameref{gvarslast} - this is important in cases
where the algorithm rearranges the variable sequence because \nameref{groebopt}
is \name{on}.

\begin{Examples}
   groebner({x**2+y**2-1,x-y})  &  \{X - Y,2*Y**2 -1\}
\end{Examples}
\begin{Related}
\item[ \nameref{groebnerf} operator]
\item[ \nameref{gvarslast} variable]
\item[ \nameref{groebopt} switch]
\item[ \nameref{groebprereduce} switch]
\item[ \nameref{groebfullreduction} switch]
\item[ \nameref{gltbasis} switch]
\item[ \nameref{gltb} variable]
\item[ \nameref{glterms} variable]
\item[ \nameref{groebstat} switch]
\item[ \nameref{trgroeb} switch]
\item[ \nameref{trgroebs} switch]
\item[ \nameref{groebprot} switch]
\item[ \nameref{groebprotfile} variable]
\item[ \nameref{groebnert} operator]
\end{Related}
\end{Operator}
%-------------------------------------------------------

\begin{Operator}{groebner\_walk}
The operator \name{groebner\_walk} computes a \nameref{lex} basis
from a given \nameref{graded} (or \nameref{weighted}) one.
\begin{Syntax}
   \name{groebner\_walk}\(\meta{g}\)
\end{Syntax}

where \meta{g} is a \nameref{graded} basis (or \nameref{weighted} basis
with a weight vector with one repeated element) of the polynomial ideal. 
\name{Groebner\_walk} computes a sequence of monomial bases, each
time lifting the full system to a complete basis.  \name{Groebner\_walk}
should be called only in cases, where a normal \nameref{kex} computation
would take too much computer time.

The operator \nameref{torder} has to be called before in order to
define the variable sequence and the term order mode of \meta{g}. 

The variable \nameref{gvarslast} is not set.

Do not call \name{groebner\_walk} with \name{on} \nameref{groebopt}.

\name{Groebner\_walk} includes some overhead (such as e. g. 
computation with division). On the other hand, sometimes
\name{groebner\_walk} is faster than a direct \nameref{lex} computation.
\end{Operator}

%-------------------------------------------------------

\begin{Switch}{groebopt}
If \name{groebopt} is set ON, the sequence of variables is optimized
with respect to execution speed of \name{groebner} calculations; 
note that the final list of variables is available in \nameref{gvarslast}.
By default \name{groebopt} is off, conserving the original variable
sequence.

An explicitly declared dependency using the \nameref{depend}
declaration  supersedes the variable optimization.
\begin{Examples}

   depend a, x, y;

\end{Examples}
guarantees that a will be placed in front of x and y.
\end{Switch}


%-------------------------------------------------------

\begin{Variable}{gvarslast}
After a \nameref{groebner} or \nameref{groebnerf} calculation
the actual variable sequence is stored in the variable 
\name{gvarslast}. If \nameref{groebopt} is \name{on}
\name{gvarslast} shows the variable sequence after reordering.
\end{Variable}

%--------------------------------------------------------------

\begin{Switch}{groebprereduce}
If \name{groebprereduce} set ON, \nameref{groebner} 
and \nameref{groebnerf} try to simplify the
input expressions: if the head term of an input expression is a
multiple of the head term of another expression, it can be reduced;
these reductions are done cyclicly as long as possible in order to
shorten the main part of the algorithm.

By default \name{groebprereduce} is off.
\end{Switch}

%---------------------------------------------------------------

\begin{Switch}{groebfullreduction}
If \name{groebfullreduction} set off, the polynomial reduction steps during
\nameref{groebner} and \nameref{groebnerf} are limited to the pure head
term reduction; subsequent terms are reduced otherwise.

By default \name{groebfullreduction} is on.
\end{Switch}

%----------------------------------------------------------------

\begin{Switch}{gltbasis}
If \name{gltbasis} set on, the leading terms of the result basis 
of a \nameref{groebner} or \nameref{groebnerf} calculation are
extracted. They are collected as a basis of monomials, which is
available as value of the global variable \nameref{gltb}.
\end{Switch}
%------------------------------------------------------------------
\begin{Variable}{gltb}
See \nameref{gltbasis}
\end{Variable}
%------------------------------------------------------------------

\begin{Variable}{glterms}
If the expressions in a \nameref{groebner} or \nameref{groebnerf} 
call contain parameters (symbols
which are not member of the variable list), the share variable
\name{glterms} is set to a list of expression which during the
calculation were assumed to be nonzero. The calculated bases 
are valid only under the assumption that all these expressions do
not vanish.
\end{Variable}

%-----------------------------------------------------------
\begin{Switch}{groebstat}
if \name{groebstat} is on, a summary of the 
\nameref{groebner} or \nameref{groebnerf} computation is printed
at the end 
including the computing time, the number of intermediate
H polynomials and the counters for the criteria hits.
\end{Switch}

%-----------------------------------------------------------
\begin{Switch}{trgroeb}
if \name{trgroeb} is on, intermediate H polynomials are 
printed during a \nameref{groebner} 
or \nameref{groebnerf} calculation.
\end{Switch}

%-----------------------------------------------------------
\begin{Switch}{trgroebs}
if \name{trgroebs} is on, intermediate H and S polynomials are 
printed during a \nameref{groebner} or \nameref{groebnerf} calculation.
\end{Switch}

%-----------------------------------------------------------
\begin{Operator}{gzerodim?} 
\begin{Syntax}

  \name{gzerodim!?}\(\meta{basis}\)

\end{Syntax}
where \meta{bas} is a Groebner basis in the current 
\nameref{term order} with the actual setting 
(see \nameref{ideal parameters}). 


\name{gzerodim!?} tests whether the ideal spanned by the given basis 
has dimension zero. If yes, the number of zeros is returned,
\nameref{nil} otherwise.
\end{Operator}

%---------------------------------------------------------------

\begin{Operator}{gdimension}
\index{ideal dimension}\index{groebner}
\begin{Syntax}

     \name{gdimension}\(\meta{bas}\) 

\end{Syntax}
where \meta{bas} is a \nameref{groebner} basis in the current
term order (see \nameref{ideal parameters}). 
\name{gdimension} computes the dimension of the ideal
spanned by the given basis and returns the dimension as an integer
number. The Kredel-Weispfenning algorithm is used: the dimension
is the length of the longest independent variable set,
see \nameref{gindependent\_sets}
\end{Operator}


%---------------------------------------------------------------

\begin{Operator}{gindependent\_sets}
\index{ideal variables}\index{ideal dimension}\index{groebner}
\index{Kredel-Weispfenning algorithm}
\begin{Syntax}

  \name{gindependent\_sets}\(\meta{bas}\)

\end{Syntax}
where \meta{bas} is a \nameref{groebner} basis in any \name{term order} 
(which must be the current \name{term order}) with the specified
variables (see \nameref{ideal parameters}). 


\name{Gindependent_sets} computes the maximal
left independent variable sets of the ideal, that are 
the variable sets which play the role of free parameters in the
current ideal basis. Each set is a list which is a subset of the
variable list. The result is a list of these sets. For an
ideal with dimension zero the list is empty.
The Kredel-Weispfenning algorithm is used.
\end{Operator}

%--------------------------------------------------------------

\begin{Operator}{dd_groebner}
For a homogeneous system of polynomials under 
\nameref{graded term order}, \nameref{gradlex term order}, 
\nameref{revgradlex term order} 
or \nameref{weighted term order} 
a Groebner Base can be computed with limiting the grade
of the intermediate S polynomials: 
\begin{Syntax}
\name{dd_groebner}\(\meta{d1},\meta{d2},\meta{plist}\)
\end{Syntax}
where \meta{d1} is a non negative integer and \meta{d2} is an integer
or ``infinity". A pair of polynomials is considered
only if the grade of the lcm of their head terms is between
\meta{d1} and \meta{d2}.
For the term orders \name{graded} or \name{weighted} the (first) weight
vector is used for the grade computation. Otherwise the total
degree of a term is used.
\end{Operator}

%--------------------------------------------------------------


\begin{Operator}{glexconvert}
\index{ideal variables}\index{term order}
\begin{Syntax}

\name{glexconvert}\(\meta{bas}[,\meta{vars}][,MAXDEG=\meta{mx}]
[,NEWVARS=\meta{nv}]\)

\end{Syntax}
where \meta{bas} is a \nameref{groebner} basis
in the current term order,  \meta{mx} (optional) is a positive
integer and \meta{nvl} (optional) is a list of variables 
(see \nameref{ideal parameters}).


The operator \name{glexconvert} converts the basis 
of a zero-dimensional ideal (finite number
of isolated solutions) from arbitrary ordering into a basis under 
\nameref{lex term order}. 


The parameter \meta{newvars} defines the new variable sequence. 
If omitted, the
original variable sequence is used. If only a subset of variables is
specified here, the partial ideal basis is evaluated. 

If \meta{newvars} is a list with one element, the minimal
\nameindex{univariate polynomial} is computed.

\meta{maxdeg} is an upper limit for the degrees. The algorithm stops with
an error message, if this limit is reached.

A warning occurs, if the ideal is not zero dimensional.
\begin{Comments}
During the call the \name{term order} of the input basis must
be active.
\end{Comments}
\end{Operator}

%--------------------------------------------------------------

\begin{Operator}{greduce}
\begin{Syntax}

\name{greduce}\(exp, \{exp1, exp2, \ldots , expm\}\)

\end{Syntax}

where exp is an expression, and \{exp1, exp2, ... , expm\} is
a list of expressions or equations.


\name{greduce} is functionally equivalent with a call to
\nameref{groebner} and then a call to \nameref{preduce}.
\end{Operator}

%---------------------------------------------------------

\begin{Operator}{preduce}
\begin{Syntax}

 \name{preduce}\(\meta{p}, \{\meta{exp}, \ldots \}\)

\end{Syntax}

where \meta{p} is an expression, and \{\meta{exp}, ... \} is
a list of expressions or equations.


\name{Preduce} computes the remainder of \name{exp}
modulo the given set of polynomials resp. equations.
This result is unique (canonical) only if the given set
is a \name{groebner} basis under the current \nameref{term order}

see also: \nameref{preducet} operator.

\end{Operator}


%-------------------------------------------

\begin{Operator}{idealquotient}
\begin{Syntax}

\name{idealquotient}\(\{\meta{exp}, ...\}, \meta{d}\)

\end{Syntax}
where \{\meta{exp},...\} is a list of 
expressions or equations,  \meta{d} is a single expression or equation.


\name{Idealquotient} computes the ideal quotient:
ideal spanned by the expressions \{\meta{exp},...\}
divided by the single polynomial/expression \meta{f}. The result
is the \nameref{groebner} basis of the quotient ideal.
\end{Operator}

%-------------------------------------------------------------

\begin{Operator}{hilbertpolynomial}
\index{Hollmann algorithm}
\begin{Syntax}

  hilbertpolynomial\(\meta{bas}\)

\end{Syntax}
where \meta{bas} is a \nameref{groebner} basis in the
current \nameref{term order}.

The degree of the \name{Hilbert polynomial} is the
dimension of the ideal spanned by the basis. For an
ideal of dimension zero the Hilbert polynomial is a
constant which is the number of common zeros of the
ideal (including eventual multiplicities).
The \name{Hollmann algorithm} is used.
\end{Operator}

%-------------------------------------------

\begin{Operator}{saturation}
\begin{Syntax}

\name{saturation}\(\{\meta{exp}, ...\}, \meta{p}\)

\end{Syntax}
where \{\meta{exp},...\} is a list of
expressions or equations,  \meta{p} is a single polynomial.

\name{Saturation} computes the quotient of the polynomial \meta{p}
and a power (with unknown but finite exponent) of the ideal built from
\{\meta{exp}, ...\}. The result is the computed quotient. \name{Saturation}
calls \nameref{idealquotient} several times until the result does not change
any more.
\end{Operator}

%-------------------------------------------------------------
\subsection{Factorizing Groebner bases}
%-------------------------------------------------------------

\begin{Operator}{groebnerf}
\begin{Syntax}

\name{groebnerf}\(\{\meta{exp}, ...\}[,\{\},\{\meta{nz}, ... \}]\);

\end{Syntax}
where \{\meta{exp}, ... \} is a list of expressions or
equations, and \{\meta{nz},... \} is
an optional list of polynomials to be considered as non zero
for this calculation. An empty list must be passed as second argument
if the non-zero list is specified.


\name{groebnerf} tries to separate polynomials into individual factors and
to branch the computation in a recursive manner (factorization tree).
The result is a list of partial Groebner bases. 
Multiplicities (one factor with a higher power, the same partial basis
twice) are deleted as early as possible in order to speed up the
calculation. 

The third parameter of \name{groebnerf} declares some polynomials
nonzero. If any of these is found in a branch of the calculation
the branch is canceled. 

\begin{Bigexample}
groebnerf({ 3*x**2*y+2*x*y+y+9*x**2+5*x = 3,  
            2*x**3*y-x*y-y+6*x**3-2*x**2-3*x = -3, 
            x**3*y+x**2*y+3*x**3+2*x**2 }, {y,x});

       {{Y - 3,X},

                      2
    {2*Y + 2*X - 1,2*X  - 5*X - 5}}
\end{Bigexample}

\begin{Related}
\item[ \nameref{groebresmax} variable]
\item[ \nameref{groebmonfac} variable]
\item[ \nameref{groebrestriction} variable]
\item[ \nameref{groebner} operator]
\item[ \nameref{gvarslast} variable]
\item[ \nameref{groebopt} switch]
\item[ \nameref{groebprereduce} switch]
\item[ \nameref{groebfullreduction} switch]
\item[ \nameref{gltbasis} switch]
\item[ \nameref{gltb} variable]
\item[ \nameref{glterms} variable]
\item[ \nameref{groebstat} switch]
\item[ \nameref{trgroeb} switch]
\item[ \nameref{trgroebs} switch]
\item[ \nameref{groebnert} operator]
\end{Related}

\end{Operator}

% ------------------------------------------------------------------

\begin{Variable}{groebmonfac}
The variable \name{groebmonfac} is connected to
the handling of monomial factors.  A monomial factor is a product
of variable powers as a factor, e.g. x**2*y  in  x**3*y -
2*x**2*y**2.  A monomial factor represents a solution of the type
 x = 0  or  y = 0 with a certain multiplicity.  With
\nameref{groebnerf} the multiplicity of monomial factors is lowered 
to the value of the shared variable \name{groebmonfac}
which by default is 1 (= monomial factors remain present, but their
multiplicity is brought down). With
\name{groebmonfac}:= 0
the monomial factors are suppressed completely.
\end{Variable}

% ----------------------------------------------------------------
\begin{Variable}{groebresmax}
The variable \name{groebresmax}
controls  during \nameref{groebnerf} calculations
the number of partial results. Its default value is 300. If
more partial results are calculated, the calculation is
terminated.
\end{Variable}

% ----------------------------------------------------------------
\begin{Variable}{groebrestriction}
During \nameref{groebnerf} calculations 
irrelevant branches can be excluded
by setting the variable \name{groebrestriction}. The
following restrictions are implemented:
\begin{Syntax} 
     \name{groebrestriction} := \name{nonnegative} \\
     \name{groebrestriction} := \name{positive}\\
     \name{groebrestriction} := \name{zeropoint}
\end{Syntax}
With \name{nonnegative} branches are excluded where one
polynomial has no nonnegative real zeros; with \name{positive}
the restriction is sharpened to positive zeros only.
The restriction \name{zeropoint} excludes all branches
which do not have the origin (0,0,...0) in their solution
set.
\end{Variable}

%---------------------------------------------------------
\subsection{Tracing Groebner bases}
%---------------------------------------------------------
\index{tracing Groebner}
\begin{Switch}{groebprot}
If \name{groebprot} is \name{ON} the computation steps during
\nameref{preduce}, \nameref{greduce} and \nameref{groebner}
are collected in a list which is assigned to the variable
\nameref{groebprotfile}.
\end{Switch}
%----------------------------------------------------------
\begin{Variable}{groebprotfile}
See \nameref{groebprot} switch.
\end{Variable}
%----------------------------------------------------------

\begin{Operator}{groebnert}
\begin{Syntax}

  \name{groebnert}\(\{\meta{v}=\meta{exp},...\}\)

\end{Syntax}
where \meta{v} are \nameref{kernel}\name{s} (simple or indexed variables),
\meta{exp} are polynomials.


\name{groebnert} is functionally equivalent to a \nameref{groebner}
call for \{\meta{exp},...\}, but the result is a set of
equations where the left-hand sides are the basis elements while
the right-hand sides are the same values expressed as combinations
of the input formulas, expressed in terms of the names \meta{v}
\begin{Bigexample}
    groebnert({p1=2*x**2+4*y**2-100,p2=2*x-y+1});

   GB1 := {2*X - Y + 1=P2,

           2
        9*Y  - 2*Y - 199= - 2*X*P2 - Y*P2 + 2*P1 + P2}
\end{Bigexample}
\end{Operator}
%----------------------------------------------------------

\begin{Operator}{preducet}
\begin{Syntax}

\name{preduce}\(\meta{p},\{\meta{v}=\meta{exp}...\}\)
\end{Syntax}
where \meta{p} is an expression, \meta{v} are kernels 
(simple or indexed variables),
\name{exp} are polynomials.

\name{preducet} computes the remainder of \meta{p} modulo \{\meta{exp},...\}
similar to \nameref{preduce}, but the result is an equation
which expresses the remainder as combination of the polynomials.
\begin{Bigexample}
                             
   GB2 := {G1=2*X - Y + 1,G2=9*Y**2  - 2*Y - 199}
   preducet(q=x**2,gb2);

 - 16*Y + 208= - 18*X*G1 - 9*Y*G1 + 36*Q + 9*G1 - G2
\end{Bigexample}
\end{Operator}

%------------------------------------------------------------
\subsection{Groebner Bases for Modules}
%------------------------------------------------------------
\begin{Concept}{Module}
Given a polynomial ring, e.g. R=Z[x,y,...] and an integer n>1.
The vectors with n elements of R form a free MODULE under
elementwise addition and multiplication with elements of R.

For a submodule given by a finite basis a Groebner basis
can be computed, and the facilities of the GROEBNER package
are available except the operators \nameref{groebnerf}
and \name{groesolve}. The vectors are encoded using auxiliary
variables which represent the unit vectors in the module.
These are declared in the share variable \nameref{gmodule}.

\end{Concept}

\begin{Variable}{gmodule}
The vectors of a free \nameref{module} over a polynomial ring R 
are encoded as linear combinations with unit vectors of
M which are represented by auxiliary variables. These
must be collected in the variable \name{gmodule} before
any call to an operator of the Groebner package.

\begin{verbatim}
   torder({x,y,v1,v2,v3})$
   gmodule := {v1,v2,v3}$
   g:=groebner({x^2*v1 + y*v2,x*y*v1 - v3,2y*v1 + y*v3});
\end{verbatim}

compute the Groebner basis of the submodule

\begin{verbatim}
      ([x^2,y,0],[xy,0,-1],[0,2y,y])
\end{verbatim}
The members of the list \name{gmodule} are automatically
appended to the end of the variable list, if they are not
yet members there. They take part in the actual term ordering.
\end{Variable}

%------------------------------------------------------------
\subsection{Computing with distributive polynomials}
%------------------------------------------------------------

\begin{Operator}{gsort}
\index{distributive polynomials}
\begin{Syntax}

 \name{gsort}\(\meta{p}\)
\end{Syntax}
where \meta{p} is a polynomial or a list of polynomials.

The polynomials are reordered and sorted corresponding to
the current \nameref{term order}.
\begin{Examples}

  torder lex;\\  
  gsort(x**2+2x*y+y**2,{y,x});  &  {y**2+2y*x+x**2}

\end{Examples}
\end{Operator}

%------------------------------------------------------------

\begin{Operator}{gsplit}
\index{distributive polynomials}
\begin{Syntax}

 \name{gsplit}\(\meta{p}[,\meta{vars}]\);
\end{Syntax}
where \meta{p} is a polynomial or a list of polynomials.

The polynomial is reordered corresponding to the 
the current \nameref{term order} and then
separated into leading term and reductum. Result is
a list with the leading term as first and the reductum
as second element.
\begin{Examples}

  torder lex;\\  
  gsplit(x**2+2x*y+y**2,{y,x});  &  \{y**2,2y*x+x**2\}

\end{Examples}
\end{Operator}
%-------------------------------------------------------

\begin{Operator}{gspoly}
\index{distributive polynomials}
\begin{Syntax}

 \name{gspoly}\(\meta{p1},\meta{p2}\);

\end{Syntax}
where \meta{p1} and \meta{p2} are polynomials.

The \name{subtraction} polynomial of p1 and p2 is computed
corresponding to the method of the Buchberger algorithm for
computing \name{groebner bases}: p1 and p2 are multiplied
with terms such that when subtracting them the leading terms 
cancel each other.
\end{Operator}

