\chapter{TPS: A truncated power series package}
\label{TPS}
\typeout{{TPS: A truncated power series package}}

{\footnotesize
\begin{center}
Alan Barnes \\ 
Dept. of Computer Science and Applied Mathematics \\
Aston University, Aston Triangle, \\
Birmingham B4 7ET, England \\[0.05in]
e--mail: barnesa@aston.ac.uk \\[0.1in]
and \\[0.1in]
Julian Padget \\
School of Mathematics, University of Bath \\
Bath, BA2 7AY, England \\[0.05in]
e--mail: jap@maths.bath.ac.uk
\end{center}
}
\ttindex{TPS}\ttindex{PS}
\index{power series}\index{truncated power series}

\index{Laurent series expansions}
This package implements formal Laurent series expansions in one
variable using the domain mechanism of \REDUCE.  This means that power
series objects can be added, multiplied, differentiated {\em etc}. like other
first class objects in the system.  A lazy evaluation scheme is used in
the package and thus terms of the series are not evaluated until they
are required for printing or for use in calculating terms in other
power series.  The series are extendible giving the user the impression
that the full infinite series is being manipulated.  The errors that
can sometimes occur using series that are truncated at some fixed depth
(for example when a term in the required series depends on terms of an
intermediate series beyond the truncation depth) are thus avoided.
\newpage
\section{Basic Truncated Power Series}
\subsection{PS Operator}

Syntax:

\noindent{\tt PS}(EXPRN:{\em algebraic},DEPVAR:{\em kernel},ABOUT:{\em algebraic}):{\em ps object}

\index{PS operator}
The {\tt PS} operator returns a  power series object
representing the univariate formal power series expansion of EXPRN with
respect to the dependent variable DEPVAR about the expansion point
ABOUT.  EXPRN may itself contain power series objects.

The algebraic expression ABOUT should simplify to an expression
which is independent of the dependent variable DEPVAR, otherwise
an error will result.  If ABOUT is the identifier {\tt INFINITY}
then the power series expansion about DEPVAR = $\infty$ is
obtained in ascending powers of 1/DEPVAR.

\index{PSEXPLIM operator}
The power series object
representing EXPRN is compiled and then a number of terms of the
power series expansion are evaluated.  The expansion is
carried out as far as the value specified by {\tt PSEXPLIM}.  If,
subsequently, the value of {\tt PSEXPLIM} is increased, sufficient
information is stored in the power series object to enable the
additional terms to be calculated without recalculating the terms
already obtained.

If the function has a pole at the expansion point then the correct
Laurent series expansion will be produced.

\noindent The following examples are valid uses of {\tt PS}:
\begin{verbatim}
    psexplim 6;
    ps(log x,x,1);
    ps(e**(sin x),x,0);
    ps(x/(1+x),x,infinity);
    ps(sin x/(1-cos x),x,0);
\end{verbatim}

\index{power series ! of user defined function}
New user-defined functions may be expanded provided the user provides
LET rules giving

\begin{enumerate}
\item the value of the function at the expansion point
\item a differentiation rule for the new function.
\end{enumerate}

\noindent For example
\begin{verbatim}
    operator sech;
    forall x let df(sech x,x)= - sech x * tanh x;
    let sech 0 = 1;
    ps(sech(x**2),x,0);
\end{verbatim}
 
\index{power series ! of integral}
The power series expansion of an integral may also be obtained (even if
\REDUCE\ cannot evaluate the integral in closed form).  An example of
this is

\begin{verbatim}
    ps(int(e**x/x,x),x,1);
\end{verbatim}
 
Note that if the integration variable is the same as the expansion
variable then \REDUCE's integration package is not called; if on the
other hand the two variables are different then the integrator is
called to integrate each of the coefficients in the power series
expansion of the integrand.  The constant of integration is zero by
default.  If another value is desired, then the shared variable {\tt
PSINTCONST} should be set to required value.\index{PSINTCONST (shared)}

\subsection{PSORDLIM Operator}

\index{PSORDLIM operator}
Syntax:

\hspace*{2em} {\tt PSORDLIM}(UPTO:{\em integer}):{\em integer}

\hspace*{4em} or

\hspace*{2em} {\tt PSORDLIM}():{\em integer}

An internal variable is set to the value of {\tt UPTO} (which should
evaluate to an integer).  The value returned is the previous value of
the variable.  The default value is 15.

If {\tt PSORDLIM} is called with no argument, the current value is
returned.

The significance of this control is that the system attempts to find
the order of the power series required, that is the order is the
degree of the first non-zero term in the power series.  If the order
is greater than the value of this variable an error message is given
and the computation aborts.  This prevents infinite loops in examples
such as

\begin{verbatim}
        ps(1 - (sin x)**2 - (cos x)**2,x,0);
\end{verbatim}

where the expression being expanded is identically zero, but is not
recognised as such by \REDUCE.


\section{Controlling Power Series}
\subsection{PSTERM Operator}

\index{PSTERM operator}
Syntax:

\hspace*{2em} {\tt PSTERM}(TPS:{\em power series object},NTH:{\em integer}):{\em algebraic}

The operator {\tt PSTERM} returns the NTH term of the existing
power series object TPS. If NTH does not evaluate to
an integer or TPS to a power series object an error results.  It
should be noted that an integer is treated as a power series.


\subsection{PSORDER Operator}

\index{PSORDER operator}
Syntax:

\hspace*{2em} {\tt PSORDER}(TPS:{\em power series object}):{\em integer}

The operator {\tt PSORDER} returns the order, that is the degree of
the first non-zero term, of the power series object TPS.
TPS should evaluate to a power series object or an error results. If
TPS is zero, the identifier {\tt UNDEFINED} is returned.

\subsection{PSSETORDER Operator}

\index{PSSETORDER operator}
Syntax:

\hspace*{2em} {\tt PSSETORDER}(TPS:{\em power series object}, ORD:{\em integer}):{\em integer}

The operator {\tt PSSETORDER} sets the order of the power series TPS to the
value ORD, which should evaluate to an integer. If
TPS does not evaluate to a power series object, then an error
occurs. The value returned by this operator is the previous order of
TPS, or 0 if the order of TPS was undefined.  This
operator is useful for setting the order of the power series of a
function defined by a differential equation in cases where the power
series package is inadequate to determine the order automatically.


\subsection{PSDEPVAR Operator}

\index{PSDEPVAR operator}
Syntax:

\hspace*{2em} {\tt PSDEPVAR}(TPS:{\em power series object}):{\em identifier}

The operator {\tt PSDEPVAR} returns the expansion variable of the
power series object TPS. TPS should evaluate to a power
series object or an integer, otherwise an error results. If TPS
is an integer, the identifier {\tt UNDEFINED} is returned.

\subsection{PSEXPANSIONPT operator}

\index{PSEXPANSIONPT operator}
Syntax:

\hspace*{2em} {\tt PSEXPANSIONPT}(TPS:{\em power series object}):{\em algebraic}

The operator {\tt PSEXPANSIONPT} returns the expansion point of the
power series object TPS. TPS should evaluate to a power
series object or an integer, otherwise an error results. If TPS
is integer, the identifier {\tt UNDEFINED} is returned. If the
expansion is about infinity, the identifier {\tt INFINITY} is
returned.

\subsection{PSFUNCTION Operator}

\index{PSFUNCTION operator}
Syntax:

\hspace*{2em} {\tt PSFUNCTION}(TPS:{\em power series object}):{\em algebraic}

The operator {\tt PSFUNCTION} returns the function whose expansion
gave rise to the power series object TPS. TPS should
evaluate to a power series object or an integer, otherwise an error
results.

\subsection{PSCHANGEVAR Operator}

\index{PSCHANGEVAR operator}
Syntax:

\hspace*{2em} {\tt PSCHANGEVAR}(TPS:{\em power series object}, X:{\em kernel}):{\em power series object}

The operator {\tt PSCHANGEVAR} changes the dependent variable of the
power series object TPS to the variable X. TPS
should evaluate to a power series object and X to a kernel,
otherwise an error results.  Also X should not appear as a
parameter in TPS. The power series with the new dependent
variable is returned.

\subsection{PSREVERSE Operator}

\index{PSREVERSE operator}
Syntax:

\hspace*{2em} {\tt PSREVERSE}(TPS:{\em power series object}):{\em power series}

Power series reversion.  The power series TPS is functionally
inverted.  Four cases arise:

\begin{enumerate}
\item If the order of the series is 1, then the expansion point of the
inverted series is 0. 

\item If the order is 0 {\em and} if the first order term in TPS
is non-zero, then the expansion point of the inverted series is taken
to be the coefficient of the zeroth order term in TPS.

\item If the order is -1 the expansion point of the inverted series
is the point at infinity.  In all other cases a \REDUCE\ error is
reported because the series cannot be inverted as a power series. Puiseux
\index{Puiseux expansion} expansion would be required to handle these cases.

\item If the expansion point of TPS is finite it becomes the
zeroth order term in the inverted series. For expansion about 0 or the
point at infinity the order of the inverted series is one.
\end{enumerate}

If TPS is not a power series object after evaluation an error results.

\noindent Here are some examples:
\begin{verbatim}
        ps(sin x,x,0);
        psreverse(ws); % produces series for asin x about x=0.
        ps(exp x,x,0);
        psreverse ws; % produces series for log x about x=1.
        ps(sin(1/x),x,infinity);
        psreverse(ws); % produces series for 1/asin(x) about x=0.
\end{verbatim}

\subsection{PSCOMPOSE Operator}

\index{PSCOMPOSE operator}
Syntax:

\hspace*{2em} {\tt PSCOMPOSE}(TPS1:{\em power series}, TPS2:{\em power series}):{\em power series}

\index{power series ! composition}
{\tt PSCOMPOSE} performs power series composition.
The power series TPS1 and TPS2 are functionally composed.
That is to say that TPS2 is substituted for the expansion
variable in TPS1 and the result expressed as a power series. The
dependent variable and expansion point of the result coincide with
those of TPS2.  The following conditions apply to power series
composition:

\begin{enumerate}
\item If the expansion point of TPS1 is 0 then the order of the
TPS2 must be at least 1.

\item If the expansion point of TPS1 is finite, it should
coincide with the coefficient of the zeroth order term in TPS2.
The order of TPS2 should also be non-negative in this case.

\item If the expansion point of TPS1 is the point at infinity
then the order of TPS2 must be less than or equal to -1.

\end{enumerate}

If these conditions do not hold the series cannot be composed (with
the current algorithm terms of the inverted series would involve
infinite sums) and a \REDUCE\ error occurs.

\noindent Examples of power series composition include the following.

\begin{verbatim}
    a:=ps(exp y,y,0);  b:=ps(sin x,x,0); 
    pscompose(a,b);
    % Produces the power series expansion of exp(sin x)
    % about x=0.

    a:=ps(exp z,z,1); b:=ps(cos x,x,0);
    pscompose(a,b);
    % Produces the power series expansion of exp(cos x)
    % about x=0.

    a:=ps(cos(1/x),x,infinity);  b:=ps(1/sin x,x,0);
    pscompose(a,b);
    % Produces the power series expansion of cos(sin x)
    % about x=0.
\end{verbatim}

\subsection{PSSUM Operator}

\index{PSSUM operator}
Syntax:

\begin{tabbing}
\hspace*{2em} {\tt PSSUM}(\=J:{\em kernel} = LOWLIM:{\em integer}, COEFF:{\em algebraic}, X:{\em kernel}, \\ 
\> ABOUT:{\em algebraic}, POWER:{\em algebraic}):{\em power series}
\end{tabbing}

The formal power series sum for J from LOWLIM to {\tt INFINITY} of 

\begin{verbatim}
      COEFF*(X-ABOUT)**POWER
\end{verbatim}

or if ABOUT is given as {\tt INFINITY}

\begin{verbatim}
      COEFF*(1/X)**POWER
\end{verbatim}

is constructed and returned. This enables power series whose general
term is known to be constructed and manipulated using the other
procedures of the power series package. 

J and X should be distinct simple kernels. The algebraics
ABOUT,  COEFF and POWER should not depend on the
expansion variable X, similarly the algebraic ABOUT should
not depend on the summation variable J.  The algebraic POWER should be
a strictly increasing integer valued function of J for J in the range
LOWLIM to {\tt INFINITY}.

\begin{verbatim}
   pssum(n=0,1,x,0,n*n);
   % Produces the power series summation for n=0 to
   % infinity of x**(n*n).

   pssum(m=1,(-1)**(m-1)/(2m-1),y,1,2m-1);
   % Produces the power series expansion of atan(y-1)
   % about y=1.

   pssum(j=1,-1/j,x,infinity,j);
   % Produces the power series expansion of log(1-1/x)
   % about the point at infinity.

   pssum(n=0,1,x,0,2n**2+3n) + pssum(n=1,1,x,0,2n**2-3n);
   % Produces the power series summation for n=-infinity
   % to +infinity of x**(2n**2+3n).
\end{verbatim}

\subsection{Arithmetic Operations}

\index{power series ! arithmetic}
As power series objects are domain elements they may be combined
together in algebraic expressions in algebraic mode of \REDUCE\ in the
normal way.
 
For example if A and B are power series objects then the commands such as:

\index{+ ! power series}\index{- ! power series}\index{/ ! power series}
\index{* ! power series}\index{** ! power series}
\begin{verbatim}
    a*b;
    a**2+b**2;
\end{verbatim}

will produce power series objects representing the product and the sum
of the squares of the power series objects A and B respectively.
 
\subsection{Differentiation}

\index{power series ! differentiation}
If A is a power series object depending on X then the input
{\tt df(a,x);} will produce the power series expansion of the derivative
of A with respect to X.


\section{Restrictions and Known Bugs}

If A and B are power series objects and X is a variable
which evaluates to itself then currently expressions such as {\tt a/b} and
{\tt a*x} do not evaluate to a single power series object (although the
results are in each case formally valid).  Instead use {\tt ps(a/b,x,0)}
and {\tt ps(a*x,x,0)} {\em etc.}. 

