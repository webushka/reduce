

\chapter*{Introduction}

\section{Opening Remarks}

 This document describes PSL (Portable Standard Lisp \footnote{
"LSP"  backwards!}),  a portable, "small" and fast LISP developed 
originally at the University  of  Utah and the Hewlett-Packard Company.
The present version of PSL is supported by the Konrad--Zuse-Zentrum Berlin. 
PSL is upward-compatible with Standard Lisp 
\footnote{J. B. Marti, A. C. Hearn, M. L. Griss, C. Griss: 
The Standard Lisp Report}.  In most cases,
Standard Lisp did not commit itself to  specific  implementation
details  (since it was to be compatible with a portion of "most"
LISPs).  PSL is more specific and provides many  more  functions
than described in that report.

The goals of PSL include:
\begin{itemize}
\item Providing implementation tools for LISP that can be used to
     implement   a   variety  of  LISP-like  systems,  including
     mini-Lisps embedded in  other  language  systems.
\item Effectively  supporting the algebra system on a
     number of machines.
\item Studying  the  utility of a LISP-based systems language for
     other applications (such as CAGD or VLSI design)  in  which
     Lisp code provides efficiency comparable to that of C or
     BCPL,  yet  enjoys  the interactive program development and
     debugging environment of LISP.
\end{itemize}

\section{Scope of the Manual}

This manual is intended to describe the syntax, semantics, and
implementation of PSL.  While  we  have  attempted  to  make  it
comprehensive,  it  is  not  intended for use as a primer.  Some
prior exposure to LISP will prove to be very helpful. 

\subsection{Typographic Conventions within the Manual}

A large portion of this manual is devoted  to  descriptions
of the functions that make up PSL.  For each function there is a
prototypical  header  line.    Each  argument  is  given a name,
followed by the type of argument expected.   If  more  than  one
type  is  allowed  then  the choices will be enclosed within the
brackets \{, and \}.  For example, the following  header  shows  a
function  named  vector-fetch  which accepts two arguments.  The
first V, is a vector, the second I, is an  integer.    The  term
expr  refers  to  a  type  of function.

\de{vector-fetch}{(vector-fetch V:vector I:integer): any}{expr}
{The value returned is the i'th element of the vector V.}\\
\\

\noindent
Within compiled code, some function calls are  replaced  by  a
sequence of instructions whose execution is equivalent to a call
on the function.  Functions of this type will have a note saying
open-compiled next to the function type.

\noindent
Some  functions  accept an arbitrary number of arguments.  The
header for these functions shows a single argument  enclosed  in
square  brackets,  indicating  that  zero or more occurrences of
that argument are allowed.\\

%\noindent
\de{and}{(and [U:form]): extra-boolean}{fexpr}
{And is a function which accepts zero or more arguments each of
which may be any form.}

\noindent
In  some  cases,  LISP  code  is   given   in   the   function
documentation  as the function's definition.  This code is given
to clarify the semantics of the function, it is not  a  copy  of
the actual PSL definition.



\section{Hints on Using the PSL System}

The following sub-sections collect a few  miscellaneous  notes
that  are further expanded on elsewhere.  They are provided here
simply to get you started.


\subsection{Loading Optional Modules}

Certain modules are not present in the "kernel" or  "bare-psl"
system,  but  can  be loaded as options.  Some of these optional
modules will be  automatically  loaded  when  first  referenced.
Other  modules may be explicitly loaded by the user, or included
by the installer when building the PSL  core  image.    Optional
modules can be loaded by executing the following.

\begin{verbatim}
(load modulename)
\end{verbatim}
When a module is loaded its name is added to a list referenced
by  the global variable options*.  The modules which make up the
BARE-PSL kernel are not included in this list.   An  application
of load will be aborted if the argument is found in the options*
list.    If  it  is necessary to load a module a second time use
reload, do not attempt to alter the value of options*.

\subsection{Error and Warning Messages}

Many functions  detect  and  signal  appropriate  errors ;  
in  many cases, an error message is
printed.  The error conditions are given as part of a function's
definition in the manual.  An error message is preceded by  five
stars (*); a warning message is preceded by three.  For example,
most  primitive  functions check the type of their arguments and
display an error message if an argument is incorrect.  The  type
mismatch  error  mentions  the  function  in which the error was
detected, gives the expected type, and prints the  actual  value
passed.

\begin{verbatim}
1 lisp> (car "STR")
***** An attempt was made to do CAR on `"STR"', which is not a pair
\end{verbatim}
The  switch {\bf usermode}\index{*usermode}\index{usermode}
is used to distinguish between functions
which comprise PSL and functions defined by the user.  When  the
value  of  this  switch  is t then the name of each user defined
function is flagged {\bf user}. When a function is about to be
defined, if *usermode is t, there is another function associated
with  the  name,  and  the  name  is  not  flagged lose then the
following prompt will be displayed.
\begin{verbatim}
Do you really want to redefine the system function `NAME'?
\end{verbatim}
\noindent
You are expected to respond with \verb+YES+, \verb+Y+, 
\verb+NO+, \verb+N+,  or \verb+B+. A response  of \verb+B+ will 
result  in a break loop (see Chapter 16). After quitting the
break loop you should respond  with  \verb+YES+, \verb+Y+, 
\verb+NO+, or \verb+N+ (see yesp in Chapter 15).  Unless you
consider yourself an  expert  it is dangerous to give an
affirmative response.  It is best to simply give your function a
different name.
\noindent
If the switch {\bf redefmsg}\index{redefmsg}\index{*redefmsg}
is set to t then whenever a function
is redefined the message

\begin{verbatim}
*** Function `NAME' has been redefined
\end{verbatim}
will be printed.

\noindent
If  an identifier is flagged LOSE\index{lose} then it cannot be
assigned a functional definition.

\begin{verbatim}
*** `NAME' has not been defined, because it is flagged LOSE
\end{verbatim}

\section{Switches and Globals}

Generally, the name of an identifier which represents a switch
will start with a "*".  For example, *redefmsg \index{*redefmsg}
and *usermode are
switches.  The value of a switch is modified with the  functions
on  and  off.  Note that in the example below, the prefix "*" is
omitted when using on and off.

\begin{verbatim}
1 lisp> (on redefmsg)
nil
2 lisp> *redefmsg
t
\end{verbatim}
The name of an identifier which represents a  global  ends  with
"*".  options* is an example of a global.

\section{Compilation Versus Interpretation}

  PSL  uses  both compiled and interpreted code.  If compiled, a
function usually executes faster and is smaller.  However, there
are some semantic differences of which the user should be aware.
For example, some recursive functions  are  made  non-recursive,
and   certain  functions  are  open-compiled.    A  call  to  an
open-compiled function is replaced, on compilation, by a  series
of  online  instructions  instead  of  just being a reference to
another function.  Functions compiled open may not  do  as  much
type checking.

\begin{verbatim}
1 lisp> (de list-first (p) (car p))
list-first
2 lisp> (list-first "STR")
***** An attempt was made to do CAR on `"STR', which is not a pair 
3 lisp> (compile '(list-first))
nil
4 lisp> (list-first "STR")
#<Unknown f7000002>
\end{verbatim}
To  avoid  this the user would have to add code which checks the
type of P before car is applied.
