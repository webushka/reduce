\documentstyle[11pt]{article}
% Pass group opnote title page and macros.
% Global definitions 
\def\opnumber#1{\number\yearonly -#1}
\def\opnote#1{\gdef\opheader
     {\xiipt Utah PASS Project OpNote \opnumber{#1}\hfill\today
      \markright{\today\hfill PASS Opnote \opnumber{#1}\hfill}}}
\def\mpr#1{\gdef\opheader
     {\xiipt Utah PASS Project Progress Report \opnumber{#1}\hfill\today
      \markright{\today\hfill PASS Progress \opnumber{#1}\hfill}}}
\def\title#1{\gdef\Xtitle{#1}}
\def\author#1{\gdef\Xauthor{#1}}
\gdef\opheader{}
\gdef\Xtitle{}
\gdef\Xauthor{}

\def\endtitlepage{\eject}

\def\maketitle{
\begin{titlepage}
%
\newcount\yearonly
\yearonly=\year
\advance\yearonly by-1900

% reset certain absurd settings for the title page.
\topmargin -12pt \headsep 0pt \headheight 0pt

\setcounter{page}{0}
\noindent\opheader
\vskip 1.25in
\begin{center}
{\xivpt\bf \Xtitle \par}
\vfil
{\xivpt\bf \Xauthor \par}
\vfil


% \begin{picture}(52,80)
% \put(0,0){\line(0,1){80}}        % left side
% \put(50,0){\line(0,1){65}}       % right side
% \put(0,0){\line(1,0){50}}        % bottom
% \put(0,80){\line(1,0){30}}       % top
% \put(30,65){\line(0,1){15}}      % left dent
% \put(30,65){\line(1,0){20}}       % bottom dent
% \put(10,45){\xpt(\( \textstyle\rm pass\))}  % pass  name
% \put(20,55){\(\textstyle \star\)}       % star
% \end{picture}

\setlength{\unitlength}{0.01in}
\begin{picture}(100,100)

\put (0,0){\line(1,2){60}}
\put (0,0){\line(1,0){120}}
\put (60,120){\line(1,-2){60}}
\put (18,0){\makebox(84,36){P A S S}}
\put (18,36){\line(1,0){84}}
\put (36,36){\makebox(48,36){( :- )}}
\put (36,72){\line(1,0){48}}

\put (56,72){\framebox(9,10){}}
\put (60,86){\line(2,1){14.2}}
\put (60,86){\line(-2,1){14.2}}
\put (60,99){\line(-2,-1){14.2}}
\put (60,99){\line(2,-1){14.2}}
\put (60,93){\circle*{4}}

\put (20,0){\oval(20,20)[b]}
\put (100,0){\oval(20,20)[b]}

\end{picture}
\vfil
Utah Portable Artificial Intelligence Support Systems Project\\
Computer Science Department\\
University of Utah\\
Salt Lake City, Utah 84112\\
\vfil
Copyright \copyright \number\year, Utah Pass Project\\
\end{center}
\bigskip
{\xpt Work supported in part by the Hewlett Packard
Corporation, the Schlumberger Palo Alto Laboratory for A.I. Research,
the National
Science foundation Under Grant Number MCS81-21750 and the
Defense Advanced Research Projects Agency under contract number
DAAK11-84-K-0017.}

\end{titlepage}
% \protect\topmargin 0pt \headsep 35pt \headheight 12pt 
\setcounter{footnote}{0}}

\headheight 14pt
\textwidth 5.75in
\oddsidemargin 0.50in
\pagestyle{myheadings}
\title{UNIX PSL 3.4/PCLS 2.1 Release Notes}

\author{Harold Carr, Leigh Stoller, \\
        Stan Shebs, Robert Kessler}

\opnote{07}

\begin{document}

\maketitle
\nonstopmode

\section{Introduction}

Portable Standard Lisp (PSL) 3.4 runs on Vax BSD Unix, Apollo DOMAIN
SR 9.5 and above, HP Series 9000 HP-UX, HP Integrated Personal
Computer, SUN BSD (except SUN 4), Iris BSD, Gould BSD, and MacIntosh
Plus.

Portable Common Lisp Subset (PCLS) 2.1 is a subset of Common Lisp
built on top of the PSL 3.4 kernel on the machines listed above.

PSL is a delivery vehicle for the REDUCE Computer Algebra System, the
ALPHA-1 Spline-based Geometric Modeling System, and an implementation
base for PCLS.  Users are encouraged to write applications in
PCLS, since PSL as an independent dialect of Lisp is no longer
supported.  Only enough of PSL to build the above applications is
supported (see the addendum to the PSL Manual).

The attached TAR format tape contains the files needed to use and
maintain the Unix PSL/PCLS systems.  Although PSL/PCLS is available for
many machines, the attached tape only contains files and directories
corresponding to your particular system, including sources and scripts
to permit a complete rebuild of all parts of the language.  The PCLS
portion of the tape does contain some files for other machines, but
these are few in number and should not affect anything.

Most people running PSL/PCLS under Unix will not be interested in all of
the files, and probably will not want to have them all on line.  Only
the executable programs, {\tt psl-names}, {\tt xxx-names}, and the
files residing on the
sub-directory {\tt dist/lap} (described below) must be on line to use
PSL.  Any other files may be deleted after the system is installed.
The full distribution tape should be retained if you plan to delete
files after installation.  PCLS Level 3 does not need any files
whatsoever, while PCLS Level 2 requires the directory {\tt
pclsdist/lap} once it has been built.

\section{DISCLAIMER}

Please be aware that this is a research system, not a program product,
and some of the files and documentation are not quite complete; we may
also have forgotten some files, or sent incorrect versions.  We are
releasing this version to you at this time to enhance our
collaborative research.

For these reasons please:

\begin{itemize}

\item Make a note of ANY problems, concerns, suggestions you have, and
send this information to us to aid in improving the system and this
distribution mechanism.

\item Please do not REDISTRIBUTE any of these files, listings or machine
readable form to anyone.

\end{itemize}

\section{CONTENTS OF THE TAPE}

Attached to this note is a listing of the files on the tape, produced
as the tape was built. The tape was written by changing the working
directory to the root of the PSL directory {\tt \$proot} ({\tt
/v/misc/psl} at Utah), using {\tt cd}.  The subdirectories {\tt dist}
and {\tt pclsdist} are written using {\tt tar} to write the tape as a
set of files and directories relative to the {\tt \$proot} node.  Note
that {\tt \$proot} is a {\em C shell} variable, established by {\tt setenv}
in the {\tt psl-names} file, described below.

The tape contains the following files and directories
(directories are indicated by {\tt */}). Some of the directories
might be empty, but are included for compatibility with the full
system. The c-shell variables (\$ variables) referencing these
directories are indicated by {\tt [\$*]}.  Some of the directories
mention {\tt MACHINE}, which refers to the specific machine.  For
example on a Vax, it is {\tt vax}, Sun it is {\tt sun}, etc.

\vskip 12pt
\noindent
EXECUTABLES: {\tt \$proot/bin/} (the bin directory is {\tt \$psys})

\begin{description}

\item[{\tt bare-psl}:] The smallest executable used mostly for
building other executables. 

\item[{\tt psl}:] {\tt bare-psl} with several modules loaded (see your
installer or look at the {\tt psl} variable {\tt options*} to see which
modules are loaded.

\item[{\tt pslcomp}:] {\tt psl} with the compiler loaded.

\item[{\tt pclslev2}:]
Level 2 PCLS, contains a PSL-sized (i.e. small) subset of Common Lisp.
Selected PCLS level 3 modules may be loaded into this executable.
{\tt pclslev2} is not always present but may be built if necessary.

\item[{\tt pcls}:]
Level 3 PCLS, contains a large subset of Common Lisp.  This is the
recommended one for users.

\item[{\tt pclscomp}:]
Invokes Level 3 PCLS to compile a single file - useful with
{\tt make} and other Unix tools.

\end{description}

\noindent
SHELL SCRIPTS: {\tt \$proot/dist/} 

\begin{description}

\item[{\tt oload}:]
A C shell script to to permit {\tt .o} files to be dynamically loaded.  (This
does not exist on all systems.)

\item[{\tt psl-names}:]
A file to be {\tt source}ed to define environment variables (such as
{\tt \$pu} for {\tt util/}, {\tt \$pl} for {\tt lap/}, etc.) to aid in
accessing the various PSL sub-directories.

\item[{\tt xxx-names}:]
The machine-dependent portion of {\tt psl-names}, {\tt source}ed by
{\tt psl-names}.

\end{description}

\noindent
DIRECTORIES: {\tt \$proot/dist/} 

\begin{description}

\item[{\tt lap/}:]
PSL binary ({\tt *.b}) files.  Contains binary files needed to
rebuild the executables plus binary files for optional loadable
modules.  PSL binary files are sometimes referred to as FASL (fast
loading) files.  [{\tt \$pl}]

\item[{\tt kernel/}:]
Machine-independent kernel sources. [{\tt \$pk}]

\item[{\tt kernel/MACHINE/}:]
Machine-specific kernel sources and object files. [{\tt \$pxk}]

\item[{\tt nonkernel/}:]
Files which are part of a standard {\tt bare-psl}, but not
compiled in the kernel build. [{\tt \$pnk}]

\item[{\tt nonkernel/MACHINE/}:]
Machine-specific files which are part of a standard {\tt bare-psl}, but
not compiled in the kernel build. [{\tt \$pxnk}]

\item[{\tt comp/}:]
Machine-independent compiler sources. [{\tt \$pc}]

\item[{\tt comp/MACHINE/}:]
Machine-specific compiler sources. [{\tt \$pxc}]

\item[{\tt util/}:]
Sources for machine-independent utilities. [{\tt \$pu}]

\item[{\tt util/MACHINE/}:]
Sources for machine-specific utilities. [{\tt \$pxu}]

\item[{\tt distrib/}:]
Scripts to help build and maintain the machine-independent parts of
PSL. [{\tt \$pdist}]

\item[{\tt distrib/MACHINE/}:]
Scripts to help build and maintain the machine-specific parts of PSL.
[{\tt \$pdist}]

\item[{\tt lpt/}:]
The PSL manual in printable form (has overprinting and underlining).

\item[{\tt doc/}:]
Versions of this document and the addendum to the PSL manual.  We keep
the source texts and printable versions in postscript ({\tt .ps}) and
laserjet format ({\tt .jep}).

\end{description}

The {\tt pclsdist} subdirectory has a similar organization.
\bigskip

\noindent
SHELL SCRIPTS: {\tt \$proot/pclsdist/} 

\begin{description}

\item[{\tt pcls-names}:]
Sets up csh variable names that PCLS will need.  This file
{\it must} be sourced {\it before} building PCLS.  Also, {\tt
\$psl/psl-names} must be sourced before sourcing {\tt pcls-names}
since {\tt pcls-names} picks up the root of the tree from {\tt psl-names}.

\end{description}

\noindent
MISCELLANEOUS: {\tt \$proot/pclsdist/} 

\begin{description}

\item[{\tt config.cl}:]
This file contains information relevant to your site and
should be edited appropriately. (see below)

\item[{\tt Makefile}:]
Has actions for creating and manipulating PCLS files.

\item[{\tt makepcls2.cl}:]
Lisp code that constructs Level 2 PCLS.

\item[{\tt makepcls3.cl}:]
Lisp code that constructs Level 3 PCLS.

\end{description}

\noindent
DIRECTORIES: {\tt \$proot/pclsdist/} 

\begin{description}

\item[{\tt kernel/}:]
Source code to reconfigure PSL into a ``bare pcls'' (also known
as ``cucumber''). [{\tt \$cuke}]

\item[{\tt level2/}:]
Source code to build a reasonably-sized PCLS.  [{\tt \$cl2}]

\item[{\tt level3/}:]
Source code for full PCLS.  [{\tt \$cl3}]

\item[{\tt comp/}:]
Source code for the frontend optimizer.  This translates Common Lisp
code into PSL code. [{\tt \$clc}]

\item[{\tt util/}:]
Useful things not part of the Common Lisp standard.  [{\tt \$clu}]

\item[{\tt lap/}:]
Compiled code, for both utilities and Level 3 modules.  [{\tt \$cll}]

\item[{\tt doc/}:]
Human-readable information about PCLS.

\item[{\tt test/}:]
Test suite for the PCLS implementation.

\item[{\tt bmark/}:]
Gabriel benchmarks sources and some results.

\end{description}

\section{A SUGGESTION TO SYSTEM MANAGERS}

PSL and PCLS have a large heap (of which only half may be used at any
given time if the copying garbage collector is being used, rather than
the compacting collector).  The default size of a running PSL core
image is roughly 4.5 megabytes.  Because of this, it is recommended
that your system be configured with at least 32 megabytes of swap
space.  If this is too great a burden on your system, see the section
on rebuilding the kernel for instructions on creating a version with a
smaller heap.

If a large number of people wish to run PSL or PCLS simultaneously, either
the swap space will have to be increased, or a smaller PSL should be built.

\section{READING THE TAPE}

\begin{enumerate}

\item The first step is to set up the PSL directory.

\item Then the contents of the tape is read into subtrees of the PSL
directory structure.

\item Finally certain files are customized for local directory conventions
and the system installed by running a script to set the root of the
PSL tree.

\end{enumerate}

\subsection{Set up the PSL/PCLS directory structure}

Unless absolutely essential, we recommend that you use exactly the same
directory structure as we do (relative to the root of the PSL tree),
even though the system is set up to be installed according to local
conventions.

We connect the entire PSL directory structure to the node {\tt
/v/misc/psl}, which we refer to as {\tt \$proot}.  The tape will be
read into the subdirectories {\tt \$proot/bin} {\tt \$proot/dist}
and {\tt \$proot/pclsdist}.  We refer to {\tt \$proot/dist} as {\tt
\$psl} below.  By only referring to {\tt
\$proot} or using
relative path names, local versions of PSL can be built on (say)
{\tt \$proot/local-version}, and older distributions retained as
{\tt \$proot/old-dist}, etc.

If possible, it is a good idea to have the {\tt \$proot} directory defined
as a ``login'', so that {\tt \verb|~|psl} may be used to reference {\tt
\$proot}, before any
c-shell (\$) variables have been defined. This will make it easier for users to
place appropriate references in their {\tt .login} and {\tt .cshrc} files.
This also improves control over source code, since protections can be
set so that only maintainers with the password can modify things.

\subsection{Read the tape}

Change your working directory to the directory which will be the root
of this PSL distribution ({\tt /v/misc/psl} at Utah), called {\tt
\$proot}.  Use the command
\begin{verbatim}
        cd ~psl
\end{verbatim}
or equivalent.

Mount the tape and give
the command:
\begin{verbatim}
        tar x
\end{verbatim}
This will retrieve all the files on the tape, into appropriate
sub-directories of {\tt \$proot}

If you do not wish to retrieve all the files on the tape,
but only a subset (listed below), then do:
\begin{verbatim}
        tar x ./dist/distrib/read-tape.csh ./dist/xxx-names
\end{verbatim}
then run the script
\begin{verbatim}
        ./dist/distrib/read-tape.csh
\end{verbatim}

It is possible to pass two arguments to the {\tt read-tape.csh} script.  The
first optional argument specifies what portions of the system you are
going to read.  It defaults to the entire PSL/PCLS tree.  The second
optional argument is the name of the device to read from (it defaults
to {\tt /dev/rmt8}).

Legal first arguments are:

\begin{description}
\item[psl]
Read the entire PSL tree (source plus binaries).
Disk space required\footnote{Disk space requirements are for a Vax.  Other
systems may require more or less space.}: 18 megabytes.

\item[psl+pcls-executables+resize]
Read the PSL and PCLS executables plus enough of the system to change
the sizes of internal structures (i.e., heap, bps, symbol table,
etc.).  Disk space required: 23 megabytes.

\item[psl-executables+resize]
Read the PSL executables plus enough of the system to change
the sizes of internal structures (i.e., heap, bps, symbol table,
etc.).  Disk space required: 14 megabytes.

\item[psl+pcls-executables]
Read the PSL and PCLS executables only.  Disk space required: 8.5 megabytes.

\item[pcls-executables]
Read the PCLS executables only.  Disk space required: 2.5 megabytes.

\item[psl-executables]
Read the PSL executables only.  Disk space required: 6 megabytes.

\item[all]
Read the entire PSL and PCLS trees (source plus binaries).  Disk space
required: 27 megabytes.

\end{description}

\section{INSTALLING PSL}

\subsection{Update the Root Location}

After reading the tape, you need to customize the files to reflect the
root location of PSL.  This can be accomplished with
{\tt newroot.csh}.  It takes one argument which is the fully rooted
path of the PSL root directory.  {\tt newroot.csh} will replace the root
in {\tt psl-names} and all of the makefiles.  For example,

\begin{verbatim}
        ./dist/distrib/newroot.csh /v/misc/psl
\end{verbatim}

\subsection{Adding the PSL Environment}

Your {\tt .cshrc} file should be edited to {\tt source} the new
{\tt psl-names}, and set up your search path.  If you have previously
installed a version of PSL, you should at this time make sure that
your {\tt .cshrc} refers to this NEW {\tt psl-names}.

To do this, insert the lines 
\begin{verbatim}
        source $~psl/dist/psl-names          # or equivalent
        set path=(. $psys /usr/bin /bin ...) # plus any others
\end{verbatim}
into your {\tt .cshrc} file at an appropriate point.

Now do
\begin{verbatim}
        source ~/.cshrc
\end{verbatim}
to set all paths to that of the new PSL.

\subsection{Make links to executables}

Do:

\begin{verbatim}
        make install
\end{verbatim}

to make symbolic links from your standard system directory to the
executables in {\tt \$psys}.  The default is {\tt /usr/site/bin/}.
This makes it unnecessary for users of PSL to put {\tt \$psys} in the
search path variables of their {\tt .login} files before they can use
PSL.  (For users on System V Unix or Apollos, this step does not work.
Users must put {\tt \$psys} on their search path.)  This default is
defined in {\tt \$psl/Makefile} as the {\tt DESTDIR} variable.  PSL
users must still {\tt source psl-names} to use PSL if they intend to
load modules from the {\tt lap/} directory or use {\tt oload} to load
{\tt .o files}.

\subsection{Delete unneeded directories}

Only the {\tt lap/} subdirectory and the {\tt \$psl/psl-names} and
{\tt \$xxx-names} files are  necessary for running PSL
programs.  If you do not wish to keep the source files on-line, you
may delete any other files after successfully testing PSL, and
rebuilding any modules or binaries that you need to change.  But DO
retain the distribution tape!  Many users of PSL find that on-line
sources are useful tutorial and reference material, and we once again
recommend that you retain the file structure as distributed.

\subsection{Announce the system}

You should only announce the system if there are users who are going
to use PSL directly, rather than PCLS.  We suggest only announcing
PCLS as discussed below.  However, if you do need to announce PSL here
is how to do it:

Send out a message to all those interested in using PSL.  The file
{\tt \$psl/doc/bboard.msg} is a suggested start.  In particular, make
sure that you inform people of the exact path needed to place a
{\tt source .../psl-names} in their {\tt .login} file, if they desire, since
{\tt psl-names} facilitates reference to the PSL source directories.

Edit as you see fit, but please REMIND people not to re-distribute the PSL
system and sources. Note the [.....] blocks in which you are to insert
the name of some person responsible to answer questions.

You may also want to set the directory protection and limit access only to
those that you feel should have access at this time.

\section{INSTALLING PCLS}

Once PSL is built and running, you may go on to the construction of PCLS.
Many of the details are parallel to those for PSL. Remember, these
steps are only applicable if you have read in the entire PCLS tree
from the distribution tape. If you have only the binaries in {\tt \$psys},
then you can skip this section entirely.

To build a PCLS system follow these steps:

\begin{itemize}
\item Source {\tt pcls-names}.

\item Edit the configuration file {\tt \$cl/config.cl}.

\item Create PCLS executables.

\item Link the executables to places where users can find them.

\item Clean up unwanted directories.

\item Announce the system.

\end{itemize}

\subsection{Source pcls-names}

Edit your {\tt .cshrc} file to {\tt source pcls-names}.  This step is
essential during construction.  It is not necessary for a user to {\tt
source pcls-names} to use PCLS, unless they plan to use {\tt
pclslev2}.  In that case, users will also {\tt source pcls-names} to
define these names in their own {\tt .cshrc} files, since they are
used during the process of loading modules.

\subsection{Edit the configuration file}

The file {\tt \$cl/config.cl} contains site-specific information, and may
be edited to reflect facts about your own site.  As distributed, the
site-specific procedures are correct only for the University of Utah CS
Dept.  The file contains instructions about how to modify it
correctly.  We recommend you be at least slightly familiar with Common
Lisp before trying; fortunately, these procedures are not critical, and
PCLS can actually be used without modifying {\tt config.cl}.

\subsection{Make PCLS Executables}

Connect to the directory {\tt \$cl}, and give the command {\tt make
executables}, which will build executables {\tt pclslev2} and {\tt
pcls}.

\subsection{Make Links to Executables}

Edit and run the script {\tt \$cl/install-pcls} to make symbolic links
from a standard system directory to the executables in {\tt \$cl}. (For
users on System V Unix, this step does not work.  Users must put {\tt
\$psys} on their search path.)  The default is {\tt /usr/site/bin},
although for many sites either {\tt /usr/local/bin} or {\tt
/usr/local} will be appropriate.  This step is optional, but if done,
users will not need to take any special measures to access PCLS.

\subsection{Delete Unneeded Files}

No files are necessary to run Level 3 PCLS - compiler and environment
are built in.  The directory {\tt \$cl/lap} contains modules that
Level 2 PCLS may load.  You may delete any other files as desired -
{\tt \$cl/Makefile} provides several kinds of deletion for common
situations.  {\tt make clean} removes log files and some intermediate
executables, while {\tt make stripped} removes everything but the
executables and loadable modules.

\subsection{Announce PCLS}

Send out a message to all those interested in using PCLS.  The file
{\tt \$cl/doc/bboard.msg} is a suggested start.  Edit as you see fit,
but be sure to insert a name into the [...] blocks, preferably of
someone who will answer questions.

\subsection{Modifying PCLS}

PCLS can be totally rebuilt by switching to {\tt \$cl}, saying {\tt
make flush} (which removes all but the sources), then {\tt make all}.
Remember to source {\tt pcls-names} before trying to rebuild any part
of the system.
Rebuilding all of PCLS takes somewhat over an hour on a Vax 750, so it's
generally a good idea to put it in the background.  For a variety of
reasons, the PCLS makefiles and actions are not linked across
directories as much as one might like.  When rebuilding a part of
PCLS, you may have to do some manual dependency tracking.  The first
action in {\tt \$cl/Makefile} lists the major dependencies - compiler
stuff depends on {\tt bare-pcls}, level 2 depends on compiler, etc.
You will have to go to the appropriate directory and and do a make
there, then switch to the next appropriate directory, do another make,
etc.  This has some advantages; if {\tt bare-pcls} gets a small hack,
it may not be necessary to recompile levels 2 and 3, and you can go
directly to reconstructing {\tt pclslev2} and {\tt pcls}.  There may
be a problem in that {\tt make} may not believe that certain
directories require recompiling, in those cases, the makefile should
have a {\tt force} action.  {\tt touch}ing files also works.

\subsection{Resizing PCLS}

If you want to change the sizes of internal structures in PCLS (such
as the heap and bps) you will need to first read the section on
resizing PSL.  Once PSL has been resized then connect to {\tt \$cl},
source {\tt pcls-names}, then do {\tt make resize}.

\section{REBUILDING PSL LOADABLE MODULES}

All of the utilities, are kept as binary FASL files (with extensions
{\tt .b}) on the {\tt lap/} subdirectory.  After modifying a utility source
file just connect to {\tt \$pu} and do {\tt make}.  This will
automatically detect which files need to be recompiled.  Once
recompiled it will place the binary version(s) on {\tt \$pl}, the
directory from which the executables obtain loadable modules.

\subsection{Using Pslcomp}

{\tt pslcomp} can be used to compile a file, regardless of whether it is
listed in one of the makefiles.  Connect to the directory where the
sources for file {\tt XXX} reside.  Do
\begin{verbatim}
       pslcomp  XXX
\end{verbatim}
This will compile the file {\tt XXX}, placing the binary (.b) file is
placed in the directory to which you are connected.  {\tt pslcomp}
knows about {\tt .sl} and {\tt .l} extensions.

\section{REBUILDING THE SYSTEM}

It is not necessary to rebuild the system after reading the tape.  The
executables should run on your system.  PSL determines the location of
its tree (necessary for finding the {\tt lap/} directory, the {\tt
oload} script, etc.) by checking the environment variables.  That is
why a user must source {\tt psl-names} to use PSL.  If you do need to
rebuild the system then this section explains the necessary steps.

To rebuild the system, you must have read the entire contents of the
tape.  If you have then change your
working directory to {\tt \$psl}, and do:
\begin{verbatim}
        make
\end{verbatim}

This will connect to several subdirectories of {\tt \$psl} and run
make in those directories.  These ``recursive'' makes will:

\begin{itemize}

\item Recompile C system interface routines used in the kernel.

\item Recompile any out-of-date PSL system source files, putting their
binary versions on {\tt \$pl} (none of the PSL source files should be
out-of-date after first reading a tape).

\item Rebuild {\tt bare-psl}, {\tt psl}, and {\tt pslcomp}, placing
them on {\tt \$psys}.

\end{itemize}

\section{REBUILDING THE PSL KERNEL}

A running {\tt xxx-cross} and {\tt pslcomp} are required to rebuild
the kernel, since the entire system is written in PSL itself.  The
kernel modules are compiled to assembly code ({\tt as}) using {\tt
xxx-cross}.  They are then linked using the system loader {\tt ld}.
After that, the rest of the system is compiled to {\tt .b} files,
which are loaded into the kernel to produce the executables.  You can
modify any file in the PSL tree.  Then connect to {\tt \$psl} and do
{\tt make}.  This will recompile any modified files and rebuild the
executables.

\subsection{Cleanup}

{\tt cd \$psl; make cleanup} will remove log files, editor backup files, and
backup copies of the executables.

\subsection{Building PSL with different static sizes}

PSL uses static sizes for Binary Program Space (BPS) and the Heap.
The BPS holds compiled code (both {\tt oload}ed C code and Lisp code),
while the heap contains dynamic Lisp data.

To determine how much bps space is left call {\tt (free-bps)}.  The
result is in words, so multiply by 4 to get total bytes left.  The
amount of available heap space can be determined by first running the
garbage collector to compact memory and then calling {\tt gtheap}
with a {\tt nil} argument: {\tt (gtheap nil)}.  This also returns the
number of words, so multiply by 4 to get bytes.  The following shows
an example of this:

\begin{verbatim}
PSL 3.4,  8-Sep-87
1 lisp> (free-bps)
216013
2 lisp> (reclaim)
*** Garbage collection starting
*** GC 5: time 1564 ms, 490 recovered, 396558 free
NIL
3 lisp> (gtheap nil)
396538
4 lisp> 
\end{verbatim}

In order to change the amount of BPS space, do the following:

\begin{enumerate}

\item Connect to the {\tt \$pxk} directory.

\item Edit {\tt Makefile} and change the value of the variable
{\tt BPSSIZE}.  This is the number of bytes. {\it It must be a
multiple of four.}

\item Connect to the {\tt \$psl} directory.

\item Run ``{\tt make resize}''.

\end{enumerate}

\noindent For example:

\begin{verbatim}
% cd $pxk
% vi Makefile ... make changes to this file
% cd $psl
% make resize
\end{verbatim}

In order to change the amount of Heap space, you must do the
following:

\begin{enumerate}

\item Connect to the {\tt \$pdist} directory.

\item Edit the {\tt make-bare-psl}, {\tt make-psl} and {\tt make-pslcomp}
scripts to change the value of the {\tt -td} argument in the
invocation of the call to {\tt bpsl}.  The value that is given is the
total space to be allocated in bytes, including bps, heap, and other
miscellaneous areas.  So, the value that you should provide is the sum
of the BPS size, two times the amount of heap that you want, and some
extra space.  The reason that you need to use two times the heap size
amount is that PSL uses a copying collector, which requires double the
heap space.  Thus, suppose that we want to allocate a heap of
1,000,000 bytes, and a BPS space of 500,000.  Then you should allocate
2,500,000 to the {\tt -td} argument to {\tt bpsl} in the scripts
mentioned above.

\item Connect to the {\tt \$psl} directory.

\item Run {\tt make resize}.

\end{enumerate}

\noindent For example:

\begin{verbatim}
% cd $pdist
% vi make-bare-psl ... make changes here
% vi make-psl      ... make changes here
  ...
% cd $psl
% make resize
\end{verbatim}

\subsection{System modification for European installations}

The {\tt (Time)} procedure gives results in milliseconds.  If you have 50hz
AC current this will return 1/850ths instead, unless a modification is
made to the file {\tt \$pxnk/timc.sl}.  A constant 17 must be changed
to a 20 as noted in a comment in that file.  This will not affect any
other part of the system, so if you are not interested in timings you
need not make this change.

\section{Differences between Berkeley Unix Versions}

In theory, there are no visible differences between 4.2 and 4.3, nor
between those and Ultrix.  We would like to hear about any situations
for which this is not true.

\section{FUTURE UPDATES}

It is currently envisioned that future updates will be complete
releases.  It is therefore suggested that you

\begin{itemize}

\item Retain this distribution tape in case you may have to compare files.

\item Do not make any changes on these distributed directories. If you must
make your own bug fixes, it is suggested that you put the changed
files on some other directories, such as {\tt \$proot/local}, etc.  We
will not send any files on these directories, so that you can compare
your local versions with our newer ones.

\end{itemize}

\section{BUG REPORTS}

If you wish, you can send bug reports, suggestions, fixes, etc. to us.
If you have Arpanet access, you can send email to {\tt
psl-bugs@orion.utah.edu}.  For those with USENET access, use {\tt
utah-cs!utah-orion!psl-bugs}.

\end{document}
