\chapter*{Prettyprinting}

\section{Introduction}

  To access the prettyprinter described in this chapter you must
load the module PP.

  A  prettyprinter  takes  as  input  a stream of characters and
prints them with aethetically appropriate indentations and  line
breaks.  For example

\begin{verbatim}
(if (vectorp data) (vector-size data) (if (stringp data)
    (string-length data) (length data)))
\end{verbatim}
\begin{verbatim}
(if (vectorp data)
  (vector-size data)
  (if (stringp data)
    (string-length data)
    (length data)))
\end{verbatim}
\section{Prettyprinting Files and Data}

  The  width  of  an output line is defined by default to be 75.
You can modify this by passing the desired width to the function
line-width.


\de{pp-file}{(pp-file SOURCE:string DESTINATION:string):\\ 
undefined}{expr}
{    This function is used to prettyprint the contents of one
    file  into  another.  There must be two arguments, the first
    is a string representing the source file name, the second is
    a string representing the destination file name.  The  first
    argument  to  top  level applications of de, df, dm, ds, dn,
    defmethod, defflavor, and defmacro is printed once the  form
    has been written to the file.
}

\de{pp-object}{(pp-object E:form): undefined}{fexpr}
{    This  function  will  prettyprint  the  PSL  data  object E.
    Pp-object will also accept  two  optional  arguments.    The
    output may be directed to a file by including a string which
    represents  the file name.  Indentation is specified with an
    integer.  Both of these optional arguments must be  preceded
    by  special  identifiers, :indent for indentation, :sink for
    redirection of output.
}
\section{Formats}

  Suppose  we  define  the  following  data  structure  (in  the
discussion  which  follows  it will be assumed that the selector
functions form, root, and suffix have been defined).

\begin{verbatim}
(NAMED-FORM <form> <root> <suffix>)
\end{verbatim}
Furthermore, we would like

\begin{verbatim}
(named-form (+ left right) argument 1)
\end{verbatim}
to be printed in the form:
\begin{verbatim}
argument1: (+ left right)
\end{verbatim}
The format definition which follows could  be  used  to  specify
that  named-forms  should  be  printed  out  in  the  above way.
Deformat  will  convert  everything  which  follows  its  second
argument into a formatting function.  In the example below, this
formatting  function will be used to format lists which have the
identifer named-form as their first element.  When such  a  list
is  passed  to  it,  the  function  creates a sequence of format
instructions, specifying what should be printed corresponding to
the list.

\begin{verbatim}
(deformat (named-form) (item)
  (template '({ * * ":" [i 1 2] * }) (root item) 
  (suffix item) (form item)))
\end{verbatim}
The  function  template  creates  a   sequence   of   formatting
instructions  for its arguments based on directions specified by
the template (the first argument).  The template in this example
can be understood as follows: The  {  and  }  specify  that  the
components  between  them  should be treated as a single logical
unit when they are printed and that an outer pair of parentheses
should  be  printed.    The  three  *'s  show  where  the  three
components  of  the  data  structure should be printed.  The ":"
specifies that a colon should be printed after the  suffix.  The
vector  [I  1  2]  corresponds  to a conditional line break.  If
there is sufficient room for the form then  a  single  space  is
printed (this is specified by the second element of the vector).
Otherwise  a line break followed by an indentation of two spaces
is inserted.\\


  A template may consist of the following symbols:
%\TABELLE***
%\begin{tabular}{lp{9.0cm}}
\begin{itemize}

\item The prettyprinter will break onto different  lines  as 
      few logical blocks as possible. The left angle bracket
						($<$),denotes the beginning of a block, the right angle
						bracket ($>$), denotes the end of a block. It is assumed 
						that these brackets balance.

\item Left  and right braces ({ and }) are also to define 
						logical blocks which should be printed within a pair 
						of parentheses.

\item There are two different types of blanks, the terms
      consistent and inconsistent  are  used  to  describe them.
      Note  that  blanks  may only appear within a logical block.
     	If each blank within a block is consistent  and  the  block
     	will  not  fit  on  the  current  line  of output then each
     	sub-block of the block will be placed on  a  new  line,  if
     	each  blank  is  inconsistent then new lines will be forced
     	only when it is necessary.  For example, the  first  result
     	will  occur  if  the blanks are inconsistent, the second if
     	they are consistent.
%\end{tabular}
\end{itemize}

\begin{verbatim}
     												(one two three
      											four)

     												(one
      											two
      											three
      											four) 
\end{verbatim}

    In general if a block will not fit  on  the  current
     line  of output then each consistent blank corresponds to a
     new line, inconsistent blanks will print as  spaces  unless
     it  is necessary to force a new line. There is a length and
     an offset associated with each blank, if a blank is treated
     as a space then the length represents the number of  spaces
     which  will be printed. If a blank is treated as a break to
     a new line then the offset is used for indentation  on  the
     new  line.  Indentation  is from the horizontal position of
     the first character of the innermost logical  block.    For
     example,

\begin{verbatim}
     (template '({ * { (* [c 1 1]) } })
               '(function (arg-one arg-two)))
\end{verbatim}

     will result in

\begin{verbatim}
     (function (arg-one
                arg-two))
\end{verbatim}

     assuming  that  it is necessary to force a line break.  The
     consistent blank within the call the template specifies  an
     offset  of  one.    The  first  character  of the innermost
     logical block is the  second  left  parenthesis,  from  its
     position  the  printer indented a single space.  A blank is
     represented by a vector, the first entry should  be  either
     an  I  or  C,  these  characters  designate the type of the
     blank, I for inconsistent and C for consistent. The  second
     entry  is  a  positive integer which represents length, and
     the third is a positive integer  representing  indentation.
     The  algorithm expects to find a blank after each occurence
     of a right angle bracket, if the formats  which  have  been
     provided do not follow this convention then an inconsistent
     blank  with  length and offset set to zero will be inserted
     after the right angle bracket.
\begin{itemize}
\item An asterisk (*) is used to make a recursive call on
     dispatch. Dispatch is responsible for applying the
     appropriate formatting function to an expression

\item Strings, the text of the string is included in the output.

\item A sublist of format symbols may also appear within the list
     of  format  symbols.   This  list  may  contain  blanks  or
     asterisks  and  is  used repetitively until a corresponding
     list of expressions is exhausted.
%\end TABELLE***
\end{itemize}

\section{Dispatch}

\begin{verbatim}
  (de dispatch (item depth)
    (cond ((instance item)
           (apply (find-instance-template item)
                  (list item depth)))
          ((atom item)
           (apply (find-atom-template item)
                  (list item depth)))
          ((> depth *maxdepth*) (add-tokens "#"))
          ((has-template item)
           (apply (get-template item)
                  (list item (add1 depth))))
          (t (default-template item (add1 depth)))))))
\end{verbatim}
  It is useful to know how this  function  associates  a  format
function with an expression. If the expression is an instance of
a  flavor  then  the  dispatch function will search for a format
function associated with the name  of  the  flavor.   For  other
atomic  expressions,  the  type  is  used  to  locate  a  format
function.  Currently there are  format  functions  for  vectors,
strings,   numbers,  code-pointers  and  identifiers.    If  the
expression is a list and the first element of this  list  is  an
identifier  then  an  attempt  is made to find a format function
associated with the identifier. Before resorting to a default  a
check  is  made  for  function  application for which there is a
separate default format function.

  Function application is recognized when the first  element  of
the  list  is a lambda expression, an application of getd to the
first element returns a non-nil value, or the first  element  is
an id which is a member of a global list named *function\_names*.
This  list of function names is useful for prettyprinting a file
since a file will normally contain a number of  applications  of
functions   defined   within   the   file  itself.    The  fexpr
add-functions accepts any number of function names and adds then
to *function\_names*.


  With respect to lists, any  sub-expression  below  a  maximum
depth  is printed as \verb$`#'$.  Once the maximum length of a list
has been reached the display  of  the  remaining  elements  will 
be `...'.    The maximum depth (initially 10) and length (initially
25) can be modified with the functions new-depth and new-length.

\section{Specifying Formats}


\de{deformat}{(deformat NAME:pair PARAM:id-list [BODY:forms]):
nil}{fexpr} 
{    
Deformat  is  used  to  define  format  functions 
for   PSL
    expressions.    The   car  of  NAME  should  be  a  list  of
    identifiers (it may also be a single  identifier),  the  cdr
    should   be   nil  for  atomic  expressions  or  an  integer
    representing a minimal length for lists.  PARAM  is  a  list
    containing  a  single  id,  when the format function is used
    PARAM will be  bound  to  the  expression  being  formatted.
    Formats are defined by calls on template (described below).
}

\de{template}{(template FORMAT:list [COMPONENT:form]): any}{macro}
{    FORMAT  is a list of format symbols (described above).  Each
    COMPONENT should correspond to a call on  dispatch  (*),  in
    FORMAT.   If a sub-list of format symbols in included within
    FORMAT  then  there  must  be  a   corresponding   list   of
    components.
}

  This   first   example   illustrates   a   format  for  quoted
expressions.  For example, 'ANY  instead  of  (QUOTE  ANY).    A
minimal  length  of  two is given, since the list (QUOTE) should
not be printed as '.  The argument to quote is  formatted  by  a
separate call on dispatch.

\begin{verbatim}
(deformat (quote . 2) (item)
  (template '("'" *) (second item)))
\end{verbatim}
  The  elements  of  a  vector  should be enclosed within square
brackets.  Since the length of  vectors  will  vary  a  list  of
format  symbols  is  supplied  to  be used repetitively for each
element.  This list will be matched against a corresponding list
of components, (vector2list item).  Each element is formatted by
a  recursive  call  on  dispatch.   Elements  are  separated  by
inconsistent blanks, therefore if the entire vector does not fit
on a single line the ouput will be something like
\begin{verbatim}
[one two three
 four]

(deformat (vector) (item)
  (template '(< "[" (* [i 1 1]) "]" >) (vector2list item)))
\end{verbatim}
Given the expression (setq n-one v-one n-two v-two)

\begin{verbatim}
(setq n-one v-one
      n-two v-two)
\end{verbatim}
is preferred over

\begin{verbatim}
(setq n-one v-one n-two
 v-two)
\end{verbatim}
  The following can be used to obtain the preferred output.

\begin{verbatim}
(deformat ((set setq) . 3) (item)
  (template '({ * " " < (* [i 1 1] * [c 1 0]) > })
            (first item) (rest item)))
\end{verbatim}
The first form of ouput is obtained by using inconsistent blanks
after  the  names  (n-one and n-two) and consistent blanks after
the values (v-one and v-two).  When this expression will not fit
on a single line the consistent blanks will correspond  to  line
breaks and unless a value is too large to fit on an output line,
the  inconsistent  blanks  will  correspond to spaces.  The zero
offset associated with the consistent blanks corresponds to  the
block of names and values.  This is why the names line up within
a column.  If the inner logical brackets were omitted the output
would look something like

\begin{verbatim}
(setq n-one v-one
 n-two v-two)
\end{verbatim}
assuming  that  the  expression  would not fit on a single line.
Notice that the format for a single pair  (name  and  value)  is
used repetitively.

  A  format function may contain more than one call on template.
The two functions below are equivalent.

\begin{verbatim}
(deformat (lambda . 3) (item)
  (template '({ * [i 1 1]) (first item))
  (if (null (second item))
    (template '("()" [c 1 1] (* [c 1 1]) })
              (rest (rest item)))
    (template '((* [c 1 1]) }) (rest item))))

(deformat (lambda . 3) (item)
  (template '({ * [i 1 1] (* [c 1 1]) })
            (first item)
            (cons (if (null (second item)) '!(!) (second item))
                  (rest (rest item)))))
\end{verbatim}
