\chapter*{Operating System Contacts}

\section{Calling the Command Shell}

The function system can be used to execute a system command.

\de{system}{(system COMMAND:string):undefined }{expr}
{ starts a (system specific) command interpreter and passes the
command to the interpreter.}

E.g. unter the UNIX operating system a Bourne shell is started and
COMMAND is interpreted following the conventions of this shell.
Of course it is possible to use e.g.
\begin{verbatim}
(system "bash")
\end{verbatim}


\section{The Working Directory}

The current working directory can be read by

\de{pwd}{(pwd):STRING}{expr}
{returns the current working directory in system specific format.}

The current working directory can be set by

\de{cd}{(cd DIR:string):BOOLEAN}{expr}
{sets the current working directory to DIR after expanding the
filename according to the rules of the operating system. If this operation
is not sucessful, the value Nil is returned.}

\section{Invoking Pipes}

Pipes are known to almost all the operating systems where PSL is implemented
for. It usually comes in two flavours, named pipes and anonymous pipes.
PSL uses the anonymous pipes for unidirectional communication.
One standard operation is the connection to the Gnuplot graphics system
from the REDUCE computer algebra system.
In the following we describe the pipes operation in PSL under Unix and
Windows.

\subsection{Pipes under Unix}

The pipe interface can be made available by loading the module {\bf pipes}.

\de{pipe-open}{(pipe-open CMD:string MODE:id):Channel}{expr}

{The Unix system tries to create the process defined by CMD using the
usual Unix mechanisms like pathes etc.
Depending on the MODE either the stdin or the stdout is connected to
the PSL program during the evaluation. The resulting LISP I/O file
is returned such that in subsequent read or write operation to this file
the command input or output is used. MODE needs to be either {\tt input}
or {\tt output}.}

A pipe can be used as a normal LISP file. A close operation on the file
will terminate the process. If the user wants to keep the process alive,
instead
of close the following function has to be used:

\de{abandonpipe}{(abandonpipe Channel):undefined}{expr}
{ This closes the pipe created by pipe-open, but the process will remain
active.}

Examples:
\begin{verbatim}

(load pipes)
(setq mail (pipe-open "mail hugo@zib.de" 'output))
(wrs mail)
(prin2 "This is the result of (expt 10 100) ") 
(prin2t (expt 10 100))
(wrs nil)
(close mail)

(load pipes)
(setq unix (pipe-open "uname -s" 'input))
(rds unix)
(setq variant (read))
(rds nil)
(close unix)
\end{verbatim}

\subsection{Pipes under MS/DOS}

The pipe interface can be made available by loading the module {\bf pipes}.

\subsection{Pipes under MS Windows}

The pipe interface can be made available by loading the module {\bf w-pipes}.

\section{Socket Interface (Unix only)}

The socket interface allows the PSL system to communiate through 
Unix socket as if they were normal LISP I/O channels.
This can be used to write
server or client processes in PSL easily. The module {\tt pslsocket}
has be be loaded in advance.

\DE{socketopen}{(socketopen HOST:(string or 0) SNumB:integer):
pair of LISP channels}{expr}
{If HOST is 0 (the server mode) the PSL system waits until a BIND request
is received on the socket SNumB. Otherwise HOST is interpreted as a string
(the client mode)
and a computer with this name is sought on the net. If the name could be
resolved successfully, a BIND
request is send to this computer with the socketnumber SNumB.
If the operation worked o.k., a pair is returned consisting of
two LISP channels, the CAR for input and the CDR for output, each of these
can be used like LISP channels furtheron.}

\DE{socketflushbuffer}{(socketflushbuffer C:channel):any}{expr}
{Sends the characters hanging around in the buffer via the socket
associated with the LISP channel.}

\section{Shared Memory Interface (Unix only)}

The PSL shared memory interface provides all function for
operating with shared memory regions and semaphores
(See e.g. Unix man pages for shmop, shmget, shmctl, semop, semctl,
semget etc.)
The definitions of these man pages are used in the paragraph.
Using the memory address map mechanism described below,it
is easy to write one's own shared memory application.

In the rest of this paragraph we describe a simple model
implementation of a 'pipe' using shared memory and a semaphore.
This code is contained in the file \$pu/shmem.sl.
\\

\de{shm-open}{(shm-open S:pair M:Mode):any}{expr}{If S = 0, a new shared
memory area is allocated. Otherwise S is expected to be a dotted pair
of shmid and semid of an existing shared memory. Legal modes are
{\bf input\_create, output\_create, input and output}.
A list consisting of channelnumber shmid and semid is returned.}

\de{independentdetachshm}{(independentdetachshm C:channel): any}{expr}
{Detaches the shared memory region used by C, closes the channel.}

\de{readfromshm}{(readfromshm C:channel):any}{expr}{Waits until
the shared memory region is readable, reads an expression and 
resets the mode to writable. Returns the expression.}

\de{twritetoshm}{(writetoshm C:channel E:expression):list}{expr}
{Waits until the shared memory region is writeable, prints the expression
using prin2. Returns the value of prin2.}


The following C program works together with the PSL part
such that it prints the messages read from shared memory
when they are ready for printing. It must be started with
two paramemters, namely the shmid and the semid to synchronize
with the PSL part.

\begin{verbatim}

#include <stdio.h>
#include <sys/types.h>
#include <sys/ipc.h>
#include <sys/sem.h>

struct sembuf ;

struct sembuf sembu;
struct sembuf *sembuptr;

main (argc,argv)
     int argc;
     char *argv[];

{ int sema , shmemid;
  char * shmaddress;

  sembuptr = &sembu;

  sscanf(argv[1],"%d",&shmemid);
  sscanf(argv[2],"%d",&sema);

  /* open shared memory */

 printf("the data is : %d %d\n",shmemid,sema);
  shmaddress = shmat(shmemid,0,0);

  while (1)
  { waitsema(sema) ; /* wait for a 0 */
    printf("the message is : %s \n",shmaddress +4);
    setsema (sema,2) ; /* ok, eaten */
  }
}

setsema (sema,val)
int sema,val;
{ semctl(sema,0,SETVAL,val); }

waitsema (sema)

int sema;
{ 
  sembu.sem_num =0; sembu.sem_op = 0; sembu.sem_flg =0;

  semop (sema, sembuptr , 1);
}
\end{verbatim}

\section{Miscellaneous Features}

It is a design goal to reduce the number of system calls
which have to be linked in to the kernel to a minimum.
The necessary information can be read elsewhere, e.g. through a pipe.
For some example, see the chapter on pipes above.

Nevertheless, some system calls have be be linked in, because they 
handle process specific information which is not available to a spawned
process.

The follwoing two system calls can be used to create unique file names.

\de{getpid}{(getpid):Integer}{expr}
{Returns the system specific number
under which the PSL process is registered in the currently running system.
If PSL is running under the control of a GUI,
this number corresponds to the PSL process, not the GUI process,
}

\de{gethostid}{(gethostid):any}{expr}{The system and hardware
dependent word (say), which identifies the processor is returned.
It is normally a good idea to store this word in a place where the
garbage collector will not come along, e.g. in a WARRAY.}

