%\begin{verbatim}
%(c) Copyright 1985 Hewlett-Packard Company, all rights reserved.
%\end{verbatim}

\chapter*{Vectors and Such}

\section{Vectors}

  A  vector  is a structured entity in which random elements may
be accessed with an integer  index.    A  vector  has  a  single
dimension.  Its maximum size is determined by the implementation
and  available  space.    A  vector  is denoted by enclosing its
elements within square brackets.
\begin{verbatim}
[10 TEN]
[COLORS (RED BLUE)]
\end{verbatim}


\subsubsection{Built-in Vector Creation and Copying Functions}

\de{mkvect}{(mkvect UPLIM:integer): vector}{expr}
{    Defines  and  allocates space for a vector with (add1 UPLIM)
    elements  accessed  as  0  ...  UPLIM.    Each  element   is
    initialized  to  nil.    If  UPLIM is -1, an empty vector is
    returned.  An error occurs if UPLIM is less than  -1  or  if
    the  amount of available memory is insufficient for a vector
    of this size.
}
\begin{verbatim}
    ***** A vector of size UPLIM cannot be allocated
\end{verbatim}

\de{make-vector}{(make-vector UPLIM:integer INITVAL:any): vector}{expr}
{    Similar to mkvect, except that each element  is  initialized
    to  INITVAL.   Note the difference between this function and
    make-string,  (see  the  section  on  creating  and  copying
    strings  in  Chapter  6).  This function creates a vector of
    (add1 UPLIM) elements where make-string creates a string  of
    UPLIM characters.
}

%\vspace{0.3cm}
    The  vector  created  by  this  function  will contain (add1
    UPLIM) references to INTVAL as opposed to creating a copy of
    UPLIM for each entry.

\begin{verbatim}
    1 lisp> (setq array (make-vector 1 (make-vector 1 0)))
    [[0 0] [0 0]]
    2 lisp> (vector-store (vector-fetch array 0) 0 1)
    1
    3 lisp> array
    [[1 0] [1 0]]
\end{verbatim}

\de{vector}{(vector [ARGS:any]): vector}{nexpr}
{    Create vector of elements from the list ARGS.  The  size  of
    the  vector  will  be equal to the number of elements in the
    list ARGS.  Each element of the vector is initialized to the
    corresponding element from ARGS
}

\de{copyvectortofrom}{(copyvectortofrom NEW:vector OLD:vector):\\ 
				NEW:vector}{expr}
{    The elements of NEW are set to the corresponding elements of
    OLD The elements are not copied.
}
\begin{verbatim}
    1 lisp> (setq a [[1 2 3]])
    [[1 2 3]]
    2 lisp> (setq b [0])
    [0]
    3 lisp> (copyvectortofrom b a)
    [[1 2 3]]
    4 lisp> (eq (getv a 0) (getv b 0))
    t
\end{verbatim}
\de{copyvector}{(copyvector V:vector): vector}{expr}
{    Create a new vector, with the elements initalized  from  the
    corresponding  elements  of  V.    The elements of V are not
    copied.
}
\begin{verbatim}
    1 lisp> (setq a "A STRING")
    "A STRING"
    2 lisp> (setq b (vector a))
    ["A STRING"]
    3 lisp> (setq c (copyvector b))
    ["A STRING"]
    4 lisp> (eq (getv b 0) (getv c 0))
    t
\end{verbatim}
\subsection{About the Basic Operations on Vectors}

  The functionality provided here overlaps what is  provided  in
some  other  ways.  The functions provided here have well-chosen
names and definitions, they provide  the  option  of  generating
efficient  code,  and  they  are  consistent  with  the esthetic
preferences of our community.

\subsection{The Operations}

  This section documents functions in the library module {\bf slow-vectors}
. There is another library module called {\bf fast-vectors}. The 
fast-vectors module provides alternate definitions for these
functions. When the switch fast-vectors is non-nil the compiler will use
these alternate definitions to produce effiecient code. However, 
there will not be any verification that arguments are of correct type
(in addition, it is assumed that numeric arguments are within a proper 
range). If invalid arguments are used, then at best your code will not
generate correct results, you may actually damage the PSL system.    
There are two side effects to loading fast-vectors. The slow-vectors 
module will be loaded and the switch fast-vectors will be set 
to t.


\de{vector-fetch}{(vector-fetch V:vector I:integer): any}{expr}
{    Accesses an element of a PSL vector.  Vector  indexes  start
    with  0.  The thing stored in that position of the vector is
    returned.
}

\de{vector-store}{(vector-store V:vector I:integer X:any): None Returned}{expr}
{    Stores into a PSL vector.  Vector indexes start with 0.
}

\de{vector-size}{(vector-size V:vector): integer}{expr}
{    Returns the number of elements  in  a  PSL  vector.    Since
    indexes  start with index 0, the size is one larger than the
    greatest legal index.  See also just below.
}

\de{vector-upper-bound}{(vector-upper-bound V:vector): integer}{expr}
{    Returns the greatest legal index for  accessing  or  storing
    into a PSL vector.  See also just above.
}

\de{vector-empty?}{(vector-empty? V:vector): boolean}{expr}
{    True  if  the  vector  has  no  elements  (its  size  is 0),
    otherwise NIL.
}
\subsection{Built-in Operations on Vectors}

These predate the fast-vectors (f-vector) and slow-vectors (s-vector)
library modules.


\de{getv}{(getv V:vector INDEX:integer): any}{expr}
{    Returns  the    value   stored  at  position  INDEX  of  the
    vector V.  The type mismatch error may   occur.    An  error
    occurs  if    the  INDEX  does not lie within 0 ... (upbv V)
    inclusive:
}
\begin{verbatim}
    ***** INDEX subscript is out of range
\end{verbatim}

\de{putv}{(putv V:vector INDEX:integer VALUE:any): any}{expr}
{    Stores VALUE in the vector V at  position  INDEX,  VALUE  is
    returned.    A  type mismatch error will occur if V is not a
    vector.  If INDEX is either negative or greater  than  (upbv
    V) then an error occurs.
}
\begin{verbatim}
    ***** Subscript `INDEX' in PutV is out of range
\end{verbatim}

\de{upbv}{(upbv U:any): {nil, integer}}{expr}
{    Returns  the upper limit of U if U is a vector, or nil if it
    is not.
}
\section{Word Vectors}

  Word-vectors or w-vectors are vector-like structures, in which
each element is a "word" sized, untagged entity.   This  can  be
thought of as a special case of fixnum vector, in which the tags
have been removed.

\de{make-words}{(make-words UPLIM:integer INITVAL:integer):\\ 
word-vector}{expr}
{    Defines  and  allocates  space  for a Word-Vector with\\ 
     (add1 UPLIM) elements, each initialized to INITVAL.
}

\de{make-bytes}{(make-bytes UPLIM:integer INITVAL:integer):\\ 
byte-vector}{expr}
{    Defines and allocates space for  a  byte-vector  with\\
     (add1 UPLIM) elements, each initialized to INITVAL.
}
\section{General X-Vector Operations}

  An   x-vector   is   either  a  vector,  string,  word-vector,
or  byte-vector.    Each  may   have   several
elements,  accessed  by  an integer index.  A valid index for an
x-vector X is from 0 to (size X).  Thus an x-vector X will  have
(add1 (size X)) elements. The functions described in this
section may also be applied to lists.

\de{size}{(size X:x-vector): integer}{expr}
{    Returns the size of x-vector X, the size is the index of the
    last element.
}

\de{indx}{(indx X:x-vector I:integer): any}{expr}
{    Access the I'th element of an x-vector.  An error occurs  if
    I is either negative or exceeds the size of X.
}
\begin{verbatim}
    ***** Index `I' out of range for X in INDX
\end{verbatim}

\de{setindx}{(setindx X:x-vector I:integer A:any): any}{expr}
{    Define  A  to  be  the I'th element of X.  If the index I is
    outside the range of X then it is an error (see indx  for  a
    description of the message).
}

\de{sub}{(sub X:x-vector B:integer S:integer): x-vector}{expr}
{    Extract  a subrange of an x-vector, starting at B, producing
    a new x-vector of size S.  Note that an x-vector of  size  0
    has one entry.
}

\de{setsub}{(setsub X:x-vector I1:integer S:integer Y:x-vector):\\ 
x-vector}{expr}
{    
    Store  subrange  of  Y  of  size  S  into  X starting at I1.
    Returns Y.
}

\de{subseq}{(subseq X:x-vector LO:integer HI:integer): x-vector}{expr}
{    Returns  an  x-vector  whose  size  is  (sub1  (-  HI  LO)),
    beginning  with  the  element  of X with index LO.  In other
    words, returns the subsequence  of  X  starting  at  LO  and
    ending just before HI.
}
\begin{verbatim}
    1 lisp> (setq a '[0 1 2 3 4 5 6])
    [0 1 2 3 4 5 6]
    2 lisp> (subseq a 4 6)
    [4 5]
\end{verbatim}

\de{setsubseq}{(setsubseq Y:x-vector):Y:x-vector}{expr}
{                                                  
    The  size  of  Y  must be (sub1 (- HI LO)) and Y must be the
    same type of x-vector as X.  Elements LO through  (sub1  HI)
    in X are replaced by the elements of Y.  Y is returned and X
    is changed destructively.
}

\begin{verbatim}
    1 lisp> (setq a "0123456")
    "0123456"
    2 lisp> (setsubseq a 3 7 "ABCD")
    "ABCD"
    3 lisp> A
    "012ABCD"
\end{verbatim}

\de{concat}{(concat X:x-vector Y:x-vector): x-vector}{expr}
{    Concatenate  2  x-vectors.    Currently they must be of same
    type.
}

\de{totalcopy}{(totalcopy S:any): any}{expr}
{    Returns a unique copy of  the  entire  structure,  i.e.,  it
    copies   everything   for   which  storage  is  allocated  -
    everything but  inums  and  ids.    Like  copy  (Chapter  5)
    totalcopy  will  not  terminate  when  applied  to  circular
    structures.
}
\begin{verbatim}
    1 lisp> (setq x '("ONE" 2)
    1 lisp>       y (totalcopy x)
    1 lisp>       z (copy x))
    ("ONE" 2)
    2 lisp> (eq (first x) (first y))
    NIL
    3 lisp> (eq (first x) (first z))
    T
\end{verbatim}
