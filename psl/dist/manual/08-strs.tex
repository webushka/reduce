\chapter*{Characters and Strings}

\section{Characters}

  In  PSL  a  character is its ASCII code representation.  Using
numeric codes to represent characters leads  to  programs  which
are  difficult  to  read.    You  are  encouraged to use char to
represent characters.


\de{char}{(char U:id): integer}{macro}
{    This macro is part of the USEFUL module.    The  char  macro
    returns  the  ASCII  code  which  corresponds  to the single
    character passed as an argument.  Char will  accept  alias's
    for  characters.    An  alias  is  established by defining a
    charconst property on the property list of the alias.    The
    value  of  this  property  should  be  the ASCII code of the
    character which is being aliased.  The following alias's are
    defined when the {\bf useful} package is loaded.
}

\vspace{0.5cm}
 \begin{center}
\begin{tabular}{ll}
    NULL\\
    BELL\\
    BACKSPACE\\
    TAB\\
    LF &         line feed\\
    EOL &        end of line\\
    FF &         form feed\\
    CR &         carriage return\\
    EOF &        end of file\\
    ESCAPE &     may be abbreviated ESC\\
    SPACE &      an alias is BLANK\\
    RUBOUT &     may be abbreviated RUB\\
    DELETE &     may be abbreviated DEL \\
%\end TABELLE***
\end{tabular}
\end{center}
\vspace{0.5cm}    
				By default, the PSL reader converts  upper  case  alphabetic
    characters to lower case.  This default is controlled by the
    value  of the switch raise. A value of t indicates that the
    conversion should be done. Assuming  raise  is  t, the
    expression (char  a) refers to the character "a". The
    character "A" is referenced by (char !A). The "!" is used as an
    escape character, see chapter 12 for  more  information.  The
    application of lower is said to modify the character.  Lower is not
    a defined  function  but it  does  have  the  attribute 
    char-prefix-function  on its property list.  The value of this 
    property  is  a  function which  will modify the ASCII code of its
    argument. Modifiers are control and meta (the modifier cntrl is an
    abbreviation  for  control). The following example is a simplified
    definition of char.

\begin{verbatim}
    (defmacro char (u)
      (cond ((idp u) (or (get u 'charconst)
                         (id2int u)))
            ((digit-char u))
            ((and (pairp u)
                  (get (first u) 'char-prefix-function))
             (list (get (first u) 'char-prefix-function)
                   (list 'char (second u))))))
\end{verbatim}
    Notice that the digits have to be treated as a special case.
    The  PSL reader converts a digit to a reference to a numeric
    constant.  The definition of the alias space is

\begin{verbatim}
    (put 'space 'charconst 32)
\end{verbatim}

    If the type of the argument to char is not correct then  the
    warning

\begin{verbatim}
    *** Unknown character constant `FORM'
\end{verbatim}
    is printed and the result will be zero.\\

    The  USEFUL  package  also  defines the read macro \#. Read
				macros are explained in detail in  chapter  12. When  the
    reader encounters \#, char is applied to the next character. 
				If  this next character is a lower case alphabetic character
    then it will be converted to upper case if the switch  raise
    is set to t.  If this next character is a read macro then it
    will  be  applied.   Such  a  read  macro  should  return an
    expression which is acceptable to char.

\begin{verbatim}
    1 lisp> #\a
    65
    2 lisp> #\`(a b c)
    *** Unknown character constant: `(BACKQUOTE (A B C))'
    0
\end{verbatim}
\subsubsection{Common LISP Functions on Characters}

  The following functions are available by loading  the  library
module {\bf chars}.

  Common  LISP  provides  a  character  data type in which every
character object has three attributes:  code,  bits,  and  font.
The  bits  attribute  allows extra flags to be associated with a
character.  The font attribute permits a  specification  of  the
style  of  the  glyphs  (such as italics).  PSL does not support
nonzero bit and font attributes.  Because of this, some  of  the
Common  LISP  character functions described below have no affect
or are not very useful as implemented in PSL.  They are  present
for compatibility.

  An  argument to any of the following functions should make use
of char or one of the two read  macros  $\#/$  and 
$\#\backslash$.    Since  a character in PSL is represented as its
ASCII code it is possible to give a numeric argument to these
functions.  However, the use of  numbers  makes  the  code 
difficult  to  read and much less transportable.  The read macro
$\#\backslash$ is described in the discussion of char above.  The
read macro \#/ returns the ASCII code of  the next  character. 
In contrast to
$\#\backslash$, the switch raise is ignored and if the next character is a read
macro it is not applied.

\begin{verbatim}
1 lisp> (eq (char (lower a)) #/a)
T
2 lisp> (eq (char (lower a)) #\a)
NIL
3 lisp> #/`
96
4 lisp> #\`
*** Unknown character constant: `(BACKQUOTE !^Z)'
0
\end{verbatim}

\de{standard-charp}{(standard-charp C:character): boolean}{expr}
{    Returns t if the argument is one of the  95  ASCII  printing
    characters.
}
\begin{verbatim}
    1 lisp> (standard-charp (char a))
    T
    2 lisp> (standard-charp (char (control a)))
    NIL
\end{verbatim}

\de{graphicp}{(graphicp C:character): boolean}{expr}
{    Returns  t  if C is a printable character and nil otherwise.
    Control and formatting characters are considered to  be  not
    printable.  The space character is a printable character.
}

\de{string-charp}{(string-charp C:character): boolean}{expr}
{    Returns  t  if  C is a character that can be an element of a
    string.  Any character  that  satisfies  standard-charp  and
    graphicp also satisfies string-charp.
}

\de{alphap}{(alphap C:character): boolean}{expr}
{    Returns  t  if  C  is  an alphabetic character.  Liter is an
    equivalent function.
}

\de{uppercasep}{(uppercasep C:character): boolean}{expr}
{    Returns t if C is an upper case letter.
}

\de{lowercasep}{(lowercasep C:character): boolean}{expr}
{    Returns t if C is a lower case letter.
}

\de{bothcasep}{(bothcasep C:character): boolean}{expr}
{    In PSL this function is the same as alphap.
}

\de{digitp}{(digitp C:character): boolean}{expr}
{    Returns t if C is a  digit  character  (optional  radix  not
    supported).  An equivalent function is digit
}

\de{alphanumericp}{(alphanumericp C:character): boolean}{expr}
{    Returns t if C is a digit or an alphabetic.
}

\de{char=}{(char= C1:character C2:character): boolean}{expr}
{    Returns t if C1 and C2 are the same in all three attributes.
}

\de{char-equal}{(char-equal C1:character C2:character): boolean}{expr}
{    Returns  t  if  C1 and C2 are similar.  Differences in case,
    bits, or font are ignored by this function.
}

\de{char$<$}{(char$<$ C1:character  C2:character): boolean}{expr}
{    Returns t if C1 is strictly less than C2. }

\de{char$>$}{(char$>$ C1:character C2:character): boolean}{expr}
{    Returns t if C1 is strictly greater than C2. }

\de{char-lessp}{(char-lessp C1:character C2:character): boolean}{expr}
{    Like char$<$ but ignores differences in case, fonts, and bits. }

\de{char-greaterp}{(char-greaterp C1:character C2:character): boolean}{expr}
{    Like char$>$ but ignores differences in case, fonts, and
bits. }

\de{char-code}{(char-code C:character): character}{expr}
{    Returns the code attribute of C.  In PSL this function is an
    identity function.
}

\de{char-bits}{(char-bits C:character): integer}{expr}
{    Returns the bits attribute of C, which is always 0 in PSL.  }

\de{char-font}{(char-font C:character): integer}{expr}
{    Returns the font attribute of C, which is always 0 in PSL.  }

\de{code-char}{(code-char I:integer): {character,nil}}{expr}
{    The purpose of this function is to be able  to  construct  a
    character  by  specifying the code, bits, and font.  Because
    bits and font attributes are not used in PSL,  code-char  is
    an identity function.
}

\de{character}{(character C:{character, string, id}): character}{expr}
{    Attempts  to  coerce  C  to  be  a  character.    If  C is a
    character, C is returned.  If C is a string, then the  first
    character  of the string is returned.  If C is a symbol, the
    first character of the print name of the symbol is returned.
    Otherwise an error occurs.}
\begin{verbatim}
    ***** `FORM' cannot be coerced to a character
\end{verbatim}
\de{char-upcase}{(char-upcase C:character): character}{expr}
{    If (lowercasep C) is true, then char-upcase returns the code
    of the upper case of C.  Otherwise it returns the code of C.
}

\de{char-downcase}{(char-downcase C:character): character}{expr}
{    If (uppercasep C) is true, then  char-downcase  returns  the
    code  of the lower case of C.  Otherwise it returns the code
    of C.
}

\de{digit-char}{(digit-char N:fixnum): integer}{expr}
{    If N corresponds to a single digit then the character  which
    represents  that digit is returned.  The Common LISP version
    will  accept  an  optional  radix  argument,  this  function
    assumes  a  radix  of  ten.    If N does not correspond to a
    single digit then nil is returned.
}

\de{char-int}{(char-int C:character): integer}{expr}
{    Converts  character  to  integer.    This  is  the  identity
    operation in PSL.
}

\de{int-char}{(int-char I:integer): character}{expr}
{    Converts  integer  to  character.    This  is  the  identity
    operation in PSL.
}
\section{Strings}

\subsection{String Creation and Copying}

  The  following  are  built-in  string  creation  and   copying
functions:


\de{allocate-string}{(allocate-string SIZE:integer): string}{expr}
{    Constructs  and  returns a string with SIZE characters.  The
    contents of the string are not initialized.
}

\de{make-string}{(make-string SIZE:integer INITVAL:integer): string}{expr}
{    Constructs and returns a string with  SIZE characters,  each
    initialized to the ASCII code INITVAL.
}

\de{mkstring}{(mkstring UPLIM:integer INITVAL:integer): string}{expr}
{    An  old form of make-string.  Returns a string of characters
    all initialized to INITVAL, with upper  bound  UPLIM.    The
    returned string contains a total of (add1 UPLIM) characters.
}

\de{string}{(string [ARGS:integer]): string}{nexpr}
{    Create string of elements from  a list of ARGS.
}
\begin{verbatim}
    1 lisp> (string 65 66 67)
    "ABC"
\end{verbatim}

\de{copystringtofrom}{(copystringtofrom NEW:string OLD:string): NEW:string}{expr}
{    Copy  all  characters from old  into new.  This function  is
    destructive.
}

\de{copystring}{(copystring S:string): string}{expr}
{    Copy to new string, allocating space.
}
\subsection{About the Basic String Operations}

  The representation of strings  is  very  similar  to  that  of
vectors.   Due to this similarity, there are functions which may
be  applied  to  either  data  type.    Such  functions  provide
primitive  operations  on  strings.    PSL  provides  many other
functions specific to  strings  which  are  defined  in  various
library modules.

\subsection{The Operations}

  This   section  documents  functions  in  the  library  module
{\bf slow-strings (s-strings)}. There is another library module called
{\bf fast-strings (f-strings)}. The fast-strings module provides alternate
definitions for these functions.  When the  switch  fast-strings
is  non-nil the compiler will use these alternate definitions to
produce  efficient  code.    However,  there  will  not  be  any
verification that arguments are of correct type (in addition, it
is  assumed  that  numeric arguments are within a proper range).
If invalid arguments are used, then at best your code  will  not
generate  correct  results,  you  may  actually  damage  the PSL
system.  There are two side  effects  to  loading  {\bf fast-strings}
The {\bf slow-strings}  module   will  be  loaded  and  the  switch
fast-strings will be set to t.  For more information on the
switch fast-strings see Chapter 19.


\de{string-fetch}{(string-fetch S:string I:integer): any}{expr}
{    Accesses  an  element  of  a  PSL  string.  Indexes are zero
    based.  The ASCII character stored in that position  of  the
    string is returned.
}

    Characters are represented by inums.  You should not rely on
    this  when  you  write  code.    Such  code cannot be easily
    transported to other systems where either  the  encoding  is
    different or where characters are a seperate data type.


\de{string-store}{(string-store S:string I:integer X:char): None Returned}{expr}
{    Stores into a PSL string.  String indexes start with 0.
}

\de{string-length}{(string-length S:string): integer}{expr}
{    Returns  the  number  of  elements  in  a PSL string.  Since
    indexes start with index 0, the size is one larger than  the
    greatest   legal   index.      Compare  this  function  with
    string-upper-bound, documented below.
}

\de{string-upper-bound}{(string-upper-bound S:string): integer}{expr}
{    Returns the greatest legal index for  accessing  or  storing
    into   a   PSL   string.      Compare   this  function  with
    string-length, documented above.
}

\de{string-empty?}{(string-empty? S:string): boolean}{expr}
{    True if  the  string  has  no  elements  (its  size  is  0),
    otherwise nil.
}

\section{Common LISP String Functions}

  A  Common LISP compatible package of string functions has been
implemented in PSL, obtained by loading the {\bf strings} module.
This  section  describes the  {\bf strings}  module, including a few
functions in it that are not Common LISP functions.

\subsection{String comparison:}

\de{string=}{(string= S1:string S2:string): boolean}{expr}
{    Compares two strings S1 and S2, case sensitive.
}

\de{string-equal}{(string-equal S1:string S2:string): boolean}{expr}
{    Compare two strings S1 and S2, ignoring case, bits and font.
}					
  
The following string comparison functions  are  extra-boolean.
If  the  comparison  results  in  a value of t, the index of the
first character position at which the strings fail to  match  is
returned.    The  result  can  also be thought of as the longest
common prefix of the two strings.

\de{string$<$}{(string$<$ S1:string S2:string): extra-boolean}{expr}
{    Lexicographic comparison of strings.  Case sensitive.
}

\de{string$>$}{(string$>$ S1:string S2:string): extra-boolean}{expr}
{    Lexicographic comparison of strings.  Case sensitive.
}

\de{string$<$=}{(string$<$= S1:string S2:string):
extra-boolean}{expr} {    Lexicographic comparison of strings. 
Case sensitive. }

\de{string$>$=}{(string$>$= S1:string S2:string):
extra-boolean}{expr} {    Lexicographic comparison of strings. 
Case sensitive. }

\DE{string$<>$}{(string$<>$ S1:string S2:string):
extra-boolean}{expr} {    Lexicographic comparison of strings. 
Case  sensitive.    In
    Common LISP the function is named string/=.
}

\DE{string-lessp}{(string-lessp S1:string S2:string): extra-boolean}{expr}
{    Lexicographic  comparison  of strings.  Case differences are ignored.
}

\DE{string-greaterp}{(string-greaterp S1:string S2:string): extra-boolean}
{expr}
{    Lexicographic comparison of strings.  Case  differences  are ignored.
}

\DE{string-not-greaterp}
{(string-not-greaterp S1:string S2:string): extra-boolean}{expr}
{    Lexicographic  comparison  of strings.  Case differences are ignored.
}

\de{string-not-lessp}{(string-not-lessp S1:string S2:string): extra-boolean }
{expr}
{    Lexicographic comparison of strings.  Case  differences  are ignored.
}

\de{string-not-equal}
{(string-not-equal S1:string S2:string): extra-boolean}{expr}
{    Lexicographic  comparison  of strings.  Case differences are ignored.
}
		
\subsection{String Concatenation:}

\de{string-concat}{(string-concat [S:string]): string}{macro}
{    Concatenates all of  its  string  arguments,  returning  the
    newly created string.  Not in Common LISP.
}

\de{string-repeat}{(string-repeat S:string I:integer): string}{expr}
{  Appends copy of S to itself total of I-1 times.  Not in Common LISP.
}\\

\subsection{Transformation of Strings:}

\de{substring}{(substring S:string LO:integer  HI:integer):
string}{expr} {    Same as subseq, but the first argument  must  be 
a  string.
    Returns a substring of S of size (sub1 (- HI LO)), beginning
    with the element with index LO.  Not in Common LISP.
}

\DE{string-trim}{(string-trim BAG:\{list, string\} S:string): string}{expr}
{    Remove  leading and trailing characters in BAG from a string
    S.
}
\begin{verbatim}
    1 lisp> (string-trim "ABC" "AABAXYZCB")
    "XYZ"
    2 lisp> (string-trim (list (char a)
    2 lisp>                    (char b)
    2 lisp>                    (char c))
    2 lisp>              "AABAXYZCB")
    "XYZ"
    3 lisp> (string-trim '(65 66 67) "ABCBAVXZCC")
    "VXZ"
\end{verbatim}

\DE{string-left-trim}{(string-left-trim BAG:\{list, string\} S:string): string}
{expr}
{    Remove leading characters from string.
}

\de{string-right-trim}{(string-right-trim BAG:\{list, string\} S:string): string}
{expr}
{    Remove trailing characters from string.
}

\de{string-upcase}{(string-upcase S:string): string}{expr}
{    Copy and raise all alphabetic characters in string.
}

\de{nstring-upcase}{(nstring-upcase S:string): string}{expr}
{    Destructively raise all alphabetic characters in string.
}

\de{string-downcase}{(string-downcase S:string): string}{expr}
{    Copy and lower all alphabetic characters in string.
}

\de{nstring-downcase}{(nstring-downcase S:string): string}{expr}
{    Destructively lower all alphabetic characters in string.
}

\de{string-capitalize}{(string-capitalize S:string): string}{expr}
{    Copy and raise first letter of all words  in  string;  other
    letters in lower case.
}

\de{nstring-capitalize}{(nstring-capitalize S:string): string}{expr}
{    Destructively raise first letter of all words; other letters
    in lower case.
}\\

\subsection{Type Conversion:}

\de{string-to-list}{(string-to-list S:string): list}{expr}
{    Unpack string characters into a list.  Not in Common LISP.
}

\de{string-to-vector}{(string-to-vector S:string): vector}{expr}
{    Unpack string characters into a vector.  Not in Common LISP.
}\\

\subsection{Other:}

\de{string-length}{(string-length S:string): integer}{expr}
{    Last  index of a string, plus one.  Not in Common LISP.  Use
    string-size.
}

\de{rplachar}{(rplachar S:string I:integer  C:character): character}{expr}
{    Store a character C in a string S at position I.
}

\subsection{Substring Comparison}

  The library module STRING-SEARCH provides efficient  functions
for comparing a string against a substring of another string.

\de{substring=}{(substring= S1:string S2:string START:integer):
boolean}{expr} { Returns true if there is a substring of S2
starting at START and string= to S1, otherwise returns nil. }

    Similar to

\begin{verbatim}
(string= S1 (substring S2 START (+ START (string-length S1))))
\end{verbatim}
    but  note  that  this returns nil (no error is signalled) if
    there are fewer than

\begin{verbatim}
    (string-length S1)
\end{verbatim}
    characters from position START through the end of S2.

\DE{substring-equal}
{(substring-equal S1:string S2:string START:integer): boolean}{expr}
{   
    Returns true if there is a substring of S2 starting at START
    and string-equal to S1, otherwise returns nil.
}

Similar to

\begin{verbatim}
(string-equal S1 (substring S2 START (+ START (string-length S1))))
\end{verbatim}
    but note that this returns nil (no error  is  signalled)  if
    there are fewer than

\begin{verbatim}
(string-length S1)
\end{verbatim}
    characters from position START through the end of S2.

\subsection{Searching for Strings}

The library module {\bf str-search} provides functions for
searching for a string within another string.   These  functions
are efficiently implemented.

  The  two  strings  involved  in these searching operations are
referred to as the "domain" and the "target".   These  functions
search  for  an  occurrence  of  the  "target" string within the
"domain" string.

  The operations for string searching return the  index  of  the
leftmost  character  in  the  first  matching part of the domain
string that is found, starting from the left.  If  no  match  is
found, nil is returned.

\DE{string-search}{(string-search TARG:string DOM:string):\{integer,
nil\}}{expr} 
{Searches  for  the  leftmost occurrence of the target
in the domain. This function is case-sensitive. If  passed  two
strings, Common LISP "search" will give the same results.
}

\DE{string-search-from}{(string-search-from TARG:string DOM:string
START:integer):\\
\{integer, nil\}}{expr} {
    Like  string-search,  but  the  search effectively starts at
    index START in the domain.
}

\DE{string-search-equal}{(string-search-equal TARG:string
DOM:string):\{integer, nil\}}{expr}
{   Like  string-search  except   that   the   comparisons   are
    case-insensitive.  }

\DE{string-search-from-equal}
{(string-search-from-equal TA:string D:string ST:integer):\\
\{integer, nil\}}
{expr}
{ Like  string-search-from  except  that  the  comparisons are
    case-insensitive.
}

\subsection{Reading and Writing Strings}

  The library module {\bf str-input} provides  some  facilities  to
support  taking  input  from  strings.  Among other things, this
permits  a  user  to  obtain   a   number   from   its   printed
representation using the PSL number parser.


\de{with-input-from-string}{(with-input-from-string HEADER:list
[BODY:form]):any}{macro} 
{    The argument HEADER should be a two element list. \\
The first element should be an identifier, the second a string.
{\tt ($<channel> <string>$)}.
    The argument $<string>$ is treated as if it were the text of a
    file.    The argument $<channel>$ is bound to an input channel
    which is opened  to  give  access  to  $<string>$.   Once  the
    channel  has  been  opened, each form BODY is evaluated (the
    forms are evaluated in a  left  to  right  order).    It  is
    expected  that  these forms will be used to read and process
    input from  $<string>$.    The  value  of  the  last  form  is
    returned.  Once the application of with-input-from-string is
    complete the input channel will be closed.
}

\begin{verbatim}
    (de string-to-words (string)
      (with-input-from-string
        (channel string)
        (do ((result nil (aconc result item))
             (item (channelread channel) (channelread channel)))
            ((eq item $eof$) result))))
\end{verbatim}
\begin{verbatim}
    1 lisp> (string-to-words "DOCUMENTATION IS FUN")
    (DOCUMENTATION IS FUN)
\end{verbatim}

\de{string-read}{(string-read STR:string): any}{expr}
{    Reads  one  s-expression  from the string STR.  The function
    channelread is used to  do  this.    Note  that  it  is  not
    necessary   to   terminate   the  string  with  a  delimiter
    character.  An end of file character is also  considered  to
    be a delimiter character.
}
\begin{verbatim}
    1 lisp> (string-read "TOKEN")
    TOKEN
    2 lisp> (string-read "TWO TOKENS")
    TWO
\end{verbatim}
    Notice  that  characters  beyond  the first s-expression are
    ignored.    This   function   is   defined   in   terms   of
    with-input-from-string.

\begin{verbatim}
    (de string-read (string)
      (with-input-from-string
        (ch string)
        (channelread ch)))
\end{verbatim}
  The  library  module {\bf string-output} provides a facility for
writing to strings.  The function bldmsg provides the capability
to construct a string using  formatting  directives.    However,
complicated  strings  can  be  constructed more easily using the
macro with-output-to-string. For example, longer strings can  be
constructed  by  including the channellinelength function; items
can be printed to the string  incrementally  (in  a  stream-like
fashion) from a loop


\de{with-output-to-string}
{(with-output-to-string HEADER:list [BODY:form]): string}{macro}
{    The argument HEADER should be a two element list.  The first
    element  should be an identifier.  The second element can be
    either a string or nil (in  which  case  a  default  initial
    string  is  allocated) -- in either case, the initial string
    is extened as necessary.
}
    The written  string  is  return  as  a  result  (this  is  a
    substring  copy  of  the  string  used  for  writing).   For
    example,
\begin{verbatim}
    (setf row
      (with-output-to-string (wchan nil)
        (channellinelength wchan 350)
        (for (in tab '(0 100 200 300))
         (in str '("A" "B" "C" "D"))
         (do (channelprintf wchan "%t%w" tab str))
         )))
\end{verbatim}
