\chapter*{Function Definition and Binding}

\section{Function Definition in PSL}

Functions in PSL are global entities.  To  avoid
function-variable  naming  clashes,  the  Standard  LISP  Report
required  that  no  variable  have  the same name as a function.
There is no conflict in PSL, as separate function cells and
value cells are used.

  The  first  major  section  in  this  chapter describes how to
define new  functions;  the  second  describes  the  binding  of
variables  in PSL.  The final section presents binding functions
useful in building new interpreter functions.

\subsection{Function Types}

  Eval-type functions are those called with evaluated arguments.
NoEval  functions  are  called   with   unevaluated   arguments.
Spread-type   functions   have   their  arguments  passed  in  a
one-to-one  correspondence   with   their   formal   parameters.
NoSpread functions receive their arguments as a single list.

There are four function types implemented in PSL:\\

\begin{tabular}{lp{12.5cm}}

expr & An Eval, Spread function, with a maximum number of  arguments. The 
maximum   depends  upon  the
              implementation.    In  referring  to  the   formal
              parameters we mean their values.  Each function of
              this   type  should  always  be  called  with  the
              expected number of parameters, as indicated in the
              function definition.\\

fexpr &         A NoEval, NoSpread function.  There is no limit on
              the number of arguments.    In  referring  to  the
              formal   parameters   we   mean   the  unevaluated
              arguments, collected as a single list, and  passed
              as a single formal parameter to the function body.\\

nexpr &         An  Eval,  NoSpread  function.   Each call on this
              kind of function may present a different number of
              arguments, which are evaluated, collected  into  a
              list,  and  passed  in  to  the function body as a
              single formal parameter.\\

macro &         The macro  is  a  function  which  creates  a  new
              S-expression    for   subsequent   evaluation   or
              compilation.  There is no limit to the  number  of
              arguments  a  macro may have.  The descriptions of
              the  eval  and  expand  functions  in  Chapter  11
              provide precise details.\\
\end{tabular}
\subsection{Notes on Code Pointers}

  A  code-pointer  may  be  displayed  by the print functions or
expanded by explode.  The value appears in the convention of the
implementation e.g. (\#$<$Code:A N$>$,  where  A  is  the  number  of
arguments of the function, and N is the function's entry point).
A  code-pointer may not be created by compress.  (See Chapter 12
for descriptions of explode and  compress.)    The  code-pointer
associated with a compiled function may be retrieved by getd and
is valid as long as PSL is in execution. Normally a code-pointer
is  stored using putd.  It may be checked for equivalence by eq.
The value may be checked for being a code-pointer by  the  codep
function.

\subsection{Functions Useful in Function Definition}

  In  PSL,  ids  have  a  function cell that usually contains an
executable instruction which either JUMPs directly to the  entry
point  of a compiled function or executes a CALL to an auxiliary
routine that handles interpreted functions, undefined functions,
or other special services (such as auto-loading functions, etc).
The user can pass anonymous function objects around either as  a
code-pointer,  which  is a tagged object referring to a compiled
code block, or a lambda expression, representing an  interpreted
function.


\de{putd}{(putd Fname:id TYPE:ftype BODY:\\
{lambda,code-pointer}):id}{expr}  
{    
Creates  a function with name Fname and type TYPE, with body
    as the function definition.  If successful, putd returns the
    name of the defined function.
}

    If the body is a lambda there  are  two  possible  outcomes.
    When  the  switch comp is set to t the body will be compiled
    and a special instruction to jump to the start of  the  code
    is  placed in the function cell.  If the switch *comp is set
    to nil then the body is saved on the property list under the
    indicator *lambdalink and a call to an interpreter  function
\begin{verbatim}
    (lambdalink) is placed in the function cell.
\end{verbatim}
    If  the body is a code-pointer then a special instruction to
    jump to the start of the code  is  placed  in  the  function
    cell.

    The  Type is recorded on the property list of Fname if it is
    not an expr.

    After using putd on Fname, getd will  return  a  pair  which
    specifies the type and the body of the definition.

  The following switches are useful when defining functions.

\variable{*redefmsg}{[Initially: t]}{switch}
{If *redefmsg is not nil, the message
%%\begin{verbatim}

    {\tt *** Function `FOO' has been redefined}

%%\end{verbatim}
    is printed whenever a function is redefined.
}

\variable{*usermode}{(Initially: t)}{switch}
{ Controls   action  on  redefinition  of  a  function.    All
    functions defined when *usermode  is  t  are  flagged  USER.
    Functions  which  are  flagged USER can be redefined freely.
    If an attempt is made to redefine a function  which  is  not
    flagged USER, the query

%%\begin{verbatim}
     {\tt Do you really want to redefine the system function FOO?}
%%\end{verbatim}

    is  made,  requiring  a  Y,  N,  YES,  NO, or B response.  B
    starts the break loop, so that one can change the setting of
    *usermode.  After exiting the break loop, one must answer Y,
    Yes, N, or No (See yesp in Chapter 12).    If  *usermode  is
    nil,   all  functions  can  be  redefined  freely,  and  all
    functions defined    have  the  USER  flag  removed.    This
    provides some protection from redefining system functions.
}

\variable{*comp}{[Initially: nil]}{switch}
{
    The  value  of *comp controls  whether or not putd  compiles
    the  function before  defining it.   If  *comp  is  nil  the
    function  is  defined  as  a lambda expression.  If *comp is
    non-nil,  the  function  is  first  compiled.    Compilation
    produces  certain changes in the semantics of functions, see
    Chapter 19 for information.
}

\de{getd}{(getd U:any): {nil, pair}}{expr}
{    If U is not the name of a defined function, nil is returned.
    otherwise
    ({expr, fexpr, macro, nexpr} . {code-pointer, lambda})    is
    returned.
}

\de{copyd}{(copyd NEW:id OLD:id): NEW:id}{expr}
{    Normally  the  function body and type of NEW become the same
    as those of OLD.  However, if the switch *comp is set  to  t
    and  the body of OLD is not compiled then NEW will be set to
    the compiled version of the body of OLD.  If  no  definition
    exists for OLD an error:
}
\begin{verbatim}
    ***** OLD has no definition in COPYD
\end{verbatim}
    is given.  NEW is returned.


\de{remd}{(remd U:id): {nil, pair}}{expr}
{    Removes the  function  named  U  from  the  set  of  defined
    functions.    Returns the (ftype . function) pair or nil, as
    does getd.  If the function type is not  expr  then  it  was
    recorded on the property list when the function was defined.
    In  such  cases  this  function removes the type information
    from the property list.
}
\subsection{Function Definition in LISP Syntax}

  The functions de, df, dn, dm, and ds are used in PSL to define
functions and  macros.    The  functions  are  compiled  if  the
compiler is loaded and the switch comp is set to t.

\de{de}{(de Fname:id PARAMS:id-list [FN:form]): id}{macro}
{    Defines  the  function named Fname, of type expr.  The forms
    FN are  made  into  a  lambda  expression  with  the  formal
    parameter  list  PARAMS, and this is used as the body of the
    function.

    Previous definitions of the function are lost.  The name  of
    the defined function, Fname, is returned.

    The   COMMON   module  defines  the  macro  DeFun  which  is
    equivalent to de.
}

\de{df}{(df Fname:id PARAM:id-list [FN:form]): id}{macro}
{    Defines the function named Fname, of type fexpr.  The  forms
    FN  are  made  into  a  lambda  expression  with  the formal
    parameter list PARAM, and this is used as the  body  of  the
    function.    The  parameter  list  should  only  contain one
    parameter.

    Previous definitions of the function are lost.  The name  of
    the defined function, Fname, is returned.
}

\de{dn}{(dn Fname:id PARAM:id-list [FN:form]): id}{macro}
{    Defines  the function named Fname, of type nexpr.  The forms
    FN are  made  into  a  lambda  expression  with  the  formal
    parameter  list  PARAM,  and this is used as the body of the
    function.   The  parameter  list  should  only  contain  one
    parameter.

    Previous  definitions of the function are lost.  The name of
    the defined function, Fname, is returned.
}

\de{dm}{(dm Mname:id PARAM:id-list [FN:form]): id}{macro}
{    Defines the function named Fname, of type macro.  The  forms
    FN  are  made  into  a  lambda  expression  with  the formal
    parameter list PARAM, and this is used as the  body  of  the
    function.    The  parameter  list  should  only  contain one
    parameter.

    Previous definitions of the function are lost.  The name  of
    the defined function, Fname, is returned.
}

    The function list can be defined as follows
\begin{verbatim}
    (dm list (a)
      (if (< (length a) 2) ()
        (cons 'cons
              (cons (second a)
                    (ncons (cons 'list (rest (rest a))))))))
\end{verbatim}
    Now  consider  what  occurs during the evaluation of (list 1
    2).  The list (list 1 2) is passed to the list  macro  which
    returns  (cons  1  (list  2))  for  further evaluation.  The
    evaluator will call the list macro again for (list 2).  This
    second call on  the  macro  will  return  (cons  2  (list)).
    Finally,  (list)  is transformed into nil by a third call on
    the list macro and the entire process will  terminate  after
    evaluating  (cons  1  (cons  2  nil)).    Notice the lack of
    distinction between program and data    The  data  structure
    representation  of  the  function  call  is  passed  to  the
    function as its parameter.


\de{ds}{(ds Sname:id PARAMS:id-list [FN:form]): id}{macro}
{    Defines the function named Sname of type macro.  The  syntax
    of  ds  is  similar  to  that  of  de except that a macro is
    defined instead of an expr. 
}

 Perhaps the  behavior  of  this
    function  is best described with an example.  The evaluation
    of

\begin{verbatim}
    (ds first (x) (car x))
\end{verbatim}
    will generate an expression similar to

\begin{verbatim}
    (dm first (a)
      (prog (x)
            (setq a (cdr a))
            (setq x (car a))
            (return (list 'car x))))
\end{verbatim}
    which  is  then  evaluated.    A  sequence   of   assignment
    statements  are  created to initialize each parameter to its
    corresponding argument.  Each id within the body FN which is
    not in the parameter list is quoted.  It is an error to call
    a macro defined  this  way  with  more  arguments  than  are
    specified by the parameter list.  An error of this type will
    cause the message

\begin{verbatim}
    ***** Argument mismatch in SMacro expansion
\end{verbatim}
    to be printed.  When a call is made without enough arguments
    the additional parameters are set to nil.

  The following macro utilities are in the {\bf useful} module.

\de{defmacro}{(defmacro A:id B:form [C:form]): id}{macro}
{    Defmacro  is a useful tool for defining macros.  The form of
    an application of defmacro is given below,  square  brackets
    are   used  to  indicate  zero  or  more  occurances  of  an
    expression.
}
\begin{verbatim}
    (DEFMACRO <name> <pattern> [<form>])
\end{verbatim}
    The pattern is an list, each element is either an  id  or  a
    pair.    All  of  the ids in the pattern are local variables
    which may be used freely in the body (the  <form>s).    When
    the  macro  is called the pattern is matched against the cdr
    of the macro call, binding the ids of the pattern  to  their
    corresponding  parts  of  the  call.    Once  the binding is
    complete the body is evaluated, the result is the  value  of
    the  last form.  Defmacro is often used with backquote.  The
    following examples illustrate the  use  of  defmacro.    The
    first is intended to provide a contrast with ds.

\begin{verbatim}
    (defmacro car (s)
      `(car ,s))
\end{verbatim}
\begin{verbatim}
    (defmacro nth (s i)
      (if (onep i)
        `(car ,s)
        `(nth (cdr ,s) ,(sub1 i))))
\end{verbatim}

\de{deflambda}{(deflambda name:id PARAMS:id-list [FN:form]): id}{macro}
{    Defines  the  function named name, of type macro.  Deflambda
    is similar to ds except that the body of the  definition  is
    enclosed   within  a  lambda  expression.    The  number  of
    parameters of the lambda  expression  is  the  same  as  the
    number  specified by PARAMS.  When the macro is applied each
    of the  arguments  are  evaluated  and  then  bound  to  the
    parameters  of  the lambda expression.  Finally, the body of
    the lambda  is  evaluated.    Note  that  each  argument  is
    evaluated  once.    This  may not be true had the macro been
    defined using ds.  The following  example  illustrates  when
    deflambda  should  be used in place of ds.  The expansion of
    the first  version  of  Check  contains  two  occurances  of
    (long-calculation  n),  where  the  expansion  of the second
    version only contains one.
}
\begin{verbatim}
    (ds check (any)
      (when (> any 0) (compute any)))

    (deflambda check (any)
      (when (> any 0) (compute any)))

    (check (long-calculation n))

    (when (> (long-calculation n) 0)    % Expansion of the first
      (compute (long-calculation n)))   % version of check

    ((lambda (any)                      % Expansion of the second
       (when (> any 0)                  % version of check
         (compute any)))
     (long-calculation n))
\end{verbatim}

\subsection{BackQuote}


\de{backquote}{(backquote A:form): form}{macro}
{    Backquote is the function name for ` (accent grave).    With
    some printers it may be difficult to distinguish between the
    quote  and accent grave characters.  For this reason, in the
    examples which  follow  'expression  is  written  as  (quote
    expression).  You must load USEFUL to define backquote.
}
    In  the  previous section the function list was defined as a
    macro.

\begin{verbatim}
    (dm list (a)
      (if (< (length a) 2)
        ()
        (cons (quote cons)
              (cons (second a)
                    (ncons (cons (quote list)(rest (rest a))))
    ))))
\end{verbatim}
    The  body  of this definition is somewhat difficult to read.
    An abbreviated syntax was developed to aid both  the  writer
    and the reader.  The notation, called "backquote" (`), works
    as  an  "anti-quote".    Unquoted forms within its scope are
    assumed to be constants.  To indicate that a form should  be
    evaluated,  one  should  prefix  the  form with a comma (,).
    Consider the redefinition of the macro list.

\begin{verbatim}
    (dm list (a)
      (if (< (length a) 2)
        ()
        `(cons ,(second a)
               ,(cons (quote list) (rest (rest a))))))
\end{verbatim}
    While this is  an  improvement,  the  explicit  construction
    ,(cons  (quote  list)  (rest  (rest  a)))  clutters  up  the
    appearance.  The application of cons is  there  to  indicate
    that  we  want to combine the result of (rest (rest a)) with
    the identifier list.  The form ,(list (rest (rest  a))  will
    not  give this effect, instead it would create a list of two
    elements.  The at-sign (@) is used in conjunction  with  the
    comma  to  mean  "splice-in"  rather  than  "cons-in".  This
    results in the following defintion.

\begin{verbatim}
    (dm list (a)
      (if (< (length a) 2)
        ()
        `(cons ,(second a) (list ,@(rest (rest a))))))
\end{verbatim}

\de{unquote}{(unquote A:any): Undefined}{fexpr}
{    Function name for comma ,.  It is  an  error  to  eval  this
    function; it should occur only inside a backquote.
}

\de{unquotel}{(unquotel A:any): Undefined}{fexpr}
{    Function  name  for comma-atsign ,@.  It is an error to eval
    this function; it should only occur inside a backquote.
}
  The two examples which follow are similar  definitions  of  an
arithmetic  if  form.   There are four arguments: a test form, a
negative form, a zero form, and a positive form.    One  of  the
last  three forms is chosen to be evaluated according to whether
the test form is positive, negative, or zero. The first uses the
traditional dm while the second  uses  defmacro  and  backquote.
Clearly  the second is much simpler.  To clarify printer output,
there are no occurances of the accent  grave  character  in  the
first  definition  and the second definition does not contain an
occurance of the quote character.

\begin{verbatim}
(dm number-if (calling-form)
  (list 'let
        '((var (gensym)))
        (list 'let
              (list (list 'var (nth calling-form 2)))
              (list 'cond
                    (list (list 'minusp 'var)
                          (nth calling-form 3))
                    (list (list 'zerop 'var)
                          (nth calling-form 4))
                    (list t (nth calling-form 5))))))
 
(defmacro number-if (test minus-form zero-form plus-form)
  (let ((var (gensym)))
    `(let ((,var ,test))
       (cond ((minusp ,var) ,minus-form)
             ((zerop ,var) ,zero-form)
             (t ,plus-form)))))
\end{verbatim}

\subsection{MacroExpand}


\de{macroexpand}{(macroexpand A:form [B:id]): form}{macro}
{    Macroexpand  is  a   useful   tool   for   debugging   macro
    definitions.   If  given  one argument, macroexpand  expands
    all the macros in that form.   Often  one  wishes  for  more
    control  over this process.  For example, if a macro expands
    into a let, we may not wish to see the let  itself  expanded
    to  a lambda expression.  Therefore additional arguments may
    be given to macroexpand.  If these are supplied, they should
    be the  names  of  macros,  and  only  those  specified  are
    expanded.
}

\subsubsection{Low Level Function Definition Primitives}

  The  following functions are used especially by putd and Getd,
defined above in Section 9.1.3, and by eval and  apply,  defined
in Chapter 11.


\de{funboundp}{(funboundp U:id): boolean}{expr}
{    Tests  whether there is a definition in the function cell of
    U; returns nil if so, t if not.

    Note:  The functional value of an identifier which does  not
    define  a  function is actually a call on undefinedfunction.
    The  evaluation  of  undefinedfunction  will  result  in   a
    continuable error.
}


\de{fboundp}{(fboundp U:id): boolean}{macro}
{    Equivalent to (not (funboundp U)), the function funboundp is
    described above.
}

\de{flambdalinkp}{(flambdalinkp U:id): boolean}{expr}
{    Tests whether U is an interpreted function;  return t if so,
    nil  if  not.  This  is  done  by  checking  for the special
    code-address of the lambdalink  function,  which  calls  the
    interpreter.
}

\de{fcodep}{(fcodep U:id): boolean}{expr}
{    Tests  whether  U  is a compiled function;  returns t if so,
    nil if not.
}

\de{makefunbound}{(makefunbound U:id): nil}{expr}
{    Makes U an undefined function by planting a special call  to
    the  function, undefinedfunction, in the function cell of U.
    See   funboundp   above   for   more    information    about
    undefinedfunction.
}

\de{makeflambdalink}{(makeflambdalink U:id): nil}{expr}
{    Makes  U  an interpreted function by planting a special call
    to an interpreter support function (lambdalink) function  in
    the function cell of U.
}

\de{makefcode}{(makefcode U:id C:code-pointer): nil}{expr}
{    Makes  U  a  compiled function by planting a special JUMP to
    the code-address associated with C.
}

\de{getfcodepointer}{(getfcodepointer U:id): code-pointer}{expr}
{    Gets the code-pointer for U.
}

\de{code-number-of-arguments}{(code-number-of-arguments C:code-pointer): 
\{nil,integer\}}{expr}
{    Some compiled functions have the argument number they expect
    stored in association with the code-pointer C. This integer,
    or nil is returned.
}
\subsection{Function Type Predicates}

  \de{exprp}{(exprp U:any): boolean}{expr}
{    Test if U is a code-pointer, lambda form, or an id with expr
    definition.
}

\de{fexprp}{(fexprp U:any): boolean}{expr}
{    Test if U is an id with fexpr definition.
}

\de{nexprp}{(nexprp U:any): boolean}{expr}
{    Test if U is an id with nexpr definition.
}

\de{macrop}{(macrop U:any): boolean}{expr}
{    Test if U is an id with macro definition.
}

\section{Wrappers}

  Once  a  function  has  been  defined,  one  may want it to do
something a  little  different,  or  just  a  little  bit  more.
Breaking  and  tracing  functions  for debugging purposes can be
thought of in this way.  Having a function  keep  track  of  the
time  spent in its execution can also be thought of in this way.
The Wrappers module provides a convenient  facility  to  support
this.

  When  a  function is wrapped, its name becomes a reference for
two different function definitions.  In PSL it  is  possible  to
create  distinct  identifiers  which  have  the  same name.  The
original name of the function is associated with a new  function
called  the  wrapper.  In general a wrapper does additional work
before and/or after  applying  the  original  definition.    The
original  definition is associated with an identifier whose name
is identical with the original name but which is  not  interned.
Application  of  getd  to  the  original  name  will  return the
original  definition.  Since  a  wrapper  is  identified  by  an
indicator  on  the  property list of its name getd knows when to
look elsewhere for the original function  definition,  and  putd
knows to alter the original definition, not the wrapper.

  Typically  a  wrapper  may  be  added  to a function and later
removed.  A function potentially may be wrapped up  inside  more
than one wrapper at the same time.  It is possible to redefine a
wrapped function but if the order or number of formal parameters
is  changed  then  it  will  be necessary to unwrap all wrappers
first.

\subsection{Notes on Writing Wrappers}

  This section describes guidelines for writing wrappers.

  If a wrapper is put on a function that is used either directly
or indirectly by the wrapper  body  an  infinite  recursion  may
result.   Functions used by the PSL interpreter are particularly
susceptible to this problem.    We  require  a  means  to  avoid
infinite  recursion.   In general it is not easy to restrict the
set of functions which can be called indirectly by  a  piece  of
code.   Many  PSL  system  functions use other system functions.
Furthermore, a modification  to  the  system  may  change  these
relationships.    Some  system  functions call functions through
functional variables or functional  values  stored  on  property
lists.  This means that some of the relationships between system
functions change over time.

  Instead  of  avoiding recursion introduced by wrappers, we can
detect and recover from  it.   Wrapper  bypasses  do  this.    A
wrapper  bypass  is implemented through a fluid variable, called
the controlling variable for  that  wrapper.    On  entry  to  a
wrapper,  the  value  of  the controlling variable is checked. A
non-nil value implies that a previous invocation of this wrapper
has not been completed.   To  avoid  a  recursive  call  on  the
wrapper,  a  call on the wrapped function replaces evaluation of
the wrapper.

  If the value of the controlling variable is nil the wrapper is
evaluated.  During evaluation of  the  wrapper  the  controlling
variable  is  set  to  t  except  during  the application of the
wrapped function, at which point it should be set to nil.

  Syntactically,  wrapper  bodies  of  the  following  form  are
supported:

\begin{verbatim}
(lambda <args>
  (cond (<var> <wrapped-fn-call>)
        (t (prog (<var> <id>*)
             (setq <var> t)
             <call-expr>*))))
 
<call-expr> ::= <expr>
                | (<call-expr>*)
                | (setq <var> nil) <setq-and-wrapped-fn-call> (setq <var> t)
 
<setq-and-wrapped-fn-call> ::=
               | <wrapped-fn-call>
               | (setq <id> <wrapped-fn-call>)
 
<wrapped-fn-call> ::= (wrapped-function <expr>*)
                  | (funcall 'wrapped-function <expr>*)

<var> ::= <id>
<id> is any PSL identifier <expr> is any PSL s-expression.
\end{verbatim}

The  controlling variable of the wrapper is $<var>$. It must be
a fluid variable whose top level value is nil.

The notation is BNF with the addition of the regular
expression "*" operator. This operator is used to indicate zero
or more repetitions of an item.

  Note  that setf is not supported in place of setq.  You should
not use apply in place of funcall or let in place of prog.   You
can  eliminate  redundant  instances of (setq $<var>$ t) and (setq
$<var>$ nil).  For example, if a \verb+<wrapped-fn-call>+ appears at the
beginning of a prog then the
form \verb+(setq <var> nil)+ can be
omitted.

\subsection{Exported Functions}


\de{wrap}{(wrap name:id wtype:id BODY:list COMPILE?:boolean): id}{expr}
{    Wrap creates a wrapper.  name is the name of the function to
    be wrapped.  wtype is an id  which  identifies  the  wrapper
    type, it must be a member of the list wrapper-standard-order
    (documented below).  BODY is a list which should be a lambda
    expression.
}
    Wrap  redefines  the functional value of name.  The original
    function is renamed.  The print representation of  this  new
    name is identical to name but the identifiers are distinct.

    When  BODY is a lambda expression (you are encouraged to use
    a lambda expression for this argument), then the new name of
    the function which is being wrapped is substituted for  each
    occurance   of   wrapper-function.    The   result  of  this
    substitution becomes the definition of the wrapper.

    When BODY is not a lambda expression a different  method  is
    used  to  create  the  wrapper.    BODY is embedded within a
    lambda expression which has the same number of arguments  as
    the  function which is being wrapped.  Each occurance of the
    form (wrapped-invocation), is replaced by an application  of
    the wrapped function.

    If COMPILE? is t then the wrapper will be compiled.

    The  wrappers  package permits a function to have any number
    of wrappers of any permitted type.   For  a  given  type  of
    wrapper,  a new wrapper always encloses preexisting wrappers
    of that type.  However, you are  advised  against  depending
    upon this ordering.


\de{remove-wrapper}{(remove-wrapper name:id TYPE:id): boolean}{expr}
{    Removes a wrapper of type TYPE from the identifier name.  If
    name  does  not  have a wrapper of this type then the return
    value will be nil, otherwise t is returned.
}

\de{wrapped?}{(wrapped? name:id): boolean}{expr}
{    Returns t if there is at least one wrapper on name.
}

\de{wrapper-of-type?}{(wrapper-of-type? name:id TYPE:id): {name:id nil}}{expr}
{    Searches for a wrapper of the given type on name.   name  is
    returned if a wrapper is found, otherwise nil is returned.
}

\de{wrapper-types}{(wrapper-types name:id): list}{expr}
{    Returns  a  list of all the types of all the wrappers around
    the function named name.
}

\de{function-lambda-list}{(function-lambda-list FN:{id, code-pointer, lambda}): 
list}{expr}
{    Returns a list suitable for use as the formal parameter list
    of a wrapper for  that  function.      If  the  function  is
    interpreted,   the   argument  list  of  the  definition  is
    returned.  If the actual argument  list  is  not  available,
    names for the arguments are invented.  If FN is not a lambda
    expression, code-pointer, or identifier then it is an error.
}
\begin{verbatim}
***** FN is not a function.
\end{verbatim}
    It  is  also  an error if FN is an identifier which does not
    have a functional value.

%\begin{verbatim}
{\tt ***** FN is not defined as a function.}
%\end{verbatim}

\de{function-basic-definition}{(function-basic-definition
name:id): name:id}{expr}
{    Returns the basic definition of name, whether or not  a  set
    or  wrappers is associated with it.  Note that getd and putd
    operate on this basic definition.  If you wish to redefine a
    wrapped  function  and  the  number  or  order   of   formal
    parameters  will  change it is necessary to first unwrap all
    wrappers first.  Applications of the wrapped function within
    the wrapper will not be updated when the wrapped function is
    redefined.
}

\variable{wrapper-standard-order}{[Initially: (trace break meter)]
}{global} 
{
    Only wrapper types which are members of this list are  legal
    arguments  to  primitive functions that operate on wrappers.
    The wrappers of  each  function  are  nested  so  that  each
    wrapper  type appears earlier in wrapper-standard-order than
    the type of any wrapper it encloses.
    Wrap wraps according to this  ordering,  but  changing  this
    list  does not cause the ordering of existing wrappers to be
    changed.
}

\subsection{Examples}

  Assume that foo is a commonly called function  whose  argument
is a large data structure.  You have found that foo is sometimes
called  with  an argument of nil and would like to cause a break
when this situation arises so that you can discover where foo is
being called from.

  Since foo is a commonly used function a break on each entry to
foo is unsatisfactory.  A trace of each entry and  exit  is  not
sufficient  --    it  doesn't  tell  you  who  called foo in the
critical case.  The  solution  is  to  wrap  foo  is  an  advice
wrapper.    Such  a  wrapper  will print a backtrace and enter a
continuable break loop when an argument of nil is discovered.

  First, load the {\bf break-trace} module.  This will result  in  the
{\bf wrappers} module  being loaded and the breaktrace function being
defined (see Chapter 17 for  more  information  on  breaktrace).
The wrapper-type advice is added as the innermost wrapper on the
ordered  list  referenced by wrapper-standard-order (assuming it
is not already on the list).

\begin{verbatim}
  (load break-trace)               % loads wrappers & 
                                     breakpoint function
  (if (not member 'advice wrapper-standard-order) then
    (setf wrapper-standard-order
          (append wrapper-standard-order '(advice))))

  (de debug-foo ()
    (wrap
     'foo                          % function to wrap
     'advice                       % wrapper type
     `(lambda (x)                  % wrapper body generator
        (cond ((null x)
               (backtrace)
               (breakpoint "Foo (x=nil)")))
        (wrapped-function x))
     nil))                          % don't compile 
                                    % the wrapper body
\end{verbatim}
  The  above  will  work fine as long as foo is an expr which is
not used by the interpreter or by the {\bf wrappers} module itself.
In the more general case, the function debug-foo must have three
changes:


\begin{enumerate}
\item  If   foo   is   not   an  expr,  then  all  occurrences  of
     (wrapped-function $<args>$)  must  be  replaced  by  (funcall
     'wrapped-function $<args>$).
\item  If  foo  is  used  by  the PSL interpreter, by the {\tt wrappers}
     module, or by expression in the wrapper body then  it  will
     be  necessary  to  know  when  foo  is  called  during  the
     evaluation of foo's wrapper.  This will allow you to bypass
     evaluation of the wrapper, you can simply apply the wrapped
     function.  Otherwise an infinite  recursion  could  result.
     The  definition  below  uses  the  fluid variable (called a
     controlling  variable)  in-foo-wrapper?  to  achieve   this
     effect.
\item  If  foo  is  possibly used within the interpreter, then the
     wrapper  must  be  compiled.  This  means  that   if   (not
     (interpretive-wrapper-ok?  'foo)), then the fourth argument
     to the function wrap must be t,  or  wrap  will  signal  an
     error.
\end{enumerate}

  The revised definition of debug-foo follows.
\begin{verbatim}

(fluid  '(in-foo-wrapper?))     % necessary initializations of 
  (setq in-foo-wrapper? nil)      % the controlling  variable
(change 2)
\end{verbatim}
\begin{verbatim}
  (de debug-foo ()
    (wrap
     'foo
     'advice
     `(lambda (x)
        (cond (in-foo-wrapper?                  % change 2
               (funcall 'wrapped-function x))   % change 1
              (t
               (prog (in-foo-wrapper?)
                 (setq in-foo-wrapper? t)
                 (cond ((null x)
                        (backtrace)
                        (breakpoint "Foo (x=nil)")))
                 (setq in-foo-wrapper? nil)
                 (return
                  (funcall 'wrapped-function x) % change 1
                 ))))
     (not (interpretive-wrapper-ok? 'foo))))    % change 3
\end{verbatim}
  The  function  break-on-condition  will  cause  entry  into  a
continuable break loop just before the function fn is applied if
the result of evaluating the form bound to bool-expr is non-nil.
A sample call might be (break-on-condition 'foo '(eq x 2)).

\begin{verbatim}
(setf in-wrapper? nil)
(fluid '(in-wrapper?))

(de break-on-condition (fn bool-expr)
    % First, get the parameter-list of the function.
  (let ((parameter-list (function-lambda-list fn)))
    (wrap
     fn
     'advice
     `(lambda ,parameter-list
        (cond (in-wrapper?   
                (funcall 'wrapped-function ,@parameter-list))
              (t
               (prog (in-wrapper?)
                 (setq in-wrapper? t)
                 (cond (,bool-expr
                        (breakpoint "In %p, condition %p=%p"
                          ',fn  ',bool-expr ,bool-expr)))
                        (setq in-wrapper? nil)
                        (return
                         (funcall 'wrapped-function
                                  ,@parameter-list))
                 ))))
     (not (interpretive-wrapper-ok? fn))
     )
   ))
\end{verbatim}
\section{Variables and Bindings}

  Variables in PSL are ids, and associated  values  are  usually
stored  in  and  retrieved  from  the value cell of this id.  If
variables appear as  parameters  in  lambda  expressions  or  in
prog's,  the  contents  of the value cell are saved on a binding
stack.  A new value or nil is stored in the value cell  and  the
computation  proceeds.   On exit from the lambda or prog the old
value is restored.  This is called the "shallow  binding"  model
of  LISP.    It  is chosen to permit compiled code to do binding
efficiently.   For  even  more  efficiency,  compiled  code  may
eliminate the variable names and simply keep values in registers
or a stack.  The scope of a variable is the range over which the
variable has a defined value.  There are three different binding
mechanisms in PSL.\\

\begin{tabular}{lp{11.0cm}}
Local Binding & Only  compiled  functions  bind variables locally.
              Local variables  occur  as  formal  parameters  in
              lambda  expressions  and  as  prog form variables.
              The binding  occurs  as  a  lambda  expression  is
              evaluated  or  as  a  prog  form is executed.  The
              scope of a local  variable  is  the  body  of  the
              function in which it is defined.\\

Fluid Binding & Fluid  variables are global in scope but may occur
              as formal parameters or prog form variables.    In
              interpreted  functions,  all formal parameters and
              local  variables  are  considered  to  have  fluid
              binding   until   changed   to  local  binding  by
              compilation.  A variable can be treated as a fluid
              only by declaration.  If fluid variables are  used
              as parameters or locals they are rebound in such a
              way  that  the  previous  binding may be restored.
              All references  to  fluid  variables  are  to  the
              currently active binding.  Access to the values is
              by name, going to the value cell.\\

Global Binding & In  theory  global variables may never be used as
              parameters  in  lambda  expressions  or  as   prog
              variables.   This  restriction  is not enforced in
              PSL.    You  are  encouraged  not  to  declare  an
              identifier  global if it is used as a parameter in
              a lambda expression or prog.  For more information
              see the Section  Fluid  and  Global  Declarations,
              Chapter 19.\\
%\end{TABELLE***)
\end{tabular}
\subsection{Binding Type Declaration}


\de{fluid}{(fluid IDLIST:id-list): nil}{expr}
{    The   ids in IDLIST are declared as fluid type variables. An
    id which has not been previously declared is initialized  to
    nil.    Variables  in  IDLIST  already  declared  fluid  are
    ignored.  It is an error to attempt to change the type of  a
    variable from global to fluid.
}

%\begin{verbatim}
    {\tt***** ID cannot be changed to FLUID}
%\end{verbatim}

\de{global}{(global IDLIST:id-list): nil}{expr}
{    The ids of IDLIST are declared global type variables.  If an
    id  has  not  been previously declared, it is initialized to
    nil.  Variables which have already been declared global  are
    ignored.   Attempting to change the type of an id from fluid
    to global will result in an error.\\

%\begin{verbatim}
    {\tt***** ID cannot be changed to GLOBAL}
%\end{verbatim}
}

\de{unfluid}{(unfluid IDLIST:id-list): nil}{expr}
{    The variables in IDLIST which have been  declared  as  fluid
    are  no  longer  considered  fluid.    Other  variables  are
    ignored.
}
\subsection{Binding Type Predicates}

\de{fluidp}{(fluidp U:id): boolean}{expr}
{    If U has been declared fluid then t is  returned,  otherwise
    nil is returned.
}

\de{globalp}{(globalp U:id): boolean}{expr}
{    If  U has been declared global then t is returned, otherwise
    nil is returned.
}

\de{unboundp}{(unboundp U:id): boolean}{expr}
{    If U does not have a value then t is returned, otherwise nil
    is returned.
}
\section{User Binding Functions}

  The following functions  are  available  to  build  one's  own
interpreter  functions  that  use  the  built-in  Fluid  binding
mechanism, and interact well with the automatic  unbinding  that
takes place during  Throw and Error calls.


\de{unbindn}{(unbindn N:integer): Undefined}{expr}
{    Used  in  user-defined  interpreter functions (like prog) to
    restore previous bindings to the last N values bound.
}

\de{lbind1}{(lbind1 IDname:id VALUETOBIND:any): Undefined}{expr}
{   	Support for lambda-like  binding.    The  current  value  of
    IDname   is  saved  on  the  binding  stack;  the  value  of
    VALUETOBIND is then bound to IDname.
}

\de{pbind1}{(pbind1 IDname:id): Undefined}{expr}
{   	Support for prog.  Binds nil to IDname after saving value on
    the binding stack.  Essentially (lbind1 IDname nil).
}
