\documentstyle[11pt,reduce]{article}
\title{{\tt ghyper}, a package for simplification of \\
generalized hypergeometric functions}
\date{}
\author{Victor S. Adamchik\\
        Wolfram Research Inc. \\
        former address : \\
        Byelorussian University, Minsk, Byelorussia\\
\\
\\
        Present \REDUCE{} form by \\
        Winfried Neun \\
        ZIB Berlin \\
        Email: {\tt Neun@sc.ZIB-Berlin.de}}
\begin{document}
\maketitle

This note describes the {\tt ghyper} package of \REDUCE{}, which is able
to do simplification of several cases of generalized hypergeometric functions.
The simplifications are performed towards polynomials, elementary or
special functions or simpler hypergeometric functions.
Therefore this package should be used together with the \REDUCE{}
special function package.

\section{Introduction}

The (generalized) hypergeometric functions

\begin{displaymath}
_pF_q \left( {{a_1, \ldots , a_p} \atop {b_1, \ldots ,b_q}} \Bigg\vert z \right)
\end{displaymath}

are defined in textbooks on special functions, e.g. in
\cite{Prudnikov:90}. Many well-known functions belong to this class,
e.g. exponentials, logarithms, trigonometric functions and Bessel functions.
In \cite{Graham:89} an introduction into the analysis of sums, basic
identities and applications can be found.

Several hundreds of particular values can be found in \cite{Prudnikov:90}.

\section{\REDUCE{} operator {\tt hypergeometric}}

The operator {\tt hypergeometric} expects 3 arguments, namely the
list of upper parameters (which may be empty), the list of lower
parameters (which may be empty too), and the argument, e.g:

\begin{verbatim}

hypergeometric ({},{},z);

 Z
E

hypergeometric ({1/2,1},{3/2},-x^2);

 ATAN(X)
---------
    X
\end{verbatim}

\section{Enlarging the {\tt hypergeometric} operator}

Since hundreds of particular cases for the generalized hypergeometric
functions can be found in the literature, one cannot expect that all
cases are known to the {\tt hypergeometric} operator.
Nevertheless the set of special cases can be augmented by adding
rules to the \REDUCE{} system, e.g.

\begin{verbatim}
let {hypergeometric({1/2,1/2},{3/2},-(~x)^2) => asinh(x)/x};
\end{verbatim}

\begin{thebibliography}{9}

\bibitem{Prudnikov:90} A.~P.~Prudnikov, Yu.~A.~Brychkov, O.~I.~Marichev,
{\em Integrals and Series, Volume 3: More special functions},
Gordon and Breach Science Publishers (1990).

\bibitem{Graham:89} R.~L.~Graham, D.~E.~Knuth, O.~Patashnik,
{\em Concrete Mathematics}, Addison-Wesley Publishing Company (1989).

\end{thebibliography}
\end{document}
