%%%
%%% The tree structure for the information browser is
%%%  given by the \section, \subsection, \subsubsection commands
%%%

\section{System interaction}

%%%
%%% Environments like Switch serve a triple purpose:
%%%   - the node is defined
%%%   - an index entry "demo switch" is generated
%%%   - a cross reference with symbolic name "switch:demo" is generated.
%%%     This can be used used the \ref, \pageref, \nameref,
%%%      and \see commands.
%%%   Additional \index entries or cross reference keys can be generated
%%%    with the \index and \label commands.
%%%

\begin{Switch}{demo}
  The \name{demo} switch is used for interactive files, causing
  the system to pause after each command in the file until you type a
  \key{Return}. Default is \name{off}.

%%%
%%% The parts of a node are given as environments. Defined are:
%%%  Comments, Examples, Related
%%%
  \begin{Comments}
    The switch \name{demo} has no effect on top level interactive
    statements. Use it when you want to slow down operations in a file
    so you can see what is happening.
    
    You can either include the \name{on demo} command in the file,
    or enter it from the top level before bringing in any file. Unlike
    the \name{pause} command, \name{on demo} does not permit you to
    interrupt the file for questions of your own.
  \end{Comments}
%%%
%%% The Related environment points to related information. It should
%%%  also use a cross ref, but that is not yet implemented
%%%
  \begin{Related}
    \item [\name{in} command] Reading from files.
    \item [\name{echo} switch] Seeing what is read in.
  \end{Related}
\end{Switch}

\section{Polynomials}

\subsection{Polynomial operators}

\begin{Operator}{den}
  The {den} operator returns the denominator of its argument.
%%%
%%% The syntax description needs perhaps a bit more work.
%%%  \name and \arg are purely for printing (i.e. selecting a different
%%%   typeface
%%%
  \begin{Syntax}
    \name{den}\(\arg{expression}\)
  \end{Syntax}
  \arg{expression} is ordinarily a rational expression, but may be
  any valid scalar \REDUCE\ expression.
  \begin{Examples}
    a := x**3 + 3*x**2 + 12*x; & A := X*(X^2 + 3*X + 12) \\
    b := 4*x*y + x*sin(x);     & B := X*(SIN(X) + 4*Y) \\
    den(a/b);                  & SIN(X) + 4*Y \\
    den(a/4 + b/5);            & 20 \\
    den(100/6);                & 3 \\
    den(sin(x));               & 1 \\
    for i := 1:3 sum part(firstlis,i)*part(secondlis,i); &
              A*X + B*Y + C*Z
  \end{Examples}
\end{Operator}
\begin{Comments}
  \name{den} returns the denominator of the expression after it has
  been simplified by \REDUCE. As seen in the examples, this includes
  putting sums of rational expressions over a common denominator, and
  reducing common factors where possible. If the expression does not
  have any other denominator, $1$ is returned.
  
  Switch settings, such as \name{mcd} or \name{rational}, have an
  effect on the denominator of an expression.
\end{Comments}

\subsection{Dependency information}

\begin{Declaration}{depend}
  \name{depend} declares that its first argument depends on the rest
  of its arguments.
  \begin{Syntax}
    \name{depend} \arg{kernel}\{,\arg{kernel}\}\repeated
  \end{Syntax}
  \arg{kernel} must be a legal variable name or a prefix operator
  \see{kernel}).
  \begin{Examples}
    depend y,x;                              \\
    df(y**2,x);        & 2*DF(Y,X)*Y         \\
    depend z,cos(x),y;                       \\
    df(sin(z),cos(x)); & COS(Z)*DF(Z,COS(X)) \\
    df(z**2,x);        & 2*DF(Z,X)*Z         \\
    nodepend z,y;                            \\
    df(z**2,x);        & 2*DF(Z,X)*Z         \\
    cc := df(y**2,x);  & CC := 2*DF(Y,x)*Y   \\
    y := tan x;        & Y := TAN(X)         \\
    cc;                & 2*TAN(X)*(TAN(X)^{2} + 1)
  \end{Examples}
  \begin{Comments}
    Dependencies can be removed by using the declaration
    \nameref{nodepend}. The differentiation opeartor uses this
    information, as shown in the examples above. Linear operators alos
    use knowledge of dependencies (see \nameref{linear}). Note that
    dependencies can be nested: Having declared $y$ to depend on $x$,
    and $z$ to depend on $y$, we see that the chain rule was applied
    to the derivative of a function of $z$ with respect to $x$. If the
    explicit function of the dependencyis later entered into the
    system, terms with \name{DF(Y,X)}, for example, are expanded when
    they are displayed again, as shown in the last example.
  \end{Comments}
\end{Declaration}

\section{The Taylor package}

\begin{Operator}{taylor}
  The \name{taylor} operator is used for expansion in power
  series\index{series}.
  \begin{Syntax}
    \name{taylor}\(\arg{expression},%
                 \{\arg{kernel},\arg{expression},\arg{integer}\}%
                    \repeated\)
  \end{Syntax}
  This returns the expansion of the first argument with respect to
  \arg{kernel} about \arg{expression} to order \arg{integer}.
  \begin{Examples}
    taylor(e^(x^2+y^2),x,0,2,y,0,2); &
               1 + Y^2 + X^2 + Y^2*X^2 + O(X^{3},Y^{3})\\
    taylor(log(1+x),x,0,2);          & X - \rfrac{1}{2}*X^{2} + O(X^{3})
  \end{Examples}
  \begin{Comments}
    The expansion is performed variable per variable, i.e.\ in the
    example above by first expanding $\exp(x^{2}+y^{2})$ with respect
    to $x$ and then expanding every coefficient with respect to $y$.

    If the switch \nameref{taylorkeeporiginal} is set to \name{on} the
    original expression is kept for later reference.

    Printing is controlled by the variable \nameref{taylorprintterms}.

  \end{Comments}
  \begin{Related}
    \item[tps]    Truncated Power Series.
    \item[Koepf] Complete power series
  \end{Related}
\end{Operator}

\subsection{Controlling the package}

\begin{Switch}{taylorkeeporiginal}
  The \name{taylorkeeporiginal} switch determines whether the
  \nameref{taylor} operator keeps the expression to be expanded for
  later use. Default is \name{on}.
\end{Switch}

\begin{Operator}{taylororiginal}
  \name{taylororiginal} extracts the original expression from a Taylor
  kernel.
  \begin{Syntax}
    \name{taylororiginal}\(\arg{taylor\_kernel}\)
  \end{Syntax}
  If the argument is not a Taylor kernel, or if the expression was not
  kept, an error is signaled.
\end{Operator}

\begin{Variable}{taylorprintterms}
 
    Only a certain number of (non-zero) coefficients of a Taylor
    kernel are printed usually. If there are more, \verb|...| is
    printed as part of the expression to indicate this. The number of
    terms printed is given by the value of the shared algebraic
    variable \nameref{taylorprintterms}.  Allowed values are integers
    and the special identifier \name{all}. The latter setting
    specifies that all terms are to be printed. The default setting is
    $5$.
 
\end{Variable}

