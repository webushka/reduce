% This is a LaTeX file
\documentstyle[12pt]{article}

%Sets size of page and margins
\oddsidemargin 10mm  \evensidemargin 10mm
\topmargin 0pt   \headheight 0pt   \headsep 0pt
\footheight 14pt  \footskip 40pt
\textheight 23cm  \textwidth 15cm
%\textheight 15cm  \textwidth 10cm
 
%spaces lines at one and a half spacing
%\def\baselinestretch{1.5}

%\parskip = \baselineskip

%Defines capital R for the reals, ...
%\font\Edth=msym10
%\def\Integer{\hbox{\Edth Z}}
%\def\Rational{\hbox{\Edth Q}}
%\def\Real{\hbox{\Edth R}}
%\def\Complex{\hbox{\Edth C}}

\title{The Computer Algebra Package {\tt CRACK} for Investigating PDEs}
\author{Thomas Wolf \\                        
        School of Mathematical Sciences \\
        Queen Mary and Westfield College \\
        University of London \\
        London E1 4NS \\
        T.Wolf@maths.qmw.ac.uk
\\ \\
Andreas Brand \\ Institut f\"{u}r 
Informatik \\ Friedrich Schiller Universit\"{a}t Jena \\ 07740 Jena
\\ Germany \\ maa@hpux.rz.uni-jena.de
}

\begin{document}
\maketitle
\tableofcontents                                  
\section{The purpose of {\tt CRACK}}
The package {\tt CRACK} attempts the solution of an overdetermined 
system of ordinary or partial differential
equations (ODEs/PDEs) with at most polynomial nonlinearities. 
Under `normal circumstances' the number of DEs which describe physical
processes matches the number of unknown functions which are involved.
Moreover none of those equations can be solved or integrated and
integrability conditions yield only identities.  Though the package
{\tt CRACK} applied to solve such `difficult' systems
directly will not be of much  help usually, it is possible to
investigate them indirectly by 
studying analytic properties which would be useful for their
solution. Such `simpler' overdetermined PDEs also result from
investigating properties of ODEs and problems in differential geometry.
The property of overdetermination (by which we only mean the fact
that there are more equations than unknown functions) is not a
prerequisite for applying the package but it simplifies the problem
and it is usually necessary for success in solving or simplifying the
system.

The following examples are quite different in nature. They demonstrate
typical applications of {\tt CRACK} and give an impression of its efficiency.
In principle it makes no difference to investigate more general
symmetries, more general coordinate transformations and so on, except that the
computational complexity may grow very rapidly for more general
ans\"atze, especially for nonlinear problems, such that a solution
becomes practically impossible.  

The following overview makes suggestions for possible
applications; the mathematics of the applications and of the
algorithms which are applied in {\tt CRACK} are described only
superficially. These applications are only examples. The user will usually
have to write own procedures to formulate his problem in form of
an overdetermined system of PDEs which is then solved or simplified with
{\tt CRACK}.

\begin{enumerate}
 \item
  DEs as well as differential geometric objects, like a metric which
  describes infinitesimal distances in a manifold, can have
  infinitesimal symmetries, i.e.\ there exist transformations of dependent
  and independent variables, which leave the DE or metric 
  form-invariant. According to a theory of Sophus Lie such
  symmetries of an ODE can be used to integrate the
  ODE and symmetries of a PDE can be used to decrease the number
  of independent variables.
 
    The determining conditions for the generators of this
  symmetry are given by a system of linear PDEs.

\item
  Other ans\"atze, e.g.\ for finding 
\begin{itemize} 
\item integrating factors of DEs, or
\item a variational principle, i.e.\ a Lagrangian which is equivalent
    to a given ODE, or
\item a factorization ansatz which transforms a given $n$-th order 
    ODE into two ODEs of order $k$ and $n-k,$
\end{itemize}
  lead to PDE-systems for the remaining free functions with good
  prospects for solution.
\item
  The ability of {\tt CRACK} to decouple systems of DEs with at most polynomial
  nonlinearity can be used to perform very general transformations of
  undetermined functions. If, for example, an equation
\begin{equation}
        0 = D_1(x, f)             \label{df}                                   
\end{equation}
  for a function $f(x)$ is given with $D_1$ as a differential
expression in $x$ and $f$, and a further function $g$ of $x$ 
is related to $f$ through a differential relation
\begin{equation}
        0 = D_2(x, g, f),          \label{dg}
\end{equation}
then, to obtain an equation equivalent to (\ref{df}) for the function $g,$
the system (\ref{df},\ref{dg}) could be decoupled w.r.t.\ $f$ to
obtain an equation 
\[        0 = D_3(x, g)     \]
which contains only $g$. 

    Such generalized transformations can in principle also be done for more
than one equation of the form (\ref{df}), more old and new functions and 
variables, and more relations of the form (\ref{dg}).

\item
  If a PDE or `well defined' system is too complicated to be solved in
  general, and one has some idea of the dependence of one of the
  functions which are to be calculated (e.g. $f$) on at least one
  variable (e.g. $x$), then one could try an ansatz for $f$ which
  involves $x$ (and possibly other variables) 
  only explicitly and furthermore only unknown functions 
  that are independent of $x$. Also other
  ans\"atze are possible where no variable occurs explicitly but all new
  functions involve only a subset of all variables. This approach
  includes separations such as the well known product ansatz $f(r,
  \theta, \varphi) = R(r)Y(\theta, \varphi)$ to solve the Laplace
  equation $\triangle f = 0$ in spherical coordinates. After such a
  substitution the resulting system is simplified and can be solved
  usually, because the
  number of functions of all independent variables is decreased and is
  now less than the number of equations.
   
\item
  A difficult DE-system is simplified usually if further conditions
  in the form of differential equations are added,
  because then the resulting system is not necessarily in involution
  and integrability conditions can be used to lower the order or to
  solve it.
\end{enumerate}

To explain the input to {\tt CRACK}, which corresponds to examples
given in the final two sections, the way to call {\tt CRACK} is described next.

\section{How to apply {\tt CRACK}}  
\subsection{The call}
{\tt CRACK} is written in the symbolic mode of REDUCE 3.4 and is loaded by
{\tt load CRACK;}. Then {\tt CRACK} is called by
\begin{tabbing}
  {\tt CRACK}(\=\{{\it equ}$_1$, {\it equ}$_2$, \ldots , {\it equ}$_m$\},  \\
              \>\{{\it ineq}$_1$, {\it ineq}$_2$, \ldots , {\it ineq}$_n$\}, \\
              \>\{{\it fun}$_1$, {\it fun}$_2$, \ldots , {\it fun}$_p$\},  \\
              \>\{{\it var}$_1$, {\it var}$_2$, \ldots , {\it var}$_q$\});
\end{tabbing}     
        $m,n,p,q$ are arbitrary.
\begin{itemize}
\item
The {\it equ}$_i$ are identically vanishing partial differential expressions, 
i.e.\
they represent equations  $0 = {\it equ}_i$, which are to be solved for the 
functions ${\it fun}_j$ as far as possible, thereby drawing only necessary
conclusions and not restricting the general solution.
\item
The {\it ineq}$_i$ are expressions which must not vanish identically for
any solution to be determined, i.e. only such solutions are computed for which
none of the {\it ineq}$_i$ vanishes identically in all independent variables.
\item
The dependence of the (scalar) functions ${\it fun}_j$ on possibly a number of 
variables is assumed to have been defined with DEPEND rather than
declaring these functions 
as operators. Their arguments may  themselves only be independent variables
and not expressions.
\item
The functions ${\it fun}_j$ and their derivatives may only occur polynomially.
Other unknown functions in ${\it equ}_i$ may be represented as operators.
\item
The ${\it var}_k$ are further independent variables, which are not
already arguments  
of any of the ${\it fun}_j$. If there are none then the third argument is 
the empty list \{\}.
\item 
The dependence of the ${\it equ}_i$ on the independent variables and on
constants and functions other than ${\it fun}_j$ is arbitrary.
\end{itemize}

\subsection{The result}  
The result is a list of solutions 
\[      \{{\it sol}_1, \ldots \}  \]
where each solution is a list of 3 lists:
\begin{tabbing}
       \{\=\{${\it con}_1, \; {\it con}_2, \ldots , \; {\it con}_q$\}, \\
         \>\{${\it fun}_a={\it ex}_a, \;\;
{\it fun}_b={\it ex}_b, \ldots , \;\; {\it fun}_p={\it ex}_p$\},\=  \\
         \>\{${\it fun}_c, \;\; {\it fun}_d, \ldots , \;\; {\it fun}_r$\} \>\}
\end{tabbing}      
with integer $a, b, c, d, p, q, r.$
If {\tt CRACK} finds a contradiction as e.g. $0=1$ then there exists no
solution and it returns the empty list \{\}.
The empty list is also returned if no solution exists 
which does not violate the inequalities
{\it ineq}$_i \neq 0.$ 
For example, in the case of a linear system as input, there is
at most one solution ${\it sol}_1$.

The expressions ${\it con}_i$ (if there are any), are the
remaining necessary and sufficient conditions for the functions
${\it fun}_c,\ldots,{\it fun}_r$ in the third list.  Those
functions can be original functions from the equations to be
solved (of the second argument of the call of {\tt CRACK}) or new
functions or constants which arose from integrations. 
The dependence of new functions on variables is declared with {\tt DEPEND}
and to visualize this dependence the algebraic mode function
${\tt FARGS({\it fun}_i)}$ can be used.
If there are no ${\it con}_i$ then all equations are solved and the
functions in the third list are unconstrained.  

The second list contains
equations ${\it fun}_i={\it ex}_i$ where each ${\it fun}_i$ is an
original function and ${\it ex}_i$ is the computed expression
for ${\it fun}_i$.

\subsection{Flags}
Possible flags which can be changed in symbolic mode
after {\tt CRACK} has been loaded are (initial values in brackets):
\begin{description}    
\item[{\tt cont\_ :}] if t then if the maximal number of terms of an expression
             is greater then {\tt fcteval\_} or {\tt odesolve\_} then
             the user is asked 
             whether or not the expression is to be substituted or integrated
             with {\tt ODESOLVE} respectively.  (default: {\tt nil}) 
\item[{\tt decouple\_ :}] maximal number of decoupling attempts for a 
             function (default: 2) 
\item[{\tt factorize\_ :}] If an equation can be factored with more than one 
             factor
             containing unknown functions then independent investigations
             start. In each investigation exactly one of these factors is set 
             to zero. If 
             many equations can be factorized then this may lead to a large 
             number of individual investigations. {\tt factorize\_} is the
             maximal number 
             of factorizations. If the starting system is linear then
             an attempt at factorization would be unsuccessful, i.e.
             {\tt factorize\_:=0} prevents this unnecessary 
             attempt. (default: 4) 
\item[{\tt fcteval\_ :}] if a function has been computed from an equation and 
             is to
             be substituted in other equations then {\tt fcteval\_} is the 
             upper limit for the number of terms of the equated expression
             for which substitution is done. (default: 100) 
\item[{\tt fname\_ :}] the name to be used for new constants and functions 
             which result from integrations  (default: {\tt c})  
\item[{\tt genint\_ :}] generalized integration disabled/enabled 
             (default: {\tt nil}) 
\item[{\tt logoprint\_ :}] If t then printing of a standard message whenever 
             {\tt CRACK} is called (default: {\tt t}) 
\item[{\tt independence\_ :}] If t then the user will be asked during
             separation whether expressions are considered to
             be linear independent or not. This should be set true if
             in a previous run the assumption of linear independence made by
             {\tt CRACK} automatically was false. (default: {\tt nil})  
\item[{\tt odesolve\_ :}] the maximal number of terms of expressions which
             are to be 
             investigated with {\tt ODESOLVE}.  (default: $100$) 
\item[{\tt print\_ :}] If nil then there is no output during calculation
             otherwise {\tt print\_} is the maximal number of terms
             of equations which are to be output. (default: 8)  
\item[{\tt sp\_cases :}] After a factorization factors are dropped
             if they do not contain functions which are to be solved. An
             exceptional case is if a factor contains other (parametric)
             functions or constants. Then a corresponding message is given
             at the first appearance of each factor. These factors are
             stored in a list {\tt special\_cases}. If {\tt sp\_cases}
             is not {\tt nil} then such factors are not dropped. 
             (default: {\tt nil})
\item[{\tt time\_ :}] If t then printing the time necessary for the last
             {\tt CRACK} run (default: {\tt t}) 
\item[{\tt tr\_gensep :}] to trace of the generalized separation 
             (default: nil)
\end{description}

\subsection{Help}
Parts of this text are provided after typing 

{\tt crackhp();}

The authors are grateful for critical comments 
on the interface, efficiency and computational errors.

\section{Contents of the {\tt CRACK} package}
The package {\tt CRACK} contains modules for decoupling PDEs, integrating
exact PDEs, separating PDEs, solving DEs containing functions of only
a subset of all variables and solving standard ODEs (of Bernoulli or
Euler type, linear, homogeneous and separable ODEs). These facilities
will be described briefly together with examples.

\subsection{Decoupling}
The decoupling module differentiates equations and combines them algebraically
to obtain if possible decoupled and simplified equations of lower order.
How this could work is demonstrated in the following example.
The integrability condition for the system
\[ \begin{array}{cccl}
f = f(x,y), \; \; & f,_{x} & = & 1   \\
                  & f,_{y} & = & (f-x-1/y)x - 1/y^2
\end{array}  \]
provides an algebraic condition for the function $f$
which turns out not only to be necessary but also sufficient to solve both
equations:
\begin{eqnarray*}
 0 = f,_{xy} - f,_{yx} & = & - xf,_x - f + 2x + 1/y \\
                       & = & - f + x + 1/y \; \; \; \; \; \;
 \mbox{(with $f,_x$ from above)}
\end{eqnarray*}
\[ \rightarrow f = x + 1/y. \]
A general algorithm to bring a system of PDEs into a standard form
where all integrability conditions are satisfied by applying
a finite number of additions, multiplications and differentiations
is based on the general theory of involutive systems \cite{Riq,Th,Ja}.
Essential to this theory is a total ordering of partial derivatives
which allows assignment to each PDE of a {\em Leading Derivative} 
(LD) according to a chosen ordering of functions
and derivatives. Examples for possible orderings are 
\begin{itemize}
\item lex.\ order of functions $>$ lex.\ order of variables 
\item lex.\ order of functions $>$ total differential order $>$ lex.\ 
      order of variables 
\item total order $>$ lex.\ order of functions $>$ lex.\ order of variables
\end{itemize}
or mixtures of them by giving weights to individual functions and variables.
Above, the `$>$' indicate ``before'' in priority of criteria. The principle
is then to
\begin{itemize}
\item take two equations at a time and differentiate them as often as 
necessary to get equal LDs,
\item regard these two equations as algebraic equations in
the common LD and calculate the remainder w.r.t.\ the LD, i.e.\ to
generate an equation without the LD by the Euclidean algorithm, and
\item add this equation to the system.
\end{itemize}
Usually pairs of equations are taken first, such that only one must be
differentiated. If in such a generation step one of both equations is not
differentiated then it is called a
simplification step and this equation will be replaced by the new equation.
The algorithm ends if each combination of two equations yields only equations
which simplify to an identity modulo the other equations.
A more detailed description is given e.g. in \cite{Alex,Reid1}.

In {\tt CRACK}, a reduced version of this algorithm has so far been
implemented, which applies the first of the above orderings with 
lexicographical ordering of functions having the highest
priority. This is done to get decoupled equations, i.e.\ a system with
a ``triangular dependence'' of the equations on the functions, 
which is usually easier to solve
successively (starting with the equation containing the fewest
functions) than are coupled DEs. To save memory not all equations
are stored but new equations replace in general older ones.
Details of the algorithm used
in {\tt CRACK} are given in \cite{Wo}.
Programs implementing the standard algorithm are described e.g. in
\cite{FS,Alex,Fush} and \cite{Reid1}.

The possible variations of orderings or even the switch between
them open possibilities for future optimization.

{\em Example:}
Applying the decoupling module alone without factorization and integration 
to the two equations 
\[ \begin{array}{rclcl}
 D_1 & := & f + f,_{yy}f,_x & = & 0 \\ \\
 D_2 & := & f,_y + f,_x^2   & = & 0
\end{array} \]
for $f(x,y)$, using the lexicographic ordering of variables $y>x>1,$
the steps would be 
\begin{tabbing}
$D_1:=$\=$D_1-D_{2y}f_x=f-2f_x^2f_{yx}$ \\
     \> $\rightarrow$  a second 1st order eqn. in $y$\\ \\
$D_1:=$\>$D_1+2f_x^2D_{2x}=f+4f_x^3f_{xx} \rightarrow$  first ODE \\
     \> to get a second ODE we need an extra equation $D_3:$ \\ \\
$D_3:=$\>$4f_x^3D_{2xx}-D_{1y}=8f_{xx}^2f_x^3+8f_{xxx}f_x^4-12f_x^2f_{xx}f_{xy}
-f_y$ \\ \\
$D_3:=$\>$D_3+12f_x^2f_{xx}D_{2x}=32f_{xx}^2f_x^3+8f_{xxx}f_x^4-f_y$
\\ \\
$D_2:=$\>$D_2+D_3=32f_{xx}^2f_x^3+8f_{xxx}f_x^4+f_x^2 \rightarrow$ 
second ODE \\ \\
$D_2:=$\>$2f_xD_{1x}-D_2=-8f_{xx}^2f_x^3+f_x^2$ \\ \\
$D_2:=$\>$D_2+2f_{xx}D_1=2ff_{xx}+f_x^2$ \\ \\
$D_2:=$\>$D_1f-2f_x^3D_2=f^2-2f_x^5$ \\ \\
$D_1:=$\>$2D_{2x}+5f_xD_1=9ff_x$ \\ \\
$D_2:=$\>$2f_x^4D_1+9fD_2=9f^3 \rightarrow f=0$ is necessary and \\ 
     \> as a test shows also sufficient, \\ \\
$D_1:=$\>$D_{2x}-3fD_2=0 \rightarrow$ end
\end{tabbing} 
Here $D_1:=\ldots$ means that the old expression for $D_1$ is replaced
by the result of the calculation of the right side. As the indices
show the calculation can be done by storing only 3 equations at a time,
which is the purpose of the chosen total ordering of derivatives. If
we have $n$ independent variables and $k$ equations at the beginning,
then the calculation can be done by storing not more than $k+n-1$
equations at a time (plus the original $k$ equations to test results for
sufficiency at the end).
In {\tt CRACK} all intermediate equations generated are checked to see
whether they can be integrated at least once directly by an integration
module.

\subsection{Integrating exact PDEs}
The technical term `exact' is adapted for PDEs from exterior calculus and
is a small abuse of language but it is useful to characterize the kind of PDEs
under consideration.

The purpose of the integration module in {\tt CRACK} is to  decide
whether a given differential
expression $D$ which involves unknown functions $f^i(x^j),\;\; 1\leq i\leq m$ 
of independent variables $x^j, 1\leq j\leq n$
is a total derivative of another expression $I$
w.r.t. any variable $x^k, 1\leq k\leq n$ 
\[ D(x^i,\; f^j,\; f^j,_p,\; f^j,_{pq}, \ldots) 
     = \frac{d I(x^i,\; f^j,\; f^j,_p,\; f^j,_{pq}, \ldots)}{d x^k} \]
and to invert the total derivative i.e. to find $I.$ The index $k$ is
reserved in the following for the integration variable $x^k.$ Because
the appropriate constant or function of integration which depends on
all variables except $x^k$ is added to $I,$ a replacement of $0 = D$
by $0 = I$ in a system of equations is no loss of generality.

Of course there
always exists a function $I$ with a total derivative equal to $D$ but
the question is whether for \underline{arbitrary} $f^i$ the integral
$I$ is functionally dependent only on the $f^i$ and their derivatives,
and not on integrals of $f^i.$ \\
\underline{Preconditions:} \\
$D$ is a polynomial in the $f^i$ and their derivatives. The number of
functions and variables is free. 
For deciding the existence of $I$ only, the explicit occurrence of the
variables $x^i$ is arbitrary. In order to actually
calculate $I$ explicitly, $D$ must have the property that all terms in $D$ 
must either contain an unknown function of $x^k$ or
must be formally integrable w.r.t. $x^k.$
That means if $I$ exists then 
only a special explicit occurrence of $x^k$ can prevent the
calculation of $I$ 
and furthermore only in those terms which do not contain
any unknown function of $x^k.$ 
If such terms occur in $D$ and $I$ exists then $I$ can still be expressed
as a polynomial in the $f^i, f^i,_j, \ldots$ and terms containing 
indefinite integrals with integrands explicit in $x^k.$ \\
\underline{Algorithm:} \\
Successive partial integration of the term with the highest
$x^k$-derivative of any $f^i.$ By that the 
differential order w.r.t. $x^k$ is reduced
successively. This procedure is always applicable because steps involve only
differentiations and the polynomial
integration $(\int h^n\frac{\partial h}{\partial x}dx =
h^{n+1}/(n+1))$ where $h$ is a partial derivative of some function
$f^i.$ For a more detailed description see \cite{WoInt}.\\
\underline{Stop:} \\
Iteration stops if no term with any $x^k$-derivative of any $f^i$ is left.
If in the remaining un-integrated terms any $f^i(x^k)$ itself occurs,
then $I$ is not expressible with $f^i$ and its derivatives only.  In
case no $f^i(x^k)$ occurs then any remaining terms can contain $x^k$ only
explicitly. Whether they can be integrated depends on their formal
integrability. For their integration the REDUCE integrator is applied. \\
\underline{Example :} \\
We apply the above algorithm to
\begin{equation}
D := 2f,_yg' + 2f,_{xy}g + gg'^3 + xg'^4 + 3xgg'^2g'' = 0
\label{D}
\end{equation}
with $f = f(x,y), \, g = g(x), \, '\equiv d/dx.$
Starting with terms containing $g$
and at first with the highest derivative $g,_{xx},$ the steps are
\[
\begin{array}{rcccl}
\int 3xgg,_x^2g,_{xx} dx 
& = & \int d(xgg,_x^3)
    & - & \int \left( \partial_x(xg) g,_x^3\right) dx \\ \\
& = & xgg,_x^3 & - & \int g,_x^3(g + xg,_x) dx,
\end{array} \]
\[ I := I + xgg,_x^3 \]
\[ D := D - g,_x^3(g + xg,_x)  \]
The new terms $- g,_x^3(g + xg,_x)$ are of lower order than $g,_{xx}$ 
and so in the expression $D$ the maximal order of $x$-derivatives 
of $g$ is lowered. The conditions that $D$ is exact are the following.
\begin{itemize}
\item The leading derivative must occur linearly before each partial
integration step. 
\item After the partial integration of the terms with first order
$x$-derivatives of $f$ the remaining $D$ must not contain $f$ 
or other derivatives of $f$, because such terms cannot
be integrated w.r.t.\ $x$ without specifying $f$.
\end{itemize}
The result of $x$- and $y$-integration in the above example is
(remember $g=g(x)$)
\begin{equation}
0 = 2fg + xygg,_x^3 + c_1(x) + c_2(y) \; \; (=I). \nonumber
\end{equation}
{\tt CRACK} can now eliminate $f$ and substitute
for it in all other equations.
\underline{Generalization:} \\
If after applying the above basic algorithm, terms are left which contain
functions of $x^k$ but each of these functions depends only on a subset of
all $x^i, 1\leq i\leq n,$ then a generalized version of the above algorithm
can still provide a formal expression for the integral $I$
(see \cite{WoInt}). The price consists of
additional differential conditions, but they are equations in less variables
than occur in the integrated equation. Integrating for example 
\begin{equation}
\tilde{D} = D + g^2(y^2 + x\sin y + x^2e^y)       \label{Dnew}
\end{equation}
by introducing as few 
new functions and additional conditions as possible gives as the integral 
$\tilde{I}$
\begin{eqnarray*}
\tilde{I} & = & 2fg + xygg,_{x}^{3} + c_1(x) + c_2(y) \\
  &   & + \frac{1}{3}y^3c_3'' - \cos y(xc_3'' - c_3)
+ e^y(x^2c_3'' - 2xc_3' + 2c_3)          
\end{eqnarray*}
with $c_3 = c_3(x), \, '\equiv d/dx$ and the single additional 
condition $g^2 = c_3'''.$
The integration of the new terms of (\ref{Dnew}) is
achieved by partial integration again. For example
\begin{eqnarray*}
\int g^2x^2 dx & = & x^2\int g^2 dx - \int (2x\!\int g^2 dx) dx \\
& = & x^2\int g^2 dx - 2x\int\!\!\int g^2 dx 
+ 2 \int\!\!\int\!\!\int g^2 dx \\
& = & x^2c_3'' - 2xc_3' + 2c_3
\end{eqnarray*}
\underline{Characterization:} \\
This algorithm is a decision algorithm which does not involve any
heuristic. It is fast because time and memory requirements grow only
linearly with the number of terms in $D$ and with the order of
the highest derivative.
After integration the new equation is still a polynomial
in $f^i$ and in the new constant or function of integration,.
Therefore the algorithms for bringing the system into standard form can still 
be applied to the PDE-system 
after the equation $D = 0$ is replaced by $I = 0.$

The complexity of algorithms for bringing a PDE-system into a standard
form depends nonlinearly on the order of these equations because of
the nonlinear increase of the number of different leading derivatives
and by that the number of equations generated intermediately by such
an algorithm. It therefore in general pays off to integrate equations (with
linear expenses) during such a standard form algorithm.  The increase
in the number of unknown constants/functions through integration does
not play a role for standard form algorithms because these functions
have less variables and do not increase the number of possible leading
derivatives and therefore not the number of equations to be maintained
during execution of the standard form algorithm.

If an $f^i,$ which depends on all variables, can be eliminated after an 
integration, then depending on its length 
it is in general helpful to substitute $f^i$ in other equations and
to reduce the number of equations and functions by one. This is especially
profitable if the replaced expression is short and 
contains only functions of less variables than $f^i.$
\underline{Test:} \\
The corresponding test input is
\begin{verbatim}
depend f,x,y;
depend g,x;
crack({2*df(f,y)*df(g,x)+2*df(f,x,y)*g+g*df(g,x)**3
       +x*df(g,x)**4+3*x*g*df(g,x)**2*df(g,x,2)
       +g**2*(y**2+x*sin y+x**2*e**y)},
      {f,g},{});
\end{verbatim}
The meaning of the REDUCE command {\tt depend} is to declare that $f$
depends in an unknown way on $x$ and $y$. For more details on the
algorithm see \cite{WoInt}, for an introduction to REDUCE see \cite{WM}.

\subsection{Direct separation of PDEs}
As a result of repeated integrations the functions in the 
remaining equations have less and less variables. It therefore may happen
that after a substitution an equation results where at least one variable
occurs only explicitly and not as an argument of an unknown function.
Consequently all coefficients of linearly independent expressions in this
variable can be set to zero individually. \\

{\em Example:}  \\
$f = f(x,y), \;\; g = g(x), \;\; x,y,z$ are independent variables. 
The equation is
\begin{equation} 
0 = f,_y + z(f^2+g,_x) + z^2(g,_x+yg^2) \label{sep}
\end{equation}
$x$-separation? $\rightarrow$ no  \\
$y$-separation? $\rightarrow$ no  \\
$z$-separation? $\rightarrow$ yes: $0 \,=\, f,_y \,=\, f^2+g,_x \,=\,
g,_x+yg^2$ \\
$y$-separation? $\rightarrow$ yes: $0 = g,_x = g^2\;\;$ 
(from the third equation from the $z$-separation)   

If $z^2$ had been replaced in (\ref{sep}) by a third
function $h(z)$ then direct separation would not have been possible.
The situation changes if $h$ is a parametric function which is
assumed to be independently given and which should not be
calculated, i.e.\ $f$ and $g$ should be calculated for any
arbitrary given $h(z)$. Then the same separation could have been
done with an extra treatment of the special case $h,_{zz} = 0,$
i.e.\ $h$ linear in $z$. This different treatment of unknown functions
makes it necessary to input explicitly the functions to be
calculated as the second argument to {\tt CRACK}. The input
in this case would be
\begin{verbatim}
depend f,x,y;
depend g,x;
depend h,z;
crack({df(f,y)+z*f**2+(z+h)*df(g,x)+h*y*g**2},{f,g},{z});
\end{verbatim}
The third parameter for {\tt CRACK} is necessary to make clear that
in addition to the variables of $f$ and $g$, $z$ is also an independent
variable.
 
If the flag {\tt independence\_} is not {\tt nil} then {\tt CRACK} will
stop if linear independence of the explicit expressions of the
separation variable (in the example $1,z,z^2$) is not clear and ask 
interactively whether separation should be done or not.

\subsection{Indirect separation of PDEs}
For the above direct separation a precondition is that at least one
variable occurs only explicitly or as an argument of parametric
functions.  The situation where each variable is an argument of at least
one function but no function contains all independent variables of an
equation needs a more elaborate treatment.

The steps are these 
\begin{itemize}
 \item A variable $x_a$ is chosen which occurs in as few functions as possible.
 This variable will be separated directly later which
 requires that all unknown functions $f_i$ containing $x_a$ are to be
 eliminated. Therefore, as long as $F:=\{f_i\}$ is not empty do the following:
 \begin{itemize}
  \item Choose the function $f_i(y_p)$ in $F$ with the smallest number of
  variables $y_p$ and with $z_{ij}$ as those variables on which $f_i$ does 
  not depend.
  \item Identify all different products $P_{ik}$ of powers of 
  $f_i$-derivatives and of $f_i$ in the equation. 
  Determine the $z_{ij}$-dependent factors $C_{ik}$ of the coefficients 
  of $P_{ik}$.
  \item Choose a $z_{ij}$ and for each $C_{il}$ ($i$ fixed, $l=1,\ldots)$:
  \begin{itemize}
   \item divide by $C_{il}$ the equation and all the $C_{im}$ with $m>l,$
   \item differentiate the equation and the $C_{im}$ w.r.t.\ $z_{ij}$
  \end{itemize}
 \end{itemize}
 \item The resulting equation no longer contains any unknown function of $x_a$
 and can be separated w.r.t.\ $x_a$ directly. The
 equations arising can be integrated, inverting the sequence of differentiation
 and division, by
 \begin{itemize}
  \item multiplying them and the $C_{im}$ with $m<l$ by the elements
  of the $C_{ik}$-lists in exactly the inverse order 
  \item integrating these exact PDEs and $C_{im}$ w.r.t.\ $z_{ij}$.
 \end{itemize}
 \item The equations originating that way are used to simplify the original 
 DE\@. For example direct separability w.r.t.\ $y_p$ could be tested, because
 those products which contain functions $f_i(y_p)$ and their derivatives 
 have been evaluated up to constants so that $y_p$ can occur only explicitly.
 \item The whole procedure is repeated for another variable $x_b$ if the
 original DE still has the property that it contains only functions of
 a subset of all variables in the equation.
\end{itemize}
The additional bookkeeping of coefficients $C_{ik}$ and their updating by
division, differentiation, integration and multiplication is done to use
them as integrating factors for the backward integration.
The following example makes this clearer. The equation is
\begin{equation}
0 = f(x) g(y) - \frac{1}{2}xf'(x) - g'(y) - (1+x^2)y. \label{isep}
\end{equation}
The steps are (equal levels of indentation correspond to the general
description of the algorithm)
\begin{itemize}
 \item $x_1:=x, \, F=\{f\}$
 \begin{itemize}
  \item Identify $f_1:=f, \; \; \; \; \; y_1:=x, \; \; \; \; \; z_{11}:=y$ 
  \item and $P_1=\{f',f\}, \; \; \; \; \; C_1=\{1,g\}$
  \begin{itemize}
   \item Divide $C_{12}$ and 
         (\ref{isep}) by $C_{11}=1$ and differentiate w.r.t. $z_{11}=y:$
         \begin{eqnarray}
         0 & = & fg' - g'' - (1+x^2)   \label{isep2}  \\
         C_{12} & = & g'    \nonumber
         \end{eqnarray}
 \item Divide (\ref{isep2}) by $C_{12}=g'$ and differentiate w.r.t. $z_{11}=y:$
\[ 0 = - (g''/g')' - (1+x^2)(1/g')' \]

  \end{itemize}
 \end{itemize}
 \item Direct separation w.r.t.\ $x$ and integration:
 \[\begin{array}{rclclcl}
  x^2: 0 & = & (1/g')' & \Rightarrow & c_1g' =  1 & \Rightarrow &
        g = y/c_1 + c_2 \\
  x^0: 0 & = & (g''/g')' & \Rightarrow & c_3g' = g'' & \Rightarrow &
        c_3 = 0
 \end{array} \]
 \item Substitution of $g$ in the original DE
       \[0 = (y/c_1+c_2)f - \frac{1}{2}xf' - 1/c_1 - (1+x^2)y \]
       and further solution by direct separation w.r.t.\ $y$
 \[\begin{array}{rclcl}
  y^1: 0 & = & f/c_1 - 1 - x^2               & \Rightarrow & f'  =  2c_1x \\
  y^0: 0 & = & c_2f - \frac{1}{2}xf' - 1/c_1 & \Rightarrow & 0   =  
       c_2c_1(1+x^2) - c_1x^2 - 1/c_1
 \end{array}\]
       and direct separation w.r.t.\ $x$:
 \begin{eqnarray*}
 x^0:  0 & = & c_2c_1 - c_1    \\
 x^2:  0 & = & c_2c_1 - 1/c_1   \\
    & \Rightarrow &  0 = c_1 - 1/c_1   \\
    & \Rightarrow & c_1 = \pm 1 \Rightarrow c_2 = 1.
 \end{eqnarray*}
\end{itemize}
We get the two solutions $f = 1 + x^2, g = 1 + y$ and 
$f = - 1 - x^2, g = 1 - y.$ The corresponding input to {\tt CRACK} would be
\begin{verbatim}
depend f,x;
depend g,y;
crack({f*g-x*df(f,x)/2-df(g,x)-(1+x**2)*y},{f,g},{});
\end{verbatim}
 
\subsection{Solving standard ODEs}
For solving standard ODEs the package {\tt ODESOLVE} by MacCallum 
\cite{Mal} is applied. This package is distributed with REDUCE 3.4
and can be used independently of {\tt CRACK}. The syntax of
{\tt ODESOLVE} is quite similar to that of {\tt CRACK} \\
\verb+depend+ {\it function}, {\it variable}; \\
\verb+odesolve(+ODE, {\it function}, {\it variable});  \\
In the present form (spring 93) it solves standard first order ODEs
(Bernoulli and Euler type, with separable variables, $\ldots$) and linear
higher order ODEs with constant coefficients. Its applicability is 
increased by a subroutine which recognizes such PDEs in which there is only
one unknown function of all variables and all occurring derivatives
are only derivatives w.r.t. one variable of only one partial derivative.
For example the PDE for $f(x,y)$
\[ 0 = f,_{xxy} + f,_{xxyy} \]
can be viewed as a first order ODE in $y$ for $f,_{xxy}.$

In preparation is a subpackage
written by Y.K.\ Man for solving first order ODEs with the Prelle-Singer
algorithm. 

\section{Examples}
\subsection{Investigating symmetries of ODEs/PDEs and systems of them}
According to a theory of Sophus Lie the knowledge of an infinitesimal symmetry
of a DE(-system), i.e. of generators $\xi, \eta$ such that the
infinitesimal transformation
\begin{eqnarray*}
\tilde{x} & = & x + \epsilon\xi   \\
\tilde{y} & = & y + \epsilon\eta
\end{eqnarray*}
keeps the DE(-system) forminvariant up to terms $O(\epsilon^2)$, can be used
to integrate the ODE. The resulting conditions for $\xi, \eta$ are a
linear system of PDEs. If the DEs are PDEs then $x$ and $\xi$ get an
index. If one investigates a system of DEs then $y$ and $\eta$ get an index.
To determine for example point-symmetries of the ODE (6.97) in \cite{Ka}
\begin{equation}
0 = x^4y'' - y'(2xy + x^3) + 4y^2  \label{k97} 
\end{equation}
for $y = y(x)$ the input would be
\begin{verbatim}
depend y,x;
de := {df(y,x,2) - (df(y,x)*(2*x*y+x**3) + 4*y**2)/x**4, y, x};
mo := {0, nil, nil};
LIEPDE(de,mo);
\end{verbatim}

The first argument to {\tt LIEPDE} is a list which includes
a single equation or a list of equations,
a single unknown function or a list of unknown functions and
a single independent variables or a list of independent variables.
The second argument to {\tt LIEPDE} specifies the mode of operation.
It is a list {\tt mo = \{od, lp, fl\}} where
\begin{itemize}
\item {\tt od} is the order of the ansatz for $\xi, \eta.$ It is = 0 for
point symmetries and = 1 for contact symmetries (only in case of
one ODE/PDE for one unknown function).
% and $>1$ for dynamical symmetries
%(only in case of one ODE for one unknown function)
\item If {\tt lp} is $nil$ then the standard ansatz for $\xi^i, \eta^\alpha$
is taken which is
\begin{itemize}
\item for point symmetries ({\tt od}=0) $\xi^i = \xi^i(x^j,u^\beta),
      u^\alpha = u^\alpha(x^j,u^\beta)$
\item for contact symmetries ({\tt od}=1)
   $ \xi^i := \Omega,_{u,_i}, \;\;\;
     \eta := u,_i\Omega,_{u_i} \; - \; \Omega, \;\;\;
     \Omega:=\Omega(x^i, u, u,_j)$
%\item for dynamical symmetries ({\tt od}$>1$)  \\
%   $ \xi := \Omega,_{u'}, \;\;\;
%     \eta := u'\Omega,_{u'} \; - \; \Omega, \;\;\;
%     \Omega:=\Omega(x, u, u',\ldots, u^{({\tt od}-1)})$
%     where {\tt od} must be less than the order of the ODE.
\end{itemize}


If {\tt lp} is not $nil$ then {\tt lp} is the ansatz for
$\xi^i, \eta^\alpha$ and must have the form
\begin{itemize}
  \item for point symmetries
        {\tt \{xi\_\mbox{$x1$} = ..., ..., eta\_\mbox{$u1$} = ..., ...\}}
        where {\tt xi\_, eta\_}
        are fixed and $x1, \ldots, u1$ are to be replaced by the actual names
        of the variables and functions.
  \item otherwise {\tt spot\_ = ...} where the expression on the right hand
        side is the ansatz for the Symmetry-POTential $\Omega.$
\end{itemize}

\item {\tt fl} is the list of free functions in the ansatz 
in case {\tt lp} is not $nil.$
\end{itemize}

The conditions for the symmetry generators $\xi, \eta$ to provide
infinitesimal transformations
which leave (\ref{k97}) form-invariant to order $O(\varepsilon^2)$ are
formulated by the program {\tt LIEPDE} to be
\begin{eqnarray}
0 & = & \frac{1}{2}x^3\eta,_{yy} - x^3\xi,_{xy} - x^2\xi,_y - 2y\xi,_y  
        \label{con1} \\
0 & = & 12y^2\xi,_y - x^4\xi,_{xx} - x^3\xi,_x - 2xy\xi,_x 
        + 2x^4\eta,_{xy} + x^2\xi + 6y\xi - 2x\eta  \\
0 & = & 2y^2\xi - xy^2\xi,_x - \frac{1}{8}x^5\eta,_{xx} 
        + \frac{1}{8}x^4\eta,_x
        + \frac{1}{4}x^2y\eta,_x + \frac{1}{2}xy^2\eta,_y 
        - xy\eta  \\
0 & = & \xi,_{yy}.   \label{con2}
\end{eqnarray}
{\tt LIEPDE} then calls {\tt CRACK} to solve this system.

First (\ref{con2}) is integrated to $0 = \xi + c_1y + c_2$
with new functions $c_1(x), c_2(x).$
Then $\xi$ is substituted into the other three equations. Now (\ref{con1})
can be integrated to
\[ 0 = x^3\eta + x^3y^2c_1' - x^3yc_3 - x^3c_4 + x^2y^2c_1 
       + \frac{2}{3}y^3c_1 \]
with new functions $c_3(x), c_4(x)$ and $\eta$ can be substituted.
Because the functions $c_1,\ldots,c_4$ are functions of $x$ only,
a separation of the remaining two equations w.r.t.\ the independent
variable $y$ becomes possible which yields 5 and 4 equations. One of
them is $c_1=0$.
After setting $c_1$ to zero, the exact second order ODE 
$0 = xc_4'' + 3c_4'$ can be integrated once and
the resulting first order ODE can be solved by 
{\tt ODESOLVE} \cite{Mal} to give $c_4 = x^2c_6 - c_5/2$
with constants $c_5, c_6$. From another equation $c_3$ can be eliminated
and substituted: $c_3 = c_2' - 3c_2/x$. The last function $c_2(x)$ is
solved by {\tt ODESOLVE} from $0 = x^2c_2'' - xc_2' + c_2$ to give
$c_2 = x(c_8\log x + c_7)$. Because now only constants are involved in
the remaining two equations, {\tt CRACK} separates them w.r.t.\ $x$ to obtain 4
conditions which are solved to give $c_5 = 0$ and $c_6 = - c_8$,
finally obtaining two symmetries (for $c_7 = 0, \, c_8 = -1$ and 
$c_7 = -1, \, c_8 = 0$): 
\[ 
\begin{array}{rclrcl}
 \xi_1 & = & x\log x,\hspace{0.5in}  & \eta_1 & = & 2y\log x + x^2 - y \\
 \xi_2 & = & x,                      & \eta_2 & = & 2y . 
\end{array} \]
Applying Lie's algorithm by hand, the general solution of (\ref{k97})
is \cite{Ka} 
\[ y = \left\{ \begin{array}{l}
                 x^2(1 + c_9\tan(c_9\log x + c_{10})) \\
                 x^2(1 - c_9\tanh(c_9\log x + c_{10})) \\
                 x^2(1 - c_9\coth(c_9\log x + c_{10})) 
               \end{array}
       \right. \]
with constant $c_9, c_{10}$.
       
If symmetries of a PDE or of a system of equations shall be determined then
{\tt LIEPDE} does this iteratively by formulating some of the conditions and
solving them by calling {\tt CRACK} to simplify the formulation of the
remaining conditions. If a system of DEs is investigated then the conditions
following from the form-invariance of each single equation are investigated
successively. If a PDE is investigated then conditions which result from
setting to zero the coefficients of the highest derivatives of $y$ are
investigated first. For more details on this iteration see \cite{Alex},
\cite{Step} and \cite{LIEPDE}.

A more difficult system of PDEs are the Karpman equations \cite{Karp}
which play a role in plasma physics. In their real form, which is used to
determine symmetries, they are (\cite{Cham})
\[\rho,_t+w_1\rho,_z+\frac{1}{2}\left[s_1(2\rho,_x\phi,_x+2\rho,_y\phi,_y
  +\rho\phi,_{xx}+\rho\phi,_{yy})+
  s_2(2\rho,_z\phi,_z+\rho\phi,_{zz})\right] = 0\]
\[\phi,_t+w_1\phi,_z-\frac{1}{2}\left[s_1\left(\frac{\rho,_{xx}}{\rho}
  +\frac{\rho,_{yy}}{\rho}-\phi,_x^2-\phi,_y^2\right)+
  s_2\left(\frac{\rho,_{zz}}{\rho}-\phi,_z^2\right)\right]+a_1\nu = 0\]
\[\nu,_{tt}-(w_2)^2(\nu,_{xx}+\nu,_{yy}+\nu,_{zz})-
2a_2\rho(\rho,_{xx}+\rho,_{yy}+\rho,_{zz})-2a_2(\rho,_x^2+\rho,_y^2+\rho,_z^2) = 0\]
with the three functions $\rho,\phi,\nu$ of four variables $t,x,y,z$ and
with constant parameters $a_i,s_i,w_i.$

At first, preliminary symmetry conditions are
derived and solved successively
for each of the three equations.
The solution
of the preliminary conditions for the
first equation
requires 31 integrations and states that all $\xi^i$ are
functions of $t,x,y,z$ only.
The preliminary conditions of the second equation do not provide new
constraints.
The result from the preliminary conditions for the third equation
is that $\nu$ is a function of $t,x,y,z$ only. This needs 4
integrations.
After 7.1/0.1 sec the full conditions for the first equation are derived
which to solve requires further 77 integrations.
The remaining full conditions for the second equation
to solve requires 6 integrations. Finally,
the full conditions for the last equation are
solved with three more integrations.

The general solution for the symmetry generators then reads
\[ \begin{array}{rclrcl}
 \xi^x & = & - y c_1 + c_2 \;\;\;\;\; &  \eta^r & = & 0  \\
 \xi^y & = & x c_1 + c_3  &  \eta^\phi & = & a_1 c_6 t^2 + a_1 c_7 t + c_8  \\
 \xi^z & = & c_4      &  \eta^v & = & - c_6 t - c_7     \\
 \xi^t & = & c_5  & & &
\end{array}
\]
with constants $c_1, \ldots, c_8.$ The corresponding 8 symmetry generators are
\begin{eqnarray*}
 X_1 = \partial_x & \;\;\;\;\;\;\;\;\;\; & X_5 = y\partial_x - x\partial_y \\
 X_2 = \partial_y & &  X_6 = \partial_\phi               \\
 X_3 = \partial_z & & X_7 = a_1t\partial_\phi - \partial_\nu \\
 X_4 = \partial_t & & X_8 = a_1t^2\partial_\phi - 2t\partial_\nu.
\end{eqnarray*}

The total time in a test run with {\tt lisp(print\_:=nil);} was
260/3.2 seconds on a PC486 with 33 MHz and 4MB RAM,
i.e. less than 5 min.
For comparison, the same calculation reported in \cite{Cham}, also split
into 6 parts but with integrations done by hand, took more than 2 hours CPU 
on a VAX 8600 to formulate and simplify the symmetry conditions using the
program {\tt SYMMGRP.MAX} written in MACSYMA.
The corresponding input for {\tt LIEPDE} is
\begin{verbatim}
depend r,x,y,z,tt;
depend f,x,y,z,tt;
depend v,x,y,z,tt;

de := {{df(r,tt)+w1*df(r,z)
        + s1*(df(r,x)*df(f,x)+df(r,y)*df(f,y)+r*df(f,x,2)/2+r*df(f,y,2)/2)
        + s2*(df(r,z)*df(f,z)+r*df(f,z,2)/2),

        df(f,tt)+w1*df(f,z)
        - (s1*(df(r,x,2)/r+df(r,y,2)/r-df(f,x)**2-df(f,y)**2) +
           s2*(df(r,z,2)/r-df(f,z)**2))/2 + a1*v,

        df(v,tt,2)-w2**2*(df(v,x,2)+df(v,y,2)+df(v,z,2))
        - 2*a2*r*(df(r,x,2)+df(r,y,2)+df(r,z,2))
        - 2*a2*(df(r,x)**2+df(r,y)**2+df(r,z)**2)},

       {r,f,v},  {x,y,z,tt}};
mo := {0, nil, nil};
LIEPDE(de,mo);
\end{verbatim}

\subsection{Determining integrating factors}
Another way to solve or simplify nonlinear DEs is to determine first
integrals by finding integrating factors. For the equation (6.37) of
\cite{Ka} (which has no point symmetry) 
\begin{equation}
0 = y'' + y'(2y + F(x)) + F(x)y^2 - H(x)  \label{k37} 
\end{equation}
for $y(x)$ and parametric functions $F, H$ of $x$ the input
consists of
\begin{verbatim}         
depend f,x;
depend h,x;
depend y,x;
de := {df(y,x,2) = -(2*y+f)*df(y,x)-f*y**2+h, y, x};
mo := {0,{},2};
FIRINT(de,mo);
\end{verbatim}
The arguments to {\tt FIRINT} are similar to those of {\tt LIEPDE}\@.
Here {\tt mo = \{fi,fl,dg\}} specifies an ansatz for the first integral.

   For {\tt fi=0} the ansatz for the first integral is determined by 
{\tt dg}: The first integral is assumed to be a 
polynomial of degree {\tt dg} in $y^{(n-1)}$ (with $n$ as the order of the ODE)
and with arbitrary functions of $x, y,\ldots,y^{(n-2)}$ as coefficients. 
In this case
{\tt fl} has no meaning and is set to \{\}.
For {\tt fi}$\neq${\tt 0, fi} is the ansatz for the first integral and
{\tt fl} is a list of
unknown functions in the ansatz, which are to be solved for. In this case
{\tt dg} has no meaning.

   The above example determines first integrals which are at most quadratic 
in $y'$ for equation (6.37) in \cite{Ka}.
{\tt FIRINT} starts with the ansatz 
\begin{equation}
{\rm const.} = h_2(x,y)y'^2 + h_1(x,y)y' + h_0(x,y)  \label{fians}
\end{equation}
and formulates the determining system by total differentiation of
(\ref{fians}), identification of $y''$ with $y''$ in (\ref{k37}) and
direct separation of $y',$ to produce
\begin{eqnarray*}
0 & = & h_2,_y  \\
0 & = & (H - F y^2)h_1 + h_0,_x  \\
0 & = & h_1,_x + h_2,x - (2F + 4y)h_2  \\
0 & = & 2(H - Fy^2)h_2 - (F - 2y)h_1 + h_0,y + h_1,x.
\end{eqnarray*} 
It then calls {\tt CRACK} which provides the solution for $h_0,h_1,h_2.$
The first integral is
\begin{equation}
\mbox{const.} = \left( \frac{dy}{dx}+y^2 \right) e^{\int \!F\,dx} - 
\int He^{\int \!F\,dx} dx
\end{equation}
which is linear in $y'$. Many other nonlinear ODEs in \cite{Ka} have quadratic
first integrals.

\subsection{Determining a Lagrangian for a given 2nd order ODE}
The knowledge of a Lagrangian $L$ for a given set of differential
equations may be useful in solving and understanding these equations.
If $L$ is known it is easier to determine symmetries, conserved
quantities and singular points. An equivalent variational principle
could also be useful for a more efficient numerical solution.
This problem is known as the inverse problem of variational calculus.
For example, the first transcendental Painlev\'{e} equation 
\begin{equation} 
 y'' = 6y^2 + x        \label{pain}
\end{equation}
has no point symmetries but proves to have a Lagrangian with the 
structure
\begin{equation}
 L = u(x,y)(y')^2 + v(x,y).   \label{lagr}
\end{equation}
(The other 5 transcendental Painlev\'{e} equations have a Lagrangian
of this structure as well.)
A program {\tt LAGRAN} formulates the conditions for $u$ and $v$ (a first
order PDE-system). The call is
\begin{verbatim}
depend f,x; depend y,x;
de := {df(y,x,2) = 6*y**2 + x, y, x};
mo := {0,{}};
LAGRAN(de,mo);
\end{verbatim}

If a system of ODEs or a PDE with at least two variables is given,
then the existence of an equivalent Lagrangian $L$ is not guaranteed.
Only in the case of
a single 2nd order ODE, which is treated by the program {\tt LAGRAN}, 
must a Lagrangian which is locally equivalent to the ODE
exist. As a consequence the determination of $L$ for a second order
ODE is in general not simpler than the solution of the ODE itself. 
Therefore the structure of
$L$ must be specified, which is done through (\ref{lagr}).

   In the call of {\tt LAGRAN} the second argument {\tt mo=\{lg,fl\}} allows 
input of an ansatz for the Lagrangian.
For {\tt lg=0} the default ansatz for the first integral is as given in
(\ref{lagr}).
For {\tt lg$\neq$0, lg} is the ansatz for the Lagrangian and {\tt fl} is a 
list of unknown functions in the ansatz which are to be solved.
For equation (\ref{pain}) the resulting equivalent Lagrangian is
\[ L = y'^2 + xy + 4y^3. \]
Because the first order system for $u$ and $v$ is normally not too
difficult, more complicated problems can also be solved, such as the 
determination of $L$ for the 6'th transcendental Painlev\'{e} equation 
\begin{eqnarray}
 y'' = & \frac{1}{2}\left(\frac{1}{y}+\frac{1}{y-1}+\frac{1}{y-x}\right) y'^2
- \left(\frac{1}{x}+\frac{1}{x-1}+\frac{1}{y-x}\right)y'  \nonumber \\
& + \frac{y(y-1)(y-x)}{x^2(x-1)^2}\left(a+\frac{bx}{y^2}+\frac{c(x-1)}{(y-1)^2}
+\frac{dx(x-1)}{(y-x)^2}\right)            
\end{eqnarray}
for which $L$ turns out to be
\[
\begin{array}{rcll}
L & = & 1/[xy(x-y)(x+1)(x-1)(y-1)] \cdot [(x+1)(x-1)^2x^2y'^2  \\
  &   & - 2a(xy+x+y)(x-y)(y-1)y + 2b(x+y+1)(x-y)(y-1)x \\
  &   & + 2c(x+y)(x-y)(x-1)y  - 2d(x-1)(y+1)(y-1)xy ].
\end{array} \]

\subsection{A factorization ansatz for a given ODE}
A further possible way to integrate an $n$'th order ODE is to decompose it
into two ODEs of order $k$ and $n-k$. If one specifies the structure of the
$k$'th order ODE and demands that these ODEs can be solved successively then 
this leads to a system of PDEs in the variables 
$x, y, \ldots, y^{k-1}$ with a good chance of solution.

To solve the ODE (6.122) in \cite{Ka}
\begin{equation}
0 = yy'' - y'^2 + F(x)yy' + Q(x)y^2       \label{6.122}
\end{equation}
the input would be
\begin{verbatim}
depend f,x;
depend q,x;
depend y,x;
de := {df(y,x,2) = df(y,x)**2/y - f*df(y,x) - y*q, y, x};
mo := {2,{}};
DECOMP(de,mo);
\end{verbatim}
In contrast to {\tt FIRINT}, {\tt LAGRAN} the ODE could here
be in the implicit form $0 = \ldots \;$ as well.

The second argument {\tt mo = \{as,fl\}} specifies an ansatz for the $k$'th
order ODE. If {\tt as} is an integer between 1 and 4 inclusive 
then a default ansatz 
is taken and {\tt fl} (the list of further unknown functions) is \{\}. 
The 4 ans\"atze for \verb+as+ $\in$ [1,4] are
\begin{eqnarray}
y' & = & a(x)\,b(y)      \nonumber   \\
y' & = & a(x)\,y + b(x)   \label{d2}     \\
y' & = & a(x)\,y^n + b(x)\,y    \nonumber   \\
y' & = & a(x)\,y^2 + b(x)\,y + c(x)      \nonumber
\end{eqnarray}

   For {\tt as}$\not\in$[1,4], {\tt as} itself is the ansatz for the $k$'th
order ODE with $y^{(k)}$ eliminated on the left side and {\tt fl} is 
a list of unknown functions in the ansatz, which are to be solved for.

The program {\tt DECOMP} formulates the determining system, in the above
example 
\begin{eqnarray*}
0 & = & b^2  \\
0 & = & - F(x)b - b' + ab  \\
0 & = & - F(x)a - Q(x) - a'
\end{eqnarray*}
and calls {\tt CRACK} to solve it. Because the result
\begin{eqnarray*}
a(x) & = & \left(c_1 - \int Qe^{-\int F dx} dx\right)e^{-\int F dx} \\
b(x) & = & 0
\end{eqnarray*}
contains $n-k = 1$ constant(s) of integration, this factorization ansatz
does not restrict the solution space  of (\ref{6.122})
and because of (\ref{d2}) the general solution is
\[y = c_2e^{\int \left[\left(c_1 - \int Qe^{-\int F dx} dx\right)e^{-\int F dx}
\right] dx}.\]

\subsection{General remarks on complexity}
In general, PDE-systems are easier to solve, the fewer arbitrary functions of
fewer variables are contained in the general solution. This in turn
is the case if more and more equations have to be satisfied for less functions,
i.e. the more restrictive is the PDE-system.
Having more equations to solve for a given number of functions
improves the
chance of finding an integrable one among them and provides
more possibilities to combine them and find a quick way to solve
them. For example, it is usually more difficult to find all 
point symmetries of a very simple ODE with many point symmetries
than to show that an ODE that is possibly more difficult 
to solve has no point symmetry.
An appropriate strategy could therefore be to start an
investigation with a very restrictive ansatz and later to try
less restrictive ones. For example, an ODE could be investigated at first
with respect to point symmetries and if none is found then one
could look for contact and then for dynamical symmetries.
   
Experience shows that nonlinear systems may lead to an
explosion in the length of generated equations much quicker 
than linear equations
even if they have more independent variables or are of higher order.
For solving ODEs one should therefore investigate infinitesimal
symmetries or integrating factors first which provide linear systems
and only later consider the factorization into two ODEs of lower order.

\subsection{Acknowledgement}
We want to thank especially Francis Wright and Herbert Melenk for
continuous help in respect with special features and bugs of REDUCE.

\newpage
\begin{thebibliography}{99}
\bibitem{Riq} C. Riquier, Les syst\`{e}mes d'\'{e}quations aux d\'{e}riv\'{e}es
partielles, Gauthier--Villars, Paris (1910).
\bibitem{Th} J. Thomas, Differential Systems, AMS, Colloquium
publications, v. 21, N.Y. (1937).
\bibitem{Ja} M. Janet, Le\c{c}ons sur les syst\`{e}mes d'\'{e}quations aux 
d\'{e}riv\'{e}es, Gauthier--Villars, Paris (1929).
\bibitem{Topu} V.L. Topunov, Reducing Systems of Linear Differential
Equations to a Passive Form, Acta Appl. Math. 16 (1989) 191--206.
\bibitem{Alex} A.V. Bocharov and M.L. Bronstein, Efficiently
Implementing Two Methods of the Geometrical Theory of Differential
Equations: An Experience in Algorithm and Software Design, Acta. Appl.
Math. 16 (1989) 143--166.
\bibitem{Reid1} G.J. Reid, A triangularization algorithm which
determines the Lie symmetry algebra of any system of PDEs, J.Phys. A:
Math. Gen. 23 (1990) L853-L859.
\bibitem{FS} F. Schwarz, Automatically Determining Symmetries of Partial
Differential Equations, Computing 34, (1985) 91-106.
\bibitem{Fush} W.I. Fushchich and V.V. Kornyak, Computer Algebra
Application for Determining Lie and Lie--B\"{a}cklund Symmetries of
Differential Equations, J. Symb. Comp. 7 (1989) 611--619.
\bibitem{Ka} E. Kamke, Differentialgleichungen, L\"{o}sungsmethoden
und L\"{o}sungen, Band 1, Gew\"{o}hnliche Differentialgleichungen,
Chelsea Publishing Company, New York, 1959.
\bibitem{Wo} T. Wolf, An Analytic Algorithm for Decoupling and Integrating
systems of Nonlinear Partial Differential Equations, J. Comp. Phys.,
no. 3, 60 (1985) 437-446 and, Zur analytischen Untersuchung und exakten
L\"{o}sung von Differentialgleichungen mit Computeralgebrasystemen,
Dissertation B, Jena (1989).
\bibitem{WoInt} T. Wolf, The Symbolic Integration of Exact PDEs,
preprint, submitted to J.\ Symb.\ Comp. (1991).
\bibitem{WM} M.A.H. MacCallum, F.J. Wright, Algebraic Computing with REDUCE,
Clarendon Press, Oxford (1991).
\bibitem{Mal} M.A.H. MacCallum, An Ordinary Differential Equation
Solver for REDUCE, Proc. ISAAC'88, Springer Lect. Notes in Comp Sci.
358, 196--205.
\bibitem{Step} H. Stephani, Differential equations, Their solution using
symmetries, Cambridge University Press (1989).
\bibitem{LIEPDE} T. Wolf, An efficiency improved program {\tt LIEPDE}
for determining Lie-symmetries of PDEs, Proceedings of the workshop on
Modern group theory methods in Acireale (Sicily) Nov. (1992)
\bibitem{Karp} V.I. Karpman, Phys. Lett. A 136, 216 (1989)
\bibitem{Cham} B. Champagne, W. Hereman and P. Winternitz, The computer
      calculation of Lie point symmetries of large systems of differential
      equation, Comp. Phys. Comm. 66, 319-340 (1991)

\end{thebibliography}
 
\end{document}
