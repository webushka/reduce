\chapter{RESIDUE: A residue package}
\label{RESIDUE}
\typeout{{RESIDUE: A residue package}}

{\footnotesize
\begin{center}
Wolfram Koepf\\
Konrad--Zuse--Zentrum f\"ur Informationstechnik Berlin \\
Takustra\"se 7 \\
D--14195 Berlin--Dahlem, Germany \\[0.05in]
e--mail: Koepf@zib.de
\end{center}
}
\ttindex{RESIDUE}
\def\Res{\mathop{\rm Res}\limits}
\newcommand{\C}{{\rm {\mbox{C{\llap{{\vrule height1.52ex}\kern.4em}}}}}}

This package supports the calculation of residues. The residue
$\Res_{z=a} f(z)$ of a function $f(z)$ at the point $a\in\C$ is defined
as
\[
\Res_{z=a} f(z)=
\frac{1}{2 \pi i}\oint f(z)\,dz
\;,
\]
with integration along a closed curve around $z=a$ with winding number 1.

It contains two \REDUCE\ operators:

\begin{itemize}
\item
{\tt residue(f,z,a)}\ttindex{residue} determines the residue of $f$ at
the point $z=a$ if $f$ is meromorphic at $z=a$.  The calculation of
residues at essential singularities of $f$ is not supported.
\item
{\tt poleorder(f,z,a)}\ttindex{poleorder} determines the pole order
of $f$ at the point $z=a$ if $f$ is meromorphic at $z=a$.
\end{itemize}
Note that both functions use the {\tt TAYLOR} package (chapter~\ref{TAYLOR}).

\begin{verbatim}
load_package residue;

residue(x/(x^2-2),x,sqrt(2));

 1
---
 2

poleorder(x/(x^2-2),x,sqrt(2));

1

residue(sin(x)/(x^2-2),x,sqrt(2));

 sqrt(2)*sin(sqrt(2))
----------------------
          4

poleorder(sin(x)/(x^2-2),x,sqrt(2));

1

residue((x^n-y^n)/(x-y)^2,x,y);

  n
 y *n
------
  y

poleorder((x^n-y^n)/(x-y)^2,x,y);

1

\end{verbatim}

