\chapter[ORTHOVEC: scalars and vectors]%
	{ORTHOVEC: Three-dimensional vector analysis}
\label{ORTHOVEC}
\typeout{{ORTHOVEC: Three-dimensional vector analysis}}

{\footnotesize
\begin{center}
James W.~Eastwood \\
AEA Technology, Culham Laboratory \\
Abingdon \\
Oxon OX14 3DB, England \\[0.05in]
e--mail: jim\_eastwood@aeat.co.uk
\end{center}
}
\ttindex{ORTHOVEC}

The ORTHOVEC package is a collection of \REDUCE\ procedures and
operations which provide a simple to use environment for the
manipulation of scalars and vectors.  Operations include addition,
subtraction, dot and cross products, division, modulus, div, grad,
curl, laplacian, differentiation, integration, ${\bf a \cdot \nabla}$
and Taylor expansion.

\section{Initialisation}\label{vstart}
\ttindex{VSTART}
The procedure \f{START} initialises ORTHOVEC. VSTART provides a
menu of standard coordinate systems:-

\begin{enumerate}
\index{cartesian coordinates}
\item cartesian $(x, y, z) = $ {\tt (x, y, z)}
\index{cylindrical coordinates}
\item cylindrical $(r, \theta, z) = $ {\tt (r, th, z)}
\index{spherical coordinates}
\item spherical $(r, \theta, \phi) = $ {\tt (r, th, ph) }
\item general $( u_1, u_2, u_3 ) = $ {\tt (u1, u2, u3) }
\item others
\end{enumerate}

which the user selects by number.  Selecting options (1)-(4)
automatically sets up the coordinates and scale factors.  Selection
option (5) shows the user how to select another coordinate system.  If
VSTART is not called, then the default cartesian coordinates are used.
ORTHOVEC may be re-initialised to a new coordinate system at any time
during a given \REDUCE\ session by typing
\begin{verbatim}
VSTART $.
\end{verbatim}

\section{Input-Output}

ORTHOVEC assumes all quantities are either scalars or 3 component
vectors.  To define a vector $a$ with components $(c_1, c_2, c_3)$ use
the procedure SVEC:\ttindex{SVEC}
\begin{verbatim}
a := svec(c1, c2, c3);
\end{verbatim}

The procedure\ttindex{VOUT} \f{VOUT} (which returns the value of its
argument) can be used to give labelled output of components
in algebraic form:
\begin{verbatim}
b := svec (sin(x)**2, y**2, z)$
vout(b)$
\end{verbatim}

The operator {\tt \_} can be used to select a particular
component (1, 2 or 3) for output {\em e.g.}
\begin{verbatim}
b_1 ;
\end{verbatim}

\section{Algebraic Operations}

Six infix operators, sum, difference, quotient, times, exponentiation
and cross product, and four prefix
operators, plus, minus, reciprocal
and  modulus are defined in ORTHOVEC.  These operators can take suitable
combinations of scalar and vector arguments,
and in the case of scalar arguments reduce to the usual definitions of
$ +, -, *, /, $ etc.

The operators are represented by symbols
\index{+ ! 3-D vector}\index{- ! 3-D vector}\index{/ ! 3-D vector}
\index{* ! 3-D vector}\index{* ! 3-D vector}\index{"\^{} ! 3-D vector}
\index{$><$ ! 3-D vector}
\begin{verbatim}
+, -, /, *, ^, ><
\end{verbatim}

\index{$><$ ! diphthong} The composite {\tt ><} is an
attempt to represent the cross product symbol
$\times$ in ASCII characters.
If we let ${\bf v}$ be a vector and $s$ be a scalar, then
valid combinations of arguments of the
procedures and operators and the type of the result
are as summarised below.  The notation used is\\
{\em result :=procedure(left argument, right argument) } or\\
{\em result :=(left operand) operator (right operand) } . \\

\underline{Vector Addition} \\
\ttindex{VECTORPLUS}\ttindex{VECTORADD}\index{vector ! addition}
\begin{tabular}{rclcrcl}
{\bf v} &:=& VECTORPLUS({\bf v})  &{\rm or}& {\bf v} &:=&  +  {\bf v} \\
     s  &:=& VECTORPLUS(s)  &{\rm or} &      s  &:=&  +       s  \\
{\bf v} &:=& VECTORADD({\bf v},{\bf v})  &{\rm or }& {\bf v} &:=&
{\bf v} +  {\bf v} \\
     s  &:=& VECTORADD(s,s)  &{\rm or }&      s  &:=&  s + s \\
\end{tabular} \\

\underline{Vector Subtraction} \\
\ttindex{VECTORMINUS}\ttindex{VECTORDIFFERENCE}\index{vector ! subtraction}
\begin{tabular}{rclcrcl}
{\bf v} &:=& VECTORMINUS({\bf v})  &{\rm or}&
 {\bf v} &:=&  -  {\bf v} \\
 s  &:=& VECTORMINUS(s)  &{\rm or} &      s  &:=&  -       s  \\
{\bf v} &:=& VECTORDIFFERENCE({\bf v},{\bf v})  &{\rm or }& {\bf v} &:=&
  {\bf v} -  {\bf v} \\
 s  &:=& VECTORDIFFERENCE(s,s)  &{\rm or }&      s  &:=&  s - s \\
\end{tabular} \\

\underline{Vector Division}\\
\ttindex{VECTORRECIP}\ttindex{VECTORQUOTIENT}\index{vector ! division}
\begin{tabular}{rclcrcl}
{\bf v} &:=& VECTORRECIP({\bf v})  &{\rm or}& {\bf v} &:=&  /
{\bf v} \\
 s  &:=& VECTORRECIP(s)  &{\rm or} &      s  &:=&  /       s  \\
{\bf v} &:=& VECTORQUOTIENT({\bf v},{\bf v})  &{\rm or }& {\bf v} &:=&
{\bf v} /  {\bf v} \\
{\bf v} &:=& VECTORQUOTIENT({\bf v},    s  )  &{\rm or }& {\bf v} &:=&
{\bf v} /     s    \\
{\bf v} &:=& VECTORQUOTIENT(   s   ,{\bf v})  &{\rm or }& {\bf v} &:=&
   s    /  {\bf v} \\
     s  &:=& VECTORQUOTIENT(s,s)  &{\rm or }&      s  &:=&  s / s
       \\
\end{tabular} \\

\underline{Vector Multiplication}\\
\ttindex{VECTORTIMES}\index{vector ! multiplication}
\begin{tabular}{rclcrcl}
{\bf v} &:=& VECTORTIMES(   s   ,{\bf v})  &{\rm or }& {\bf v} &:=&
s    *  {\bf v} \\
{\bf v} &:=& VECTORTIMES({\bf v},   s   )  &{\rm or }& {\bf v} &:=& {\bf
 v}  *     s    \\
   s    &:=& VECTORTIMES({\bf v},{\bf v})  &{\rm or }&    s    &:=& {\bf
 v}  *  {\bf v} \\
   s    &:=& VECTORTIMES(   s   ,   s   )  &{\rm or }&    s    &:=&
s    *     s    \\
\end{tabular} \\

\underline{Vector Cross Product} \\
\ttindex{VECTORCROSS}\index{cross product}\index{vector ! cross product}
\begin{tabular}{rclcrcl}
{\bf v} &:=& VECTORCROSS({\bf v},{\bf v})  &{\rm or }& {\bf v} &:=& {\bf
 v} $\times$   {\bf v} \\
\end{tabular} \\

\underline{Vector Exponentiation}\\
\ttindex{VECTOREXPT}\index{vector ! exponentiation}
\begin{tabular}{rclcrcl}
   s    &:=& VECTOREXPT ({\bf v},   s   )  &{\rm or }&    s    &:=& {\bf
 v}  \^{} s   \\
   s    &:=& VECTOREXPT (   s   ,   s   )  &{\rm or }&    s    &:=&    s
     \^{} s   \\
\end{tabular} \\

\underline{Vector Modulus}\\
\ttindex{VMOD}\index{vector ! modulus}
\begin{tabular}{rcl}
   s    &:=& VMOD (s)\\
   s    &:=& VMOD ({\bf v}) \\
\end{tabular} \\

All other combinations of operands for these operators lead to error
messages being issued.  The first two instances of vector
multiplication are scalar multiplication of vectors, the third is the
\index{vector ! dot product}\index{vector ! inner product}
\index{inner product}\index{dot product}
product of two scalars and the last is the inner (dot) product.  The
prefix operators  {\tt +, -, /} can take either scalar or vector
arguments and return results of the same type as their arguments.
VMOD returns a scalar.

In compound expressions, parentheses may be used to specify the order of
combination.  If parentheses are omitted the ordering of the
operators, in increasing order of precedence is
\begin{verbatim}
+ | - | dotgrad | * | >< | ^ | _
\end{verbatim}
and these are placed in the precedence list defined in \REDUCE{}
after $<$.

Vector divisions are defined as follows:  If ${\bf a}$ and ${\bf b}$ are
vectors and $c$ is a scalar, then
\begin{eqnarray*}
{\bf a} /  {\bf b} & = &  \frac{{\bf a} \cdot {\bf b}}{  \mid {\bf b}
\mid^2}\\
c / {\bf a}   & = &  \frac{c {\bf a}  }{ \mid {\bf a} \mid^2}
\end{eqnarray*}

Both scalar multiplication and dot products are given by the same symbol,
braces are advisable to ensure the correct
precedences in expressions such as $({\bf a} \cdot {\bf b})
({\bf c} \cdot {\bf d})$.

Vector exponentiation is defined as the power of the modulus:\\
${\bf a}^n \equiv  {\rm VMOD}(a)^n =   \mid {\bf a} \mid^n$

\section{Differential Operations}
Differential operators provided are div, grad, curl, delsq, and dotgrad.
\index{div operator}\index{grad operator}\index{curl operator}
\index{delsq operator}\index{dotgrad operator}
All but the last of these are prefix operators having a single
vector or scalar argument as appropriate.  Valid combinations of
operator and argument, and the type of the result are shown in
table~\ref{vvecttable}.


\begin{table}
\begin{center}
\begin{tabular}{rcl}
s & := & div ({\bf v})  \\
{\bf v} & := & grad(s) \\
{\bf v} & := & curl({\bf v})  \\
{\bf v} & := & delsq({\bf v}) \\
 s  & := & delsq(s) \\
{\bf v} & := & {\bf v}  dotgrad {\bf v}  \\
 s & := & {\bf v}  dotgrad  s
\end{tabular}
\end{center}
\caption{ORTHOVEC valid combinations of operator and argument}\label{vvecttable}
\end{table}

All other combinations of operator and argument type cause error
messages to be issued.  The differential operators have their usual
meanings.  The coordinate system used by these operators is
set by invoking  VSTART (cf. Sec.~\ref{vstart}).  The names {\tt h1},
{\tt h2}  and {\tt h3 } are
reserved for the scale factors, and {\tt u1}, {\tt u2} and {\tt u3} are
used for the coordinates.

A vector extension, VDF, of the \REDUCE\ procedure DF allows the
differentiation of a vector (scalar) with respect to a scalar to be
performed.  Allowed forms are\ttindex{VDF}
VDF({\bf v}, s)  $\rightarrow$  {\bf v}   and
VDF(s, s)  $\rightarrow$   s ,
where, for example\\
\begin{eqnarray*}
{\tt vdf( B,x)} \equiv \frac{\partial {\bf B}}{\partial x}
\end{eqnarray*}

The standard \REDUCE\ procedures DEPEND and NODEPEND have been redefined
to allow dependences of vectors to be compactly
defined.  For example\index{DEPEND statement}\index{NODEPEND statement}
\begin{verbatim}
a := svec(a1,a2,a3)$;
depend a,x,y;
\end{verbatim}
causes all three components {\tt a1},{\tt a2} and {\tt a3} of {\tt a}
to be treated as functions of {\tt x} and {\tt y}.
Individual component dependences can still be defined if desired.
\begin{verbatim}
depend a3,z;
\end{verbatim}

The procedure VTAYLOR gives truncated Taylor series expansions of scalar
or vector functions:-\ttindex{VTAYLOR}
\begin{verbatim}
vtaylor(vex,vx,vpt,vorder);
\end{verbatim}
returns the series expansion of the expression
VEX  with respect to variable VX\ttindex{VORDER}
about point VPT  to order VORDER.  Valid
combinations of argument types are shown in table~\ref{ORTHOVEC:validexp}. \\

\begin{table}
\begin{center}
\begin{tabular}{cccc}
VEX & VX & VPT &  VORDER \\[2ex]
{\bf v} & {\bf v} &  {\bf v} &  {\bf v}\\
{\bf v} &  {\bf v} & {\bf v} & s\\
{\bf v} & s & s & s \\
s & {\bf v} &  {\bf v} & {\bf v}   \\
s & {\bf v} & {\bf v} & s\\
s & s & s & s\\
\end{tabular}
\end{center}
\caption{ORTHOVEC valid combination of argument types.}\label{ORTHOVEC:validexp}
\end{table}

Any other combinations cause error messages to be issued.  Elements of
VORDER must be non-negative integers, otherwise error messages are
issued.  If scalar VORDER is given for a vector expansion, expansions
in each component are truncated at the same order, VORDER.

The new version of Taylor expansion applies\index{l'H\^opital's rule}
l'H\^opital's rule in evaluating coefficients, so handle cases such as
$\sin(x) / (x) $ , etc.  which the original version of ORTHOVEC could
not. The procedure used for this is LIMIT,\ttindex{LIMIT} which can
be used directly to find the limit of a scalar function {\tt ex} of
variable {\tt x} at point {\tt pt}:-

\begin{verbatim}
ans := limit(ex,x,pt);
\end{verbatim}

\section{Integral Operations}
Definite and indefinite vector, volume and scalar line integration
procedures are included in ORTHOVEC.  They are defined as follows:
\ttindex{VINT}\ttindex{DVINT}
\ttindex{VOLINT}\ttindex{DVOLINT}\ttindex{LINEINT}\ttindex{DLINEINT}
\begin{eqnarray*}
{\rm VINT} ({\bf v},x) & = & \int {\bf v}(x)dx\\
%
{\rm DVINT} ({\bf v},x, a, b) & = & \int^b_a {\bf v} (x) dx\\
%
{\rm VOLINT} ({\bf v}) & = & \int {\bf v} h_1 h_2 h_3 du_1 du_2 du_3\\
%
{\rm DVOLINT}({\bf v},{\bf l},{\bf u},n) & = & \int^{\bf u}_{\bf l}
{\bf v} h_1 h_2 h_3 du_1 du_2 du_3\\
%
{\rm LINEINT} ({\bf v, \omega}, t) & = & \int {\bf v} \cdot {\bf dr}
\equiv \int v_i h_i \frac{\partial \omega_i}{\partial t} dt\\
%
{\rm DLINEINT} ({\bf v, \omega} t, a, b) & = & \int^b_a v_i h_i
\frac{\partial \omega_i}{\partial t} dt\\
\end{eqnarray*}

In the vector and volume integrals, ${\bf v}$ are vector or scalar,
$a, b,x$ and $n$ are scalar.  Vectors ${\bf l}$ and ${\bf u}$ contain
expressions for lower and upper bounds to the integrals.  The integer
index $n$ defines the order in which the integrals over $u_1, u_2$ and
$u_3$ are performed in order to allow for functional dependencies in
the integral bounds:

\begin{center}
\begin{tabular}{ll}
n & order\\ 1 & $u_1~u_2~u_3$\\
%
2 & $u_3~u_1~u_2$\\
%
3 & $u_2~u_3~u_1$\\
%
4 & $u_1~u_3~u_2$\\
%
5 & $u_2~u_1~u_3$\\ otherwise & $u_3~u_2~u_1$\\
\end{tabular}
\end{center}


The vector ${\bf \omega}$ in the line integral's arguments contain
explicit parameterisation of the coordinates $u_1, u_2, u_3$ of the
line ${\bf u}(t)$ along which the integral is taken.

