\chapter{Substitution Commands}\index{Substitution}
An important class of commands in {\REDUCE} define
substitutions for variables and expressions to be made during the
evaluation of expressions.  Such substitutions use the prefix operator
{\tt SUB}, various forms of the command {\tt LET}, and rule sets.

\section{SUB Operator}\ttindex{SUB}

Syntax:
\begin{verbatim}
        SUB(<substitution_list>,EXPRN1:algebraic):algebraic
\end{verbatim}
where {\tt <substitution\_list>} is a list of one or more equations of the
form
\begin{verbatim}
        VAR:kernel=EXPRN:algebraic
\end{verbatim}
or a kernel that evaluates to such a list.

The {\tt SUB} operator gives the algebraic result of replacing every
occurrence of the variable {\tt VAR} in the expression {\tt EXPRN1} by the
expression {\tt EXPRN}.  Specifically, {\tt EXPRN1} is first evaluated
using all available rules.  Next the substitutions are made, and finally
the substituted expression is reevaluated.  When more than one variable
occurs in the substitution list, the substitution is performed by
recursively walking down the tree representing {\tt EXPRN1}, and replacing
every {\tt VAR} found by the appropriate {\tt EXPRN}.  The {\tt EXPRN} are
not themselves searched for any occurrences of the various {\tt VAR}s.
The trivial case {\tt SUB(EXPRN1)} returns the algebraic value of
{\tt EXPRN1}.

{\it Examples:}
\begin{verbatim}
				    2              2
     sub({x=a+y,y=y+1},x^2+y^2) -> A  + 2*A*Y + 2*Y  + 2*Y + 1
\end{verbatim}
and with {\tt s := \{x=a+y,y=y+1\}},
\begin{verbatim}
				    2              2
     sub(s,x^2+y^2)             -> A  + 2*A*Y + 2*Y  + 2*Y + 1
\end{verbatim}

Note that the global assignments {\tt x:=a+y}, etc., do not take place.

{\tt EXPRN1} can be any valid algebraic expression whose type is such that
a substitution process is defined for it (e.g., scalar expressions, lists
and matrices).  An error will occur if an expression of an invalid type
for substitution occurs either in {\tt EXPRN} or {\tt EXPRN1}.

The braces around the substitution list may also be omitted, as in:

\begin{verbatim}
				    2              2
     sub(x=a+y,y=y+1,x^2+y^2)   -> A  + 2*A*Y + 2*Y  + 2*Y + 1
\end{verbatim}

\section{LET Rules}\ttindex{LET}
Unlike substitutions introduced via {\tt SUB}, {\tt LET}
rules are global in scope and stay in effect until replaced or {\tt CLEAR}ed.

The simplest use of the {\tt LET} statement is in the form
\begin{verbatim}
	LET <substitution list>
\end{verbatim}
where {\tt <substitution list>} is a list of rules separated by commas, each
of the form:
\begin{verbatim}
	<variable> = <expression>
\end{verbatim}
or
\begin{verbatim}
    <prefix operator>(<argument>,...,<argument>) = <expression>
\end{verbatim}
or
\begin{verbatim}
    <argument> <infix operator>,..., <argument> = <expression>
\end{verbatim}
For example,
\begin{verbatim}
	let {x => y^2,
	     h(u,v) => u - v,
	     cos(pi/3) => 1/2,
	     a*b => c,
	     l+m => n,
	     w^3 => 2*z - 3,
	     z^10 => 0}
\end{verbatim}
The list brackets can be left out if preferred.  The above rules could
also have been entered as seven separate {\tt LET} statements.

After such {\tt LET} rules have been input, {\tt X} will always be
evaluated as the square of {\tt Y}, and so on.  This is so even if at the
time the {\tt LET} rule was input, the variable {\tt Y} had a value other
than {\tt Y}. (In contrast, the assignment {\tt x:=y\verb|^|2} will set {\tt X}
equal to the square of the current value of {\tt Y}, which could be quite
different.)

The rule {\tt let a*b=c} means that whenever {\tt A} and {\tt B} are both
factors in an expression their product will be replaced by {\tt C}.  For
example, {\tt a\verb|^|5*b\verb|^|7*w} would be replaced by
{\tt c\verb|^|5*b\verb|^|2*w}.

The rule for {\tt l+m} will not only replace all occurrences of {\tt l+m}
by {\tt N}, but will also normally replace {\tt L} by {\tt n-m}, but not
{\tt M} by {\tt n-l}.  A more complete description of this case is given
in Section~\ref{sec-gensubs}.

The rule pertaining to {\tt w\verb|^|3} will apply to any power of {\tt W}
greater than or equal to the third.

Note especially the last example, {\tt let z\verb|^|10=0}.  This declaration
means, in effect: ignore the tenth or any higher power of {\tt Z}.  Such
declarations, when appropriate, often speed up a computation to a
considerable degree. (See\index{Asymptotic command}
Section~\ref{sec-asymp} for more details.)

Any new operators occurring in such {\tt LET} rules will be automatically
declared {\tt OPERATOR} by the system, if the rules are being read from a
file.  If they are being entered interactively, the system will ask
{\tt DECLARE} ... {\tt OPERATOR?} .  Answer {\tt Y} or {\tt N} and hit
\key{Return}.

In each of these examples, substitutions are only made for the explicit
expressions given; i.e., none of the variables may be considered arbitrary
in any sense. For example, the command
\begin{verbatim}
        let h(u,v) = u - v;
\end{verbatim}
will cause {\tt h(u,v)} to evaluate to {\tt U - V}, but will not affect
{\tt h(u,z)} or {\tt H} with any arguments other than precisely the
symbols {\tt U,V}.

These simple {\tt LET} rules are on the same logical level as assignments
made with the := operator.  An assignment {\tt x := p+q} cancels a rule
{\tt let x = y\verb|^|2} made earlier, and vice versa.

{\it CAUTION:} A recursive rule such as
\begin{verbatim}
        let x = x + 1;
\end{verbatim}
is erroneous, since any subsequent evaluation of {\tt X} would lead to a
non-terminating chain of substitutions:
\begin{verbatim}
      x -> x + 1 -> (x + 1) + 1 -> ((x + 1) + 1) + 1 -> ...
\end{verbatim}
Similarly, coupled substitutions such as
\begin{verbatim}
        let l = m + n, n = l + r;
\end{verbatim}
would lead to the same error. As a result, if you try to evaluate an {\tt X},
{\tt L} or {\tt N} defined as above, you will get an error such as
\begin{verbatim}
        X improperly defined in terms of itself
\end{verbatim}

Array and matrix elements can appear on the left-hand side of a {\tt LET}
statement. However, because of their {\em instant evaluation\/}
\index{Instant evaluation} property, it is the value of the element that
is substituted for, rather than the element itself.  E.g.,
\begin{verbatim}
        array a(5);
        a(2) := b;
        let a(2) = c;
\end{verbatim}
results in {\tt B} being substituted by {\tt C}; the assignment for
{\tt a(2)} does not change.

Finally, if an error occurs in any equation in a {\tt LET} statement
(including generalized statements involving {\tt FOR ALL} and {\tt SUCH
THAT)}, the remaining rules are not evaluated.

\subsection{FOR ALL \ldots LET}\ttindex{FOR ALL}
If a substitution for all possible values of a given argument of an
operator is required, the declaration {\tt FOR ALL} may be used. The
syntax of such a command is
\begin{verbatim}
        FOR ALL <variable>,...,<variable>
                <LET statement> <terminator>
\end{verbatim}
e.g.,
\begin{verbatim}
        for all x,y let h(x,y) = x-y;
        for all x let k(x,y) = x^y;
\end{verbatim}
The first of these declarations would cause {\tt h(a,b)} to be evaluated
as {\tt A-B}, {\tt h(u+v,u+w)} to be {\tt V-W}, etc.  If the operator
symbol {\tt H} is used with more or fewer argument places, not two, the
{\tt LET} would have no effect, and no error would result.

The second declaration would cause {\tt k(a,y)} to be evaluated as
{\tt a\verb|^|y}, but would have no effect on {\tt k(a,z)} since the rule
didn't say {\tt FOR ALL Y} ... .

Where we used {\tt X} and {\tt Y} in the examples, any variables could
have been used.  This use of a variable doesn't affect the value it may
have outside the {\tt LET} statement.  However, you should remember what
variables you actually used.  If you want to delete the rule subsequently,
you must use the same variables in the {\tt CLEAR} command.

It is possible to use more complicated expressions as a template for a
{\tt LET} statement, as explained in the section on substitutions for
general expressions.  In nearly all cases, the rule will be accepted, and
a consistent application made by the system.  However, if there is a sole
constant or a sole free variable on the left-hand side of a rule (e.g.,
{\tt let 2=3} or {\tt for all x let x=2)}, then the system is unable to
handle the rule, and the error message
\begin{verbatim}
        Substitution for ... not allowed
\end{verbatim}
will be issued.  Any variable listed in the {\tt FOR ALL} part will have
its symbol preceded by an equal sign: {\tt X} in the above example will
appear as {\tt =X}.  An error will also occur if a variable in the
{\tt FOR ALL} part is not properly matched on both sides of the {\tt LET}
equation.

\subsection{FOR ALL \ldots SUCH THAT \ldots LET}
\ttindex{FOR ALL}\ttindex{SUCH THAT}

If a substitution is desired for more than a single value of a variable in
an operator or other expression, but not all values, a conditional form of
the {\tt FOR ALL \ldots LET} declaration can be used.

{\it Example:}
\begin{verbatim}
        for all x such that numberp x and x<0 let h(x)=0;
\end{verbatim}
will cause {\tt h(-5)} to be evaluated as 0, but {\tt H} of a positive
integer, or of an argument that is not an integer at all, would not be
affected.  Any boolean expression can follow the {\tt SUCH THAT} keywords.

\subsection{Removing Assignments and Substitution Rules}\ttindex{CLEAR}

The user may remove all assignments and substitution rules from any
expression by the command {\tt CLEAR}, in the form
\begin{verbatim}
        CLEAR <expression>,...,<expression><terminator>
\end{verbatim}
e.g.
\begin{verbatim}
        clear x, h(x,y);
\end{verbatim}
Because of their {\em instant evaluation\/} property, array and matrix elements
cannot be cleared with {\tt CLEAR}.  For example, if {\tt A} is an array,
you must say
\begin{verbatim}
        a(3) := 0;
\end{verbatim}
rather than
\begin{verbatim}
        clear a(3);
\end{verbatim}
to ``clear'' element {\tt a(3)}.

On the other hand, a whole array (or matrix) {\tt A} can be cleared by the
command {\tt clear a};  This means much more than resetting to 0 all the
elements of {\tt A}.  The fact that {\tt A} is an array, and what its
dimensions are, are forgotten, so {\tt A} can be redefined as another type
of object, for example an operator.

The more general types of {\tt LET} declarations can also be deleted by
using {\tt CLEAR}.  Simply repeat the {\tt LET} rule to be deleted, using
{\tt CLEAR} in place of {\tt LET}, and omitting the equal sign and
right-hand part.  The same dummy variables must be used in the {\tt FOR
ALL} part, and the boolean expression in the {\tt SUCH THAT} part must be
written the same way. (The placing of blanks doesn't have to be
identical.)

{\it Example:} The {\tt LET} rule
\begin{verbatim}
        for all x such that numberp x and x<0 let h(x)=0;
\end{verbatim}
can be erased by the command
\begin{verbatim}
        for all x such that numberp x and x<0 clear h(x);
\end{verbatim}

\subsection{Overlapping LET Rules}
{\tt CLEAR} is not the only way to delete a {\tt LET} rule.  A new {\tt
LET} rule identical to the first, but with a different expression after
the equal sign, replaces the first.  Replacements are also made in other
cases where the existing rule would be in conflict with the new rule.  For
example, a rule for {\tt x\verb|^|4} would replace a rule for {\tt x\verb|^|5}.
The user should however be cautioned against having several {\tt LET}
rules in effect that relate to the same expression.  No guarantee can be
given as to which rules will be applied by {\REDUCE} or in what order.  It
is best to {\tt CLEAR} an old rule before entering a new related {\tt LET}
rule.

\subsection{Substitutions for General Expressions}
\label{sec-gensubs}
The examples of substitutions discussed in other sections have involved
very simple rules. However, the substitution mechanism used in {\REDUCE} is
very general, and can handle arbitrarily complicated rules without
difficulty.

The general substitution mechanism used in {\REDUCE} is discussed in Hearn, A.
C., ``{\REDUCE}, A User-Oriented Interactive System for Algebraic
Simplification,'' Interactive Systems for Experimental Applied Mathematics,
(edited by M. Klerer and J. Reinfelds), Academic Press, New York (1968),
79-90, and Hearn. A. C., ``The Problem of Substitution,'' Proc. 1968 Summer
Institute on Symbolic Mathematical Computation, IBM Programming Laboratory
Report FSC 69-0312 (1969). For the reasons given in these
references, {\REDUCE} does not attempt to implement a general pattern
matching algorithm.  However, the present system uses far more sophisticated
techniques than those discussed in the above papers.  It is now possible for
the rules appearing in arguments of {\tt LET} to have the form
\begin{verbatim}
        <substitution expression> = <expression>
\end{verbatim}
where any rule to which a sensible meaning can be assigned is permitted.
However, this meaning can vary according to the form of {\tt <substitution
expression>}. The semantic rules associated with the application of the
substitution are completely consistent, but somewhat complicated by the
pragmatic need to perform such substitutions as efficiently as possible.
The following rules explain how the majority of the cases are handled.

To begin with, the {\tt <substitution expression>} is first partly
simplified by collecting like terms and putting identifiers (and kernels)
in the system order.  However, no substitutions are performed on any part
of the expression with the exception of expressions with the {\em instant
evaluation\/} property, such as array and matrix elements, whose actual
values are used.  It should also be noted that the system order used is
not changeable by the user, even with the {\tt KORDER} command.  Specific
cases are then handled as follows:
\begin{enumerate}
\item If the resulting simplified rule has a left-hand side that is an
identifier, an expression with a top-level algebraic operator or a power,
then the rule is added without further change to the appropriate table.

\item If the operator * appears at the top level of the simplified left-hand
side, then any constant arguments in that expression are moved to the
right-hand side of the rule.  The remaining left-hand side is then added
to the appropriate table.  For example,
\begin{verbatim}
        let 2*x*y=3
\end{verbatim}
becomes
\begin{verbatim}
        let x*y=3/2
\end{verbatim}
so that {\tt x*y} is added to the product substitution table, and when
this rule is applied, the expression {\tt x*y} becomes 3/2, but {\tt X} or
{\tt Y} by themselves are not replaced.

\item If the operators {\tt +}, {\tt -} or {\tt /} appear at the top level
of the simplified left-hand side, all but the first term is moved to the
right-hand side of the rule.  Thus the rules
\begin{verbatim}
        let l+m=n, x/2=y, a-b=c
\end{verbatim}
become
\begin{verbatim}
        let l=n-m, x=2*y, a=c+b.
\end{verbatim}
\end{enumerate}
One problem that can occur in this case is that if a quantified expression
is moved to the right-hand side, a given free variable might no longer
appear on the left-hand side, resulting in an error because of the
unmatched free variable. E.g.,
\begin{verbatim}
        for all x,y let f(x)+f(y)=x*y
\end{verbatim}
would become
\begin{verbatim}
        for all x,y let f(x)=x*y-f(y)
\end{verbatim}
which no longer has {\tt Y} on both sides.

The fact that array and matrix elements are evaluated in the left-hand side
of rules can lead to confusion at times. Consider for example the
statements
\begin{verbatim}
        array a(5); let x+a(2)=3; let a(3)=4;
\end{verbatim}
The left-hand side of the first rule will become {\tt X}, and the second
0.  Thus the first rule will be instantiated as a substitution for
{\tt X}, and the second will result in an error.

The order in which a list of rules is applied is not easily understandable
without a detailed knowledge of the system simplification protocol. It is
also possible for this order to change from release to release, as improved
substitution techniques are implemented. Users should therefore assume
that the order of application of rules is arbitrary, and program
accordingly.

After a substitution has been made, the expression being evaluated is
reexamined in case a new allowed substitution has been generated. This
process is continued until no more substitutions can be made.

As mentioned elsewhere, when a substitution expression appears in a
product, the substitution is made if that expression divides the product.
For example, the rule
\begin{verbatim}
	let a^2*c = 3*z;
\end{verbatim}
would cause {\tt a\verb|^|2*c*x} to be replaced by {\tt 3*Z*X} and
{\tt a\verb|^|2*c\verb|^|2} by {\tt 3*Z*C}.  If the substitution is desired only
when the substitution expression appears in a product with the explicit
powers supplied in the rule, the command {\tt MATCH} should be used
instead.\ttindex{MATCH}

For example,
\begin{verbatim}
        match a^2*c = 3*z;
\end{verbatim}
would cause {\tt a\verb|^|2*c*x} to be replaced by {\tt 3*Z*X}, but
{\tt a\verb|^|2*c\verb|^|2} would not be replaced. {\tt MATCH} can also be used
with the {\tt FOR ALL} constructions described above.

To remove substitution rules of the type discussed in this section, the
{\tt CLEAR}\ttindex{CLEAR} command can be used, combined, if necessary,
with the same {\tt FOR ALL} clause with which the rule was defined, for
example:
\begin{verbatim}
        for all x clear log(e^x),e^log(x),cos(w*t+theta(x));
\end{verbatim}
Note, however, that the arbitrary variable names in this case {\em must\/}
be the same as those used in defining the substitution.

\section{Rule Lists} \index{Rule lists}

Rule lists offer an alternative approach to defining substitutions that is
different from either {\tt SUB} or {\tt LET}.  In fact, they provide the
best features of both, since they have all the capabilities of {\tt LET},
but the rules can also be applied locally as is possible with {\tt SUB}.
In time, they will be used more and more in {\REDUCE}.  However, since they
are relatively new, much of the {\REDUCE} code you see uses the older
constructs.

A rule list is a list of {\em rules\/} that have the syntax
\begin{verbatim}
     <expression> => <expression> (WHEN <boolean expression>)
\end{verbatim}
For example,
\begin{verbatim}
	{cos(~x)*cos(~y) => (cos(x+y)+cos(x-y))/2,
	 cos(~n*pi)      => (-1)^n when remainder(n,2)=0}
\end{verbatim}

The tilde preceding a variable marks that variable as {\em free\/} for that
rule, much as a variable in a {\tt FOR ALL} clause in a {\tt LET}
statement.  The first occurrence of that variable in each relevant rule
must be so marked on input, otherwise inconsistent results can occur.
For example, the rule list
\begin{verbatim}
        {cos(~x)*cos(~y) => (cos(x+y)+cos(x-y))/2,
	 cos(x)^2        => (1+cos(2x))/2}
\end{verbatim}
designed to replace products of cosines, would not be correct, since the
second rule would only apply to the explicit argument {\tt X}.  Later
occurrences in the same rule may also be marked, but this is optional
(internally, all such rules are stored with each relevant variable
explicitly marked).  The optional {\tt WHEN}\ttindex{WHEN} clause allows
constraints to be placed on the application of the rule, much as the {\tt
SUCH THAT} clause in a {\tt LET} statement.

A rule list may be named, for example
\begin{verbatim}
        trig1 := {cos(~x)*cos(~y) => (cos(x+y)+cos(x-y))/2,
                  cos(~x)*sin(~y) => (sin(x+y)-sin(x-y))/2,
                  sin(~x)*sin(~y) => (cos(x-y)-cos(x+y))/2,
                  cos(~x)^2       => (1+cos(2*x))/2,
                  sin(~x)^2       => (1-cos(2*x))/2};
\end{verbatim}

Such named rule lists may be inspected as needed. E.g., the command
{\tt trig1;} would cause the above list to be printed.

Rule lists may be used in two ways.  They can be globally instantiated by
means of the command {\tt LET}.\ttindex{LET} For example,
\begin{verbatim}
        let trig1;
\end{verbatim}
would cause the above list of rules to be globally active from then on until
cancelled by the command {\tt CLEARRULES},\ttindex{CLEARRULES} as in
\begin{verbatim}
        clearrules trig1;
\end{verbatim}
{\tt CLEARRULES} has the syntax
\begin{verbatim}
        CLEARRULES <rule list>|<name of rule list>(,...) .
\end{verbatim}

The second way to use rule lists is to invoke them locally by means of a
{\tt WHERE}\ttindex{WHERE} clause.  For example
\begin{verbatim}
        cos(a)*cos(b+c)
	   where {cos(~x)*cos(~y) => (cos(x+y)+cos(x-y))/2};
\end{verbatim}
or
\begin{verbatim}
        cos(a)*sin(b) where trigrules;
\end{verbatim}

The syntax of an expression with a {\tt WHERE} clause is:
\begin{verbatim}
        <expression>
            WHERE <rule>|<rule list>(,<rule>|<rule list> ...)
\end{verbatim}
so the first example above could also be written
\begin{verbatim}
        cos(a)*cos(b+c)
           where cos(~x)*cos(~y) => (cos(x+y)+cos(x-y))/2;
\end{verbatim}

The effect of this construct is that the rule list(s) in the {\tt WHERE}
clause only apply to the expression on the left of {\tt WHERE}.  They have
no effect outside the expression.  In particular, they do not affect
previously defined {\tt WHERE} clauses or {\tt LET} statements.  For
example, the sequence
\begin{verbatim}
     let a=2;
     a where a=>4;
     a;
\end{verbatim}
would result in the output
\begin{verbatim}
     4

     2
\end{verbatim}

Although {\tt WHERE} has a precedence less than any other infix operator,
it still binds higher than keywords such as {\tt ELSE}, {\tt THEN},
{\tt DO}, {\tt REPEAT} and so on.  Thus the expression
\begin{verbatim}
        if a=2 then 3 else a+2 where a=3
\end{verbatim}
will parse as
\begin{verbatim}
        if a=2 then 3 else (a+2 where a=3)
\end{verbatim}

{\tt WHERE} may be used to introduce auxiliary variables in symbolic mode
expressions, as described in Section~\ref{sec-lambda}.  However, the
symbolic mode use has different semantics, so expressions do not carry
from one mode to the other.

\COMPATNOTE In order to provide compatibility with older versions of rule
lists released through the Network Library, it is currently possible to use
an equal sign interchangeably with the replacement sign {\tt =>} in rules
and {\tt LET} statements.  However, since this will change in future
versions, the replacement sign is preferable in rules and the equal sign
in non-rule-based {\tt LET} statements.

\subsection*{Advanced Use of Rule Lists}

Some advanced features of the rule list mechanism make it possible to
write more complicated rules than those discussed so far, and in many
cases to write more compact rule lists.  These features are:

\begin{itemize}
\item Free operators
\item Double slash operator
\item Double tilde variables.
\end{itemize}
A {\bf free operator} in the left hand side of a pattern will match any
operator with the same number of arguments.  The free operator is written
in the same style as a variable.  For example, the implementation of the
product rule of differentiation can be written as:
\begin{verbatim}
operator diff, !~f, !~g;

prule := {diff(~f(~x) * ~g(~x),x) =>
	     diff(f(x),x) * g(x) + diff(g(x),x) * f(x)};

let prule;

diff(sin(z)*cos(z),z);

         cos(z)*diff(sin(z),z) + diff(cos(z),z)*sin(z)
\end{verbatim}

The {\bf double slash operator} may be used as an alternative to a single
slash (quotient) in order to match quotients properly.  E.g., in the
example of the Gamma function above, one can use:
\begin{verbatim}
gammarule :=
   {gamma(~z)//(~c*gamma(~zz))  => gamma(z)/(c*gamma(zz-1)*zz)
		  when fixp(zz -z) and (zz -z) >0,
    gamma(~z)//gamma(~zz) => gamma(z)/(gamma(zz-1)*zz)
		  when fixp(zz -z) and (zz -z) >0};

let gammarule;

gamma(z)/gamma(z+3);

          1
----------------------
  3      2
 z  + 6*z  + 11*z + 6
\end{verbatim}
The above example suffers from the fact that two rules had to be
written in order to perform the required operation. This can be simplified
by the use of {\bf double tilde variables}. E.g. the rule list
\begin{verbatim}
 GGrule :=  {
    gamma(~z)//(~~c*gamma(~zz))  => gamma(z)/(c*gamma(zz-1)*zz)
     when fixp(zz -z) and (zz -z) >0};
\end{verbatim}
will implement the same operation in a much more compact way.
In general, double tilde variables are bound to the neutral element
with respect to the operation in which they are used.

\begin{tabular}{lll}

Pattern given & Argument used & Binding  \\
\\
\symbol{126}z + \symbol{126}\symbol{126}y  &   x   &  z=x; y=0  \\   
\symbol{126}z + \symbol{126}\symbol{126}y  &   x+3 &  z=x; y=3  or  z=3; y=x \\ 
\\
\symbol{126}z * \symbol{126}\symbol{126}y  &   x   &  z=x; y=1\\
\symbol{126}z * \symbol{126}\symbol{126}y  &   x*3 &  z=x; y=3  or  z=3; y=x\\
\\
\symbol{126}z / \symbol{126}\symbol{126}y  &    x   &  z=x; y=1\\
\symbol{126}z / \symbol{126}\symbol{126}y  &    x/3 &  z=x; y=3  \\
\\
\end{tabular}

Remarks: A double tilde variable as the numerator of a pattern is not allowed.
Also, using double tilde variables may lead to recursion errors when the
zero case is not handled properly.
\begin{verbatim}
let f(~~a * ~x,x)  => a * f(x,x) when freeof (a,x);

f(z,z);

***** f(z,z) improperly defined in terms of itself

% BUT:

let ff(~~a * ~x,x)
       => a * ff(x,x) when freeof (a,x) and a neq 1;

ff(z,z);
                 ff(z,z)

ff(3*z,z);
                 3*ff(z,z)
\end{verbatim}

\subsection*{Displaying Rules Associated with an Operator}

The operator {\tt SHOWRULES}\ttindex{SHOWRULES} takes a single identifier
as argument, and returns in rule-list form the operator rules associated
with that argument.  For example:
\begin{verbatim}
showrules log;

{LOG(E) => 1,

 LOG(1) => 0,

      ~X
 LOG(E  ) => ~X,

		    1
 DF(LOG(~X),~X) => ----}
		    ~X
\end{verbatim}

Such rules can then be manipulated further as with any list.  For example
{\tt rhs first ws;} has the value {\tt 1}.  Note that an operator may
have other properties that cannot be displayed in such a form, such as the
fact it is an odd function, or has a definition defined as a procedure.

\subsection*{Order of Application of Rules}

If rules have overlapping domains, their order of application is
important.  In general, it is very difficult to specify this order
precisely, so that it is best to assume that the order is arbitrary.
However, if only one operator is involved, the order of application of the
rules for this operator can be determined from the following:

\begin{enumerate}
\item Rules containing at least one free variable apply before all rules
without free variables.
\item Rules activated in the most recent {\tt LET}
command are applied first.
\item {\tt LET} with several entries generate
the same order of application as a corresponding sequence of commands with
one rule or rule set each.
\item Within a rule set, the rules containing at least
one free variable are applied in their given order.
In other words, the first member of the list is applied first.
\item Consistent with the first item, any rule in a rule list that
contains no free variables is applied after all rules containing free
variables.
\end{enumerate}
{\it Example:} The following rule set enables the computation of exact
values of the Gamma function:
\begin{verbatim}
        operator gamma,gamma_error;
        gamma_rules :=
        {gamma(~x)=>sqrt(pi)/2 when x=1/2,
         gamma(~n)=>factorial(n-1) when fixp n and n>0,
         gamma(~n)=>gamma_error(n) when fixp n,
         gamma(~x)=>(x-1)*gamma(x-1) when fixp(2*x) and x>1,
         gamma(~x)=>gamma(x+1)/x when fixp(2*x)};
\end{verbatim}
Here, rule by rule, cases of known or definitely uncomputable values
are sorted out; e.g. the rule leading to the error expression
will be applied for negative integers only, since the positive
integers are caught by the preceding rule, and the
last rule will apply for negative odd multiples of $1/2$ only.
Alternatively the first rule could have been written as
\begin{verbatim}
        gamma(1/2) => sqrt(pi)/2,
\end{verbatim}
but then the case $x=1/2$ should be excluded in the {\tt WHEN} part of the
last rule explicitly because a rule without free variables cannot take
precedence over the other rules.

\section{Asymptotic Commands} \index{Asymptotic command}
\label{sec-asymp}
In expansions of polynomials involving variables that are known to be
small, it is often desirable to throw away all powers of these variables
beyond a certain point to avoid unnecessary computation.  The command {\tt
LET} may be used to do this.  For example, if only powers of {\tt X} up to
{\tt x\verb|^|7} are needed, the command
\begin{verbatim}
        let x^8 = 0;
\end{verbatim}
will cause the system to delete all powers of {\tt X} higher than 7.

{\it CAUTION:}  This particular simplification works differently from most
substitution mechanisms in {\REDUCE} in that it is applied during
polynomial manipulation rather than to the whole evaluated expression.
Thus, with the above rule in effect, {\tt x\verb|^|10/x\verb|^|5} would give the
result zero, since the numerator would simplify to zero.  Similarly
{\tt x\verb|^|20/x\verb|^|10} would give a {\tt Zero divisor} error message,
since both numerator and denominator would first simplify to zero.

The method just described is not adequate when expressions involve several
variables having different degrees of smallness. In this case, it is
necessary to supply an asymptotic weight to each variable and count up the
total weight of each product in an expanded expression before deciding
whether to keep the term or not. There are two associated commands in the
system to permit this type of asymptotic constraint. The command {\tt WEIGHT}
\ttindex{WEIGHT}
takes a list of equations of the form
\begin{verbatim}
        <kernel form> = <number>
\end{verbatim}
where {\tt <number>} must be a positive integer (not just evaluate to a
positive integer).  This command assigns the weight {\tt <number>} to the
relevant kernel form.  A check is then made in all algebraic evaluations
to see if the total weight of the term is greater than the weight level
assigned to the calculation.  If it is, the term is deleted.  To compute
the total weight of a product, the individual weights of each kernel form
are multiplied by their corresponding powers and then added.

The weight level of the system is initially set to 1. The user may change
this setting by the command\ttindex{WTLEVEL}
\begin{verbatim}
        wtlevel <number>;
\end{verbatim}
which sets {\tt <number>} as the new weight level of the system.
{\tt <number>} must evaluate to a positive integer.  WTLEVEL will also
allow NIL as an argument, in which case the current weight level is returned.
